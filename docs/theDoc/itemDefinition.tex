
\newpage
%+++++++++++++++++++++++++++++++
%+++++++++++++++++++++++++++++++
\mysubsection{Nodes}

%+++++++++++++++++++++++++++++++++++
\mysubsubsection{NodePoint}
A 3D point node for point masses or solid finite elements which has 3 displacement degrees of freedom for second order differential equations (ODE2).
 \\\vspace{12pt} \noindent The item {\bf NodePoint} with type = 'Point' has the following parameters:\vspace{-1cm}\\ 
%reference manual TABLE
\begin{center}
  \footnotesize
  \begin{longtable}{| p{4.5cm} | p{2.5cm} | p{0.5cm} | p{2.5cm} | p{6cm} |}
    \hline
    \bf Name & \bf type & \bf size & \bf default value & \bf description \\ \hline
    name &     String &      &     '' &     node"s unique name\\ \hline
    referenceCoordinates &     Vector3D &     3 &     [0.,0.,0.] &     reference coordinates of node, e.g. ref. coordinates for finite elements; global position of node without displacement\\ \hline
    initialCoordinates &     Vector3D &     3 &     [0.,0.,0.] &     initial displacement coordinate\\ \hline
    initialVelocities &     Vector3D &     3 &     [0.,0.,0.] &     initial velocity coordinate\\ \hline
    visualization & VNodePoint & & & parameters for visualization of item \\ \hline
	  \end{longtable}
	\end{center}
The item VNodePoint has the following parameters:\vspace{-1cm}\\ 
%reference manual TABLE
\begin{center}
  \footnotesize
  \begin{longtable}{| p{4.5cm} | p{2.5cm} | p{0.5cm} | p{2.5cm} | p{6cm} |}
    \hline
    \bf Name & \bf type & \bf size & \bf default value & \bf description \\ \hline
    show &     bool &      &     True &     set true, if item is shown in visualization and false if it is not shown\\ \hline
    drawSize &     float &      &     -1. &     drawing size (diameter, dimensions of underlying cube, etc.)  for item; size == -1.f means that default size is used\\ \hline
    color &     Float4 &     4 &     [-1.,-1.,-1.,-1.] &     Default RGBA color for nodes; 4th value is alpha-transparency; R=-1.f means, that default color is used\\ \hline
	  \end{longtable}
	\end{center}

\noindent{\bf Short name} for Python: {\bf Point}
 \vspace{6pt}\\{\bf Definition of quantities}:\\
\startTable{input parameter}{symbol}{description see tables above}
\rowTable{referenceCoordinates}{$\qv\cRef = [q_0,\,q_1,\,q_2]\tp\cRef = \pv\cRef = [r_0,\,r_1,\,r_2]\tp$}{}
\rowTable{initialCoordinates}{$\qv\cIni = [q_0,\,q_1,\,q_2]\cIni\tp = \uv\cIni = [u_0,\,u_1,\,u_2]\cIni\tp$}{}
\rowTable{initialVelocities}{$\dot\qv\cIni = \vv\cIni = [\dot q_0,\,\dot q_1,\,\dot q_2]\cIni\tp$}{}
\finishTable
{\bf The following output parameters are available as OutputVariableType in sensors and other functions}:\\ 
\startTable{output parameters}{symbol}{description}
\rowTable{Position}{$\pv\cConfig = [p_0,\,p_1,\,p_2]\cConfig\tp= \uv\cConfig + \pv_{ref}$}{global 3D position vector of node; $\uv\cRef=0$}
\rowTable{Displacement}{$\uv\cConfig = [q_0,\,q_1,\,q_2]\cConfig\tp$}{global 3D displacement vector of node}
\rowTable{Velocity}{$\vv\cConfig = [\dot q_0,\,\dot q_1,\,\dot q_2]\cConfig\tp$}{global 3D velocity vector of node}
\rowTable{Coordinates}{$\cv\cConfig = \uv\cConfig = [q_0,\,q_1,\,q_2]\tp\cConfig$}{ coordinate vector of node}
\rowTable{Coordinates\_t}{$\dot\cv\cConfig = \vv\cConfig = [\dot q_0,\,\dot q_1,\,\dot q_2]\tp\cConfig$}{ velocity coordinates vector of node}
\finishTable
{\bf Description of Item}:
 \noindent
    %\startTable{intermediate variables}{symbol}{description}
    %  \rowTable{marker m0 position}{$\LU{0}{\pv}_{m0}$}{current global position which is provided by marker m0}
    %\finishTable
    The node provides $n_c=3$ displacement coordinates. According equations need to be provided by an according object (e.g., MassPoint, finite elements, ...).
    Usually, the nodal coordinates are provided in the global frame. However, the coordinate system is defined by the object (e.g. MassPoint uses global coordinates, but floating frame of reference objects use local frames).
    Note that for this very simple node, coordinates are identical to the nodal displacements, same for time derivatives. This is not the case, e.g. for nodes with orientation. \vspace{6pt}\\
    \noindent {\bf Example} for NodePoint: see ObjectMassPoint
\newpage

%+++++++++++++++++++++++++++++++++++
\mysubsubsection{NodePoint2D}
A 2D point node for point masses or solid finite elements which has 2 displacement degrees of freedom for second order differential equations.
 \\\vspace{12pt} \noindent The item {\bf NodePoint2D} with type = 'Point2D' has the following parameters:\vspace{-1cm}\\ 
%reference manual TABLE
\begin{center}
  \footnotesize
  \begin{longtable}{| p{4.5cm} | p{2.5cm} | p{0.5cm} | p{2.5cm} | p{6cm} |}
    \hline
    \bf Name & \bf type & \bf size & \bf default value & \bf description \\ \hline
    name &     String &      &     '' &     node"s unique name\\ \hline
    referenceCoordinates &     Vector2D &     2 &     [0.,0.] &     reference coordinates of node ==> e.g. ref. coordinates for finite elements; global position of node without displacement\\ \hline
    initialCoordinates &     Vector2D &     2 &     [0.,0.] &     initial displacement coordinate\\ \hline
    initialVelocities &     Vector2D &     2 &     [0.,0.] &     initial velocity coordinate\\ \hline
    visualization & VNodePoint2D & & & parameters for visualization of item \\ \hline
	  \end{longtable}
	\end{center}
The item VNodePoint2D has the following parameters:\vspace{-1cm}\\ 
%reference manual TABLE
\begin{center}
  \footnotesize
  \begin{longtable}{| p{4.5cm} | p{2.5cm} | p{0.5cm} | p{2.5cm} | p{6cm} |}
    \hline
    \bf Name & \bf type & \bf size & \bf default value & \bf description \\ \hline
    show &     bool &      &     True &     set true, if item is shown in visualization and false if it is not shown\\ \hline
    drawSize &     float &      &     -1. &     drawing size (diameter, dimensions of underlying cube, etc.)  for item; size == -1.f means that default size is used\\ \hline
    color &     Float4 &     4 &     [-1.,-1.,-1.,-1.] &     Default RGBA color for nodes; 4th value is alpha-transparency; R=-1.f means, that default color is used\\ \hline
	  \end{longtable}
	\end{center}

\noindent{\bf Short name} for Python: {\bf Point2D}
 \vspace{6pt}\\{\bf Definition of quantities}:\\
\startTable{input parameter}{symbol}{description see tables above}
\rowTable{referenceCoordinates}{$\qv\cRef = [q_0,\,q_1]\tp\cRef = \pv\cRef = [r_0,\,r_1]\tp$}{}
\rowTable{initialCoordinates}{$\qv\cIni = [q_0,\,q_1]\cIni\tp = [u_0,\,u_1]\cIni\tp$}{}
\rowTable{initialVelocities}{$\dot\qv\cIni = \vv\cIni = [\dot q_0,\,\dot q_1]\cIni\tp$}{}
\finishTable
{\bf The following output parameters are available as OutputVariableType in sensors and other functions}:\\ 
\startTable{output parameters}{symbol}{description}
\rowTable{Position}{$\pv\cConfig = [p_0,\,p_1,\,0]\cConfig\tp= \uv\cConfig + \pv_{ref}$}{global 3D position vector of node; $\uv\cRef=0$}
\rowTable{Displacement}{$\uv\cConfig = [q_0,\,q_1,\,0]\cConfig\tp$}{global 3D displacement vector of node}
\rowTable{Velocity}{$\vv\cConfig = [\dot q_0,\,\dot q_1,\,0]\cConfig\tp$}{global 3D velocity vector of node}
\rowTable{Coordinates}{$\cv\cConfig = [q_0,\,q_1]\tp\cConfig$}{ coordinate vector of node}
\rowTable{Coordinates\_t}{$\dot\cv\cConfig = [\dot q_0,\,\dot q_1]\tp\cConfig$}{ velocity coordinates vector of node}
\finishTable
{\bf Description of Item}:
 \noindent
    \vspace{6pt}\\
    {\bf Note the difference of coordinate vectors and displacement or position vectors}:
    \startTable{quantity}{symbol}{description}
      \rowTable{Coordinates}{$\cv\cConfig = \qv\cConfig = [q_0,\,q_1]\cConfig\tp = [u_0,\,u_1]\cConfig\tp \ldots$}{displacement coordinates}
      \rowTable{Displacement}{$\uv\cConfig = [u_0,\,u_1,\,0]\cConfig\tp$}{displacement vector, 0 in third component}
      \rowTable{Position}{$\pv\cConfig = [p_0,\,p_1,\,0]\cConfig\tp = [u_0,\,u_1,\,0]\cConfig\tp + [r_0,\,r_1,\,0]\cRef\tp$}{displacement vector, 0 in third component}
    \finishTable
    The node provides $n_c=2$ displacement coordinates. According equations need to be provided by an according object (e.g., MassPoint2D).
    Coordinates are identical to the nodal displacements, except for the third coordinate $u_2$, which is zero, because $q_2$ does not exist. \vspace{6pt}\\
    \noindent {\bf Example} for NodePoint: see ObjectMassPoint
\newpage

%+++++++++++++++++++++++++++++++++++
\mysubsubsection{NodeRigidBodyEP}
A 3D rigid body node based on Euler parameters for rigid bodies or beams; the node has 3 displacement coordinates (displacements of center of mass - COM: ux,uy,uz) and four rotation coordinates (Euler parameters = quaternions).
 \\\vspace{12pt} \noindent The item {\bf NodeRigidBodyEP} with type = 'RigidBodyEP' has the following parameters:\vspace{-1cm}\\ 
%reference manual TABLE
\begin{center}
  \footnotesize
  \begin{longtable}{| p{4.5cm} | p{2.5cm} | p{0.5cm} | p{2.5cm} | p{6cm} |}
    \hline
    \bf Name & \bf type & \bf size & \bf default value & \bf description \\ \hline
    name &     String &      &     '' &     node"s unique name\\ \hline
    referenceCoordinates &     Vector7D &     7 &     [0.,0.,0., 0.,0.,0.,0.] &     reference coordinates (3 position coordinates and 4 Euler parameters) of node ==> e.g. ref. coordinates for finite elements or reference position of rigid body (e.g. for definition of joints)\\ \hline
    initialCoordinates &     Vector7D &     7 &     [0.,0.,0., 0.,0.,0.,0.] &     initial displacement coordinates and 4 Euler parameters relative to reference coordinates\\ \hline
    initialVelocities &     Vector7D &     7 &     [0.,0.,0., 0.,0.,0.,0.] &     initial velocity coordinates: time derivatives of initial displacements and Euler parameters\\ \hline
    visualization & VNodeRigidBodyEP & & & parameters for visualization of item \\ \hline
	  \end{longtable}
	\end{center}
The item VNodeRigidBodyEP has the following parameters:\vspace{-1cm}\\ 
%reference manual TABLE
\begin{center}
  \footnotesize
  \begin{longtable}{| p{4.5cm} | p{2.5cm} | p{0.5cm} | p{2.5cm} | p{6cm} |}
    \hline
    \bf Name & \bf type & \bf size & \bf default value & \bf description \\ \hline
    show &     bool &      &     True &     set true, if item is shown in visualization and false if it is not shown\\ \hline
    drawSize &     float &      &     -1. &     drawing size (diameter, dimensions of underlying cube, etc.)  for item; size == -1.f means that default size is used\\ \hline
    color &     Float4 &     4 &     [-1.,-1.,-1.,-1.] &     Default RGBA color for nodes; 4th value is alpha-transparency; R=-1.f means, that default color is used\\ \hline
	  \end{longtable}
	\end{center}

\noindent{\bf Short name} for Python: {\bf RigidEP}
 \vspace{6pt}\\{\bf Definition of quantities}:\\
\startTable{input parameter}{symbol}{description see tables above}
\rowTable{referenceCoordinates}{$\qv\cRef = [q_0,\,q_1,\,q_2,\,\psi_0,\,\psi_1,\,\psi_2,\,\psi_3]\tp\cRef = [\pv\tp\cRef,\,\tpsi\tp\cRef]\tp$}{}
\rowTable{initialCoordinates}{$\qv\cIni = [q_0,\,q_1,\,q_2,\,\psi_0,\,\psi_1,\,\psi_2,\,\psi_3]\tp\cIni = [\uv\tp\cIni,\,\tpsi\tp\cIni]\tp$}{}
\rowTable{initialVelocities}{$\dot \qv\cIni = [\dot q_0,\,\dot q_1,\,\dot q_2,\,\dot \psi_0,\,\dot \psi_1,\,\dot \psi_2,\,\dot \psi_3]\tp\cIni = [\dot \uv\tp\cIni,\,\dot \tpsi\tp\cIni]\tp$}{}
\finishTable
{\bf The following output parameters are available as OutputVariableType in sensors and other functions}:\\ 
\startTable{output parameters}{symbol}{description}
\rowTable{Position}{$\LU{0}{\pv}\cConfig = \LU{0}{[p_0,\,p_1,\,p_2]}\cConfig\tp= \LU{0}{\uv}\cConfig + \LU{0}{\pv}_{ref}$}{global 3D position vector of node; $\uv\cRef=0$}
\rowTable{Displacement}{$\LU{0}{\uv}\cConfig = [q_0,\,q_1,\,q_2]\cConfig\tp$}{global 3D displacement vector of node}
\rowTable{Velocity}{$\LU{0}{\vv}\cConfig = [\dot q_0,\,\dot q_1,\,\dot q_2]\cConfig\tp$}{global 3D velocity vector of node}
\rowTable{Coordinates}{$\cv\cConfig = [q_0,\,q_1,\,q_2, \,\psi_0,\,\psi_1,\,\psi_2,\,\psi_3]\tp\cConfig$}{ coordinate vector of node, having 3 displacement coordinates and 4 Euler parameters}
\rowTable{Coordinates\_t}{$\dot\cv\cConfig = [\dot q_0,\,\dot q_1,\,\dot q_2, \,\dot \psi_0,\,\dot \psi_1,\,\dot \psi_2,\,\dot \psi_3]\tp\cConfig$}{ velocity coordinates vector of node}
\rowTable{RotationMatrix}{$[A_{00},\,A_{01},\,A_{02},\,A_{10},\,\ldots,\,A_{21},\,A_{22}]\cConfig\tp$}{vector with 9 components of the rotation matrix $\LU{0b}{\Rot}\cConfig$ in row-major format, in any configuration; the rotation matrix transforms local ($b$) to global (0) coordinates}
\rowTable{Rotation}{$[\varphi_0,\,\varphi_1,\,\varphi_2]\tp\cConfig$}{vector with 3 components of the Euler angles in xyz-sequence ($\LU{0b}{\Rot}\cConfig=:\Rot_0(\varphi_0) \cdot \Rot_1(\varphi_1) \cdot \Rot_2(\varphi_2)$), recomputed from rotation matrix}
\rowTable{AngularVelocity}{$\LU{0}{\tomega}\cConfig = \LU{0}{[\omega_0,\,\omega_1,\,\omega_2]}\cConfig\tp$}{global 3D angular velocity vector of node}
\rowTable{AngularVelocityLocal}{$\LU{b}{\tomega}\cConfig = \LU{b}{[\omega_0,\,\omega_1,\,\omega_2]}\cConfig\tp$}{local (body-fixed)  3D angular velocity vector of node}
\finishTable
{\bf Description of Item}:
 \noindent
    All coordinates $\cv\cConfig$ lead to second order differential equations, but there is one additional constraint equation for the quaternions.
    The additional constraint equation, which needs to be provided by the object, reads
    \be
      1 - \sum_{i=0}^{3} \theta_i^2 = 0.
    \ee
    The rotation matrix $\LU{0b}{\Rot}\cConfig$ transforms local (body-fixed) 3D positions $\LU{b}{\pv} = \LU{b}{[p_0,\,p_1,\,p_2]}\tp$ to global 3D positions,
    \be
      \LU{0}{\pv}\cConfig = \LU{0b}{\Rot}\cConfig \LU{b}{\pv} 
    \ee
    Note that the Euler parameters $\ttheta\cCur$ are computed as sum of current coordinates plus reference coordinates,
    \be
      \ttheta\cCur = \tpsi\cCur + \tpsi\cRef.
    \ee
    The rotation matrix is defined as function of the rotation parameters $\ttheta=[\theta_0,\,\theta_1,\,\theta_2,\,\theta_3]\tp$
    \be
      \LU{0b}{\Rot} = \mr{-2\theta_3^2 - 2\theta_2^2+1}{-2\theta_3\theta_0+2\theta_2\theta_1}{2*\theta_3\theta_1+2*\theta_2\theta_0} 
                         {2\theta_3\theta_0+2\theta_2\theta_1}{-2\theta_3^2-2\theta_1^2+1}{2\theta_3\theta_2-2\theta_1\theta_0}
                         {-2\theta_2\theta_0+2\theta_3\theta_1}{2\theta_3\theta_2+2\theta_1\theta_0}{-2\theta_2^2-2\theta_1^2+1}
    \ee
    The derivatives of the angular velocity vectors w.r.t.\ the rotation velocity coordinates $\dot \ttheta=[\dot \theta_0,\,\dot \theta_1,\,\dot \theta_2,\,\dot \theta_3]\tp$ lead to the $\Gm$ matrices, as used in the equations of motion for rigid bodies,
    \bea
      \LU{0}{\tomega} &=& \LU{0}{\Gm} \dot \ttheta, \\
      \LU{b}{\tomega} &=& \LU{b}{\Gm} \dot \ttheta.
		%return ConstSizeMatrix<3*maxRotCoordinates>(3, 4, {  -2.*ep[1], 2.*ep[0],-2.*ep[3], 2.*ep[2],
		%									-2.*ep[2], 2.*ep[3], 2.*ep[0],-2.*ep[1],
		%									-2.*ep[3],-2.*ep[2], 2.*ep[1], 2.*ep[0] });
		%return ConstSizeMatrix<3*maxRotCoordinates>(3, 4, {  -2.*ep[1], 2.*ep[0], 2.*ep[3],-2.*ep[2],
		%									-2.*ep[2],-2.*ep[3], 2.*ep[0], 2.*ep[1],
		%									-2.*ep[3], 2.*ep[2],-2.*ep[1], 2.*ep[0] });
    \eea
\newpage

%+++++++++++++++++++++++++++++++++++
\mysubsubsection{NodeRigidBodyRxyz}
A 3D rigid body node based on Euler / Tait-Bryan angles for rigid bodies or beams; all coordinates lead to second order differential equations; NOTE that this node has a singularity if the second rotation parameter reaches $\psi_1 = (2k-1) \pi/2$, with $k \in \Ncal$ or $-k \in \Ncal$.
 \\\vspace{12pt} \noindent The item {\bf NodeRigidBodyRxyz} with type = 'RigidBodyRxyz' has the following parameters:\vspace{-1cm}\\ 
%reference manual TABLE
\begin{center}
  \footnotesize
  \begin{longtable}{| p{4.5cm} | p{2.5cm} | p{0.5cm} | p{2.5cm} | p{6cm} |}
    \hline
    \bf Name & \bf type & \bf size & \bf default value & \bf description \\ \hline
    name &     String &      &     '' &     node"s unique name\\ \hline
    referenceCoordinates &     Vector6D &     6 &     [0.,0.,0., 0.,0.,0.] &     reference coordinates (3 position and 3 xyz Euler angles) of node ==> e.g. ref. coordinates for finite elements or reference position of rigid body (e.g. for definition of joints)\\ \hline
    initialCoordinates &     Vector6D &     6 &     [0.,0.,0., 0.,0.,0.] &     initial displacement coordinates: ux,uy,uz and 3 Euler angles (xyz) relative to reference coordinates\\ \hline
    initialVelocities &     Vector6D &     6 &     [0.,0.,0., 0.,0.,0.] &     initial velocity coordinate: time derivatives of ux,uy,uz and of 3 Euler angles (xyz)\\ \hline
    visualization & VNodeRigidBodyRxyz & & & parameters for visualization of item \\ \hline
	  \end{longtable}
	\end{center}
The item VNodeRigidBodyRxyz has the following parameters:\vspace{-1cm}\\ 
%reference manual TABLE
\begin{center}
  \footnotesize
  \begin{longtable}{| p{4.5cm} | p{2.5cm} | p{0.5cm} | p{2.5cm} | p{6cm} |}
    \hline
    \bf Name & \bf type & \bf size & \bf default value & \bf description \\ \hline
    show &     bool &      &     True &     set true, if item is shown in visualization and false if it is not shown\\ \hline
    drawSize &     float &      &     -1. &     drawing size (diameter, dimensions of underlying cube, etc.)  for item; size == -1.f means that default size is used\\ \hline
    color &     Float4 &     4 &     [-1.,-1.,-1.,-1.] &     Default RGBA color for nodes; 4th value is alpha-transparency; R=-1.f means, that default color is used\\ \hline
	  \end{longtable}
	\end{center}

\noindent{\bf Short name} for Python: {\bf RigidRxyz}
 \vspace{6pt}\\{\bf Definition of quantities}:\\
\startTable{input parameter}{symbol}{description see tables above}
\rowTable{referenceCoordinates}{$\qv\cRef = [q_0,\,q_1,\,q_2,\,\psi_0,\,\psi_1,\,\psi_2]\tp\cRef = [\pv\tp\cRef,\,\tpsi\tp\cRef]\tp$}{}
\rowTable{initialCoordinates}{$\qv\cIni = [q_0,\,q_1,\,q_2,\,\psi_0,\,\psi_1,\,\psi_2]\tp\cIni = [\uv\tp\cIni,\,\tpsi\tp\cIni]\tp$}{}
\rowTable{initialVelocities}{$\dot \qv\cIni = [\dot q_0,\,\dot q_1,\,\dot q_2,\,\dot \psi_0,\,\dot \psi_1,\,\dot \psi_2]\tp\cIni = [\dot \uv\tp\cIni,\,\dot \tpsi\tp\cIni]\tp$}{}
\finishTable
{\bf The following output parameters are available as OutputVariableType in sensors and other functions}:\\ 
\startTable{output parameters}{symbol}{description}
\rowTable{Position}{$\LU{0}{\pv}\cConfig = \LU{0}{[p_0,\,p_1,\,p_2]}\cConfig\tp= \LU{0}{\uv}\cConfig + \LU{0}{\pv}_{ref}$}{global 3D position vector of node; $\uv\cRef=0$}
\rowTable{Displacement}{$\LU{0}{\uv}\cConfig = [q_0,\,q_1,\,q_2]\cConfig\tp$}{global 3D displacement vector of node}
\rowTable{Velocity}{$\LU{0}{\vv}\cConfig = [\dot q_0,\,\dot q_1,\,\dot q_2]\cConfig\tp$}{global 3D velocity vector of node}
\rowTable{Coordinates}{$\cv\cConfig = [q_0,\,q_1,\,q_2, \,\psi_0,\,\psi_1,\,\psi_2]\tp\cConfig$}{ coordinate vector of node, having 3 displacement coordinates and 3 Euler angles}
\rowTable{Coordinates\_t}{$\dot\cv\cConfig = [\dot q_0,\,\dot q_1,\,\dot q_2, \,\dot \psi_0,\,\dot \psi_1,\,\dot \psi_2]\tp\cConfig$}{ velocity coordinates vector of node}
\rowTable{RotationMatrix}{$[A_{00},\,A_{01},\,A_{02},\,A_{10},\,\ldots,\,A_{21},\,A_{22}]\cConfig\tp$}{vector with 9 components of the rotation matrix $\LU{0b}{\Rot}\cConfig$ in row-major format, in any configuration; the rotation matrix transforms local ($b$) to global (0) coordinates}
\rowTable{Rotation}{$[\varphi_0,\,\varphi_1,\,\varphi_2]\tp\cConfig = [\psi_0,\,\psi_1,\,\psi_2]\tp\cRef + [\psi_0,\,\psi_1,\,\psi_2]\tp\cConfig$}{vector with 3 components of the Euler / Tait-Bryan angles in xyz-sequence ($\LU{0b}{\Rot}\cConfig=:\Rot_0(\varphi_0) \cdot \Rot_1(\varphi_1) \cdot \Rot_2(\varphi_2)$), recomputed from rotation matrix}
\rowTable{AngularVelocity}{$\LU{0}{\tomega}\cConfig = \LU{0}{[\omega_0,\,\omega_1,\,\omega_2]}\cConfig\tp$}{global 3D angular velocity vector of node}
\rowTable{AngularVelocityLocal}{$\LU{b}{\tomega}\cConfig = \LU{b}{[\omega_0,\,\omega_1,\,\omega_2]}\cConfig\tp$}{local (body-fixed)  3D angular velocity vector of node}
\finishTable
{\bf Description of Item}:
 \noindent
    The node has 3 displacement coordinates $[q_0,\,q_1,\,q_2]\tp$ and three rotation coordinates $[\psi_0,\,\psi_1,\,\psi_2]\tp$ for consecutive rotations around the 0, 1 and 2-axis ($x$, $y$ and $z$).
    All coordinates $\cv\cConfig$ lead to second order differential equations.
    The rotation matrix $\LU{0b}{\Rot}\cConfig$ transforms local (body-fixed) 3D positions $\LU{b}{\pv} = \LU{b}{[p_0,\,p_1,\,p_2]}\tp$ to global 3D positions,
    \be
      \LU{0}{\pv}\cConfig = \LU{0b}{\Rot}\cConfig \LU{b}{\pv} 
    \ee
    Note that the Euler angles $\ttheta\cCur$ are computed as sum of current coordinates plus reference coordinates,
    \be
      \ttheta\cCur = \tpsi\cCur + \tpsi\cRef.
    \ee
    The rotation matrix is defined as function of the rotation parameters $\ttheta=[\theta_0,\,\theta_1,\,\theta_2]\tp$
    \be
      \LU{0b}{\Rot} = \Rot_0(\theta_0)\Rot_1(\theta_1)\Rot_2(\theta_2)
    \ee
    The derivatives of the angular velocity vectors w.r.t.\ the rotation velocity coordinates $\dot \ttheta=[\dot \theta_0,\,\dot \theta_1,\,\dot \theta_2,\,\dot \theta_3]\tp$ lead to the $\Gm$ matrices, as used in the equations of motion for rigid bodies,
    \bea
      \LU{0}{\tomega} &=& \LU{0}{\Gm} \dot \ttheta, \\
      \LU{b}{\tomega} &=& \LU{b}{\Gm} \dot \ttheta.
    \eea
\newpage

%+++++++++++++++++++++++++++++++++++
\mysubsubsection{NodeRigidBodyRotVecLG}
A 3D rigid body node based on rotation vector and Lie group methods for rigid bodies or beams; the node has 3 displacement coordinates and three rotation coordinates.
 \\\vspace{12pt} \noindent The item {\bf NodeRigidBodyRotVecLG} with type = 'RigidBodyRotVecLG' has the following parameters:\vspace{-1cm}\\ 
%reference manual TABLE
\begin{center}
  \footnotesize
  \begin{longtable}{| p{4.5cm} | p{2.5cm} | p{0.5cm} | p{2.5cm} | p{6cm} |}
    \hline
    \bf Name & \bf type & \bf size & \bf default value & \bf description \\ \hline
    name &     String &      &     '' &     node"s unique name\\ \hline
    referenceCoordinates &     Vector6D &     3 &     [0.,0.,0., 0.,0.,0.] &     reference coordinates (position and rotation vector $\tnu$) of node ==> e.g. ref. coordinates for finite elements or reference position of rigid body (e.g. for definition of joints)\\ \hline
    initialCoordinates &     Vector6D &     3 &     [0.,0.,0., 0.,0.,0.] &     initial displacement coordinates $\uv$ and rotation vector $\tnu$ relative to reference coordinates\\ \hline
    initialVelocities &     Vector6D &     3 &     [0.,0.,0., 0.,0.,0.] &     initial velocity coordinate: time derivatives of displacement and angular velocity vector\\ \hline
    visualization & VNodeRigidBodyRotVecLG & & & parameters for visualization of item \\ \hline
	  \end{longtable}
	\end{center}
The item VNodeRigidBodyRotVecLG has the following parameters:\vspace{-1cm}\\ 
%reference manual TABLE
\begin{center}
  \footnotesize
  \begin{longtable}{| p{4.5cm} | p{2.5cm} | p{0.5cm} | p{2.5cm} | p{6cm} |}
    \hline
    \bf Name & \bf type & \bf size & \bf default value & \bf description \\ \hline
    show &     bool &      &     True &     set true, if item is shown in visualization and false if it is not shown\\ \hline
    drawSize &     float &      &     -1. &     drawing size (diameter, dimensions of underlying cube, etc.)  for item; size == -1.f means that default size is used\\ \hline
    color &     Float4 &     4 &     [-1.,-1.,-1.,-1.] &     Default RGBA color for nodes; 4th value is alpha-transparency; R=-1.f means, that default color is used\\ \hline
	  \end{longtable}
	\end{center}

\noindent{\bf Short name} for Python: {\bf RigidRotVecLG}
 \vspace{6pt}\\{\bf Definition of quantities}:\\
\startTable{input parameter}{symbol}{description see tables above}
\rowTable{referenceCoordinates}{$\qv\cRef = [q_0,\,q_1,\,q_2,\,\nu_0,\,\nu_1,\,\nu_2]\tp\cRef = [\pv\tp\cRef,\,\tnu\tp\cRef]\tp$}{}
\rowTable{initialCoordinates}{$\qv\cIni = [q_0,\,q_1,\,q_2,\,\nu_0,\,\nu_1,\,\nu_2]\tp\cIni = [\uv\tp\cIni,\,\tnu\tp\cIni]\tp$}{}
\rowTable{initialVelocities}{$\dot \qv\cIni = [\dot q_0,\,\dot q_1,\,\dot q_2,\,\dot \nu_0,\,\dot \nu_1,\,\dot \nu_2]\tp\cIni = [\dot \uv\tp\cIni,\,\dot \tnu\tp\cIni]\tp$}{}
\finishTable
{\bf The following output parameters are available as OutputVariableType in sensors and other functions}:\\ 
\startTable{output parameters}{symbol}{description}
\rowTable{Position}{$\LU{0}{\pv}\cConfig = \LU{0}{[p_0,\,p_1,\,p_2]}\cConfig\tp= \LU{0}{\uv}\cConfig + \LU{0}{\pv}_{ref}$}{global 3D position vector of node; $\uv\cRef=0$}
\rowTable{Displacement}{$\LU{0}{\uv}\cConfig = [q_0,\,q_1,\,q_2]\cConfig\tp$}{global 3D displacement vector of node}
\rowTable{Velocity}{$\LU{0}{\vv}\cConfig = [\dot q_0,\,\dot q_1,\,\dot q_2]\cConfig\tp$}{global 3D velocity vector of node}
\rowTable{Coordinates}{$\cv\cConfig = [q_0,\,q_1,\,q_2, \,\nu_0,\,\nu_1,\,\nu_2]\tp\cConfig$}{ coordinate vector of node, having 3 displacement coordinates and 3 Euler angles}
\rowTable{Coordinates\_t}{$\dot\cv\cConfig = [\dot q_0,\,\dot q_1,\,\dot q_2, \,\dot \nu_0,\,\dot \nu_1,\,\dot \nu_2]\tp\cConfig$}{ velocity coordinates vector of node}
\rowTable{RotationMatrix}{$[A_{00},\,A_{01},\,A_{02},\,A_{10},\,\ldots,\,A_{21},\,A_{22}]\cConfig\tp$}{vector with 9 components of the rotation matrix $\LU{0b}{\Rot}\cConfig$ in row-major format, in any configuration; the rotation matrix transforms local ($b$) to global (0) coordinates}
\rowTable{Rotation}{$[\varphi_0,\,\varphi_1,\,\varphi_2]\tp\cConfig$}{vector with 3 components of the Euler / Tait-Bryan angles in xyz-sequence ($\LU{0b}{\Rot}\cConfig=:\Rot_0(\varphi_0) \cdot \Rot_1(\varphi_1) \cdot \Rot_2(\varphi_2)$), recomputed from rotation matrix}
\rowTable{AngularVelocity}{$\LU{0}{\tomega}\cConfig = \LU{0}{[\omega_0,\,\omega_1,\,\omega_2]}\cConfig\tp$}{global 3D angular velocity vector of node}
\rowTable{AngularVelocityLocal}{$\LU{b}{\tomega}\cConfig = \LU{b}{[\omega_0,\,\omega_1,\,\omega_2]}\cConfig\tp$}{local (body-fixed)  3D angular velocity vector of node}
\finishTable
{\bf Description of Item}:
 \noindent
    The node has 3 displacement coordinates $[q_0,\,q_1,\,q_2]\tp$ and three rotation coordinates, which is the rotation vector 
    \be
      \tnu = \varphi \nv = \tnu\cConfig + \tnu\cRef,
    \ee
    with the rotation angle $\varphi$ and the rotation axis $\nv$.
    All coordinates $\cv\cConfig$ lead to second order differential equations, however the rotation vector cannot be used as a conventional parameterization. It must be computed within a nonlinear update, using appropriate Lie group methods.

    The rotation matrix $\LU{0b}{\Rot}\cConfig$ transforms local (body-fixed) 3D positions $\LU{b}{\pv} = \LU{b}{[p_0,\,p_1,\,p_2]}\tp$ to global 3D positions,
    \be
      \LU{0}{\pv}\cConfig = \LU{0b}{\Rot}\cConfig \LU{b}{\pv} 
    \ee
    Note that the rotation vector $\tnu\cCur$ are computed as sum of current coordinates plus reference coordinates,
    \be
      \ttheta\cCur = \tnu\cCur + \tnu\cRef \quad \mathrm{with}.
    \ee
    %The rotation matrix is defined as function of the rotation parameters $\ttheta=[\theta_0,\,\theta_1,\,\theta_2]\tp$
    %\be
    %  \LU{0b}{\Rot} = \Rot_0(\theta_0)\Rot_1(\theta_1)\Rot_2(\theta_2)
    %\ee
    The derivatives of the angular velocity vectors w.r.t.\ the rotation velocity coordinates $\dot \ttheta=[\dot \theta_0,\,\dot \theta_1,\,\dot \theta_2,\,\dot \theta_3]\tp$ lead to the $\Gm$ matrices, as used in the equations of motion for rigid bodies,
    \bea
      \LU{0}{\tomega} &=& \LU{0}{\Gm} \dot \ttheta, \\
      \LU{b}{\tomega} &=& \LU{b}{\Gm} \dot \ttheta.
    \eea
\newpage

%+++++++++++++++++++++++++++++++++++
\mysubsubsection{NodeRigidBody2D}
A 2D rigid body node for rigid bodies or beams; the node has 2 displacement degrees of freedom and one rotation coordinate (rotation around z-axis: uphi). All coordinates are ODE2, used for second order differetial equations.
 \\\vspace{12pt} \noindent The item {\bf NodeRigidBody2D} with type = 'RigidBody2D' has the following parameters:\vspace{-1cm}\\ 
%reference manual TABLE
\begin{center}
  \footnotesize
  \begin{longtable}{| p{4.5cm} | p{2.5cm} | p{0.5cm} | p{2.5cm} | p{6cm} |}
    \hline
    \bf Name & \bf type & \bf size & \bf default value & \bf description \\ \hline
    name &     String &      &     '' &     node"s unique name\\ \hline
    referenceCoordinates &     Vector3D &     3 &     [0.,0.,0.] &     reference coordinates (x-pos,y-pos and rotation) of node ==> e.g. ref. coordinates for finite elements; global position of node without displacement\\ \hline
    initialCoordinates &     Vector3D &     3 &     [0.,0.,0.] &     initial displacement coordinates and angle (relative to reference coordinates)\\ \hline
    initialVelocities &     Vector3D &     3 &     [0.,0.,0.] &     initial velocity coordinates\\ \hline
    visualization & VNodeRigidBody2D & & & parameters for visualization of item \\ \hline
	  \end{longtable}
	\end{center}
The item VNodeRigidBody2D has the following parameters:\vspace{-1cm}\\ 
%reference manual TABLE
\begin{center}
  \footnotesize
  \begin{longtable}{| p{4.5cm} | p{2.5cm} | p{0.5cm} | p{2.5cm} | p{6cm} |}
    \hline
    \bf Name & \bf type & \bf size & \bf default value & \bf description \\ \hline
    show &     bool &      &     True &     set true, if item is shown in visualization and false if it is not shown\\ \hline
    drawSize &     float &      &     -1. &     drawing size (diameter, dimensions of underlying cube, etc.)  for item; size == -1.f means that default size is used\\ \hline
    color &     Float4 &     4 &     [-1.,-1.,-1.,-1.] &     Default RGBA color for nodes; 4th value is alpha-transparency; R=-1.f means, that default color is used\\ \hline
	  \end{longtable}
	\end{center}

\noindent{\bf Short name} for Python: {\bf Rigid2D}
 \vspace{6pt}\\{\bf Definition of quantities}:\\
\startTable{input parameter}{symbol}{description see tables above}
\rowTable{referenceCoordinates}{$\qv\cRef = [q_0,\,q_1,\,\psi_0]\tp\cRef$}{}
\rowTable{initialCoordinates}{$\qv\cIni = [q_0,\,q_1,\,\psi_0]\tp\cIni$}{}
\rowTable{initialVelocities}{$\dot \qv\cIni = [\dot q_0,\,\dot q_1,\,\dot \psi_0]\tp\cIni =  [v_0,\,v_1,\,\omega_2]\tp\cIni$}{}
\finishTable
{\bf The following output parameters are available as OutputVariableType in sensors and other functions}:\\ 
\startTable{output parameters}{symbol}{description}
\rowTable{Position}{$\LU{0}{\pv}\cConfig = \LU{0}{[p_0,\,p_1,\,0]}\cConfig\tp= \LU{0}{\uv}\cConfig + \LU{0}{\pv}_{ref}$}{global 3D position vector of node; $\uv\cRef=0$}
\rowTable{Displacement}{$\LU{0}{\uv}\cConfig = [q_0,\,q_1,\,0]\cConfig\tp$}{global 3D displacement vector of node}
\rowTable{Velocity}{$\LU{0}{\vv}\cConfig = [\dot q_0,\,\dot q_1,\,0]\cConfig\tp$}{global 3D velocity vector of node}
\rowTable{Coordinates}{$\cv\cConfig = [q_0,\,q_1,\,\psi_0]\tp\cConfig$}{ coordinate vector of node, having 2 displacement coordinates and 1 angle}
\rowTable{Coordinates\_t}{$\dot\cv\cConfig = [\dot q_0,\,\dot q_1,\,\dot \psi_0]\tp\cConfig$}{ velocity coordinates vector of node}
\rowTable{RotationMatrix}{$[A_{00},\,A_{01},\,A_{02},\,A_{10},\,\ldots,\,A_{21},\,A_{22}]\cConfig\tp$}{vector with 9 components of the rotation matrix $\LU{0b}{\Rot}\cConfig$ in row-major format, in any configuration; the rotation matrix transforms local ($b$) to global (0) coordinates}
\rowTable{Rotation}{$[\theta_0]\tp\cConfig = [\psi_0]\tp\cRef + [\psi_0]\tp\cConfig$}{vector with 1 angle around out of plane axix}
\rowTable{AngularVelocity}{$\LU{0}{\tomega}\cConfig = \LU{0}{[\omega_0,\,\omega_1,\,\omega_2]}\cConfig\tp$}{global 3D angular velocity vector of node}
\rowTable{AngularVelocityLocal}{$\LU{b}{\tomega}\cConfig = \LU{b}{[\omega_0,\,\omega_1,\,\omega_2]}\cConfig\tp$}{local (body-fixed)  3D angular velocity vector of node}
\finishTable
{\bf Description of Item}:
 \noindent
    The node provides 2 displacement coordinates (displacement of center of mass, COM, ($q_0,q_1$) ) and 1 rotation parameter ($\theta_0$). According equations need to be provided by an according object (e.g., RigidBody2D).
    Using the rotation parameter $\theta_{0\mathrm{config}} = \psi_{0ref} + \psi_{0\mathrm{config}}$, the rotation matrix is defined as
    \be
      \LU{0b}{\Rot}\cConfig = \mr{\cos(\theta_0)}{-\sin(\theta_0)}{0}{\sin(\theta_0)}{\cos(\theta_0)}{0} {0}{0}{1}\cConfig
    \ee
    \noindent {\bf Example} for NodeRigidBody2D: see ObjectRigidBody2D
\newpage

%+++++++++++++++++++++++++++++++++++
\mysubsubsection{NodePoint2DSlope1}
A 2D point/slope vector node for planar Bernoulli-Euler ANCF (absolute nodal coordinate formulation) beam elements; the node has 4 displacement degrees of freedom (2 for displacement of point node and 2 for the slope vector 'slopex'); all coordinates lead to second order differential equations; the slope vector defines the directional derivative w.r.t the local axial (x) coordinate, denoted as $()^\prime$; in straight configuration aligned at the global x-axis, the slope vector reads $\rv^\prime=[r_x^\prime\;\;r_y^\prime]^T=[1\;\;0]^T$.
 \\\vspace{12pt} \noindent The item {\bf NodePoint2DSlope1} with type = 'Point2DSlope1' has the following parameters:\vspace{-1cm}\\ 
%reference manual TABLE
\begin{center}
  \footnotesize
  \begin{longtable}{| p{4.5cm} | p{2.5cm} | p{0.5cm} | p{2.5cm} | p{6cm} |}
    \hline
    \bf Name & \bf type & \bf size & \bf default value & \bf description \\ \hline
    name &     String &      &     '' &     node"s unique name\\ \hline
    referenceCoordinates &     Vector4D &     4 &     [0.,0.,1.,0.] &     reference coordinates (x-pos,y-pos; x-slopex, y-slopex) of node; global position of node without displacement\\ \hline
    initialCoordinates &     Vector4D &     4 &     [0.,0.,0.,0.] &     initial displacement coordinates: ux, uy and x/y "displacements" of slopex\\ \hline
    initialVelocities &     Vector4D &     4 &     [0.,0.,0.,0.] &     initial velocity coordinates\\ \hline
    visualization & VNodePoint2DSlope1 & & & parameters for visualization of item \\ \hline
	  \end{longtable}
	\end{center}
The item VNodePoint2DSlope1 has the following parameters:\vspace{-1cm}\\ 
%reference manual TABLE
\begin{center}
  \footnotesize
  \begin{longtable}{| p{4.5cm} | p{2.5cm} | p{0.5cm} | p{2.5cm} | p{6cm} |}
    \hline
    \bf Name & \bf type & \bf size & \bf default value & \bf description \\ \hline
    show &     bool &      &     True &     set true, if item is shown in visualization and false if it is not shown\\ \hline
    drawSize &     float &      &     -1. &     drawing size (diameter, dimensions of underlying cube, etc.)  for item; size == -1.f means that default size is used\\ \hline
    color &     Float4 &     4 &     [-1.,-1.,-1.,-1.] &     Default RGBA color for nodes; 4th value is alpha-transparency; R=-1.f means, that default color is used\\ \hline
	  \end{longtable}
	\end{center}

\noindent{\bf Short name} for Python: {\bf Point2DS1}
 \vspace{6pt}\\{\bf Definition of quantities}:\\
{\bf The following output parameters are available as OutputVariableType in sensors and other functions}:\\ 
\startTable{output parameters}{symbol}{description}
\rowTable{Position}{}{global 3D position vector of node (=displacement+reference position)}
\rowTable{Displacement}{}{global 3D displacement vector of node}
\rowTable{Velocity}{}{global 3D velocity vector of node}
\rowTable{Coordinates}{}{coordinates vector of node (2 displacement coordinates + 2 slope vector coordinates)}
\rowTable{Coordinates\_t}{}{velocity coordinates vector of node (derivative of the 2 displacement coordinates + 2 slope vector coordinates)}
\finishTable
\newpage

%+++++++++++++++++++++++++++++++++++
\mysubsubsection{NodeGenericODE2}
A node containing a number of ODE2 variables; use e.g. for scalar dynamic equations (Mass1D) or for the ALECable element.
 \\\vspace{12pt} \noindent The item {\bf NodeGenericODE2} with type = 'GenericODE2' has the following parameters:\vspace{-1cm}\\ 
%reference manual TABLE
\begin{center}
  \footnotesize
  \begin{longtable}{| p{4.5cm} | p{2.5cm} | p{0.5cm} | p{2.5cm} | p{6cm} |}
    \hline
    \bf Name & \bf type & \bf size & \bf default value & \bf description \\ \hline
    name &     String &      &     '' &     node"s unique name\\ \hline
    referenceCoordinates &     Vector &      &     [] &     generic reference coordinates of node; must be consistent with numberOfODE2Coordinates\\ \hline
    initialCoordinates &     Vector &      &     [] &     initial displacement coordinates; must be consistent with numberOfODE2Coordinates\\ \hline
    initialCoordinates\_t &     Vector &      &     [] &     initial velocity coordinates; must be consistent with numberOfODE2Coordinates\\ \hline
    numberOfODE2Coordinates &     Index &      &     0 &     number of generic ODE2 coordinates\\ \hline
    visualization & VNodeGenericODE2 & & & parameters for visualization of item \\ \hline
	  \end{longtable}
	\end{center}
The item VNodeGenericODE2 has the following parameters:\vspace{-1cm}\\ 
%reference manual TABLE
\begin{center}
  \footnotesize
  \begin{longtable}{| p{4.5cm} | p{2.5cm} | p{0.5cm} | p{2.5cm} | p{6cm} |}
    \hline
    \bf Name & \bf type & \bf size & \bf default value & \bf description \\ \hline
    show &     bool &      &     False &     set true, if item is shown in visualization and false if it is not shown\\ \hline
	  \end{longtable}
	\end{center}
{\bf Definition of quantities}:\\
\startTable{input parameter}{symbol}{description see tables above}
\rowTable{referenceCoordinates}{$\qv\cRef = [q_0,\,\ldots,\,q_{nc}]\tp\cRef$}{}
\rowTable{initialCoordinates}{$\qv\cIni = [q_0,\,\ldots,\,q_{nc}]\tp\cIni$}{}
\rowTable{initialCoordinates\_t}{$\dot \qv\cIni = [\dot q_0,\,\ldots,\,\dot q_{nc}]\tp\cIni$}{}
\rowTable{numberOfODE2Coordinates}{$n_c$}{}
\finishTable
{\bf The following output parameters are available as OutputVariableType in sensors and other functions}:\\ 
\startTable{output parameters}{symbol}{description}
\rowTable{Coordinates}{$\qv\cConfig = [q_0,\,\ldots,\,q_{nc}]\tp\cConfig$}{coordinates vector of node}
\rowTable{Coordinates\_t}{$\dot \qv\cConfig = [\dot q_0,\,\ldots,\,\dot q_{nc}]\tp\cConfig$}{velocity coordinates vector of node}
\finishTable
\newpage

%+++++++++++++++++++++++++++++++++++
\mysubsubsection{NodeGenericData}
A node containing a number of data (history) variables; use e.g. for contact (active set), friction or plasticity (history variable).
 \\\vspace{12pt} \noindent The item {\bf NodeGenericData} with type = 'GenericData' has the following parameters:\vspace{-1cm}\\ 
%reference manual TABLE
\begin{center}
  \footnotesize
  \begin{longtable}{| p{4.5cm} | p{2.5cm} | p{0.5cm} | p{2.5cm} | p{6cm} |}
    \hline
    \bf Name & \bf type & \bf size & \bf default value & \bf description \\ \hline
    name &     String &      &     '' &     node"s unique name\\ \hline
    initialCoordinates &     Vector &      &     [] &     initial data coordinates\\ \hline
    numberOfDataCoordinates &     Index &      &     0 &     number of generic data coordinates (history variables)\\ \hline
    visualization & VNodeGenericData & & & parameters for visualization of item \\ \hline
	  \end{longtable}
	\end{center}
The item VNodeGenericData has the following parameters:\vspace{-1cm}\\ 
%reference manual TABLE
\begin{center}
  \footnotesize
  \begin{longtable}{| p{4.5cm} | p{2.5cm} | p{0.5cm} | p{2.5cm} | p{6cm} |}
    \hline
    \bf Name & \bf type & \bf size & \bf default value & \bf description \\ \hline
    show &     bool &      &     False &     set true, if item is shown in visualization and false if it is not shown\\ \hline
	  \end{longtable}
	\end{center}
{\bf Definition of quantities}:\\
{\bf The following output parameters are available as OutputVariableType in sensors and other functions}:\\ 
\startTable{output parameters}{symbol}{description}
\rowTable{Coordinates}{}{data coordinates (history variables) vector of node}
\finishTable
\newpage

%+++++++++++++++++++++++++++++++++++
\mysubsubsection{NodePointGround}
A 3D point node fixed to ground. The node can be used as NodePoint, but it does not generate coordinates. Applied or reaction forces do not have any effect.
 \\\vspace{12pt} \noindent The item {\bf NodePointGround} with type = 'PointGround' has the following parameters:\vspace{-1cm}\\ 
%reference manual TABLE
\begin{center}
  \footnotesize
  \begin{longtable}{| p{4.5cm} | p{2.5cm} | p{0.5cm} | p{2.5cm} | p{6cm} |}
    \hline
    \bf Name & \bf type & \bf size & \bf default value & \bf description \\ \hline
    name &     String &      &     '' &     node"s unique name\\ \hline
    referenceCoordinates &     Vector3D &     3 &     [0.,0.,0.] &     reference coordinates of node ==> e.g. ref. coordinates for finite elements; global position of node without displacement\\ \hline
    visualization & VNodePointGround & & & parameters for visualization of item \\ \hline
	  \end{longtable}
	\end{center}
The item VNodePointGround has the following parameters:\vspace{-1cm}\\ 
%reference manual TABLE
\begin{center}
  \footnotesize
  \begin{longtable}{| p{4.5cm} | p{2.5cm} | p{0.5cm} | p{2.5cm} | p{6cm} |}
    \hline
    \bf Name & \bf type & \bf size & \bf default value & \bf description \\ \hline
    show &     bool &      &     True &     set true, if item is shown in visualization and false if it is not shown\\ \hline
    drawSize &     float &      &     -1. &     drawing size (diameter, dimensions of underlying cube, etc.)  for item; size == -1.f means that default size is used\\ \hline
    color &     Float4 &     4 &     [-1.,-1.,-1.,-1.] &     Default RGBA color for nodes; 4th value is alpha-transparency; R=-1.f means, that default color is used\\ \hline
	  \end{longtable}
	\end{center}

\noindent{\bf Short name} for Python: {\bf PointGround}
 \vspace{6pt}\\{\bf Definition of quantities}:\\
{\bf The following output parameters are available as OutputVariableType in sensors and other functions}:\\ 
\startTable{output parameters}{symbol}{description}
\rowTable{Position}{}{global 3D position vector of node (=reference position)}
\rowTable{Displacement}{}{zero 3D vector}
\rowTable{Velocity}{}{zero 3D vector}
\rowTable{Coordinates}{}{vector of length zero}
\rowTable{Coordinates\_t}{}{vector of length zero}
\finishTable

\newpage
%+++++++++++++++++++++++++++++++
%+++++++++++++++++++++++++++++++
\mysubsection{Objects}

%+++++++++++++++++++++++++++++++++++
\mysubsubsection{ObjectMassPoint}
A 3D mass point which is attached to a position-based node, usually NodePoint.
 \\\vspace{12pt} \noindent The item {\bf ObjectMassPoint} with type = 'MassPoint' has the following parameters:\vspace{-1cm}\\ 
%reference manual TABLE
\begin{center}
  \footnotesize
  \begin{longtable}{| p{4.5cm} | p{2.5cm} | p{0.5cm} | p{2.5cm} | p{6cm} |}
    \hline
    \bf Name & \bf type & \bf size & \bf default value & \bf description \\ \hline
    name &     String &      &     '' &     objects"s unique name\\ \hline
    physicsMass &     UReal &      &     0. &     mass [SI:kg] of mass point\\ \hline
    nodeNumber &     Index &      &     MAXINT &     node number for mass point\\ \hline
    visualization & VObjectMassPoint & & & parameters for visualization of item \\ \hline
	  \end{longtable}
	\end{center}
The item VObjectMassPoint has the following parameters:\vspace{-1cm}\\ 
%reference manual TABLE
\begin{center}
  \footnotesize
  \begin{longtable}{| p{4.5cm} | p{2.5cm} | p{0.5cm} | p{2.5cm} | p{6cm} |}
    \hline
    \bf Name & \bf type & \bf size & \bf default value & \bf description \\ \hline
    show &     bool &      &     True &     set true, if item is shown in visualization and false if it is not shown\\ \hline
    graphicsData &     BodyGraphicsData &     \tabnewline  &      &     Structure contains data for body visualization; data is defined in special list / dictionary structure\\ \hline
	  \end{longtable}
	\end{center}

\noindent{\bf Short name} for Python: {\bf MassPoint}
 \vspace{6pt}\\{\bf Definition of quantities}:\\
{\bf The following output parameters are available as OutputVariableType in sensors and other functions}:\\ 
\startTable{output parameters}{symbol}{description}
\rowTable{Position}{}{global position vector of translated local position}
\rowTable{Displacement}{}{global displacement vector of center point}
\rowTable{Velocity}{}{global velocity vector of center point}
\finishTable
{\bf Description of Item}:
 \noindent
    \vspace{6pt}\\
    {\bf Definition of quantities}:
    \bi
      \item $m \ldots$ physicsMass
      \item $n0 \ldots$ node number
      \item $\cv\body  = \cv_{n0} (= [q_0,\;q_1,\;q_2]\tp) \ldots$ displacement coordinates of body (taken from NodePoint)
      \item $\fv = [f_0,\;f_1,\;f_2]\tp \ldots$ residual of all forces (loads, constraints, springs, ...)
      \item $\pv\cRef = \cv\cRef = [q_0,\;q_1,\;q_2]\cRef\tp \ldots$ reference position = reference coordinates of node
      \item $\pv\cConfig = \uv\cConfig + \pv\cRef \ldots$ position in any configuration ($\uv\cRef = 0$)
      \item $\pv\cCur = \uv\cCur + \pv\cRef \ldots$ current position, equals to node's reference position + current coordinates
    \ei
    {\bf Equations of motion}:
    \be 
      \mr{m}{0}{0} {0}{m}{0} {0}{0}{m} \vr{\ddot q_0}{\ddot q_1}{\ddot q_2} = \vr{f_0}{f_1}{f_2}.
    \ee
    For example, a LoadCoordinate on coordinate 1 of the node would add a term in $f_1$ on the RHS.
    
    Position-based markers can measure position $\pv\cConfig$. The {\bf position jacobian}  
    \be
      \Jm_{pos} = \partial \pv\cCur / \partial \cv\cCur = \mr{1}{0}{0} {0}{1}{0} {0}{0}{0}
    \ee
    transforms the action of global forces $\LU{0}{\fv}$ of position-based markers on the coordinates $\cv$
    \be
      \Qm = \Jm_{pos} \LU{0}{\fv}.
    \ee
\noindent {\bf Example} for ObjectMassPoint:
\pythonstyle
\begin{lstlisting}[language=Python, firstnumber=1]
    node = mbs.AddNode(NodePoint(referenceCoordinates = [1,1,0], 
                                 initialCoordinates=[0.5,0,0],
                                 initialVelocities=[0.5,0,0]))
    mbs.AddObject(MassPoint(nodeNumber = node, physicsMass=1))

    #assemble and solve system for default parameters
    mbs.Assemble()
    SC.TimeIntegrationSolve(mbs, 'GeneralizedAlpha', exu.SimulationSettings())

    #check result
    testError = mbs.GetNodeOutput(node, exu.OutputVariableType.Position)[0] - 2 
    #final x-coordinate of position shall be 2

\end{lstlisting}

\newpage

%+++++++++++++++++++++++++++++++++++
\mysubsubsection{ObjectMassPoint2D}
A 2D mass point which is attached to a position-based 2D node. Equations of motion with the displacements $[u_x\;\; u_y]^T$, the mass $m$ and the residual of all forces $[R_x\;\; R_y]^T$ are given as \be \vp{m \cdot \ddot u_x}{m \cdot \ddot u_y} = \vp{R_x}{R_y}.\ee
 \\\vspace{12pt} \noindent The item {\bf ObjectMassPoint2D} with type = 'MassPoint2D' has the following parameters:\vspace{-1cm}\\ 
%reference manual TABLE
\begin{center}
  \footnotesize
  \begin{longtable}{| p{4.5cm} | p{2.5cm} | p{0.5cm} | p{2.5cm} | p{6cm} |}
    \hline
    \bf Name & \bf type & \bf size & \bf default value & \bf description \\ \hline
    name &     String &      &     '' &     objects"s unique name\\ \hline
    physicsMass &     UReal &      &     0. &     mass [SI:kg] of mass point\\ \hline
    nodeNumber &     Index &      &     MAXINT &     node number for mass point\\ \hline
    visualization & VObjectMassPoint2D & & & parameters for visualization of item \\ \hline
	  \end{longtable}
	\end{center}
The item VObjectMassPoint2D has the following parameters:\vspace{-1cm}\\ 
%reference manual TABLE
\begin{center}
  \footnotesize
  \begin{longtable}{| p{4.5cm} | p{2.5cm} | p{0.5cm} | p{2.5cm} | p{6cm} |}
    \hline
    \bf Name & \bf type & \bf size & \bf default value & \bf description \\ \hline
    show &     bool &      &     True &     set true, if item is shown in visualization and false if it is not shown\\ \hline
    graphicsData &     BodyGraphicsData &     \tabnewline  &      &     Structure contains data for body visualization; data is defined in special list / dictionary structure\\ \hline
	  \end{longtable}
	\end{center}

\noindent{\bf Short name} for Python: {\bf MassPoint2D}
 \vspace{6pt}\\{\bf Definition of quantities}:\\
{\bf The following output parameters are available as OutputVariableType in sensors and other functions}:\\ 
\startTable{output parameters}{symbol}{description}
\rowTable{Position}{}{global position vector of translated local position}
\rowTable{Displacement}{}{global displacement vector of center point}
\rowTable{Velocity}{}{global velocity vector of center point}
\finishTable
\newpage

%+++++++++++++++++++++++++++++++++++
\mysubsubsection{ObjectRigidBody}
A 3D rigid body which is attached to a 3D rigid body node. Equations of motion with the displacements $[u_x\;\; u_y\;\; u_z]^T$ of the center of mass and the rotation parameters (Euler parameters) $\mathbf{q}$, the mass $m$, inertia $\mathbf{J} = [J_{xx}, J_{xy}, J_{xz}; J_{yx}, J_{yy}, J_{yz}; J_{zx}, J_{zy}, J_{zz}]$ and the residual of all forces and moments $[R_x\;\; R_y\;\; R_z\;\; R_{q0}\;\; R_{q1}\;\; R_{q2}\;\; R_{q3}]^T$ are given as ...
 \\\vspace{12pt} \noindent The item {\bf ObjectRigidBody} with type = 'RigidBody' has the following parameters:\vspace{-1cm}\\ 
%reference manual TABLE
\begin{center}
  \footnotesize
  \begin{longtable}{| p{4.5cm} | p{2.5cm} | p{0.5cm} | p{2.5cm} | p{6cm} |}
    \hline
    \bf Name & \bf type & \bf size & \bf default value & \bf description \\ \hline
    name &     String &      &     '' &     objects"s unique name\\ \hline
    physicsMass &     UReal &      &     0. &     mass [SI:kg] of mass point\\ \hline
    physicsInertia &     Vector6D &      &     [0.,0.,0., 0.,0.,0.] &     inertia components [SI:kgm$^2$]: $[J_{xx}, J_{yy}, J_{zz}, J_{yz}, J_{xz}, J_{xy}]$ of rigid body w.r.t. center of mass\\ \hline
    nodeNumber &     Index &      &     MAXINT &     node number for rigid body node\\ \hline
    visualization & VObjectRigidBody & & & parameters for visualization of item \\ \hline
	  \end{longtable}
	\end{center}
The item VObjectRigidBody has the following parameters:\vspace{-1cm}\\ 
%reference manual TABLE
\begin{center}
  \footnotesize
  \begin{longtable}{| p{4.5cm} | p{2.5cm} | p{0.5cm} | p{2.5cm} | p{6cm} |}
    \hline
    \bf Name & \bf type & \bf size & \bf default value & \bf description \\ \hline
    show &     bool &      &     True &     set true, if item is shown in visualization and false if it is not shown\\ \hline
    graphicsData &     BodyGraphicsData &     \tabnewline  &      &     Structure contains data for body visualization; data is defined in special list / dictionary structure\\ \hline
	  \end{longtable}
	\end{center}

\noindent{\bf Short name} for Python: {\bf RigidBody}
 \vspace{6pt}\\{\bf Definition of quantities}:\\
{\bf The following output parameters are available as OutputVariableType in sensors and other functions}:\\ 
\startTable{output parameters}{symbol}{description}
\rowTable{Position}{}{global position vector of rotated and translated local position}
\rowTable{Displacement}{}{global displacement vector of local position}
\rowTable{RotationMatrix}{}{vector with 9 components of the rotation matrix (row-major format)}
\rowTable{Rotation}{}{vector with 3 components of the Euler angles in xyz-sequence (R=Rx*Ry*Rz), recomputed from rotation matrix}
\rowTable{Velocity}{}{global velocity vector of local position}
\rowTable{AngularVelocity}{}{angular velocity of body}
\rowTable{AngularVelocityLocal}{}{local (body-fixed) 3D velocity vector of node}
\finishTable
\newpage

%+++++++++++++++++++++++++++++++++++
\mysubsubsection{ObjectRigidBody2D}
A 2D rigid body which is attached to a rigid body 2D node. The body obtains coordinates, position, velocity, etc. from the underlying 2D node
 \\\vspace{12pt} \noindent The item {\bf ObjectRigidBody2D} with type = 'RigidBody2D' has the following parameters:\vspace{-1cm}\\ 
%reference manual TABLE
\begin{center}
  \footnotesize
  \begin{longtable}{| p{4.5cm} | p{2.5cm} | p{0.5cm} | p{2.5cm} | p{6cm} |}
    \hline
    \bf Name & \bf type & \bf size & \bf default value & \bf description \\ \hline
    name &     String &      &     '' &     objects"s unique name\\ \hline
    physicsMass &     UReal &      &     0. &     mass [SI:kg] of mass point\\ \hline
    physicsInertia &     UReal &      &     0. &     inertia [SI:kgm$^2$] of rigid body w.r.t. center of mass\\ \hline
    nodeNumber &     Index &      &     MAXINT &     node number for 2D rigid body node\\ \hline
    visualization & VObjectRigidBody2D & & & parameters for visualization of item \\ \hline
	  \end{longtable}
	\end{center}
The item VObjectRigidBody2D has the following parameters:\vspace{-1cm}\\ 
%reference manual TABLE
\begin{center}
  \footnotesize
  \begin{longtable}{| p{4.5cm} | p{2.5cm} | p{0.5cm} | p{2.5cm} | p{6cm} |}
    \hline
    \bf Name & \bf type & \bf size & \bf default value & \bf description \\ \hline
    show &     bool &      &     True &     set true, if item is shown in visualization and false if it is not shown\\ \hline
    graphicsData &     BodyGraphicsData &     \tabnewline  &      &     Structure contains data for body visualization; data is defined in special list / dictionary structure\\ \hline
	  \end{longtable}
	\end{center}

\noindent{\bf Short name} for Python: {\bf RigidBody2D}
 \vspace{6pt}\\{\bf Definition of quantities}:\\
{\bf The following output parameters are available as OutputVariableType in sensors and other functions}:\\ 
\startTable{output parameters}{symbol}{description}
\rowTable{Position}{}{global position vector of rotated and translated local position}
\rowTable{Displacement}{}{global displacement vector of local position}
\rowTable{Velocity}{}{global velocity vector of local position}
\rowTable{Rotation}{}{scalar rotation angle of body}
\rowTable{AngularVelocity}{}{angular velocity of body}
\rowTable{RotationMatrix}{}{rotation matrix in vector form (stored in row-major order)}
\finishTable
{\bf Description of Item}:
 \noindent
    \vspace{6pt}\\
    {\bf Definition of quantities}:
    \bi
      \item $m \ldots$ physicsMass: total body mass
      \item $J \ldots$ physicsInertia: momentinertia w.r.t.\ axis 2
      \item $n0 \ldots$ node number
      \item $\cv\body  = \cv_{n0} (= [q_0,\;q_1,\;\psi_2]_{n0}\tp) \ldots$ displacement coordinates of body taken from NodeRigidBody2D
      \item $\Qm = [f_0,\;f_1,\;\tau_2] \ldots$ residual of all generalized forces (incl.~torques), e.g., loads, constraints, springs, ...
    \ei
    Global position of a local body-fixed position, in any configuration
    \be
      \LU{0}{\pv}\cConfig = \LU{0}{\pv}\cRef + \LU{0}{\uv}\cConfig + \LU{0b}{\Rot}\cConfig \LU{b}{\pv}
    \ee
    {\bf Equations of motion}:
    \be 
      \mr{m}{0}{0} {0}{m}{0} {0}{0}{J} \vr{\ddot q_0}{\ddot q_1}{\ddot \psi_2} = \vr{f_0}{f_1}{\tau_2} = \Qm.
    \ee
    For example, a LoadCoordinate on coordinate 1 of the node would add a term in $f_1$ on the RHS.
    
    Position-based markers can measure position $\pv\cConfig$. 
    With the body rotation $\theta = \theta_{2,n0}$ and the localPosition $\LU{b}{\pv} = [\LU{b}{p_0},\,\LU{b}{p_1},\,0]\tp$,
    the {\bf position jacobian} is defined as,
    \be
      \Jm_{pos} = \partial \pv\cCur / \partial \cv\cCur = \mr{1}{0}{-\sin(\theta)\LU{b}{p_0} - \cos(\theta)\LU{b}{p_1}} 
                                                             {0}{1}{\cos(\theta)\LU{b}{p_0}-\sin(\theta)\LU{b}{p_1}} {0}{0}{0}
    \ee
    which transforms the action of global forces $\LU{0}{\fv}$  of position-based markers on the coordinates $\cv$,
    \be
      \Qm = \Jm_{pos} \LU{0}{\fv}
    \ee
    Orientation-based markers can measure the rotation matrix $\Rot\cConfig$. 
    The {\bf rotation jacobian}  
    \be
      \Jm_{rot} = \partial \pv\cCur / \partial \cv\cCur = \mr{0}{0}{0} {0}{0}{0} {0}{0}{1}
    \ee
    transforms the action of global torques $\LU{0}{\ttau}$ of orientation-based markers on the coordinates $\cv$,
    \be
      \Qm = \Jm_{rot} \LU{0}{\ttau}
    \ee
\noindent {\bf Example} for ObjectRigidBody2D:
\pythonstyle
\begin{lstlisting}[language=Python, firstnumber=1]
    node = mbs.AddNode(NodeRigidBody2D(referenceCoordinates = [1,1,0.25*np.pi], 
                                       initialCoordinates=[0.5,0,0],
                                       initialVelocities=[0.5,0,0.75*np.pi]))
    mbs.AddObject(RigidBody2D(nodeNumber = node, physicsMass=1, physicsInertia=2))

    #assemble and solve system for default parameters
    mbs.Assemble()
    SC.TimeIntegrationSolve(mbs, 'GeneralizedAlpha', exu.SimulationSettings())

    #check result
    testError = mbs.GetNodeOutput(node, exu.OutputVariableType.Position)[0] - 2
    testError+= mbs.GetNodeOutput(node, exu.OutputVariableType.Coordinates)[2] - 0.75*np.pi
    #final x-coordinate of position shall be 2, angle theta shall be np.pi

\end{lstlisting}

\newpage

%+++++++++++++++++++++++++++++++++++
\mysubsubsection{ObjectGenericODE2}
A system of $n$ second order ordinary differential equations (ODE2), having a mass matrix, damping/gyroscopic matrix, stiffness matrix and generalized forces. It can combine generic nodes, or node points. User functions can be used to compute mass matrix and generalized forces depending on given coordinates. NOTE that all matrices, vectors, etc. must have the same dimensions $n$ or $(n \times n)$, or they must be empty $(0 \times 0)$, except for the mass matrix which always needs to have dimensions $(n \times n)$.
 \\\vspace{12pt} \noindent The item {\bf ObjectGenericODE2} with type = 'GenericODE2' has the following parameters:\vspace{-1cm}\\ 
%reference manual TABLE
\begin{center}
  \footnotesize
  \begin{longtable}{| p{4.5cm} | p{2.5cm} | p{0.5cm} | p{2.5cm} | p{6cm} |}
    \hline
    \bf Name & \bf type & \bf size & \bf default value & \bf description \\ \hline
    name &     String &      &     '' &     objects"s unique name\\ \hline
    nodeNumbers &     ArrayIndex &      &     [] &     node numbers which provide the coordinates for the object (consecutively as provided in this list)\\ \hline
    massMatrix &     NumpyMatrix &      &     Matrix[] &     mass matrix of object in python numpy format\\ \hline
    stiffnessMatrix &     NumpyMatrix &      &     Matrix[] &     stiffness matrix of object in python numpy format\\ \hline
    dampingMatrix &     NumpyMatrix &      &     Matrix[] &     damping matrix of object in python numpy format\\ \hline
    forceVector &     NumpyVector &      &     [] &     generalized force vector added to RHS\\ \hline
    forceUserFunction &     PyFunctionVectorScalar2Vector &     \tabnewline  &     \tabnewline 0 &     A python user function which computes the generalized user force vector for the ODE2 equations; The function takes the time, coordinates q (without reference values) and coordinate velocities q\_t; Example for python function with numpy stiffness matrix K: def f(t, q, q\_t): return np.dot(K, q)\\ \hline
    massMatrixUserFunction &     PyFunctionMatrixScalar2Vector &     \tabnewline  &     \tabnewline 0 &     A python user function which computes the mass matrix instead of the constant mass matrix; The function takes the time, coordinates q (without reference values) and coordinate velocities q\_t; Example (academic) for python function with numpy stiffness matrix M: def f(t, q, q\_t): return (q[0]+1)*M\\ \hline
    visualization & VObjectGenericODE2 & & & parameters for visualization of item \\ \hline
	  \end{longtable}
	\end{center}
The item VObjectGenericODE2 has the following parameters:\vspace{-1cm}\\ 
%reference manual TABLE
\begin{center}
  \footnotesize
  \begin{longtable}{| p{4.5cm} | p{2.5cm} | p{0.5cm} | p{2.5cm} | p{6cm} |}
    \hline
    \bf Name & \bf type & \bf size & \bf default value & \bf description \\ \hline
    show &     bool &      &     True &     set true, if item is shown in visualization and false if it is not shown\\ \hline
	  \end{longtable}
	\end{center}
{\bf Definition of quantities}:\\
\startTable{input parameter}{symbol}{description see tables above}
\rowTable{nodeNumbers}{$\mathbf{n}_n = [n_0,\,\ldots,\,n_n]\tp$}{}
\rowTable{massMatrix}{$\Mm \in \Rcal^{n \times n}$}{}
\rowTable{stiffnessMatrix}{$\Km \in \Rcal^{n \times n}$}{}
\rowTable{dampingMatrix}{$\Dm \in \Rcal^{n \times n}$}{}
\rowTable{forceVector}{$\fv \in \Rcal^{n}$}{}
\rowTable{forceUserFunction}{$\fv_{user} \in \Rcal^{n}$}{}
\rowTable{massMatrixUserFunction}{$\Mm_{user} \in \Rcal^{n\times n}$}{}
\finishTable
{\bf The following output parameters are available as OutputVariableType in sensors and other functions}:\\ 
\startTable{output parameters}{symbol}{description}
\rowTable{Coordinates}{}{all ODE2 coordinates}
\rowTable{Coordinates\_t}{}{all ODE2 velocity coordinates}
\rowTable{Force}{}{generalized forces for all coordinates (residual of all forces except mass*accleration; corresponds to ComputeODE2RHS)}
\finishTable
{\bf Description of Item}:
 \noindent
    An object with node numbers $[n_0,\,\ldots,\,n_n]$ and according numbers of nodal coordinates $[n_{c_0},\,\ldots,\,n_{c_n}]$, the total number of equations (=coordinates) of the object is
    \be
      n = \sum_{i} n_{c_i}.
    \ee
    {\bf Equations of motion}:
    \be \label{eq_ObjectGenericODE2_EOM}
      \Mm \ddot \qv + \Dm \dot \qv + \Km \qv = \fv + \fv_{user}(t,\qv,\dot \qv)
    \ee
    Note that the user function $\fv_{user}(t,\qv,\dot \qv)$ may be empty (=0). 
    
    In case that a user mass matrix is specified, \eq{eq_ObjectGenericODE2_EOM} is replaced with
    \be
      \Mm_{user}(t,\qv,\dot \qv) \ddot \qv + \Dm \dot \qv + \Km \qv = \fv + \fv_{user}(t,\qv,\dot \qv)
    \ee
    CoordinateLoads are integrated for each ODE2 coordinate on the RHS of the latter equation.
\newpage

%+++++++++++++++++++++++++++++++++++
\mysubsubsection{ObjectANCFCable2D}
A 2D cable finite element using 2 nodes of type NodePoint2DSlope1; the element has 8 coordinates and uses cubic polynomials for position interpolation; the Bernoulli-Euler beam is capable of large deformation as it employs the material measure of curvature for the bending.
 \\\vspace{12pt} \noindent The item {\bf ObjectANCFCable2D} with type = 'ANCFCable2D' has the following parameters:\vspace{-1cm}\\ 
%reference manual TABLE
\begin{center}
  \footnotesize
  \begin{longtable}{| p{4.5cm} | p{2.5cm} | p{0.5cm} | p{2.5cm} | p{6cm} |}
    \hline
    \bf Name & \bf type & \bf size & \bf default value & \bf description \\ \hline
    name &     String &      &     '' &     objects"s unique name\\ \hline
    physicsLength &     UReal &      &     0. &     reference length $L$ [SI:m] of beam; such that the total volume (e.g. for volume load) gives $\rho A L$\\ \hline
    physicsMassPerLength &     UReal &      &     0. &     mass $\rho A$ [SI:kg/m$^2$] of beam\\ \hline
    physicsBendingStiffness &     UReal &      &     0. &     bending stiffness $EI$ [SI:Nm$^2$] of beam; the bending moment is $m = EI (\kappa - \kappa_0)$, in which $\kappa$ is the material measure of curvature\\ \hline
    physicsAxialStiffness &     UReal &      &     0. &     axial stiffness $EA$ [SI:N] of beam; the axial force is $f_{ax} = EA (\varepsilon -\varepsilon_0)$, in which $\varepsilon = |\rv^\prime|-1$ is the axial strain\\ \hline
    physicsBendingDamping &     UReal &      &     0. &     bending damping $d_{EI}$ [SI:Nm$^2$/s] of beam; the additional virtual work due to damping is $\delta W_{\dot \kappa} = \int_0^L \dot \kappa \delta \kappa dx$\\ \hline
    physicsAxialDamping &     UReal &      &     0. &     axial stiffness $d_{EA}$ [SI:N/s] of beam; the additional virtual work due to damping is $\delta W_{\dot\varepsilon} = \int_0^L \dot \varepsilon \delta \varepsilon dx$\\ \hline
    physicsReferenceAxialStrain &     UReal &      &     0. &     reference axial strain of beam (pre-deformation) $\varepsilon_0$ [SI:1] of beam; without external loading the beam will statically keep the reference axial strain value\\ \hline
    physicsReferenceCurvature &     UReal &      &     0. &     reference curvature of beam (pre-deformation) $\kappa_0$ [SI:1/m] of beam; without external loading the beam will statically keep the reference curvature value\\ \hline
    nodeNumbers &     Index2 &      &     [MAXINT, MAXINT] &     two node numbers ANCF cable element\\ \hline
    useReducedOrderIntegration &     Bool &      &     False &     false: use Gauss order 9 integration for virtual work of axial forces, order 5 for virtual work of bending moments; true: use Gauss order 7 integration for virtual work of axial forces, order 3 for virtual work of bending moments\\ \hline
    visualization & VObjectANCFCable2D & & & parameters for visualization of item \\ \hline
	  \end{longtable}
	\end{center}
The item VObjectANCFCable2D has the following parameters:\vspace{-1cm}\\ 
%reference manual TABLE
\begin{center}
  \footnotesize
  \begin{longtable}{| p{4.5cm} | p{2.5cm} | p{0.5cm} | p{2.5cm} | p{6cm} |}
    \hline
    \bf Name & \bf type & \bf size & \bf default value & \bf description \\ \hline
    show &     bool &      &     True &     set true, if item is shown in visualization and false if it is not shown\\ \hline
    drawHeight &     float &      &     0. &     if beam is drawn with rectangular shape, this is the drawing height\\ \hline
    color &     Float4 &      &     [-1.,-1.,-1.,-1.] &     RGBA color of the object; if R==-1, use default color\\ \hline
	  \end{longtable}
	\end{center}

\noindent{\bf Short name} for Python: {\bf Cable2D}
 \vspace{6pt}\\{\bf Definition of quantities}:\\
{\bf The following output parameters are available as OutputVariableType in sensors and other functions}:\\ 
\startTable{output parameters}{symbol}{description}
\rowTable{Position}{}{global position vector of local axis (1) and cross section (2) position}
\rowTable{Displacement}{}{global displacement vector of local axis (1) and cross section (2) position}
\rowTable{Velocity}{}{global velocity vector of local axis (1) and cross section (2) position}
\rowTable{Director1}{}{(axial) slope vector of local axis position}
\rowTable{Strain}{}{axial strain (scalar)}
\rowTable{Curvature}{}{axial strain (scalar)}
\rowTable{Force}{}{(local) section normal force (scalar)}
\rowTable{Torque}{}{(local) bending moment (scalar)}
\finishTable
\newpage

%+++++++++++++++++++++++++++++++++++
\mysubsubsection{ObjectALEANCFCable2D}
A 2D cable finite element using 2 nodes of type NodePoint2DSlope1 and a axially moving coordinate of type NodeGenericODE2; the element has 8+1 coordinates and uses cubic polynomials for position interpolation; the element in addition to ANCFCable2D adds an Eulerian axial velocity by the GenericODE2 coordiante
 \\\vspace{12pt} \noindent The item {\bf ObjectALEANCFCable2D} with type = 'ALEANCFCable2D' has the following parameters:\vspace{-1cm}\\ 
%reference manual TABLE
\begin{center}
  \footnotesize
  \begin{longtable}{| p{4.5cm} | p{2.5cm} | p{0.5cm} | p{2.5cm} | p{6cm} |}
    \hline
    \bf Name & \bf type & \bf size & \bf default value & \bf description \\ \hline
    name &     String &      &     '' &     objects"s unique name\\ \hline
    physicsLength &     UReal &      &     0. &     reference length $L$ [SI:m] of beam; such that the total volume (e.g. for volume load) gives $\rho A L$\\ \hline
    physicsMassPerLength &     UReal &      &     0. &     mass $\rho A$ [SI:kg/m$^2$] of beam\\ \hline
    physicsMovingMassFactor &     UReal &      &     1. &     this factor denotes the amount of $\rho A$ which is moving; physicsMovingMassFactor=1 means, that all mass is moving; physicsMovingMassFactor=0 means, that no mass is moving; factor can be used to simulate e.g. pipe conveying fluid, in which $\rho A$ is the mass of the pipe+fluid, while $physicsMovingMassFactor \cdot \rho A$ is the mass per unit length of the fluid\\ \hline
    physicsBendingStiffness &     UReal &      &     0. &     bending stiffness $EI$ [SI:Nm$^2$] of beam; the bending moment is $m = EI (\kappa - \kappa_0)$, in which $\kappa$ is the material measure of curvature\\ \hline
    physicsAxialStiffness &     UReal &      &     0. &     axial stiffness $EA$ [SI:N] of beam; the axial force is $f_{ax} = EA (\varepsilon -\varepsilon_0)$, in which $\varepsilon = |\rv^\prime|-1$ is the axial strain\\ \hline
    physicsBendingDamping &     UReal &      &     0. &     bending damping $d_{EI}$ [SI:Nm$^2$/s] of beam; the additional virtual work due to damping is $\delta W_{\dot \kappa} = \int_0^L \dot \kappa \delta \kappa dx$\\ \hline
    physicsAxialDamping &     UReal &      &     0. &     axial stiffness $d_{EA}$ [SI:N/s] of beam; the additional virtual work due to damping is $\delta W_{\dot\varepsilon} = \int_0^L \dot \varepsilon \delta \varepsilon dx$\\ \hline
    physicsReferenceAxialStrain &     UReal &      &     0. &     reference axial strain of beam (pre-deformation) $\varepsilon_0$ [SI:1] of beam; without external loading the beam will statically keep the reference axial strain value\\ \hline
    physicsReferenceCurvature &     UReal &      &     0. &     reference curvature of beam (pre-deformation) $\kappa_0$ [SI:1/m] of beam; without external loading the beam will statically keep the reference curvature value\\ \hline
    physicsUseCouplingTerms &     bool &      &     True &     true: correct case, where all coupling terms due to moving mass are respected; false: only include constant mass for ALE node coordinate, but deactivate other coupling terms (behaves like ANCFCable2D then)\\ \hline
    nodeNumbers &     Index3 &      &     [MAXINT, MAXINT, MAXINT] &     two node numbers ANCF cable element, third node=ALE GenericODE2 node\\ \hline
    useReducedOrderIntegration &     Bool &      &     False &     false: use Gauss order 9 integration for virtual work of axial forces, order 5 for virtual work of bending moments; true: use Gauss order 7 integration for virtual work of axial forces, order 3 for virtual work of bending moments\\ \hline
    visualization & VObjectALEANCFCable2D & & & parameters for visualization of item \\ \hline
	  \end{longtable}
	\end{center}
The item VObjectALEANCFCable2D has the following parameters:\vspace{-1cm}\\ 
%reference manual TABLE
\begin{center}
  \footnotesize
  \begin{longtable}{| p{4.5cm} | p{2.5cm} | p{0.5cm} | p{2.5cm} | p{6cm} |}
    \hline
    \bf Name & \bf type & \bf size & \bf default value & \bf description \\ \hline
    show &     bool &      &     True &     set true, if item is shown in visualization and false if it is not shown\\ \hline
    drawHeight &     float &      &     0. &     if beam is drawn with rectangular shape, this is the drawing height\\ \hline
    color &     Float4 &      &     [-1.,-1.,-1.,-1.] &     RGBA color of the object; if R==-1, use default color\\ \hline
	  \end{longtable}
	\end{center}

\noindent{\bf Short name} for Python: {\bf ALECable2D}
 \vspace{6pt}\\{\bf Definition of quantities}:\\
{\bf The following output parameters are available as OutputVariableType in sensors and other functions}:\\ 
\startTable{output parameters}{symbol}{description}
\rowTable{Position}{}{global position vector of local axis (1) and cross section (2) position}
\rowTable{Displacement}{}{global displacement vector of local axis (1) and cross section (2) position}
\rowTable{Velocity}{}{global velocity vector of local axis (1) and cross section (2) position}
\rowTable{Director1}{}{(axial) slope vector of local axis position}
\rowTable{Strain}{}{axial strain (scalar)}
\rowTable{Curvature}{}{axial strain (scalar)}
\rowTable{Force}{}{(local) section normal force (scalar)}
\rowTable{Torque}{}{(local) bending moment (scalar)}
\finishTable
\newpage

%+++++++++++++++++++++++++++++++++++
\mysubsubsection{ObjectGround}
A ground object behaving like a rigid body, but having no degrees of freedom; used to attach body-connectors without an action. For examples see spring dampers and joints.
 \\\vspace{12pt} \noindent The item {\bf ObjectGround} with type = 'Ground' has the following parameters:\vspace{-1cm}\\ 
%reference manual TABLE
\begin{center}
  \footnotesize
  \begin{longtable}{| p{4.5cm} | p{2.5cm} | p{0.5cm} | p{2.5cm} | p{6cm} |}
    \hline
    \bf Name & \bf type & \bf size & \bf default value & \bf description \\ \hline
    name &     String &      &     '' &     objects"s unique name\\ \hline
    referencePosition &     Vector3D &     3 &     [0.,0.,0.] &     reference position for ground object; local position is added on top of reference position for a ground object\\ \hline
    visualization & VObjectGround & & & parameters for visualization of item \\ \hline
	  \end{longtable}
	\end{center}
The item VObjectGround has the following parameters:\vspace{-1cm}\\ 
%reference manual TABLE
\begin{center}
  \footnotesize
  \begin{longtable}{| p{4.5cm} | p{2.5cm} | p{0.5cm} | p{2.5cm} | p{6cm} |}
    \hline
    \bf Name & \bf type & \bf size & \bf default value & \bf description \\ \hline
    show &     bool &      &     True &     set true, if item is shown in visualization and false if it is not shown\\ \hline
    color &     Float4 &      &     [-1.,-1.,-1.,-1.] &     RGB node color; if R==-1, use default color\\ \hline
    graphicsData &     BodyGraphicsData &     \tabnewline  &      &     Structure contains data for body visualization; data is defined in special list / dictionary structure\\ \hline
	  \end{longtable}
	\end{center}
{\bf Definition of quantities}:\\
{\bf The following output parameters are available as OutputVariableType in sensors and other functions}:\\ 
\startTable{output parameters}{symbol}{description}
\rowTable{Position}{}{global position vector of rotated and translated local position}
\rowTable{Displacement}{}{global displacement vector of local position}
\rowTable{Velocity}{}{global velocity vector of local position}
\rowTable{AngularVelocity}{}{angular velocity of body}
\rowTable{RotationMatrix}{}{rotation matrix in vector form (stored in row-major order)}
\finishTable
\newpage

%+++++++++++++++++++++++++++++++++++
\mysubsubsection{ObjectConnectorSpringDamper}
An simple spring-damper element with additional force; connects to position-based markers.
 \\  {\bf Requested marker type} = Marker::Position \\ 
\vspace{12pt} \noindent The item {\bf ObjectConnectorSpringDamper} with type = 'ConnectorSpringDamper' has the following parameters:\vspace{-1cm}\\ 
%reference manual TABLE
\begin{center}
  \footnotesize
  \begin{longtable}{| p{4.5cm} | p{2.5cm} | p{0.5cm} | p{2.5cm} | p{6cm} |}
    \hline
    \bf Name & \bf type & \bf size & \bf default value & \bf description \\ \hline
    name &     String &      &     '' &     connector"s unique name\\ \hline
    markerNumbers &     ArrayIndex &      &     [ MAXINT, MAXINT ] &     list of markers used in connector\\ \hline
    referenceLength &     UReal &      &     0. &     reference length [SI:m] of spring\\ \hline
    stiffness &     UReal &      &     0. &     stiffness [SI:N/m] of spring; acts against (length-initialLength)\\ \hline
    damping &     UReal &      &     0. &     damping [SI:N/(m s)] of damper; acts against d/dt(length)\\ \hline
    force &     UReal &      &     0. &     added constant force [SI:N] of spring; scalar force; f=1 is equivalent to reducing initialLength by 1/stiffness; f > 0: tension; f < 0: compression\\ \hline
    activeConnector &     bool &      &     True &     flag, which determines, if the connector is active; used to deactivate (temorarily) a connector or constraint\\ \hline
    springForceUserFunction &     PyFunctionScalar6 &     \tabnewline  &     0 &     A python function which defines the spring force with parameters (time, deltaL, deltaL\_t, Real stiffness, Real damping, Real springForce); the parameters are provided to the function using the current values of the SpringDamper object; The python function will only be evaluated, if activeConnector is true, otherwise the SpringDamper is inactive; Example for python function: def f(t, u, v, k, d, F0): return k*u + d*v + F0\\ \hline
    visualization & VObjectConnectorSpringDamper & & & parameters for visualization of item \\ \hline
	  \end{longtable}
	\end{center}
The item VObjectConnectorSpringDamper has the following parameters:\vspace{-1cm}\\ 
%reference manual TABLE
\begin{center}
  \footnotesize
  \begin{longtable}{| p{4.5cm} | p{2.5cm} | p{0.5cm} | p{2.5cm} | p{6cm} |}
    \hline
    \bf Name & \bf type & \bf size & \bf default value & \bf description \\ \hline
    show &     Bool &      &     True &     set true, if item is shown in visualization and false if it is not shown\\ \hline
    drawSize &     float &      &     -1. &     drawing size = diameter of spring; size == -1.f means that default connector size is used\\ \hline
    color &     Float4 &      &     [-1.,-1.,-1.,-1.] &     RGB connector color; if R==-1, use default color\\ \hline
	  \end{longtable}
	\end{center}

\noindent{\bf Short name} for Python: {\bf SpringDamper}
 \vspace{6pt}\\{\bf Definition of quantities}:\\
{\bf The following output parameters are available as OutputVariableType in sensors and other functions}:\\ 
\startTable{output parameters}{symbol}{description}
\rowTable{Distance}{}{distance between both points}
\rowTable{Displacement}{}{relative displacement between both points}
\rowTable{Velocity}{}{relative velocity between both points}
\rowTable{Force}{}{spring-damper force}
\finishTable
{\bf Description of Item}:
 \noindent
    \vspace{6pt}\\
    {\bf Definition of quantities}:
    \startTable{input parameter}{symbol}{description}
    \rowTable{referenceLength}{$L_0$}{}
    \rowTable{stiffness}{$k$}{}
    \rowTable{damping}{$d$}{}
    \rowTable{force}{$f_{a}$}{additional force (e.g., actuator force)}
    \rowTable{markerNumbers[0]}{$m0$}{global marker number m0}
    \rowTable{markerNumbers[1]}{$m1$}{global marker number m1}
    \finishTable
    \startTable{intermediate variables}{symbol}{description}
    \rowTable{marker m0 position}{$\LU{0}{\pv}_{m0}$}{current global position which is provided by marker m0}
    \rowTable{marker m1 position}{$\LU{0}{\pv}_{m1}$}{}
    \rowTable{marker m0 velocity}{$\LU{0}{\vv}_{m0}$}{current global velocity which is provided by marker m0}
    \rowTable{marker m1 velocity}{$\LU{0}{\vv}_{m1}$}{}
    \finishTable
    \startTable{output variables}{symbol}{formula}
    \rowTable{Distance}{$L$}{$|\Delta\! \LU{0}{\pv}|$}
    \rowTable{Displacement}{$\Delta\! \LU{0}{\pv}$}{$\LU{0}{\pv}_{m1} - \LU{0}{\pv}_{m0}$}
    \rowTable{Velocity}{$\Delta\! \LU{0}{\vv}$}{$\LU{0}{\vv}_{m1} - \LU{0}{\vv}_{m0}$}
    \rowTable{Force}{$\fv$}{see below}
    \finishTable

    \noindent {\bf Connector equations}:
    %Displacement between marker m0 to marker m1 positions,
    %\be
    %  \Delta\! \LU{0}{\pv}= \LU{0}{\pv}_{m1} - \LU{0}{\pv}_{m0}
    %\ee
    %and relative velocity,
    %\be
    %  \Delta\! \LU{0}{\vv}= \LU{0}{\vv}_{m1} - \LU{0}{\vv}_{m0}
    %\ee
    %With the current spring length (distance) $L = |\Delta\! \LU{0}{\pv}|$, 
    The unit vector in force direction reads (raises SysError if $L=0$),
    \be
      \vv_{f} = \frac{1}{L} \Delta\! \LU{0}{\pv}
    \ee
    If \texttt{activeConnector = True}, the scalar spring force is computed as
    \be
      f_{SD} = k\cdot(L-L_0) + d \cdot\Delta\! \LU{0}{\vv}\tp \vv_{f} + f_{a}
    \ee
    If the springForceUserFunction $\mathrm{UF}$ is defined, $\fv$ instead becomes ($t$ is current time)
    \be
      f_{SD} = \mathrm{UF}(t, L-L_0, \Delta\! \LU{0}{\vv}\tp \vv_{f}, k, d, f_{a})
    \ee
    if \texttt{activeConnector = False}, $f_{SD}$ is set to zero.:
    The vector of the spring force applied at both markers finally reads
    \be
      \fv = f_{SD}\vv_{f}
    \ee
\noindent {\bf Example} for ObjectConnectorSpringDamper:
\pythonstyle
\begin{lstlisting}[language=Python, firstnumber=1]
    node = mbs.AddNode(NodePoint(referenceCoordinates = [1.05,0,0]))
    oMassPoint = mbs.AddObject(MassPoint(nodeNumber = node, physicsMass=1))
    
    m0 = mbs.AddMarker(MarkerBodyPosition(bodyNumber=oGround, localPosition=[0,0,0]))
    m1 = mbs.AddMarker(MarkerBodyPosition(bodyNumber=oMassPoint, localPosition=[0,0,0]))
    
    mbs.AddObject(ObjectConnectorSpringDamper(markerNumbers=[m0,m1],
                                              referenceLength = 1, #shorter than initial distance
                                              stiffness = 100,
                                              damping = 1))

    #assemble and solve system for default parameters
    mbs.Assemble()
    SC.TimeIntegrationSolve(mbs, 'GeneralizedAlpha', exu.SimulationSettings())

    #check result at default integration time
    testError = mbs.GetNodeOutput(node, exu.OutputVariableType.Position)[0] - 0.9736596225944887

\end{lstlisting}

\newpage

%+++++++++++++++++++++++++++++++++++
\mysubsubsection{ObjectConnectorCartesianSpringDamper}
An 3D spring-damper element acting accordingly in three (global) directions (x,y,z) which connects to position-based markers.
 \\  {\bf Requested marker type} = Marker::Position \\ 
\vspace{12pt} \noindent The item {\bf ObjectConnectorCartesianSpringDamper} with type = 'ConnectorCartesianSpringDamper' has the following parameters:\vspace{-1cm}\\ 
%reference manual TABLE
\begin{center}
  \footnotesize
  \begin{longtable}{| p{4.5cm} | p{2.5cm} | p{0.5cm} | p{2.5cm} | p{6cm} |}
    \hline
    \bf Name & \bf type & \bf size & \bf default value & \bf description \\ \hline
    name &     String &      &     '' &     connector"s unique name\\ \hline
    markerNumbers &     ArrayIndex &      &     [ MAXINT, MAXINT ] &     list of markers used in connector\\ \hline
    stiffness &     Vector3D &      &     [0.,0.,0.] &     stiffness [SI:N/m] of springs; act against relative displacements in 0, 1, and 2-direction\\ \hline
    damping &     Vector3D &      &     [0.,0.,0.] &     damping [SI:N/(m s)] of dampers; act against relative velocities in 0, 1, and 2-direction\\ \hline
    offset &     Vector3D &      &     [0.,0.,0.] &     offset between two springs\\ \hline
    springForceUserFunction &     PyFunctionVector3DScalar5Vector3D &     \tabnewline  &     \tabnewline 0 &     A python function which computes the 3D force vector between the two marker points, if activeConnector=True;  The function takes the relative displacement (3D) vector (m1.position-m0.position, etc.) and the relative velocity vector (3D), the spring striffness vector 3D, damping and offset parameter vectors (3D): f(time, displacement, velocity, stiffness, damping, offset); Example for python function: def f(t, u, v, k, d, offset): return [u[0]*k[0],u[1]*k[1],u[2]*k[2]]\\ \hline
    activeConnector &     bool &      &     True &     flag, which determines, if the connector is active; used to deactivate (temorarily) a connector or constraint\\ \hline
    visualization & VObjectConnectorCartesianSpringDamper & & & parameters for visualization of item \\ \hline
	  \end{longtable}
	\end{center}
The item VObjectConnectorCartesianSpringDamper has the following parameters:\vspace{-1cm}\\ 
%reference manual TABLE
\begin{center}
  \footnotesize
  \begin{longtable}{| p{4.5cm} | p{2.5cm} | p{0.5cm} | p{2.5cm} | p{6cm} |}
    \hline
    \bf Name & \bf type & \bf size & \bf default value & \bf description \\ \hline
    show &     Bool &      &     True &     set true, if item is shown in visualization and false if it is not shown\\ \hline
    drawSize &     float &      &     -1. &     drawing size = diameter of spring; size == -1.f means that default connector size is used\\ \hline
    color &     Float4 &      &     [-1.,-1.,-1.,-1.] &     RGB connector color; if R==-1, use default color\\ \hline
	  \end{longtable}
	\end{center}

\noindent{\bf Short name} for Python: {\bf CartesianSpringDamper}
 \vspace{6pt}\\{\bf Definition of quantities}:\\
\startTable{input parameter}{symbol}{description see tables above}
\rowTable{markerNumbers}{$[m0,m1]\tp$}{}
\rowTable{stiffness}{$\kv$}{}
\rowTable{damping}{$\dv$}{}
\rowTable{offset}{$\vv_{\mathrm{off}}$}{}
\finishTable
{\bf The following output parameters are available as OutputVariableType in sensors and other functions}:\\ 
\startTable{output parameters}{symbol}{description}
\rowTable{Displacement}{$\Delta\! \LU{0}{\pv} = \LU{0}{\pv}_{m1} - \LU{0}{\pv}_{m0}$}{relative displacement in global coordinates}
\rowTable{Distance}{$L=|\Delta\! \LU{0}{\pv}|$}{scalar distance between both marker points}
\rowTable{Velocity}{$\Delta\! \LU{0}{\vv} = \LU{0}{\vv}_{m1} - \LU{0}{\vv}_{m0}$}{relative translational velocity in global coordinates}
\rowTable{Force}{$\fv_{SD}$}{joint force in global coordinates, see equations}
\finishTable
{\bf Description of Item}:
 \noindent
    \vspace{6pt}\\
    {\bf Definition of quantities}:
    \startTable{intermediate variables}{symbol}{description}
    \rowTable{marker m0 position}{$\LU{0}{\pv}_{m0}$}{current global position which is provided by marker m0}
    \rowTable{marker m1 position}{$\LU{0}{\pv}_{m1}$}{}
    \rowTable{marker m0 velocity}{$\LU{0}{\vv}_{m0}$}{current global velocity which is provided by marker m0}
    \rowTable{marker m1 velocity}{$\LU{0}{\vv}_{m1}$}{}
    \finishTable
    \noindent {\bf Connector equations}:
    Displacement between marker m0 to marker m1 positions,
    \be
      \Delta\! \LU{0}{\pv}= \LU{0}{\pv}_{m1} - \LU{0}{\pv}_{m0}
    \ee
    and relative velocity,
    \be
      \Delta\! \LU{0}{\vv}= \LU{0}{\vv}_{m1} - \LU{0}{\vv}_{m0}
    \ee
    If \texttt{activeConnector = True}, the spring force vector is computed as
    \be
      \fv_{SD} = \left(\kv\cdot(\Delta\! \LU{0}{\pv}-\vv_{\mathrm{off}}) + \dv \Delta\! \LU{0}{\vv} \right)
    \ee
    If the springForceUserFunction $\mathrm{UF}$ is defined, $\fv_{SD}$ instead becomes ($t$ is current time)
    \be
      \fv_{SD} = \mathrm{UF}(t, \Delta\! \LU{0}{\pv}, \Delta\! \LU{0}{\vv}, \kv, \dv, \vv_{\mathrm{off}})
    \ee
    if \texttt{activeConnector = False}, $\fv_{SD}$ is set to zero.
\noindent {\bf Example} for ObjectConnectorCartesianSpringDamper:
\pythonstyle
\begin{lstlisting}[language=Python, firstnumber=1]
    #example with mass at [1,1,0], 5kg under load 5N in -y direction
    k=5000
    nMass = mbs.AddNode(NodePoint(referenceCoordinates=[1,1,0]))
    oMass = mbs.AddObject(MassPoint(physicsMass = 5, nodeNumber = nMass))
    
    mMass = mbs.AddMarker(MarkerNodePosition(nodeNumber=nMass))
    mGround = mbs.AddMarker(MarkerBodyPosition(bodyNumber=oGround, localPosition = [1,1,0]))
    mbs.AddObject(CartesianSpringDamper(markerNumbers = [mGround, mMass], 
                                        stiffness = [k,k,k], 
                                        damping = [0,k*0.05,0], offset = [0,0,0]))
    mbs.AddLoad(Force(markerNumber = mMass, loadVector = [0, -5, 0])) #static solution=-5/5000=-0.001m

    #assemble and solve system for default parameters
    mbs.Assemble()
    SC.TimeIntegrationSolve(mbs, 'GeneralizedAlpha', exu.SimulationSettings())

    #check result at default integration time
    testError = mbs.GetNodeOutput(nMass, exu.OutputVariableType.Displacement)[1] - (-0.00099999999997058)

\end{lstlisting}

\newpage

%+++++++++++++++++++++++++++++++++++
\mysubsubsection{ObjectConnectorRigidBodySpringDamper}
An 3D spring-damper element acting on relative displacements and relative rotations of two rigid body (position+orientation) markers; connects to (position+orientation)-based markers and represents a penalty-based rigid joint (or prismatic, revolute, etc.)
 \\  {\bf Requested marker type} = (Marker::Type)((Index)Marker::Position + (Index)Marker::Orientation) \\ 
\vspace{12pt} \noindent The item {\bf ObjectConnectorRigidBodySpringDamper} with type = 'ConnectorRigidBodySpringDamper' has the following parameters:\vspace{-1cm}\\ 
%reference manual TABLE
\begin{center}
  \footnotesize
  \begin{longtable}{| p{4.5cm} | p{2.5cm} | p{0.5cm} | p{2.5cm} | p{6cm} |}
    \hline
    \bf Name & \bf type & \bf size & \bf default value & \bf description \\ \hline
    name &     String &      &     '' &     connector"s unique name\\ \hline
    markerNumbers &     ArrayIndex &      &     [ MAXINT, MAXINT ] &     list of markers used in connector\\ \hline
    stiffness &     Matrix6D &      &     np.zeros([6,6]) &     stiffness [SI:N/m or Nm/rad] of translational, torsional and coupled springs; act against relative displacements in x, y, and z-direction as well as the relative angles (calculated as Euler angles); in the simplest case, the first 3 diagonal values correspond to the local stiffness in x,y,z direction and the last 3 diagonal values correspond to the rotational stiffness around x,y and z axis\\ \hline
    damping &     Matrix6D &      &     np.zeros([6,6]) &     damping [SI:N/(m/s) or Nm/(rad/s)] of translational, torsional and coupled dampers; very similar to stiffness, however, the rotational velocity is computed from the angular velocity vector\\ \hline
    rotationMarker0 &     Matrix3D &      &     [[1,0,0], [0,1,0], [0,0,1]] &     local rotation matrix for marker 0; stiffness, damping, etc. components are measured in local coordinates relative to rotationMarker0\\ \hline
    rotationMarker1 &     Matrix3D &      &     [[1,0,0], [0,1,0], [0,0,1]] &     local rotation matrix for marker 1; stiffness, damping, etc. components are measured in local coordinates relative to rotationMarker1\\ \hline
    offset &     Vector6D &      &     [0.,0.,0.,0.,0.,0.] &     translational and rotational offset considered in the spring force calculation\\ \hline
    activeConnector &     bool &      &     True &     flag, which determines, if the connector is active; used to deactivate (temorarily) a connector or constraint\\ \hline
    visualization & VObjectConnectorRigidBodySpringDamper & & & parameters for visualization of item \\ \hline
	  \end{longtable}
	\end{center}
The item VObjectConnectorRigidBodySpringDamper has the following parameters:\vspace{-1cm}\\ 
%reference manual TABLE
\begin{center}
  \footnotesize
  \begin{longtable}{| p{4.5cm} | p{2.5cm} | p{0.5cm} | p{2.5cm} | p{6cm} |}
    \hline
    \bf Name & \bf type & \bf size & \bf default value & \bf description \\ \hline
    show &     Bool &      &     True &     set true, if item is shown in visualization and false if it is not shown\\ \hline
    drawSize &     float &      &     -1. &     drawing size = diameter of spring; size == -1.f means that default connector size is used\\ \hline
    color &     Float4 &      &     [-1.,-1.,-1.,-1.] &     RGB connector color; if R==-1, use default color\\ \hline
	  \end{longtable}
	\end{center}

\noindent{\bf Short name} for Python: {\bf RigidBodySpringDamper}
 \vspace{6pt}\\{\bf Definition of quantities}:\\
{\bf The following output parameters are available as OutputVariableType in sensors and other functions}:\\ 
\startTable{output parameters}{symbol}{description}
\rowTable{DisplacementLocal}{}{relative displacement in local marker0 coordinates}
\rowTable{VelocityLocal}{}{relative translational velocity in local marker0 coordinates}
\rowTable{Rotation}{}{relative rotation parameters (Tait Bryan Rxyz); these are the angles used for calculation of joint torques (e.g. if cX is the diagonal rotational stiffness, the moment for axis X reads mX=cX*phiX, etc.)}
\rowTable{AngularVelocityLocal}{}{relative angular velocity in local marker0 coordinates}
\rowTable{ForceLocal}{}{joint force in local marker0 coordinates}
\rowTable{TorqueLocal}{}{joint torque in in local marker0 coordinates}
\finishTable
{\bf Description of Item}:
 \noindent
    \vspace{6pt}\\
    {\bf Definition of quantities}:
    \startTable{input parameter}{symbol}{description}
    \rowTable{stiffness}{$\kv \in \mathbb{R}^{6\times 6}$}{stiffness in $J0$ coordinates}
    \rowTable{damping}{$\dv \in \mathbb{R}^{6\times 6}$}{damping in $J0$ coordinates}
    \rowTable{offset}{$\LUR{J0}{\vv}{\mathrm{off}} \in \mathbb{R}^{6}$}{offset in $J0$ coordinates}
    \rowTable{rotationMarker0}{$\LU{m0,J0}{\Rot}$}{rotation matrix which transforms from joint 0 into marker 0 coordinates}
    \rowTable{rotationMarker1}{$\LU{m1,J1}{\Rot}$}{rotation matrix which transforms from joint 1 into marker 1 coordinates}
    \rowTable{markerNumbers[0]}{$m0$}{global marker number m0}
    \rowTable{markerNumbers[1]}{$m1$}{global marker number m1}
    \finishTable
    \startTable{intermediate variables}{symbol}{description}
    \rowTable{marker m0 position}{$\LU{0}{\pv}_{m0}$}{current global position which is provided by marker m0}
    \rowTable{marker m0 orientation}{$\LU{0,m0}{\Rot}$}{current rotation matrix provided by marker m0}
    \rowTable{marker m1 position}{$\LU{0}{\pv}_{m1}$}{accordingly}
    \rowTable{marker m1 orientation}{$\LU{0,m1}{\Rot}$}{current rotation matrix provided by marker m1}
%
    \rowTable{marker m0 velocity}{$\LU{0}{\vv}_{m0}$}{current global velocity which is provided by marker m0}
    \rowTable{marker m1 velocity}{$\LU{0}{\vv}_{m1}$}{accordingly}
    \rowTable{marker m0 velocity}{$\LU{b}{\tomega}_{m0}$}{current local angular velocity vector provided by marker m0}
    \rowTable{marker m1 velocity}{$\LU{b}{\tomega}_{m1}$}{current local angular velocity vector provided by marker m1}
    \rowTable{Displacement}{$\LU{0}{\Delta\pv}$} {$\LU{0}{\pv}_{m1} - \LU{0}{\pv}_{m0}$}
    \rowTable{Velocity}{$\LU{0}{\Delta\vv}$}{$\LU{0}{\vv}_{m1} - \LU{0}{\vv}_{m0}$}
    \finishTable
%
    \startTable{output variables}{symbol}{formula}
    \rowTable{DisplacementLocal}{$\LU{J0}{\Delta\pv}$}{$\left(\LU{0,m0}{\Rot}\LU{m0,J0}{\Rot}\right)\tp \LU{0}{\Delta\pv}$}
    \rowTable{VelocityLocal}{$\LU{J0}{\Delta\vv}$}{$\left(\LU{0,m0}{\Rot}\LU{m0,J0}{\Rot}\right)\tp \LU{0}{\Delta\vv}$}
%
    \rowTable{AngularVelocityLocal}{$\LU{J0}{\Delta\omega}$}{$\left(\LU{0,m0}{\Rot}\LU{m0,J0}{\Rot}\right)\tp \left( \LU{0,m1}{\Rot} \LU{m1}{\omega} - \LU{0,m0}{\Rot} \LU{m0}{\omega} \right)$}
    \rowTable{Rotation}{$\LU{J0}{\ttheta} = [\theta_0,\theta_1,\theta_2]$}{angles retrieved from relative rotation matrix, ...}
    \rowTable{ForceLocal}{$\LU{J0}{\fv}$}{see below}
    \rowTable{TorqueLocal}{$\LU{J0}{\mv}$}{see below}
    \finishTable

    \noindent {\bf Connector equations}:
    If \texttt{activeConnector = True}, the vector spring force is computed as
    \be
      \vp{\LU{J0}{\fv_{SD}}}{\LU{J0}{\mv_{SD}}} = \kv \left( \vp{\LU{J0}{\Delta\pv}}{\LU{J0}{\ttheta}} - \LUR{J0}{\vv}{\mathrm{off}}\right) + 
            \dv \vp{\LU{J0}{\Delta\vv}}{\LU{J0}{\Delta\omega}}
    \ee
    For the application of joint forces to markers, $[\LU{J0}{\fv_{SD}},\,\LU{J0}{\mv_{SD}}]\tp$ is transformed into global coordinates.
    if \texttt{activeConnector = False}, $\LU{J0}{\fv_{SD}}$ and  $\LU{J0}{\mv_{SD}}$ are set to zero.
\newpage

%+++++++++++++++++++++++++++++++++++
\mysubsubsection{ObjectConnectorCoordinateSpringDamper}
A 1D (scalar) spring-damper element acting on single ODE2 coordinates; connects to coordinate-based markers; NOTE that the coordinate markers only measure the coordinate (=displacement), but the reference position is not included as compared to position-based markers!; the spring-damper can also act on rotational coordinates.
 \\  {\bf Requested marker type} = Marker::Coordinate \\ 
\vspace{12pt} \noindent The item {\bf ObjectConnectorCoordinateSpringDamper} with type = 'ConnectorCoordinateSpringDamper' has the following parameters:\vspace{-1cm}\\ 
%reference manual TABLE
\begin{center}
  \footnotesize
  \begin{longtable}{| p{4.5cm} | p{2.5cm} | p{0.5cm} | p{2.5cm} | p{6cm} |}
    \hline
    \bf Name & \bf type & \bf size & \bf default value & \bf description \\ \hline
    name &     String &      &     '' &     connector"s unique name\\ \hline
    markerNumbers &     ArrayIndex &      &     [ MAXINT, MAXINT ] &     list of markers used in connector\\ \hline
    stiffness &     Real &      &     0. &     stiffness [SI:N/m] of spring; acts against relative value of coordinates\\ \hline
    damping &     Real &      &     0. &     damping [SI:N/(m s)] of damper; acts against relative velocity of coordinates\\ \hline
    offset &     Real &      &     0. &     offset between two coordinates (reference length of springs), see equation\\ \hline
    dryFriction &     Real &      &     0. &     dry friction force [SI:N] against relative velocity; assuming a normal force $f_N$, the friction force can be interpreted as $f_\mu = \mu f_N$\\ \hline
    dryFrictionProportionalZone &     Real &      &     0. &     limit velocity [m/s] up to which the friction is proportional to velocity (for regularization / avoid numerical oscillations)\\ \hline
    activeConnector &     bool &      &     True &     flag, which determines, if the connector is active; used to deactivate (temorarily) a connector or constraint\\ \hline
    springForceUserFunction &     PyFunctionScalar8 &     \tabnewline  &     0 &     A python function which defines the spring force with 8 parameters, see equations section; Example for python function: def f(t, u, v, k, d, offset, frictionForce, frictionProportionalZone): return k*(u-offset) + d*v\\ \hline
    visualization & VObjectConnectorCoordinateSpringDamper & & & parameters for visualization of item \\ \hline
	  \end{longtable}
	\end{center}
The item VObjectConnectorCoordinateSpringDamper has the following parameters:\vspace{-1cm}\\ 
%reference manual TABLE
\begin{center}
  \footnotesize
  \begin{longtable}{| p{4.5cm} | p{2.5cm} | p{0.5cm} | p{2.5cm} | p{6cm} |}
    \hline
    \bf Name & \bf type & \bf size & \bf default value & \bf description \\ \hline
    show &     Bool &      &     True &     set true, if item is shown in visualization and false if it is not shown\\ \hline
    drawSize &     float &      &     -1. &     drawing size = diameter of spring; size == -1.f means that default connector size is used\\ \hline
    color &     Float4 &      &     [-1.,-1.,-1.,-1.] &     RGB connector color; if R==-1, use default color\\ \hline
	  \end{longtable}
	\end{center}

\noindent{\bf Short name} for Python: {\bf CoordinateSpringDamper}
 \vspace{6pt}\\{\bf Definition of quantities}:\\
\startTable{input parameter}{symbol}{description see tables above}
\rowTable{stiffness}{$k$}{}
\rowTable{damping}{$d$}{}
\rowTable{offset}{$l_\mathrm{off}$}{}
\rowTable{dryFriction}{$f_\mu$}{}
\rowTable{dryFrictionProportionalZone}{$v_\mu$}{}
\finishTable
{\bf The following output parameters are available as OutputVariableType in sensors and other functions}:\\ 
\startTable{output parameters}{symbol}{description}
\rowTable{Displacement}{$\Delta q$}{relative scalar displacement of marker coordinates}
\rowTable{Velocity}{$\Delta v$}{difference of scalar marker velocity coordinates}
\rowTable{Force}{$f_{SD}$}{scalar spring force}
\finishTable
{\bf Description of Item}:
 \noindent
    \vspace{6pt}\\
    {\bf Definition of quantities}:
    \startTable{intermediate variables}{symbol}{description}
    \rowTable{marker m0 coordinate}{$q_{m0}$}{current displacement coordinate which is provided by marker m0; does NOT include reference coordinate!}
    \rowTable{marker m1 coordinate}{$q_{m1}$}{}
    \rowTable{marker m0 velocity coordinate}{$v_{m0}$}{current velocity coordinate which is provided by marker m0}
    \rowTable{marker m1 velocity coordinate}{$v_{m1}$}{}
    \finishTable
    \noindent {\bf Connector equations}:
    Displacement between marker m0 to marker m1 coordinates (does NOT include reference coordinates),
    \be
      \Delta q= q_{m1} - q_{m0}
    \ee
    and relative velocity,
    \be
      \Delta v= v_{m1} - v_{m0}
    \ee
    If $f_\mu > 0$, the friction force is computed as 
    \be
      f_\mathrm{friction} = \left\{ 
              \begin{aligned} \mathrm{Sgn}(\Delta v) \cdot f_\mu \quad \mathrm{if} \quad |\Delta v| \ge v_\mu \\
              \frac{\Delta v}{v_\mu} f_\mu \quad \mathrm{if} \quad |\Delta v| < v_\mu 
              \end{aligned}  \right.
    \ee
    If \texttt{activeConnector = True}, the scalar spring force vector is computed as
    \be
      f_{SD} = k \left( \Delta q - l_\mathrm{off} \right) + d \cdot \Delta v + f_\mathrm{friction}
    \ee
    If the springForceUserFunction $\mathrm{UF}$ is defined, $\fv_{SD}$ instead becomes ($t$ is current time)
    \be
      f_{SD} = \mathrm{UF}(t, \Delta q, \Delta v, k, d, l_\mathrm{off}, f_\mu, v_\mu)
    \ee
    if \texttt{activeConnector = False}, $f_{SD}$ is set to zero.
\noindent {\bf Example} for ObjectConnectorCoordinateSpringDamper:
\pythonstyle
\begin{lstlisting}[language=Python, firstnumber=1]
    def springForce(t, u, v, k, d, offset, frictionForce, frictionProportionalZone):
        return 0.1*k*u+k*u**3+v*d

    nMass=mbs.AddNode(Point(referenceCoordinates = [2,0,0]))
    massPoint = mbs.AddObject(MassPoint(physicsMass = 5, nodeNumber = nMass))
    
    groundMarker=mbs.AddMarker(MarkerNodeCoordinate(nodeNumber= nGround, coordinate = 0))
    nodeMarker  =mbs.AddMarker(MarkerNodeCoordinate(nodeNumber= nMass, coordinate = 0))
    
    #Spring-Damper between two marker coordinates
    mbs.AddObject(CoordinateSpringDamper(markerNumbers = [groundMarker, nodeMarker], 
                                         stiffness = 5000, damping = 80, springForceUserFunction = springForce)) 
    loadCoord = mbs.AddLoad(LoadCoordinate(markerNumber = nodeMarker, load = 1)) #static linear solution:0.002

    #assemble and solve system for default parameters
    mbs.Assemble()
    SC.TimeIntegrationSolve(mbs, 'GeneralizedAlpha', exu.SimulationSettings())

    #check result at default integration time
    testError = mbs.GetNodeOutput(nMass, exu.OutputVariableType.Displacement)[0] - 0.0019995158325691875

\end{lstlisting}

\newpage

%+++++++++++++++++++++++++++++++++++
\mysubsubsection{ObjectConnectorDistance}
Connector which enforces constant or prescribed distance between two bodies/nodes.
 \\  {\bf Requested marker type} = Marker::Position \\ 
\vspace{12pt} \noindent The item {\bf ObjectConnectorDistance} with type = 'ConnectorDistance' has the following parameters:\vspace{-1cm}\\ 
%reference manual TABLE
\begin{center}
  \footnotesize
  \begin{longtable}{| p{4.5cm} | p{2.5cm} | p{0.5cm} | p{2.5cm} | p{6cm} |}
    \hline
    \bf Name & \bf type & \bf size & \bf default value & \bf description \\ \hline
    name &     String &      &     '' &     constraints"s unique name\\ \hline
    markerNumbers &     ArrayIndex &      &     [ MAXINT, MAXINT ] &     list of markers used in connector\\ \hline
    distance &     UReal &      &     0. &     prescribed distance [SI:m] of the used markers\\ \hline
    activeConnector &     bool &      &     True &     flag, which determines, if the connector is active; used to deactivate (temorarily) a connector or constraint\\ \hline
    visualization & VObjectConnectorDistance & & & parameters for visualization of item \\ \hline
	  \end{longtable}
	\end{center}
The item VObjectConnectorDistance has the following parameters:\vspace{-1cm}\\ 
%reference manual TABLE
\begin{center}
  \footnotesize
  \begin{longtable}{| p{4.5cm} | p{2.5cm} | p{0.5cm} | p{2.5cm} | p{6cm} |}
    \hline
    \bf Name & \bf type & \bf size & \bf default value & \bf description \\ \hline
    show &     bool &      &     True &     set true, if item is shown in visualization and false if it is not shown\\ \hline
    drawSize &     float &      &     -1. &     drawing size = link size; size == -1.f means that default connector size is used\\ \hline
    color &     Float4 &      &     [-1.,-1.,-1.,-1.] &     RGB connector color; if R==-1, use default color\\ \hline
	  \end{longtable}
	\end{center}

\noindent{\bf Short name} for Python: {\bf DistanceConstraint}
 \vspace{6pt}\\{\bf Definition of quantities}:\\
\startTable{input parameter}{symbol}{description see tables above}
\rowTable{markerNumbers}{$[m0,m1]\tp$}{}
\rowTable{distance}{$d_0$}{}
\finishTable
{\bf The following output parameters are available as OutputVariableType in sensors and other functions}:\\ 
\startTable{output parameters}{symbol}{description}
\rowTable{Displacement}{$\LU{0}{\Delta\pv}$}{relative displacement in global coordinates}
\rowTable{Velocity}{$\LU{0}{\Delta\vv}$}{relative translational velocity in global coordinates}
\rowTable{Distance}{$|\LU{0}{\Delta\pv}|$}{distance between markers (should stay constant; shows constraint deviation)}
\rowTable{Force}{$\lambda_0$}{joint force in global coordinates}
\finishTable
{\bf Description of Item}:
 \noindent
    {\bf Definition of quantities}:
    \startTable{intermediate variables}{symbol}{description}
        \rowTable{marker m0 position}{$\LU{0}{\pv}_{m0}$}{current global position which is provided by marker m0}
        \rowTable{marker m1 position}{$\LU{0}{\pv}_{m1}$}{accordingly}
    %
        \rowTable{marker m0 velocity}{$\LU{0}{\vv}_{m0}$}{current global velocity which is provided by marker m0}
        \rowTable{marker m1 velocity}{$\LU{0}{\vv}_{m1}$}{accordingly}
        \rowTable{relative displacement}{$\LU{0}{\Delta\pv}$} {$\LU{0}{\pv}_{m1} - \LU{0}{\pv}_{m0}$}
        \rowTable{relative velocity}{$\LU{0}{\Delta\vv}$}{$\LU{0}{\vv}_{m1} - \LU{0}{\vv}_{m0}$}
    %
        \rowTable{algebraicVariable}{$\lambda_0$}{Lagrange multiplier = force in constraint}
    \finishTable

    \noindent {\bf Algebraic constraint equations}:
    If \texttt{activeConnector = True}, the index 3 algebraic equation reads
    \be
      \left|\LU{0}{\Delta\pv}\right| - d_0 = 0
    \ee
    The index 2 (velocity level) algebraic equation reads
    \be
      \left(\frac{\LU{0}{\Delta\pv}}{\left|\LU{0}{\Delta\pv}\right|}\right)\tp \Delta\vv = 0
    \ee
    if \texttt{activeConnector = False}, the algebraic equation reads
    \be
      \lambda_0 = 0
    \ee
\noindent {\bf Example} for ObjectConnectorDistance:
\pythonstyle
\begin{lstlisting}[language=Python, firstnumber=1]
    #example with 1m pendulum, 50kg under gravity
    nMass = mbs.AddNode(NodePoint2D(referenceCoordinates=[1,0]))
    oMass = mbs.AddObject(MassPoint2D(physicsMass = 50, nodeNumber = nMass))
    
    mMass = mbs.AddMarker(MarkerNodePosition(nodeNumber=nMass))
    mGround = mbs.AddMarker(MarkerBodyPosition(bodyNumber=oGround, localPosition = [0,0,0]))
    oDistance = mbs.AddObject(DistanceConstraint(markerNumbers = [mGround, mMass], distance = 1))
    
    mbs.AddLoad(Force(markerNumber = mMass, loadVector = [0, -50*9.81, 0])) 

    #assemble and solve system for default parameters
    mbs.Assemble()
    
    sims=exu.SimulationSettings()
    sims.timeIntegration.generalizedAlpha.spectralRadius=0.7
    SC.TimeIntegrationSolve(mbs, 'GeneralizedAlpha', sims)

    #check result at default integration time
    testError = mbs.GetNodeOutput(nMass, exu.OutputVariableType.Position)[0] - (-0.9845225086606828)

\end{lstlisting}

\newpage

%+++++++++++++++++++++++++++++++++++
\mysubsubsection{ObjectConnectorCoordinate}
A coordinate constraint which constrains two (scalar) coordinates of Marker[Node|Body]Coordinates attached to nodes or bodies. The constraint acts directly on coordinates, but does not include reference values, e.g., of nodal values.
 \\  {\bf Requested marker type} = Marker::Coordinate \\ 
\vspace{12pt} \noindent The item {\bf ObjectConnectorCoordinate} with type = 'ConnectorCoordinate' has the following parameters:\vspace{-1cm}\\ 
%reference manual TABLE
\begin{center}
  \footnotesize
  \begin{longtable}{| p{4.5cm} | p{2.5cm} | p{0.5cm} | p{2.5cm} | p{6cm} |}
    \hline
    \bf Name & \bf type & \bf size & \bf default value & \bf description \\ \hline
    name &     String &      &     '' &     constraints"s unique name\\ \hline
    markerNumbers &     ArrayIndex &      &     [ MAXINT, MAXINT ] &     list of markers used in connector\\ \hline
    offset &     UReal &      &     0. &     An offset between the two values\\ \hline
    factorValue1 &     UReal &      &     1. &     An additional factor multiplied with value1 used in algebraic equation\\ \hline
    velocityLevel &     bool &      &     False &     If true: connector constrains velocities (only works for ODE2 coordinates!); offset is used between velocities; in this case, the offsetUserFunction\_t is considered and offsetUserFunction is ignored\\ \hline
    offsetUserFunction &     PyFunctionScalar2 &     \tabnewline  &     0 &     A python function which defines the time-dependent offset; it is highly RECOMMENDED to use sufficiently smooth functions, having consistent initial offsets with initial configuration of bodies, zero or compatible initial offset-velocity, and no accelerations; Example for python function: def UF(t, l\_offset): return l\_offset*(1-np.cos(t*10*2*np.pi))\\ \hline
    offsetUserFunction\_t &     PyFunctionScalar2 &     \tabnewline  &     0 &     time derivative of offsetUserFunction; needed for "velocityLevel=True", or for index2 time integration and for computation of initial accelerations in SecondOrderImplicit integrators\\ \hline
    activeConnector &     bool &      &     True &     flag, which determines, if the connector is active; used to deactivate (temorarily) a connector or constraint\\ \hline
    visualization & VObjectConnectorCoordinate & & & parameters for visualization of item \\ \hline
	  \end{longtable}
	\end{center}
The item VObjectConnectorCoordinate has the following parameters:\vspace{-1cm}\\ 
%reference manual TABLE
\begin{center}
  \footnotesize
  \begin{longtable}{| p{4.5cm} | p{2.5cm} | p{0.5cm} | p{2.5cm} | p{6cm} |}
    \hline
    \bf Name & \bf type & \bf size & \bf default value & \bf description \\ \hline
    show &     bool &      &     True &     set true, if item is shown in visualization and false if it is not shown\\ \hline
    drawSize &     float &      &     -1. &     drawing size = link size; size == -1.f means that default connector size is used\\ \hline
    color &     Float4 &      &     [-1.,-1.,-1.,-1.] &     RGB connector color; if R==-1, use default color\\ \hline
	  \end{longtable}
	\end{center}

\noindent{\bf Short name} for Python: {\bf CoordinateConstraint}
 \vspace{6pt}\\{\bf Definition of quantities}:\\
\startTable{input parameter}{symbol}{description see tables above}
\rowTable{markerNumbers}{$[m0,m1]\tp$}{}
\rowTable{offset}{$l_\mathrm{off}$}{}
\rowTable{factorValue1}{$k_{m1}$}{}
\rowTable{offsetUserFunction}{$\mathrm{UF}(t,l_\mathrm{off})$}{}
\rowTable{offsetUserFunction\_t}{$\mathrm{UF}_t(t,l_\mathrm{off})$}{}
\finishTable
{\bf The following output parameters are available as OutputVariableType in sensors and other functions}:\\ 
\startTable{output parameters}{symbol}{description}
\rowTable{Displacement}{$\Delta q$}{relative scalar displacement of marker coordinates, not including factorValue1}
\rowTable{Velocity}{$\Delta v$}{difference of scalar marker velocity coordinates, not including factorValue1}
\rowTable{ConstraintEquation}{$\cv$}{(residuum of) constraint equation}
\rowTable{Force}{$\lambda_0$}{scalar constraint force (Lagrange multiplier)}
\finishTable
{\bf Description of Item}:
 \noindent
    {\bf Definition of quantities}:
    \startTable{intermediate variables}{symbol}{description}
    \rowTable{marker m0 coordinate}{$q_{m0}$}{current displacement coordinate which is provided by marker m0; does NOT include reference coordinate!}
    \rowTable{marker m1 coordinate}{$q_{m1}$}{}
    \rowTable{marker m0 velocity coordinate}{$v_{m0}$}{current velocity coordinate which is provided by marker m0}
    \rowTable{marker m1 velocity coordinate}{$v_{m1}$}{}
    \rowTable{difference of coordinates}{$\Delta q = q_{m1} - q_{m0}$}{Displacement between marker m0 to marker m1 coordinates (does NOT include reference coordinates)}
    \rowTable{difference of velocity coordinates}{$\Delta v= v_{m1} - v_{m0}$}{}
    \finishTable
    \noindent {\bf Algebraic constraint equations}:
    If \texttt{activeConnector = True}, the index 3 algebraic equation reads
    \be
      \cv(q_{m0}, q_{m1}) = k_{m1} \cdot q_{m1} - q_{m0} - l_\mathrm{off} = 0
    \ee
    If the offsetUserFunction $\mathrm{UF}$ is defined, $\cv$ instead becomes ($t$ is current time)
    \be
      \cv(q_{m0}, q_{m1}) = k_{m1} \cdot q_{m1} - q_{m0} -  \mathrm{UF}(t,l_\mathrm{off}) = 0
    \ee
    The \texttt{activeConnector = True}, index 2 (velocity level) algebraic equation reads
    \be
      \dot \cv(\dot q_{m0}, \dot q_{m1}) = k_{m1} \cdot \dot q_{m1} - \dot q_{m0} - d = 0
    \ee
    The factor $d$ in velocity level equations is zero, or if parameters.velocityLevel = True, $d=l_\mathrm{off}$.
    If velocity level constraints are active and the velocity level offsetUserFunction\_t $\mathrm{UF}_t$ is defined, $\dot \cv$ instead becomes ($t$ is current time)
    \be
      \dot \cv(\dot q_{m0}, \dot q_{m1}) = k_{m1} \cdot \dot q_{m1} - \dot q_{m0} - \mathrm{UF}_t(t,l_\mathrm{off}) = 0
    \ee
    Note that the index 2 equations are used, if the solver uses index 2 formulation OR if the flag parameters.velocityLevel = True (or both).
    The user functions include dependency on time $t$, but this time dependency is not respected in the computation of initial accelerations. Therefore,
    it is recommended that $\mathrm{UF}$ and $\mathrm{UF}_t$ does not include initial accelerations.

    If \texttt{activeConnector = False}, the (index 1) algebraic equation reads for ALL cases:
    \be
      \cv(\lambda_0) = \lambda_0 = 0
    \ee
\noindent {\bf Example} for ObjectConnectorCoordinate:
\pythonstyle
\begin{lstlisting}[language=Python, firstnumber=1]
    def OffsetUF(t, lOffset): #gives 0.05 at t=1
        return 0.5*(1-np.cos(2*3.141592653589793*0.25*t))*lOffset

    nMass=mbs.AddNode(Point(referenceCoordinates = [2,0,0]))
    massPoint = mbs.AddObject(MassPoint(physicsMass = 5, nodeNumber = nMass))
    
    groundMarker=mbs.AddMarker(MarkerNodeCoordinate(nodeNumber= nGround, coordinate = 0))
    nodeMarker  =mbs.AddMarker(MarkerNodeCoordinate(nodeNumber= nMass, coordinate = 0))
    
    #Spring-Damper between two marker coordinates
    mbs.AddObject(CoordinateConstraint(markerNumbers = [groundMarker, nodeMarker], 
                                       offset = 0.1, offsetUserFunction = OffsetUF)) 

    #assemble and solve system for default parameters
    mbs.Assemble()
    SC.TimeIntegrationSolve(mbs, 'GeneralizedAlpha', exu.SimulationSettings())

    #check result at default integration time
    testError = mbs.GetNodeOutput(nMass, exu.OutputVariableType.Displacement)[0] - 0.049999999999272404

\end{lstlisting}

\newpage

%+++++++++++++++++++++++++++++++++++
\mysubsubsection{ObjectContactCoordinate}
A penalty-based contact condition for one coordinate; the contact gap $g$ is defined as $g=marker.value[1]- marker.value[0] - offset$; the contact force $f_c$ is zero for $gap>0$ and otherwise computed from $f_c = g*contactStiffness + \dot g*contactDamping$; during Newton iterations, the contact force is actived only, if $dataCoordinate[0] <= 0$; dataCoordinate is set equal to gap in nonlinear iterations, but not modified in Newton iterations.
 \\  {\bf Requested marker type} = Marker::Coordinate \\ 
\vspace{12pt} \noindent The item {\bf ObjectContactCoordinate} with type = 'ContactCoordinate' has the following parameters:\vspace{-1cm}\\ 
%reference manual TABLE
\begin{center}
  \footnotesize
  \begin{longtable}{| p{4.5cm} | p{2.5cm} | p{0.5cm} | p{2.5cm} | p{6cm} |}
    \hline
    \bf Name & \bf type & \bf size & \bf default value & \bf description \\ \hline
    name &     String &      &     '' &     connector"s unique name\\ \hline
    markerNumbers &     ArrayIndex &      &     [ MAXINT, MAXINT ] &     markers define contact gap\\ \hline
    nodeNumber &     Index &      &     MAXINT &     node number of a NodeGenericData for 1 dataCoordinate (used for active set strategy ==> holds the gap of the last discontinuous iteration)\\ \hline
    contactStiffness &     UReal &      &     0. &     contact (penalty) stiffness [SI:N/m]; acts only upon penetration\\ \hline
    contactDamping &     UReal &      &     0. &     contact damping [SI:N/(m s)]; acts only upon penetration\\ \hline
    offset &     UReal &      &     0. &     offset [SI:m] of contact\\ \hline
    activeConnector &     bool &      &     True &     flag, which determines, if the connector is active; used to deactivate (temorarily) a connector or constraint\\ \hline
    visualization & VObjectContactCoordinate & & & parameters for visualization of item \\ \hline
	  \end{longtable}
	\end{center}
The item VObjectContactCoordinate has the following parameters:\vspace{-1cm}\\ 
%reference manual TABLE
\begin{center}
  \footnotesize
  \begin{longtable}{| p{4.5cm} | p{2.5cm} | p{0.5cm} | p{2.5cm} | p{6cm} |}
    \hline
    \bf Name & \bf type & \bf size & \bf default value & \bf description \\ \hline
    show &     Bool &      &     True &     set true, if item is shown in visualization and false if it is not shown\\ \hline
    drawSize &     float &      &     -1. &     drawing size = diameter of spring; size == -1.f means that default connector size is used\\ \hline
    color &     Float4 &      &     [-1.,-1.,-1.,-1.] &     RGB connector color; if R==-1, use default color\\ \hline
	  \end{longtable}
	\end{center}
{\bf Definition of quantities}:\\
\newpage

%+++++++++++++++++++++++++++++++++++
\mysubsubsection{ObjectContactCircleCable2D}
A very specialized penalty-based contact condition between a 2D circle (=marker0, any Position-marker) on a body and an ANCFCable2DShape (=marker1, Marker: BodyCable2DShape), in xy-plane; a node NodeGenericData is required with the number of cordinates according to the number of contact segments; the contact gap $g$ is integrated (piecewise linear) along the cable and circle; the contact force $f_c$ is zero for $gap>0$ and otherwise computed from $f_c = g*contactStiffness + \dot g*contactDamping$; during Newton iterations, the contact force is actived only, if $dataCoordinate[0] <= 0$; dataCoordinate is set equal to gap in nonlinear iterations, but not modified in Newton iterations.
 \\  {\bf Requested marker type} = Marker::\_None \\ 
\vspace{12pt} \noindent The item {\bf ObjectContactCircleCable2D} with type = 'ContactCircleCable2D' has the following parameters:\vspace{-1cm}\\ 
%reference manual TABLE
\begin{center}
  \footnotesize
  \begin{longtable}{| p{4.5cm} | p{2.5cm} | p{0.5cm} | p{2.5cm} | p{6cm} |}
    \hline
    \bf Name & \bf type & \bf size & \bf default value & \bf description \\ \hline
    name &     String &      &     '' &     connector"s unique name\\ \hline
    markerNumbers &     ArrayIndex &      &     [ MAXINT, MAXINT ] &     markers define contact gap\\ \hline
    nodeNumber &     Index &      &     MAXINT &     node number of a NodeGenericData for nSegments dataCoordinates (used for active set strategy ==> hold the gap of the last discontinuous iteration and the friction state)\\ \hline
    numberOfContactSegments &     Index &      &     3 &     number of linear contact segments to determine contact; each segment is a line and is associated to a data (history) variable; must be same as in according marker\\ \hline
    contactStiffness &     UReal &      &     0. &     contact (penalty) stiffness [SI:N/m/(contact segment)]; the stiffness is per length of the beam axis; specific contact forces (per length) $f_N$ act in contact normal direction only upon penetration\\ \hline
    contactDamping &     UReal &      &     0. &     contact damping [SI:N/(m s)/(contact segment)]; the damping is per length of the beam axis; acts in contact normal direction only upon penetration\\ \hline
    circleRadius &     UReal &      &     0. &     radius [SI:m] of contact circle\\ \hline
    offset &     UReal &      &     0. &     offset [SI:m] of contact, e.g. to include thickness of cable element\\ \hline
    activeConnector &     bool &      &     True &     flag, which determines, if the connector is active; used to deactivate (temorarily) a connector or constraint\\ \hline
    visualization & VObjectContactCircleCable2D & & & parameters for visualization of item \\ \hline
	  \end{longtable}
	\end{center}
The item VObjectContactCircleCable2D has the following parameters:\vspace{-1cm}\\ 
%reference manual TABLE
\begin{center}
  \footnotesize
  \begin{longtable}{| p{4.5cm} | p{2.5cm} | p{0.5cm} | p{2.5cm} | p{6cm} |}
    \hline
    \bf Name & \bf type & \bf size & \bf default value & \bf description \\ \hline
    show &     Bool &      &     True &     set true, if item is shown in visualization and false if it is not shown\\ \hline
    drawSize &     float &      &     -1. &     drawing size = diameter of spring; size == -1.f means that default connector size is used\\ \hline
    color &     Float4 &      &     [-1.,-1.,-1.,-1.] &     RGB connector color; if R==-1, use default color\\ \hline
	  \end{longtable}
	\end{center}
{\bf Definition of quantities}:\\
\newpage

%+++++++++++++++++++++++++++++++++++
\mysubsubsection{ObjectContactFrictionCircleCable2D}
A very specialized penalty-based contact/friction condition between a 2D circle (=marker0, any Position-marker) on a body and an ANCFCable2DShape (=marker1, Marker: BodyCable2DShape), in xy-plane; a node NodeGenericData is required with 3$\times$(number of contact segments) -- containing per segment: [contact gap, stick/slip (stick=1), last friction position]; the contact gap $g$ is integrated (piecewise linear) along the cable and circle; the contact force $f_c$ is zero for $gap>0$ and otherwise computed from $f_c = g*contactStiffness + \dot g*contactDamping$; during Newton iterations, the contact force is actived only, if $dataCoordinate[0] <= 0$; dataCoordinate is set equal to gap in nonlinear iterations, but not modified in Newton iterations.
 \\  {\bf Requested marker type} = Marker::\_None \\ 
\vspace{12pt} \noindent The item {\bf ObjectContactFrictionCircleCable2D} with type = 'ContactFrictionCircleCable2D' has the following parameters:\vspace{-1cm}\\ 
%reference manual TABLE
\begin{center}
  \footnotesize
  \begin{longtable}{| p{4.5cm} | p{2.5cm} | p{0.5cm} | p{2.5cm} | p{6cm} |}
    \hline
    \bf Name & \bf type & \bf size & \bf default value & \bf description \\ \hline
    name &     String &      &     '' &     connector"s unique name\\ \hline
    markerNumbers &     ArrayIndex &      &     [ MAXINT, MAXINT ] &     markers define contact gap\\ \hline
    nodeNumber &     Index &      &     MAXINT &     node number of a NodeGenericData with 3 $\times$ nSegments dataCoordinates (used for active set strategy ==> hold the gap of the last discontinuous iteration and the friction state)\\ \hline
    numberOfContactSegments &     Index &      &     3 &     number of linear contact segments to determine contact; each segment is a line and is associated to a data (history) variable; must be same as in according marker\\ \hline
    contactStiffness &     UReal &      &     0. &     contact (penalty) stiffness [SI:N/m/(contact segment)]; the stiffness is per length of the beam axis; specific contact forces (per length) $f_N$ act in contact normal direction only upon penetration\\ \hline
    contactDamping &     UReal &      &     0. &     contact damping [SI:N/(m s)/(contact segment)]; the damping is per length of the beam axis; acts in contact normal direction only upon penetration\\ \hline
    frictionVelocityPenalty &     UReal &      &     0. &     velocity dependent penalty coefficient for friction [SI:N/(m s)/(contact segment)]; the coefficient causes tangential (contact) forces against relative tangential velocities in the contact area\\ \hline
    frictionStiffness &     UReal &      &     0. &     CURRENTLY NOT IMPLEMENTED: displacement dependent penalty/stiffness coefficient for friction [SI:N/m/(contact segment)]; the coefficient causes tangential (contact) forces against relative tangential displacements in the contact area\\ \hline
    frictionCoefficient &     UReal &      &     0. &     friction coefficient $\mu$ [SI: 1]; tangential specific friction forces (per length) $f_T$ must fulfill the condition $f_T \le \mu f_N$\\ \hline
    circleRadius &     UReal &      &     0. &     radius [SI:m] of contact circle\\ \hline
    offset &     UReal &      &     0. &     offset [SI:m] of contact, e.g. to include thickness of cable element\\ \hline
    activeConnector &     bool &      &     True &     flag, which determines, if the connector is active; used to deactivate (temorarily) a connector or constraint\\ \hline
    visualization & VObjectContactFrictionCircleCable2D & & & parameters for visualization of item \\ \hline
	  \end{longtable}
	\end{center}
The item VObjectContactFrictionCircleCable2D has the following parameters:\vspace{-1cm}\\ 
%reference manual TABLE
\begin{center}
  \footnotesize
  \begin{longtable}{| p{4.5cm} | p{2.5cm} | p{0.5cm} | p{2.5cm} | p{6cm} |}
    \hline
    \bf Name & \bf type & \bf size & \bf default value & \bf description \\ \hline
    show &     Bool &      &     True &     set true, if item is shown in visualization and false if it is not shown\\ \hline
    drawSize &     float &      &     -1. &     drawing size = diameter of spring; size == -1.f means that default connector size is used\\ \hline
    color &     Float4 &      &     [-1.,-1.,-1.,-1.] &     RGB connector color; if R==-1, use default color\\ \hline
	  \end{longtable}
	\end{center}
{\bf Definition of quantities}:\\
\newpage

%+++++++++++++++++++++++++++++++++++
\mysubsubsection{ObjectJointSliding2D}
A specialized sliding joint (without rotation) in 2D between a Cable2D (marker1) and a position-based marker (marker0); the data coordinate x[0] provides the current index in slidingMarkerNumbers, and x[1] the local position in the cable element at the beginning of the timestep.
 \\  {\bf Requested marker type} = Marker::\_None \\ 
\vspace{12pt} \noindent The item {\bf ObjectJointSliding2D} with type = 'JointSliding2D' has the following parameters:\vspace{-1cm}\\ 
%reference manual TABLE
\begin{center}
  \footnotesize
  \begin{longtable}{| p{4.5cm} | p{2.5cm} | p{0.5cm} | p{2.5cm} | p{6cm} |}
    \hline
    \bf Name & \bf type & \bf size & \bf default value & \bf description \\ \hline
    name &     String &      &     '' &     constraints"s unique name\\ \hline
    markerNumbers &     ArrayIndex &      &     [ MAXINT, MAXINT ] &     marker0: position-marker of mass point or rigid body; marker1: updated marker to Cable2D element, where the sliding joint currently is attached to; must be initialized with an appropriate (global) marker number according to the starting position of the sliding object; this marker changes with time (PostNewtonStep)\\ \hline
    slidingMarkerNumbers &     ArrayIndex &      &     [] &     these markers are used to update marker1, if the sliding position exceeds the current cable"s range; the markers must be sorted such that marker(i) at x=cable.length is equal to marker(i+1) at x=0\\ \hline
    slidingMarkerOffsets &     Vector &      &     [] &     this list contains the offsets of every sliding object (given by slidingMarkerNumbers) w.r.t. to the initial position (0): marker0: offset=0, marker1: offset=Length(cable0), marker2: offset=Length(cable0)+Length(cable1), ...\\ \hline
    nodeNumber &     Index &      &     MAXINT &     node number of a NodeGenericData for 1 dataCoordinate showing the according marker number which is currently active and the start-of-step (global) sliding position\\ \hline
    classicalFormulation &     bool &      &     True &     uses a formulation with 3 equations, including the force in sliding direction to be zero; forces in global coordinates, only index 3; alternatively: use local formulation, which only needs two equations and can be used with index 2 formulation\\ \hline
    activeConnector &     bool &      &     True &     flag, which determines, if the connector is active; used to deactivate (temorarily) a connector or constraint\\ \hline
    visualization & VObjectJointSliding2D & & & parameters for visualization of item \\ \hline
	  \end{longtable}
	\end{center}
The item VObjectJointSliding2D has the following parameters:\vspace{-1cm}\\ 
%reference manual TABLE
\begin{center}
  \footnotesize
  \begin{longtable}{| p{4.5cm} | p{2.5cm} | p{0.5cm} | p{2.5cm} | p{6cm} |}
    \hline
    \bf Name & \bf type & \bf size & \bf default value & \bf description \\ \hline
    show &     bool &      &     True &     set true, if item is shown in visualization and false if it is not shown\\ \hline
    drawSize &     float &      &     -1. &     drawing size = radius of revolute joint; size == -1.f means that default connector size is used\\ \hline
    color &     Float4 &      &     [-1.,-1.,-1.,-1.] &     RGB connector color; if R==-1, use default color\\ \hline
	  \end{longtable}
	\end{center}

\noindent{\bf Short name} for Python: {\bf SlidingJoint2D}
 \vspace{6pt}\\{\bf Definition of quantities}:\\
{\bf The following output parameters are available as OutputVariableType in sensors and other functions}:\\ 
\startTable{output parameters}{symbol}{description}
\rowTable{Position}{}{position vector of joint given by marker0}
\rowTable{Velocity}{}{velocity vector of joint given by marker0}
\rowTable{SlidingCoordinate}{}{global sliding coordinate along all elements; the maximum sliding coordinate is equivalent to the reference lengths of all sliding elements}
\rowTable{Force}{}{joint force vector (3D)}
\finishTable
{\bf Description of Item}:
 \noindent
    \vspace{6pt}\\
    {\bf Definition of quantities}:
    \startTable{input parameter}{symbol}{description}
    \rowTable{nodeNumber}{$n_{GD}$}{node number of generic data node}
    \rowTable{markerNumbers[0]}{$m0$}{position-marker of mass point or rigid body}
    \rowTable{markerNumbers[1]}{$m1$}{marker to a Cable2D element, which is {\bf updated} in every PostNewtonStep; if the sliding body ($m0$) is in the range of all sliding cable elements, $m1$ contains the current marker number, which is active for the sliding joint}
    \rowTable{slidingMarkerNumbers}{$[m_{s0}, \ldots, m_{sn}]\tp$}{a list of $sn$ (global) marker numbers which are are used to update marker1}
    \rowTable{slidingMarkerOffsets}{$[d_{s0}, \ldots, d_{sn}]$}{a list of $sn$ scalar offsets, which represent the (reference arc) length of all previous sliding cable elements}
%    
    \finishTable
    \startTable{intermediate variables}{symbol}{description}
    \rowTable{data node}{$\xv=[x_{data0},\,x_{data1}]\tp$}{coordinates of node with node number $n_{GD}$}
    \rowTable{data coordinate 0}{$x_{data0}$}{the current index in slidingMarkerNumbers}
    \rowTable{data coordinate 1}{$x_{data1}$}{the global sliding coordinate (ranging from 0 to the total length of all sliding elements) at {\bf start-of-step} - beginning of the timestep}
    \rowTable{marker m0 position}{$\LU{0}{\pv}_{m0}$}{current global position which is provided by marker m0}
    \rowTable{marker m0 velocity}{$\LU{0}{\vv}_{m0}$}{current global velocity which is provided by marker m0}
%
    \rowTable{cable coordinates}{$\qv_{ANCF,m1}$}{current coordiantes of the ANCF cable element with the current marker $m1$ is referring to}
    \rowTable{sliding position}{$\LUR{0}{\rv}{ANCF} = \Sm(s_{el})\qv_{ANCF,m1}$}{current global position at the ANCF cable element, evaluated at local sliding position $s_{el}$}
    \rowTable{sliding position slope}{$\LURU{0}{\rv}{ANCF}{\prime} = \Sm^\prime(s_{el})\qv_{ANCF,m1} = [r^\prime_0,\,r^\prime_1]\tp$}{current global slope vector of the ANCF cable element, evaluated at local sliding position $s_{el}$}
    \rowTable{sliding velocity}{$\LUR{0}{\vv}{ANCF} = \Sm(s_{el})\dot\qv_{ANCF,m1}$}{current global velocity at the ANCF cable element, evaluated at local sliding position $s_{el}$ ($s_{el}$ not differentiated!!!)}
    \rowTable{sliding velocity slope}{$\LURU{0}{\vv}{ANCF}{\prime} = \Sm^\prime(s_{el})\dot\qv_{ANCF,m1}$}{current global slope velocity vector of the ANCF cable element, evaluated at local sliding position $s_{el}$}
%
    \rowTable{sliding normal vector}{$\LU{0}{\nv} = [-r^\prime_1,\,r^\prime_0]$}{2D normal vector computed from slope $\rv^\prime=\LURU{0}{\rv}{ANCF}{\prime}$}
    \rowTable{sliding normal velocity vector}{$\LU{0}{\dot\nv} = [-\dot r^\prime_1,\,\dot r^\prime_0]$}{time derivative of 2D normal vector computed from slope velocity $\dot \rv^\prime=\LURU{0}{\dot \rv}{ANCF}{\prime}$}
%
    \rowTable{algebraic coordinates}{$\zv=[\lambda_0,\,\lambda_1,\, s]\tp$}{algebraic coordinates composed of Lagrange multipliers $\lambda_0$ and $\lambda_1$ (in local cable coordinates: $\lambda_0$ is in axis direction) and the current sliding coordinate $s$, which is local in the current cable element. }
    \rowTable{local sliding coordinate}{$s$}{local incremental sliding coordinate $s$: the (algebraic) sliding coordinate {\bf relative to the start-of-step value}. Thus, $s$ only contains small local increments.}
    \finishTable
    \startTable{output variables}{symbol}{formula}
    \rowTable{Position}{$\LU{0}{\pv}_{m0}$}{current global position of position marker $m0$}
    \rowTable{Velocity}{$\LU{0}{\vv}_{m0}$}{current global velocity of position marker $m0$}
    \rowTable{SlidingCoordinate}{$s_g = s + x_{data1}$}{current value of the global sliding coordinate}
    \rowTable{Force}{$\fv$}{see below}
    \finishTable

    %cable
    Assume we have given the sliding coordinate $s$ (e.g., as a guess of the Newton method or beginning of the time step). 
    The element sliding coordinate (in the local coordinates of the current sliding element) is computed as
    \be
      s_{el} = s + x_{data1} - d_{m1} = s_g - d_{m1}.
    \ee
    The vector (=difference; error) between the marker $m0$ and the marker $m1$ (=$\rv_{ANCF}$) positions reads
    \be
      \LU{0}{\Delta\pv} = \LUR{0}{\rv}{ANCF} - \LU{0}{\pv}_{m0}
    \ee
    The vector (=difference; error) between the marker $m0$ and the marker $m1$ velocities reads
    \be
      \LU{0}{\Delta\vv} = \LUR{0}{\dot\rv}{ANCF} - \LU{0}{\vv}_{m0}
    \ee
%
    %+++++++++++++++++++++++++++++++++++++++++++++
    \noindent {\bf Algebraic constraint equations (classicalFormulation=True)}:
    The 2D sliding joint is implemented having 3 equations, using the special algebraic coordinates $\zv$.
    The algebraic equations read
    \bea
      \LU{0}{\Delta\pv\tp} &=& \Null, \quad \mbox{... 2 index 3 equations, ensuring the sliding body to stay at the cable}\\
      \left[\lambda_0,\lambda_1\right] \cdot  \LURU{0}{\rv}{ANCF}{\prime} &=& 0, \quad \mbox{... 1 index 1 equation, ensuring the force in sliding direction = 0}  \\
    \eea
    No index 2 case exists, because no time derivative exists for $s_{el}$. The jacobian matrices for algebraic and ODE2 coordinates read
    \be
      J_{AE} = \mr{0}{0}{r^\prime_0} {0}{0}{r^\prime_1} {r^\prime_0}{r^\prime_1}{r^{\prime\prime}_0\lambda_0 + r^{\prime\prime}_1\lambda_1}    %\LURU{0}{\rv}{ANCF}{\prime\prime \mathrm{T}} \vp{\lambda_0}{\lambda_1}}
    \ee
    \be
      J_{ODE2} = \mp{-J_{pos,m0}}{\Sm(s_{el})} {\Null\tp}{\left[\lambda_0,\,\lambda_1\right]\cdot\Sm^\prime(s_{el}) }
    \ee
    if \texttt{activeConnector = False}, the algebraic equations are changed to:
    \bea
      \lambda_0 &=& 0,   \\
      \lambda_1 &=& 0,   \\
      s &=& 0
    \eea
    %the algebraic variables are \be \qv_{AE}=[\lambda_x\;\; \lambda_y \;\; s]^T \ee in which $\lambda_x$ and $\lambda_y$ are the Lagrange multipliers for the position of the sliding joint; 
    %+++++++++++++++++++++++++++++++++++++++++++++
    \noindent {\bf Algebraic constraint equations (classicalFormulation=False)}:
    The 2D sliding joint is implemented having 3 equations (first equation is dummy and could be eliminated), using the special algebraic coordinates $\zv$. 
    The algebraic equations read
    \bea
      \lambda_0 &=& 0, \quad \mbox{... this equation is not necessary, but can be used for switching to other modes}  \\
      \LU{0}{\Delta\pv\tp} \LU{0}{\nv} &=& 0, \quad \mbox{... equation ensures that sliding body stays at cable centerline; index3 equation}\\
      \LU{0}{\Delta\pv\tp} \LURU{0}{\rv}{ANCF}{\prime} &=& 0. \quad \mbox{... resolves the sliding coordinate $s$; index1 equation!}
    \eea
    In the index 2 case, the second equation reads
    \be
      \LU{0}{\Delta\vv\tp} \LU{0}{\nv}  + \LU{0}{\Delta\pv\tp} \LU{0}{\dot\nv}  = 0
    \ee
    if \texttt{activeConnector = False}, the algebraic equations are changed to:
    \bea
      \lambda_0 &=& 0,   \\
      \lambda_1 &=& 0,   \\
      s &=& 0
    \eea   
    %the algebraic variables are \be \qv_{AE}=[\lambda_x\;\; \lambda_y \;\; s]^T \ee in which $\lambda_x$ and $\lambda_y$ are the Lagrange multipliers for the position of the sliding joint; 
%
    %+++++++++++++++++++++++++++++++++++++++++++++
    \noindent {\bf Post Newton Step}:
    After the Newton solver has converged, a PostNewtonStep is performed for the element, which
    updates the marker $m1$ index if necessary.
    \bea
      s_{el} < 0 \quad \ra \quad x_{data0}\;-\!\!=1 \nonumber\\
      s_{el} > L \quad \ra \quad x_{data0}\;+\!\!=1
    \eea
    Furthermore, it is checked, if $x_{data0}$ becomes smaller than zero, which raises a warning and keeps $x_{data0}=0$.
    The same results if $x_{data0}\ge sn$, then $x_{data0} = sn$.
    Finally, the data coordinate is updated in order to provide the starting value for the next step,
    \be
      x_{data1} \;+\!\!= s.
    \ee
    %the data coordinates are \be \qv_{Data} = [i_{marker} \;\; s_{0}]^T \ee in which $i_{marker}$ is the current local index to the slidingMarkerNumber list and  $s_{0}$ is the sliding coordinate (which is the total sliding length along all cable elements in the cableMarkerNumber list) at the beginning of the solution step.
%
    {\bf Examples}: see TestModels!
\newpage

%+++++++++++++++++++++++++++++++++++
\mysubsubsection{ObjectJointALEMoving2D}
A specialized axially moving joint (without rotation) in 2D between a ALE Cable2D (marker1) and a position-based marker (marker0); ALE=Arbitrary Lagrangian Eulerian; the data coordinate x[0] provides the current index in slidingMarkerNumbers, and the ODE2 coordinate q[0] provides the (given) moving coordinate in the cable element.
 \\  {\bf Requested marker type} = Marker::\_None \\ 
\vspace{12pt} \noindent The item {\bf ObjectJointALEMoving2D} with type = 'JointALEMoving2D' has the following parameters:\vspace{-1cm}\\ 
%reference manual TABLE
\begin{center}
  \footnotesize
  \begin{longtable}{| p{4.5cm} | p{2.5cm} | p{0.5cm} | p{2.5cm} | p{6cm} |}
    \hline
    \bf Name & \bf type & \bf size & \bf default value & \bf description \\ \hline
    name &     String &      &     '' &     constraints"s unique name\\ \hline
    markerNumbers &     ArrayIndex &      &     [ MAXINT, MAXINT ] &     marker m0: position-marker of mass point or rigid body; marker m1: updated marker to ANCF Cable2D element, where the sliding joint currently is attached to; must be initialized with an appropriate (global) marker number according to the starting position of the sliding object; this marker changes with time (PostNewtonStep)\\ \hline
    slidingMarkerNumbers &     ArrayIndex &      &     [] &     a list of sn (global) marker numbers which are are used to update marker1\\ \hline
    slidingMarkerOffsets &     Vector &      &     [] &     this list contains the offsets of every sliding object (given by slidingMarkerNumbers) w.r.t. to the initial position (0): marker0: offset=0, marker1: offset=Length(cable0), marker2: offset=Length(cable0)+Length(cable1), ...\\ \hline
    slidingOffset &     Real &      &     0. &     sliding offset list [SI:m]: a list of sn scalar offsets, which represent the (reference arc) length of all previous sliding cable elements\\ \hline
    nodeNumbers &     ArrayIndex &      &     [ MAXINT, MAXINT ] &     node number of NodeGenericData (GD) with one data coordinate and of NodeGenericODE2 (ALE) with one ODE2 coordinate\\ \hline
    usePenaltyFormulation &     bool &      &     False &     flag, which determines, if the connector is formulated with penalty, but still using algebraic equations (IsPenaltyConnector() still false)\\ \hline
    penaltyStiffness &     Real &      &     0. &     penalty stiffness [SI:N/m] used if usePenaltyFormulation=True\\ \hline
    activeConnector &     bool &      &     True &     flag, which determines, if the connector is active; used to deactivate (temorarily) a connector or constraint\\ \hline
    visualization & VObjectJointALEMoving2D & & & parameters for visualization of item \\ \hline
	  \end{longtable}
	\end{center}
The item VObjectJointALEMoving2D has the following parameters:\vspace{-1cm}\\ 
%reference manual TABLE
\begin{center}
  \footnotesize
  \begin{longtable}{| p{4.5cm} | p{2.5cm} | p{0.5cm} | p{2.5cm} | p{6cm} |}
    \hline
    \bf Name & \bf type & \bf size & \bf default value & \bf description \\ \hline
    show &     bool &      &     True &     set true, if item is shown in visualization and false if it is not shown\\ \hline
    drawSize &     float &      &     -1. &     drawing size = radius of revolute joint; size == -1.f means that default connector size is used\\ \hline
    color &     Float4 &      &     [-1.,-1.,-1.,-1.] &     RGB connector color; if R==-1, use default color\\ \hline
	  \end{longtable}
	\end{center}

\noindent{\bf Short name} for Python: {\bf ALEMovingJoint2D}
 \vspace{6pt}\\{\bf Definition of quantities}:\\
\startTable{input parameter}{symbol}{description see tables above}
\rowTable{markerNumbers}{$[m0,\,m1]\tp$}{}
\rowTable{slidingMarkerNumbers}{$[m_{s0}, \ldots, m_{sn}]\tp$}{}
\rowTable{slidingMarkerOffsets}{$[d_{s0}, \ldots, d_{sn}]$}{}
\rowTable{slidingOffset}{$s_{off}$}{}
\rowTable{nodeNumbers}{$[n_{GD}, n_{ALE}]$}{}
\rowTable{penaltyStiffness}{$k$}{}
\finishTable
{\bf The following output parameters are available as OutputVariableType in sensors and other functions}:\\ 
\startTable{output parameters}{symbol}{description}
\rowTable{Position}{$\LU{0}{\pv}_{m0}$}{current global position of position marker $m0$}
\rowTable{Velocity}{$\LU{0}{\vv}_{m0}$}{current global velocity of position marker $m0$}
\rowTable{SlidingCoordinate}{$s_g = q_{ALE} + s_{off}$}{current value of the global sliding ALE coordinate, including offset; note that reference coordinate of $q_{ALE}$ is ignored!}
\rowTable{Coordinates}{$[x_{data0},\,q_{ALE}]\tp$}{provides two values: [0] = current sliding marker index, [1] = ALE sliding coordinate}
\rowTable{Coordinates\_t}{$[\dot q_{ALE}]\tp$}{provides ALE sliding velocity}
\rowTable{Force}{$\fv$}{joint force vector (3D)}
\finishTable
{\bf Description of Item}:
 \noindent
    %\vspace{6pt}\\
    %{\bf Definition of quantities}:
    %\startTable{input parameter}{symbol}{description}
    %\rowTable{nodeNumbers}{$[n_{GD}, n_{ALE}]$}{node number of NodeGenericData $n_{GD}$ with one data coordinate and $n_{ALE}$ of NodeGenericODE2 with one ODE2 coordinate}
    %\rowTable{slidingOffset}{$s_{off}$}{sliding offset, which is added to the current value of the ALE node coordinate}
    %\rowTable{markerNumbers[0]}{$m0$}{position-marker of mass point or rigid body}
    %\rowTable{markerNumbers[1]}{$m1$}{marker to a Cable2D element, which is {\bf updated} in every PostNewtonStep; $m1$ contains the current marker number, which is active for the sliding joint; must be initialized appropriately}
    %\rowTable{slidingMarkerNumbers}{$[m_{s0}, \ldots, m_{sn}]\tp$}{a list of $sn$ (global) marker numbers which are are used to update marker1}
    %\rowTable{slidingMarkerOffsets}{$[d_{s0}, \ldots, d_{sn}]$}{a list of $sn$ scalar offsets, which represent the (reference arc) length of all previous sliding cable elements}
    %\rowTable{penaltyStiffness}{$k$}{penalty stiffness coefficients for usePenaltyFormulation=True}
    %\finishTable
%    
    \startTable{intermediate variables}{symbol}{description}
    \rowTable{generic data node}{$\xv=[x_{data0}]\tp$}{coordinates of node with node number $n_{GD}$}
    \rowTable{generic ODE2 node}{$\qv=[q_{0}]\tp$}{coordinates of node with node number $n_{ALE}$, which is shared with all ALE-ANCF and ALE sliding joint objects}
    \rowTable{data coordinate}{$x_{data0}$}{the current index in slidingMarkerNumbers}
    \rowTable{ALE coordinate}{$q_{ALE} = q_{0}$}{current ALE coordinate (in fact this is the Eulerian coordinate in the ALE formulation); note that reference coordinate of $q_{ALE}$ is ignored!}
    \rowTable{marker m0 position}{$\LU{0}{\pv}_{m0}$}{current global position which is provided by marker m0}
    \rowTable{marker m0 velocity}{$\LU{0}{\vv}_{m0}$}{current global velocity which is provided by marker m0}
%
    \rowTable{cable coordinates}{$\qv_{ANCF,m1}$}{current coordiantes of the ANCF cable element with the current marker $m1$ is referring to}
    \rowTable{sliding position}{$\LUR{0}{\rv}{ANCF} = \Sm(s_{el})\qv_{ANCF,m1}$}{current global position at the ANCF cable element, evaluated at local sliding position $s_{el}$}
    \rowTable{sliding position slope}{$\LURU{0}{\rv}{ANCF}{\prime} = \Sm^\prime(s_{el})\qv_{ANCF,m1}$}{current global slope vector of the ANCF cable element, evaluated at local sliding position $s_{el}$}
    \rowTable{sliding velocity}{$\LUR{0}{\vv}{ANCF} = \Sm(s_{el})\dot\qv_{ANCF,m1} + \dot q_{ALE} \LURU{0}{\rv}{ANCF}{\prime}$}{current global velocity at the ANCF cable element, evaluated at local sliding position $s_{el}$, including convective term}
%
    \rowTable{sliding normal vector}{$\LU{0}{\nv} = [-r^\prime_1,\,r^\prime_0]$}{2D normal vector computed from slope $\rv^\prime=\LURU{0}{\rv}{ANCF}{\prime}$}
    %\rowTable{sliding normal vector}{$\LU{0}{\dot\nv} = [-\dot r^\prime_1,\,\dot r^\prime_0]$}{time derivative of 2D normal vector computed from slope velocity $\dot \rv^\prime=\LURU{0}{\dot \rv}{ANCF}{\prime}$}
%
    \rowTable{algebraic coordinates}{$\zv=[\lambda_0,\,\lambda_1]\tp$}{algebraic coordinates composed of Lagrange multipliers $\lambda_0$ and $\lambda_1$, according to the algebraic equations }
    \finishTable
    %\startTable{output variables}{symbol}{formula}
    %\rowTable{Position}{$\LU{0}{\pv}_{m0}$}{current global position of position marker $m0$}
    %\rowTable{Velocity}{$\LU{0}{\vv}_{m0}$}{current global velocity of position marker $m0$}
    %\rowTable{SlidingCoordinate}{$s_g = q_{ALE} + s_{off}$}{current value of the global sliding ALE coordinate, including offset; note that reference coordinate of $q_{ALE}$ is ignored!}
    %\rowTable{Coordinates}{$[x_{data0},\,q_{ALE}]\tp$}{}
    %\rowTable{Coordinates\_t}{$[\dot q_{ALE}]\tp$}{}
    %\rowTable{Force}{$\fv$}{see below}
    %\finishTable

    The element sliding coordinate (in the local coordinates of the current sliding element) is computed from the ALE coordinate
    \be
      s_{el} = q_{ALE} + s_{off} - d_{m1} = s_g - d_{m1}.
    \ee
    The vector (=difference; error) between the marker $m0$ and the marker $m1$ (=$\rv_{ANCF}$) positions reads
    \be
      \LU{0}{\Delta\pv} = \LUR{0}{\rv}{ANCF} - \LU{0}{\pv}_{m0}
    \ee
    The vector (=difference; error) between the marker $m0$ and the marker $m1$ velocities reads
    \be
      \LU{0}{\Delta\vv} = \LUR{0}{\vv}{ANCF} - \LU{0}{\vv}_{m0}
    \ee
%
    %+++++++++++++++++++++++++++++++++++++++++++++
    \noindent {\bf Algebraic constraint equations}:
    The 2D sliding joint is implemented having 2 equations, using the Lagrange multipliers $\zv$. 
    The algebraic (index 3) equations read
    \be
      \LU{0}{\Delta\pv} = 0
    \ee
    Note that the Lagrange multipliers $[\lambda_0,\,\lambda_1]\tp$are the global forces in the joint.
    In the index 2 case the algebraic equations read
    \be
      \LU{0}{\Delta\vv} = 0
    \ee
    If \texttt{usePenalty = True}, the algebraic equations are changed to:
    \be
      \LU{0}{\Delta \pv} - \frac 1 k \zv = 0.
    \ee
%
    %not realized yet, because AE Jacobian becomes involved:
    %If \texttt{usePenaltyFormulation = True}, the algebraic equations are changed to:
    %\bea
    %  k_1 \LURU{0}{\rv}{ANCF}{\prime \mathrm{T}}   \LU{0}{\Delta\pv} - \lambda_0 &=& 0, \nonumber \\
    %  k_2 \LU{0}{\nv\tp}   \LU{0}{\Delta\pv}  - \lambda_1 &=& 0.
    %\eea
    %Note that in this case, the Lagrange multipliers $[\lambda_0,\,\lambda_1]\tp$are the local ($m1$) forces in the joint.

    \noindent If \texttt{activeConnector = False}, the algebraic equations are changed to:
    \bea
      \lambda_0 &=& 0,   \\
      \lambda_1 &=& 0.
    \eea   
%
    %+++++++++++++++++++++++++++++++++++++++++++++
    \noindent {\bf Post Newton Step}:
    After the Newton solver has converged, a PostNewtonStep is performed for the element, which
    updates the marker $m1$ index if necessary.
    \bea
      s_{el} < 0 \quad \ra \quad x_{data0} \;-\!\!=1 \nonumber\\
      s_{el} > L \quad \ra \quad x_{data0} \;+\!\!=1
    \eea
    Furthermore, it is checked, if $x_{data0}$ becomes smaller than zero, which raises a warning and keeps $x_{data0}=0$.
    The same results if $x_{data0}\ge sn$, then $x_{data0} = sn$.
    Finally, the data coordinate is updated in order to provide the starting value for the next step,
    \be
      x_{data1} \;+\!\!= s.
    \ee
%
    {\bf Examples}: see TestModels!
\newpage

%+++++++++++++++++++++++++++++++++++
\mysubsubsection{ObjectJointGeneric}
A generic joint in 3D; constrains components of the absolute position and rotations of two points given by PointMarkers or RigidMarkers; an additional local rotation can be used to define three rotation axes and/or sliding axes
 \\  {\bf Requested marker type} = (Marker::Type)((Index)Marker::Position + (Index)Marker::Orientation) \\ 
\vspace{12pt} \noindent The item {\bf ObjectJointGeneric} with type = 'JointGeneric' has the following parameters:\vspace{-1cm}\\ 
%reference manual TABLE
\begin{center}
  \footnotesize
  \begin{longtable}{| p{4.5cm} | p{2.5cm} | p{0.5cm} | p{2.5cm} | p{6cm} |}
    \hline
    \bf Name & \bf type & \bf size & \bf default value & \bf description \\ \hline
    name &     String &      &     '' &     constraints"s unique name\\ \hline
    markerNumbers &     ArrayIndex &     2 &     [ MAXINT, MAXINT ] &     list of markers used in connector\\ \hline
    constrainedAxes &     ArrayIndex &     6 &     [1,1,1,1,1,1] &     flag, which determines which translation (0,1,2) and rotation (3,4,5) axes are constrained; 0=free, 1=constrained\\ \hline
    rotationMarker0 &     Matrix3D &      &     [[1,0,0], [0,1,0], [0,0,1]] &     local rotation matrix for marker 0; translation and rotation axes for marker0 are defined in the local body coordinate system and additionally transformed by rotationMarker0\\ \hline
    rotationMarker1 &     Matrix3D &      &     [[1,0,0], [0,1,0], [0,0,1]] &     local rotation matrix for marker 1; translation and rotation axes for marker1 are defined in the local body coordinate system and additionally transformed by rotationMarker1\\ \hline
    activeConnector &     bool &      &     True &     flag, which determines, if the connector is active; used to deactivate (temorarily) a connector or constraint\\ \hline
    offsetUserFunctionParameters &     Vector6D &      &     [0.,0.,0.,0.,0.,0.] &     vector of 6 parameters for joint"s offsetUserFunction\\ \hline
    offsetUserFunction &     PyFunctionVector6DScalarVector6D &     \tabnewline  &     \tabnewline 0 &     A python function which defines the time-dependent (fixed) offset of translation (indices 0,1,2) and rotation (indices 3,4,5) joint coordinates with parameters (t, offsetUserFunctionParameters); the offset represents the current value of the object; it is highly RECOMMENDED to use sufficiently smooth functions, having consistent initial offsets with initial configuration of bodies, zero or compatible initial offset-velocity, and no accelerations; Example for python function: def f(t, offsetUserFunctionParameters): return [offsetUserFunctionParameters[0]*(1 - np.cos(t*10*2*np.pi)), 0,0,0,0,0]\\ \hline
    offsetUserFunction\_t &     PyFunctionVector6DScalarVector6D &     \tabnewline  &     \tabnewline 0 &     time derivative of offsetUserFunction using the same parameters; needed for "velocityLevel=True", or for index2 time integration and for computation of initial accelerations in SecondOrderImplicit integrators\\ \hline
    visualization & VObjectJointGeneric & & & parameters for visualization of item \\ \hline
	  \end{longtable}
	\end{center}
The item VObjectJointGeneric has the following parameters:\vspace{-1cm}\\ 
%reference manual TABLE
\begin{center}
  \footnotesize
  \begin{longtable}{| p{4.5cm} | p{2.5cm} | p{0.5cm} | p{2.5cm} | p{6cm} |}
    \hline
    \bf Name & \bf type & \bf size & \bf default value & \bf description \\ \hline
    show &     bool &      &     True &     set true, if item is shown in visualization and false if it is not shown\\ \hline
    axesRadius &     float &      &     0.1 &     radius of joint axes to draw\\ \hline
    axesLength &     float &      &     0.4 &     length of joint axes to draw\\ \hline
    color &     Float4 &      &     [-1.,-1.,-1.,-1.] &     RGB connector color; if R==-1, use default color\\ \hline
	  \end{longtable}
	\end{center}

\noindent{\bf Short name} for Python: {\bf GenericJoint}
 \vspace{6pt}\\{\bf Definition of quantities}:\\
\newpage

%+++++++++++++++++++++++++++++++++++
\mysubsubsection{ObjectJointRevolute2D}
A revolute joint in 2D; constrains the absolute 2D position of two points given by PointMarkers or RigidMarkers
 \\  {\bf Requested marker type} = Marker::Position \\ 
\vspace{12pt} \noindent The item {\bf ObjectJointRevolute2D} with type = 'JointRevolute2D' has the following parameters:\vspace{-1cm}\\ 
%reference manual TABLE
\begin{center}
  \footnotesize
  \begin{longtable}{| p{4.5cm} | p{2.5cm} | p{0.5cm} | p{2.5cm} | p{6cm} |}
    \hline
    \bf Name & \bf type & \bf size & \bf default value & \bf description \\ \hline
    name &     String &      &     '' &     constraints"s unique name\\ \hline
    markerNumbers &     ArrayIndex &      &     [ MAXINT, MAXINT ] &     list of markers used in connector\\ \hline
    activeConnector &     bool &      &     True &     flag, which determines, if the connector is active; used to deactivate (temorarily) a connector or constraint\\ \hline
    visualization & VObjectJointRevolute2D & & & parameters for visualization of item \\ \hline
	  \end{longtable}
	\end{center}
The item VObjectJointRevolute2D has the following parameters:\vspace{-1cm}\\ 
%reference manual TABLE
\begin{center}
  \footnotesize
  \begin{longtable}{| p{4.5cm} | p{2.5cm} | p{0.5cm} | p{2.5cm} | p{6cm} |}
    \hline
    \bf Name & \bf type & \bf size & \bf default value & \bf description \\ \hline
    show &     bool &      &     True &     set true, if item is shown in visualization and false if it is not shown\\ \hline
    drawSize &     float &      &     -1. &     drawing size = radius of revolute joint; size == -1.f means that default connector size is used\\ \hline
    color &     Float4 &      &     [-1.,-1.,-1.,-1.] &     RGB connector color; if R==-1, use default color\\ \hline
	  \end{longtable}
	\end{center}

\noindent{\bf Short name} for Python: {\bf RevoluteJoint2D}
 \vspace{6pt}\\{\bf Definition of quantities}:\\
\newpage

%+++++++++++++++++++++++++++++++++++
\mysubsubsection{ObjectJointPrismatic2D}
A prismatic joint in 2D; allows the relative motion of two bodies, using two RigidMarkers; the vector $\tv_0$ = axisMarker0 is given in local coordinates of the first marker's (body) frame and defines the prismatic axis; the vector $\mathbf{n}_1$ = normalMarker1 is given in the second marker's (body) frame and is the normal vector to the prismatic axis; using the global position vector $\pv_0$ and rotation matrix $\Am_0$ of marker0 and the global position vector $\pv_1$ rotation matrix $\Am_1$ of marker1, the equations for the prismatic joint follow as \be (\pv_1-\pv_0)^T\cdot \Am_1 \cdot \mathbf{n}_1 = 0 \ee  \be (\Am_0 \cdot \tv_0)^T \cdot \Am_1 \cdot \mathbf{n}_1 = 0\ee The lagrange multipliers follow for these two equations $[\lambda_0,\lambda_1]$, in which $\lambda_0$ is the transverse force and $\lambda_1$ is the torque in the joint.
 \\  {\bf Requested marker type} = (Marker::Type)(Marker::Position + Marker::Orientation) \\ 
\vspace{12pt} \noindent The item {\bf ObjectJointPrismatic2D} with type = 'JointPrismatic2D' has the following parameters:\vspace{-1cm}\\ 
%reference manual TABLE
\begin{center}
  \footnotesize
  \begin{longtable}{| p{4.5cm} | p{2.5cm} | p{0.5cm} | p{2.5cm} | p{6cm} |}
    \hline
    \bf Name & \bf type & \bf size & \bf default value & \bf description \\ \hline
    name &     String &      &     '' &     constraints"s unique name\\ \hline
    markerNumbers &     ArrayIndex &      &     [ MAXINT, MAXINT ] &     list of markers used in connector\\ \hline
    axisMarker0 &     Vector3D &      &     [1.,0.,0.] &     direction of prismatic axis, given as a 3D vector in Marker0 frame\\ \hline
    normalMarker1 &     Vector3D &      &     [0.,1.,0.] &     direction of normal to prismatic axis, given as a 3D vector in Marker1 frame\\ \hline
    constrainRotation &     bool &      &     True &     flag, which determines, if the connector also constrains the relative rotation of the two objects; if set to false, the constraint will keep an algebraic equation set equal zero\\ \hline
    activeConnector &     bool &      &     True &     flag, which determines, if the connector is active; used to deactivate (temorarily) a connector or constraint\\ \hline
    visualization & VObjectJointPrismatic2D & & & parameters for visualization of item \\ \hline
	  \end{longtable}
	\end{center}
The item VObjectJointPrismatic2D has the following parameters:\vspace{-1cm}\\ 
%reference manual TABLE
\begin{center}
  \footnotesize
  \begin{longtable}{| p{4.5cm} | p{2.5cm} | p{0.5cm} | p{2.5cm} | p{6cm} |}
    \hline
    \bf Name & \bf type & \bf size & \bf default value & \bf description \\ \hline
    show &     bool &      &     True &     set true, if item is shown in visualization and false if it is not shown\\ \hline
    drawSize &     float &      &     -1. &     drawing size = radius of revolute joint; size == -1.f means that default connector size is used\\ \hline
    color &     Float4 &      &     [-1.,-1.,-1.,-1.] &     RGB connector color; if R==-1, use default color\\ \hline
	  \end{longtable}
	\end{center}

\noindent{\bf Short name} for Python: {\bf PrismaticJoint2D}
 \vspace{6pt}\\{\bf Definition of quantities}:\\

\newpage
%+++++++++++++++++++++++++++++++
%+++++++++++++++++++++++++++++++
\mysubsection{Markers}

%+++++++++++++++++++++++++++++++++++
\mysubsubsection{MarkerBodyMass}
A marker attached to the body mass; use this marker to apply a body-load (e.g. gravitational force).
 \\\vspace{12pt} \noindent The item {\bf MarkerBodyMass} with type = 'BodyMass' has the following parameters:\vspace{-1cm}\\ 
%reference manual TABLE
\begin{center}
  \footnotesize
  \begin{longtable}{| p{4.5cm} | p{2.5cm} | p{0.5cm} | p{2.5cm} | p{6cm} |}
    \hline
    \bf Name & \bf type & \bf size & \bf default value & \bf description \\ \hline
    name &     String &      &     '' &     marker"s unique name\\ \hline
    bodyNumber &     Index &      &     MAXINT &     body number to which marker is attached to\\ \hline
    visualization & VMarkerBodyMass & & & parameters for visualization of item \\ \hline
	  \end{longtable}
	\end{center}
The item VMarkerBodyMass has the following parameters:\vspace{-1cm}\\ 
%reference manual TABLE
\begin{center}
  \footnotesize
  \begin{longtable}{| p{4.5cm} | p{2.5cm} | p{0.5cm} | p{2.5cm} | p{6cm} |}
    \hline
    \bf Name & \bf type & \bf size & \bf default value & \bf description \\ \hline
    show &     bool &      &     True &     set true, if item is shown in visualization and false if it is not shown\\ \hline
	  \end{longtable}
	\end{center}
{\bf Definition of quantities}:\\
\newpage

%+++++++++++++++++++++++++++++++++++
\mysubsubsection{MarkerBodyPosition}
A position body-marker attached to local position (x,y,z) of the body.
 \\\vspace{12pt} \noindent The item {\bf MarkerBodyPosition} with type = 'BodyPosition' has the following parameters:\vspace{-1cm}\\ 
%reference manual TABLE
\begin{center}
  \footnotesize
  \begin{longtable}{| p{4.5cm} | p{2.5cm} | p{0.5cm} | p{2.5cm} | p{6cm} |}
    \hline
    \bf Name & \bf type & \bf size & \bf default value & \bf description \\ \hline
    name &     String &      &     '' &     marker"s unique name\\ \hline
    bodyNumber &     Index &      &     MAXINT &     body number to which marker is attached to\\ \hline
    localPosition &     Vector3D &     3 &     [0.,0.,0.] &     local body position of marker; e.g. local (body-fixed) position where force is applied to\\ \hline
    visualization & VMarkerBodyPosition & & & parameters for visualization of item \\ \hline
	  \end{longtable}
	\end{center}
The item VMarkerBodyPosition has the following parameters:\vspace{-1cm}\\ 
%reference manual TABLE
\begin{center}
  \footnotesize
  \begin{longtable}{| p{4.5cm} | p{2.5cm} | p{0.5cm} | p{2.5cm} | p{6cm} |}
    \hline
    \bf Name & \bf type & \bf size & \bf default value & \bf description \\ \hline
    show &     bool &      &     True &     set true, if item is shown in visualization and false if it is not shown\\ \hline
	  \end{longtable}
	\end{center}
{\bf Definition of quantities}:\\
\newpage

%+++++++++++++++++++++++++++++++++++
\mysubsubsection{MarkerBodyRigid}
A rigid-body (position+orientation) body-marker attached to local position (x,y,z) of the body.
 \\\vspace{12pt} \noindent The item {\bf MarkerBodyRigid} with type = 'BodyRigid' has the following parameters:\vspace{-1cm}\\ 
%reference manual TABLE
\begin{center}
  \footnotesize
  \begin{longtable}{| p{4.5cm} | p{2.5cm} | p{0.5cm} | p{2.5cm} | p{6cm} |}
    \hline
    \bf Name & \bf type & \bf size & \bf default value & \bf description \\ \hline
    name &     String &      &     '' &     marker"s unique name\\ \hline
    bodyNumber &     Index &      &     MAXINT &     body number to which marker is attached to\\ \hline
    localPosition &     Vector3D &     3 &     [0.,0.,0.] &     local body position of marker; e.g. local (body-fixed) position where force is applied to\\ \hline
    visualization & VMarkerBodyRigid & & & parameters for visualization of item \\ \hline
	  \end{longtable}
	\end{center}
The item VMarkerBodyRigid has the following parameters:\vspace{-1cm}\\ 
%reference manual TABLE
\begin{center}
  \footnotesize
  \begin{longtable}{| p{4.5cm} | p{2.5cm} | p{0.5cm} | p{2.5cm} | p{6cm} |}
    \hline
    \bf Name & \bf type & \bf size & \bf default value & \bf description \\ \hline
    show &     bool &      &     True &     set true, if item is shown in visualization and false if it is not shown\\ \hline
	  \end{longtable}
	\end{center}
{\bf Definition of quantities}:\\
\newpage

%+++++++++++++++++++++++++++++++++++
\mysubsubsection{MarkerNodePosition}
A node-Marker attached to a position-based node.
 \\\vspace{12pt} \noindent The item {\bf MarkerNodePosition} with type = 'NodePosition' has the following parameters:\vspace{-1cm}\\ 
%reference manual TABLE
\begin{center}
  \footnotesize
  \begin{longtable}{| p{4.5cm} | p{2.5cm} | p{0.5cm} | p{2.5cm} | p{6cm} |}
    \hline
    \bf Name & \bf type & \bf size & \bf default value & \bf description \\ \hline
    name &     String &      &     '' &     marker"s unique name\\ \hline
    nodeNumber &     Index &      &     MAXINT &     node number to which marker is attached to\\ \hline
    visualization & VMarkerNodePosition & & & parameters for visualization of item \\ \hline
	  \end{longtable}
	\end{center}
The item VMarkerNodePosition has the following parameters:\vspace{-1cm}\\ 
%reference manual TABLE
\begin{center}
  \footnotesize
  \begin{longtable}{| p{4.5cm} | p{2.5cm} | p{0.5cm} | p{2.5cm} | p{6cm} |}
    \hline
    \bf Name & \bf type & \bf size & \bf default value & \bf description \\ \hline
    show &     bool &      &     True &     set true, if item is shown in visualization and false if it is not shown\\ \hline
	  \end{longtable}
	\end{center}
{\bf Definition of quantities}:\\
\newpage

%+++++++++++++++++++++++++++++++++++
\mysubsubsection{MarkerNodeRigid}
A rigid-body (position+orientation) node-marker attached to a rigid-body node.
 \\\vspace{12pt} \noindent The item {\bf MarkerNodeRigid} with type = 'NodeRigid' has the following parameters:\vspace{-1cm}\\ 
%reference manual TABLE
\begin{center}
  \footnotesize
  \begin{longtable}{| p{4.5cm} | p{2.5cm} | p{0.5cm} | p{2.5cm} | p{6cm} |}
    \hline
    \bf Name & \bf type & \bf size & \bf default value & \bf description \\ \hline
    name &     String &      &     '' &     marker"s unique name\\ \hline
    nodeNumber &     Index &      &     MAXINT &     node number to which marker is attached to\\ \hline
    visualization & VMarkerNodeRigid & & & parameters for visualization of item \\ \hline
	  \end{longtable}
	\end{center}
The item VMarkerNodeRigid has the following parameters:\vspace{-1cm}\\ 
%reference manual TABLE
\begin{center}
  \footnotesize
  \begin{longtable}{| p{4.5cm} | p{2.5cm} | p{0.5cm} | p{2.5cm} | p{6cm} |}
    \hline
    \bf Name & \bf type & \bf size & \bf default value & \bf description \\ \hline
    show &     bool &      &     True &     set true, if item is shown in visualization and false if it is not shown\\ \hline
	  \end{longtable}
	\end{center}
{\bf Definition of quantities}:\\
\newpage

%+++++++++++++++++++++++++++++++++++
\mysubsubsection{MarkerNodeCoordinate}
A node-Marker attached to a ODE2 coordinate of a node; for other coordinates (ODE1,...) other markers need to be defined.
 \\\vspace{12pt} \noindent The item {\bf MarkerNodeCoordinate} with type = 'NodeCoordinate' has the following parameters:\vspace{-1cm}\\ 
%reference manual TABLE
\begin{center}
  \footnotesize
  \begin{longtable}{| p{4.5cm} | p{2.5cm} | p{0.5cm} | p{2.5cm} | p{6cm} |}
    \hline
    \bf Name & \bf type & \bf size & \bf default value & \bf description \\ \hline
    name &     String &      &     '' &     marker"s unique name\\ \hline
    nodeNumber &     Index &      &     MAXINT &     node number to which marker is attached to\\ \hline
    coordinate &     Index &      &     MAXINT &     coordinate of node to which marker is attached to\\ \hline
    visualization & VMarkerNodeCoordinate & & & parameters for visualization of item \\ \hline
	  \end{longtable}
	\end{center}
The item VMarkerNodeCoordinate has the following parameters:\vspace{-1cm}\\ 
%reference manual TABLE
\begin{center}
  \footnotesize
  \begin{longtable}{| p{4.5cm} | p{2.5cm} | p{0.5cm} | p{2.5cm} | p{6cm} |}
    \hline
    \bf Name & \bf type & \bf size & \bf default value & \bf description \\ \hline
    show &     bool &      &     True &     set true, if item is shown in visualization and false if it is not shown\\ \hline
	  \end{longtable}
	\end{center}
{\bf Definition of quantities}:\\
\newpage

%+++++++++++++++++++++++++++++++++++
\mysubsubsection{MarkerBodyCable2DShape}
A special Marker attached to a 2D ANCF beam finite element with cubic interpolation and 8 coordinates.
 \\\vspace{12pt} \noindent The item {\bf MarkerBodyCable2DShape} with type = 'BodyCable2DShape' has the following parameters:\vspace{-1cm}\\ 
%reference manual TABLE
\begin{center}
  \footnotesize
  \begin{longtable}{| p{4.5cm} | p{2.5cm} | p{0.5cm} | p{2.5cm} | p{6cm} |}
    \hline
    \bf Name & \bf type & \bf size & \bf default value & \bf description \\ \hline
    name &     String &      &     '' &     marker"s unique name\\ \hline
    bodyNumber &     Index &      &     MAXINT &     body number to which marker is attached to\\ \hline
    numberOfSegments &     Index &      &     3 &     number of number of segments; each segment is a line and is associated to a data (history) variable; must be same as in according contact element\\ \hline
    visualization & VMarkerBodyCable2DShape & & & parameters for visualization of item \\ \hline
	  \end{longtable}
	\end{center}
The item VMarkerBodyCable2DShape has the following parameters:\vspace{-1cm}\\ 
%reference manual TABLE
\begin{center}
  \footnotesize
  \begin{longtable}{| p{4.5cm} | p{2.5cm} | p{0.5cm} | p{2.5cm} | p{6cm} |}
    \hline
    \bf Name & \bf type & \bf size & \bf default value & \bf description \\ \hline
    show &     bool &      &     True &     set true, if item is shown in visualization and false if it is not shown\\ \hline
	  \end{longtable}
	\end{center}
{\bf Definition of quantities}:\\
\newpage

%+++++++++++++++++++++++++++++++++++
\mysubsubsection{MarkerBodyCable2DCoordinates}
A special Marker attached to the coordinates of a 2D ANCF beam finite element with cubic interpolation.
 \\\vspace{12pt} \noindent The item {\bf MarkerBodyCable2DCoordinates} with type = 'BodyCable2DCoordinates' has the following parameters:\vspace{-1cm}\\ 
%reference manual TABLE
\begin{center}
  \footnotesize
  \begin{longtable}{| p{4.5cm} | p{2.5cm} | p{0.5cm} | p{2.5cm} | p{6cm} |}
    \hline
    \bf Name & \bf type & \bf size & \bf default value & \bf description \\ \hline
    name &     String &      &     '' &     marker"s unique name\\ \hline
    bodyNumber &     Index &      &     MAXINT &     body number to which marker is attached to\\ \hline
    visualization & VMarkerBodyCable2DCoordinates & & & parameters for visualization of item \\ \hline
	  \end{longtable}
	\end{center}
The item VMarkerBodyCable2DCoordinates has the following parameters:\vspace{-1cm}\\ 
%reference manual TABLE
\begin{center}
  \footnotesize
  \begin{longtable}{| p{4.5cm} | p{2.5cm} | p{0.5cm} | p{2.5cm} | p{6cm} |}
    \hline
    \bf Name & \bf type & \bf size & \bf default value & \bf description \\ \hline
    show &     bool &      &     True &     set true, if item is shown in visualization and false if it is not shown\\ \hline
	  \end{longtable}
	\end{center}
{\bf Definition of quantities}:\\

\newpage
%+++++++++++++++++++++++++++++++
%+++++++++++++++++++++++++++++++
\mysubsection{Loads}

%+++++++++++++++++++++++++++++++++++
\mysubsubsection{LoadForceVector}
Load with (3D) force vector; attached to position-based marker.
 \\  {\bf Requested marker type} = Marker::Position \\ 
\vspace{12pt} \noindent The item {\bf LoadForceVector} with type = 'ForceVector' has the following parameters:\vspace{-1cm}\\ 
%reference manual TABLE
\begin{center}
  \footnotesize
  \begin{longtable}{| p{4.5cm} | p{2.5cm} | p{0.5cm} | p{2.5cm} | p{6cm} |}
    \hline
    \bf Name & \bf type & \bf size & \bf default value & \bf description \\ \hline
    name &     String &      &     '' &     load"s unique name\\ \hline
    markerNumber &     Index &      &     MAXINT &     marker"s number to which load is applied\\ \hline
    loadVector &     Vector3D &      &     [0.,0.,0.] &     vector-valued load [SI:N]\\ \hline
    bodyFixed &     Bool &      &     False &     if bodyFixed is true, the load is defined in body-fixed (local) coordinates, leading to a follower force; if false: global coordinates are used\\ \hline
    loadVectorUserFunction &     PyFunctionVector3DScalarVector3D &     \tabnewline  &     \tabnewline 0 &     A python function which defines the time-dependent load with parameters (Real t, Vector3D load); the load represents the current value of the load; WARNING: this factor does not work in combination with static computation (loadFactor); Example for python function: def f(t, loadVector): return [loadVector[0]*np.sin(t*10*2*3.1415),0,0]\\ \hline
    visualization & VLoadForceVector & & & parameters for visualization of item \\ \hline
	  \end{longtable}
	\end{center}
The item VLoadForceVector has the following parameters:\vspace{-1cm}\\ 
%reference manual TABLE
\begin{center}
  \footnotesize
  \begin{longtable}{| p{4.5cm} | p{2.5cm} | p{0.5cm} | p{2.5cm} | p{6cm} |}
    \hline
    \bf Name & \bf type & \bf size & \bf default value & \bf description \\ \hline
    show &     bool &      &     True &     set true, if item is shown in visualization and false if it is not shown\\ \hline
	  \end{longtable}
	\end{center}

\noindent{\bf Short name} for Python: {\bf Force}
 \vspace{6pt}\\{\bf Definition of quantities}:\\
\newpage

%+++++++++++++++++++++++++++++++++++
\mysubsubsection{LoadTorqueVector}
Load with (3D) torque vector; attached to rigidbody-based marker.
 \\  {\bf Requested marker type} = Marker::Orientation \\ 
\vspace{12pt} \noindent The item {\bf LoadTorqueVector} with type = 'TorqueVector' has the following parameters:\vspace{-1cm}\\ 
%reference manual TABLE
\begin{center}
  \footnotesize
  \begin{longtable}{| p{4.5cm} | p{2.5cm} | p{0.5cm} | p{2.5cm} | p{6cm} |}
    \hline
    \bf Name & \bf type & \bf size & \bf default value & \bf description \\ \hline
    name &     String &      &     '' &     load"s unique name\\ \hline
    markerNumber &     Index &      &     MAXINT &     marker"s number to which load is applied\\ \hline
    loadVector &     Vector3D &      &     [0.,0.,0.] &     vector-valued load [SI:N]\\ \hline
    bodyFixed &     Bool &      &     False &     if bodyFixed is true, the load is defined in body-fixed (local) coordinates, leading to a follower torque; if false: global coordinates are used\\ \hline
    loadVectorUserFunction &     PyFunctionVector3DScalarVector3D &     \tabnewline  &     \tabnewline 0 &     A python function which defines the time-dependent load with parameters (Real t, Vector3D load); the load represents the current value of the load; WARNING: this factor does not work in combination with static computation (loadFactor); Example for python function: def f(t, loadVector): return [loadVector[0]*np.sin(t*10*2*3.1415),0,0]\\ \hline
    visualization & VLoadTorqueVector & & & parameters for visualization of item \\ \hline
	  \end{longtable}
	\end{center}
The item VLoadTorqueVector has the following parameters:\vspace{-1cm}\\ 
%reference manual TABLE
\begin{center}
  \footnotesize
  \begin{longtable}{| p{4.5cm} | p{2.5cm} | p{0.5cm} | p{2.5cm} | p{6cm} |}
    \hline
    \bf Name & \bf type & \bf size & \bf default value & \bf description \\ \hline
    show &     bool &      &     True &     set true, if item is shown in visualization and false if it is not shown\\ \hline
	  \end{longtable}
	\end{center}

\noindent{\bf Short name} for Python: {\bf Torque}
 \vspace{6pt}\\{\bf Definition of quantities}:\\
\newpage

%+++++++++++++++++++++++++++++++++++
\mysubsubsection{LoadMassProportional}
Load attached to BodyMass-based marker, applying a 3D vector load (e.g. the vector [0,-g,0] is used to apply gravitational loading of size g in negative y-direction).
 \\  {\bf Requested marker type} = Marker::BodyMass \\ 
\vspace{12pt} \noindent The item {\bf LoadMassProportional} with type = 'MassProportional' has the following parameters:\vspace{-1cm}\\ 
%reference manual TABLE
\begin{center}
  \footnotesize
  \begin{longtable}{| p{4.5cm} | p{2.5cm} | p{0.5cm} | p{2.5cm} | p{6cm} |}
    \hline
    \bf Name & \bf type & \bf size & \bf default value & \bf description \\ \hline
    name &     String &      &     '' &     load"s unique name\\ \hline
    markerNumber &     Index &      &     MAXINT &     marker"s number to which load is applied\\ \hline
    loadVector &     Vector3D &      &     [0.,0.,0.] &     vector-valued load [SI:N/kg = m/s$^2$] \\ \hline
    loadVectorUserFunction &     PyFunctionVector3DScalarVector3D &     \tabnewline  &     \tabnewline 0 &     A python function which defines the time-dependent load with parameters (Real t, Vector3D load); the load represents the current value of the load; WARNING: this factor does not work in combination with static computation (loadFactor); Example for python function: def f(t, loadVector): return [loadVector[0]*np.sin(t*10*2*3.1415),0,0]\\ \hline
    visualization & VLoadMassProportional & & & parameters for visualization of item \\ \hline
	  \end{longtable}
	\end{center}
The item VLoadMassProportional has the following parameters:\vspace{-1cm}\\ 
%reference manual TABLE
\begin{center}
  \footnotesize
  \begin{longtable}{| p{4.5cm} | p{2.5cm} | p{0.5cm} | p{2.5cm} | p{6cm} |}
    \hline
    \bf Name & \bf type & \bf size & \bf default value & \bf description \\ \hline
    show &     bool &      &     True &     set true, if item is shown in visualization and false if it is not shown\\ \hline
	  \end{longtable}
	\end{center}

\noindent{\bf Short name} for Python: {\bf Gravity}
 \vspace{6pt}\\{\bf Definition of quantities}:\\
\newpage

%+++++++++++++++++++++++++++++++++++
\mysubsubsection{LoadCoordinate}
Load with scalar value, which is attached to a coordinate-based marker; the load can be used e.g. to apply a force to a single axis of a body, a nodal coordinate of a finite element  or a torque to the rotatory DOF of a rigid body.
 \\  {\bf Requested marker type} = Marker::Coordinate \\ 
\vspace{12pt} \noindent The item {\bf LoadCoordinate} with type = 'Coordinate' has the following parameters:\vspace{-1cm}\\ 
%reference manual TABLE
\begin{center}
  \footnotesize
  \begin{longtable}{| p{4.5cm} | p{2.5cm} | p{0.5cm} | p{2.5cm} | p{6cm} |}
    \hline
    \bf Name & \bf type & \bf size & \bf default value & \bf description \\ \hline
    name &     String &      &     '' &     load"s unique name\\ \hline
    markerNumber &     Index &      &     MAXINT &     marker"s number to which load is applied\\ \hline
    load &     Real &      &     0. &     scalar load [SI:N]\\ \hline
    loadUserFunction &     PyFunctionScalar2 &     \tabnewline  &     0 &     A python function which defines the time-dependent load with parameters (Real t, Real load); the load represents the current value of the load; WARNING: this factor does not work in combination with static computation (loadFactor); Example for python function: def f(t, load): return load*np.sin(t*10*2*3.1415)\\ \hline
    visualization & VLoadCoordinate & & & parameters for visualization of item \\ \hline
	  \end{longtable}
	\end{center}
The item VLoadCoordinate has the following parameters:\vspace{-1cm}\\ 
%reference manual TABLE
\begin{center}
  \footnotesize
  \begin{longtable}{| p{4.5cm} | p{2.5cm} | p{0.5cm} | p{2.5cm} | p{6cm} |}
    \hline
    \bf Name & \bf type & \bf size & \bf default value & \bf description \\ \hline
    show &     bool &      &     True &     set true, if item is shown in visualization and false if it is not shown\\ \hline
	  \end{longtable}
	\end{center}
{\bf Definition of quantities}:\\

\newpage
%+++++++++++++++++++++++++++++++
%+++++++++++++++++++++++++++++++
\mysubsection{Sensors}

%+++++++++++++++++++++++++++++++++++
\mysubsubsection{SensorNode}
A sensor attached to a node. The sensor measures OutputVariables and outputs values into a file, showing time, sensorValue[0], sensorValue[1], ... . A user function can be attached to modify sensor values accordingly.
 \\\vspace{12pt} \noindent The item {\bf SensorNode} with type = 'Node' has the following parameters:\vspace{-1cm}\\ 
%reference manual TABLE
\begin{center}
  \footnotesize
  \begin{longtable}{| p{4.5cm} | p{2.5cm} | p{0.5cm} | p{2.5cm} | p{6cm} |}
    \hline
    \bf Name & \bf type & \bf size & \bf default value & \bf description \\ \hline
    name &     String &      &     '' &     marker"s unique name\\ \hline
    nodeNumber &     Index &      &     MAXINT &     node number to which sensor is attached to\\ \hline
    writeToFile &     bool &      &     True &     true: write sensor output to file\\ \hline
    fileName &     String &      &     '' &     directory and file name for sensor file output; default: empty string generates sensor + sensorNumber + outputVariableType\\ \hline
    outputVariableType &     OutputVariableType &     \tabnewline  &     OutputVariableType::\_None &     OutputVariableType for sensor\\ \hline
    visualization & VSensorNode & & & parameters for visualization of item \\ \hline
	  \end{longtable}
	\end{center}
The item VSensorNode has the following parameters:\vspace{-1cm}\\ 
%reference manual TABLE
\begin{center}
  \footnotesize
  \begin{longtable}{| p{4.5cm} | p{2.5cm} | p{0.5cm} | p{2.5cm} | p{6cm} |}
    \hline
    \bf Name & \bf type & \bf size & \bf default value & \bf description \\ \hline
    show &     bool &      &     True &     set true, if item is shown in visualization and false if it is not shown\\ \hline
	  \end{longtable}
	\end{center}
{\bf Definition of quantities}:\\
\newpage

%+++++++++++++++++++++++++++++++++++
\mysubsubsection{SensorBody}
A sensor attached to a body-object with local position. As a difference to other ObjectSensors, the body sensor has a local position at which the sensor is attached to. The sensor measures OutputVariableBody and outputs values into a file, showing time, sensorValue[0], sensorValue[1], ... . A user function can be attached to postprocess sensor values accordingly.
 \\\vspace{12pt} \noindent The item {\bf SensorBody} with type = 'Body' has the following parameters:\vspace{-1cm}\\ 
%reference manual TABLE
\begin{center}
  \footnotesize
  \begin{longtable}{| p{4.5cm} | p{2.5cm} | p{0.5cm} | p{2.5cm} | p{6cm} |}
    \hline
    \bf Name & \bf type & \bf size & \bf default value & \bf description \\ \hline
    name &     String &      &     '' &     marker"s unique name\\ \hline
    bodyNumber &     Index &      &     MAXINT &     body (=object) number to which sensor is attached to\\ \hline
    localPosition &     Vector3D &     3 &     [0.,0.,0.] &     local (body-fixed) body position of sensor\\ \hline
    writeToFile &     bool &      &     True &     true: write sensor output to file\\ \hline
    fileName &     String &      &     '' &     directory and file name for sensor file output; default: empty string generates sensor + sensorNumber + outputVariableType\\ \hline
    outputVariableType &     OutputVariableType &     \tabnewline  &     OutputVariableType::\_None &     OutputVariableType for sensor\\ \hline
    visualization & VSensorBody & & & parameters for visualization of item \\ \hline
	  \end{longtable}
	\end{center}
The item VSensorBody has the following parameters:\vspace{-1cm}\\ 
%reference manual TABLE
\begin{center}
  \footnotesize
  \begin{longtable}{| p{4.5cm} | p{2.5cm} | p{0.5cm} | p{2.5cm} | p{6cm} |}
    \hline
    \bf Name & \bf type & \bf size & \bf default value & \bf description \\ \hline
    show &     bool &      &     True &     set true, if item is shown in visualization and false if it is not shown\\ \hline
	  \end{longtable}
	\end{center}
{\bf Definition of quantities}:\\
\newpage

%+++++++++++++++++++++++++++++++++++
\mysubsubsection{SensorObject}
A sensor attached to any object except bodies  (connectors, constraint, spring-damper, etc). As a difference to other SensorBody, the connector sensor measures quantities without a local position. The sensor measures OutputVariable and outputs values into a file, showing time, sensorValue[0], sensorValue[1], ... . A user function can be attached to postprocess sensor values accordingly.
 \\\vspace{12pt} \noindent The item {\bf SensorObject} with type = 'Object' has the following parameters:\vspace{-1cm}\\ 
%reference manual TABLE
\begin{center}
  \footnotesize
  \begin{longtable}{| p{4.5cm} | p{2.5cm} | p{0.5cm} | p{2.5cm} | p{6cm} |}
    \hline
    \bf Name & \bf type & \bf size & \bf default value & \bf description \\ \hline
    name &     String &      &     '' &     marker"s unique name\\ \hline
    objectNumber &     Index &      &     MAXINT &     object (e.g. connector) number to which sensor is attached to\\ \hline
    writeToFile &     bool &      &     True &     true: write sensor output to file\\ \hline
    fileName &     String &      &     '' &     directory and file name for sensor file output; default: empty string generates sensor + sensorNumber + outputVariableType\\ \hline
    outputVariableType &     OutputVariableType &     \tabnewline  &     OutputVariableType::\_None &     OutputVariableType for sensor\\ \hline
    visualization & VSensorObject & & & parameters for visualization of item \\ \hline
	  \end{longtable}
	\end{center}
The item VSensorObject has the following parameters:\vspace{-1cm}\\ 
%reference manual TABLE
\begin{center}
  \footnotesize
  \begin{longtable}{| p{4.5cm} | p{2.5cm} | p{0.5cm} | p{2.5cm} | p{6cm} |}
    \hline
    \bf Name & \bf type & \bf size & \bf default value & \bf description \\ \hline
    show &     bool &      &     True &     set true, if item is shown in visualization and false if it is not shown; sensors can be shown at the position assiciated with the object - note that in some cases, there might be no such position (e.g. data object)!\\ \hline
	  \end{longtable}
	\end{center}
{\bf Definition of quantities}:\\
\newpage

%+++++++++++++++++++++++++++++++++++
\mysubsubsection{SensorLoad}
A sensor attached to a load. The sensor measures the load values and outputs values into a file, showing time, sensorValue[0], sensorValue[1], ... .
 \\\vspace{12pt} \noindent The item {\bf SensorLoad} with type = 'Load' has the following parameters:\vspace{-1cm}\\ 
%reference manual TABLE
\begin{center}
  \footnotesize
  \begin{longtable}{| p{4.5cm} | p{2.5cm} | p{0.5cm} | p{2.5cm} | p{6cm} |}
    \hline
    \bf Name & \bf type & \bf size & \bf default value & \bf description \\ \hline
    name &     String &      &     '' &     marker"s unique name\\ \hline
    loadNumber &     Index &      &     MAXINT &     load number to which sensor is attached to\\ \hline
    writeToFile &     bool &      &     True &     true: write sensor output to file\\ \hline
    fileName &     String &      &     '' &     directory and file name for sensor file output; default: empty string generates sensor + sensorNumber + outputVariableType\\ \hline
    visualization & VSensorLoad & & & parameters for visualization of item \\ \hline
	  \end{longtable}
	\end{center}
The item VSensorLoad has the following parameters:\vspace{-1cm}\\ 
%reference manual TABLE
\begin{center}
  \footnotesize
  \begin{longtable}{| p{4.5cm} | p{2.5cm} | p{0.5cm} | p{2.5cm} | p{6cm} |}
    \hline
    \bf Name & \bf type & \bf size & \bf default value & \bf description \\ \hline
    show &     bool &      &     True &     set true, if item is shown in visualization and false if it is not shown; CURRENTLY NOT AVAILABLE\\ \hline
	  \end{longtable}
	\end{center}
{\bf Definition of quantities}:\\
