
\mysubsection{List of Items}
The following items are available in \codeName:
\begin{itemize}
  \item NodePoint
  \item NodePoint2D
  \item NodeRigidBodyEP
  \item NodeRigidBodyRxyz
  \item NodeRigidBodyRotVecLG
  \item NodeRigidBody2D
  \item NodePoint2DSlope1
  \item NodeGenericODE2
  \item NodeGenericData
  \item NodePointGround
  \item ObjectMassPoint
  \item ObjectMassPoint2D
  \item ObjectRigidBody
  \item ObjectRigidBody2D
  \item ObjectANCFCable2D
  \item ObjectALEANCFCable2D
  \item ObjectGround
  \item ObjectConnectorSpringDamper
  \item ObjectConnectorCartesianSpringDamper
  \item ObjectConnectorRigidBodySpringDamper
  \item ObjectConnectorCoordinateSpringDamper
  \item ObjectConnectorDistance
  \item ObjectConnectorCoordinate
  \item ObjectContactCoordinate
  \item ObjectContactCircleCable2D
  \item ObjectContactFrictionCircleCable2D
  \item ObjectJointSliding2D
  \item ObjectJointALEMoving2D
  \item ObjectJointGeneric
  \item ObjectJointRevolute2D
  \item ObjectJointPrismatic2D
  \item MarkerBodyMass
  \item MarkerBodyPosition
  \item MarkerBodyRigid
  \item MarkerNodePosition
  \item MarkerNodeRigid
  \item MarkerNodeCoordinate
  \item MarkerBodyCable2DShape
  \item MarkerBodyCable2DCoordinates
  \item LoadForceVector
  \item LoadTorqueVector
  \item LoadMassProportional
  \item LoadCoordinate
  \item SensorNode
  \item SensorBody
\end{itemize}

%+++++++++++++++++++++++++++++++
%+++++++++++++++++++++++++++++++
\mysubsection{Nodes}

%+++++++++++++++++++++++++++++++++++
\mysubsubsection{NodePoint}
A 3D point node for point masses or solid finite elements which has 3 displacement degrees of freedom for second order differential equations.
 \\
{\bf Short name} for Python: {\bf Point}
 \\\\ 
{\bf Output variables} (chose type, e.g., OutputVariableType.Position): 
\begin{itemize}
    \item {\bf Position}: global 3D position vector of node (=displacement+reference position)
    \item {\bf Displacement}: global 3D displacement vector of node
    \item {\bf Velocity}: global 3D velocity vector of node
    \item {\bf Coordinates}: coordinates vector of node
    \item {\bf Coordinates\_t}: velocity coordinates vector of node
\end{itemize}
The item NodePoint has the following parameters:
%reference manual TABLE
\begin{center}
  \footnotesize
  \begin{longtable}{| p{4.5cm} | p{2.5cm} | p{0.5cm} | p{2.5cm} | p{6cm} |}
    \hline
    \bf Name & \bf type & \bf size & \bf default value & \bf description \\ \hline
    \multicolumn{4}{l}{\parbox{10cm}{type = 'Point'}} & \multicolumn{1}{l}{\parbox{6cm}{\it item typename for initialization}}\\ \hline
    name &     String &      &     '' &     node"s unique name\\ \hline
    referenceCoordinates &     Vector3D &     3 &     [0.,0.,0.] &     reference coordinates of node ==> e.g. ref. coordinates for finite elements; global position of node without displacement\\ \hline
    initialDisplacements &     Vector3D &     3 &     [0.,0.,0.] &     initial displacement coordinate\\ \hline
    initialVelocities &     Vector3D &     3 &     [0.,0.,0.] &     initial velocity coordinate\\ \hline
    visualization & VNodePoint & & & parameters for visualization of item \\ \hline
	  \end{longtable}
	\end{center}
The item VNodePoint has the following parameters:
%reference manual TABLE
\begin{center}
  \footnotesize
  \begin{longtable}{| p{4.5cm} | p{2.5cm} | p{0.5cm} | p{2.5cm} | p{6cm} |}
    \hline
    \bf Name & \bf type & \bf size & \bf default value & \bf description \\ \hline
    show &     bool &      &     True &     set true, if item is shown in visualization and false if it is not shown\\ \hline
    drawSize &     float &      &     -1. &     drawing size (diameter, dimensions of underlying cube, etc.)  for item; size == -1.f means that default size is used\\ \hline
    color &     Float4 &     4 &     [-1.,-1.,-1.,-1.] &     Default RGBA color for nodes; 4th value is alpha-transparency; R=-1.f means, that default color is used\\ \hline
	  \end{longtable}
	\end{center}

%+++++++++++++++++++++++++++++++++++
\mysubsubsection{NodePoint2D}
A 2D point node for point masses or solid finite elements which has 2 displacement degrees of freedom for second order differential equations.
 \\
{\bf Short name} for Python: {\bf Point2D}
 \\\\ 
{\bf Output variables} (chose type, e.g., OutputVariableType.Position): 
\begin{itemize}
    \item {\bf Position}: global 3D position vector of node (=displacement+reference position)
    \item {\bf Displacement}: global 3D displacement vector of node
    \item {\bf Velocity}: global 3D velocity vector of node
    \item {\bf Coordinates}: coordinates vector of node
    \item {\bf Coordinates\_t}: velocity coordinates vector of node
\end{itemize}
The item NodePoint2D has the following parameters:
%reference manual TABLE
\begin{center}
  \footnotesize
  \begin{longtable}{| p{4.5cm} | p{2.5cm} | p{0.5cm} | p{2.5cm} | p{6cm} |}
    \hline
    \bf Name & \bf type & \bf size & \bf default value & \bf description \\ \hline
    \multicolumn{4}{l}{\parbox{10cm}{type = 'Point2D'}} & \multicolumn{1}{l}{\parbox{6cm}{\it item typename for initialization}}\\ \hline
    name &     String &      &     '' &     node"s unique name\\ \hline
    referenceCoordinates &     Vector2D &     2 &     [0.,0.] &     reference coordinates of node ==> e.g. ref. coordinates for finite elements; global position of node without displacement\\ \hline
    initialDisplacements &     Vector2D &     2 &     [0.,0.] &     initial displacement coordinate\\ \hline
    initialVelocities &     Vector2D &     2 &     [0.,0.] &     initial velocity coordinate\\ \hline
    visualization & VNodePoint2D & & & parameters for visualization of item \\ \hline
	  \end{longtable}
	\end{center}
The item VNodePoint2D has the following parameters:
%reference manual TABLE
\begin{center}
  \footnotesize
  \begin{longtable}{| p{4.5cm} | p{2.5cm} | p{0.5cm} | p{2.5cm} | p{6cm} |}
    \hline
    \bf Name & \bf type & \bf size & \bf default value & \bf description \\ \hline
    show &     bool &      &     True &     set true, if item is shown in visualization and false if it is not shown\\ \hline
    drawSize &     float &      &     -1. &     drawing size (diameter, dimensions of underlying cube, etc.)  for item; size == -1.f means that default size is used\\ \hline
    color &     Float4 &     4 &     [-1.,-1.,-1.,-1.] &     Default RGBA color for nodes; 4th value is alpha-transparency; R=-1.f means, that default color is used\\ \hline
	  \end{longtable}
	\end{center}

%+++++++++++++++++++++++++++++++++++
\mysubsubsection{NodeRigidBodyEP}
A 3D rigid body node based on Euler parameters for rigid bodies or beams; the node has 3 displacement coordinates (displacements of center of mass - COM: ux,uy,uz) and four rotation coordinates (Euler parameters = quaternions); all coordinates lead to second order differential equations; additionally there is one constraint equation for quaternions; The rotation matrix $\Am$, transforming local (body-fixed) 3D positions $\pv_{loc} = [p^x_{loc}\;\;p^y_{loc}\;\;p^z_{loc}]^T$ to global 3D positions $\pv_{glob} = [p^x_{glob}\;\;p^y_{glob}\;\;p^z_{glob}]^T$, \be \pv_{glob} = \Am \pv_{loc}, \ee is defined according to the book of Shabana, same with the transformation matrix $\mathbf{G}$ between time derivatives of Euler parameters and angular velocities.
 \\
{\bf Short name} for Python: {\bf RigidEP}
 \\\\ 
{\bf Output variables} (chose type, e.g., OutputVariableType.Position): 
\begin{itemize}
    \item {\bf Position}: global 3D position vector of node (=displacement+reference position)
    \item {\bf Displacement}: global 3D displacement vector of node
    \item {\bf RotationMatrix}: vector with 9 components of the rotation matrix (row-major format)
    \item {\bf Rotation}: vector with 3 components of the Euler angles in xyz-sequence (R=Rx*Ry*Rz), recomputed from rotation matrix
    \item {\bf Velocity}: global 3D velocity vector of node
    \item {\bf AngularVelocity}: global 3D velocity vector of node
    \item {\bf AngularVelocityLocal}: local (body-fixed) 3D velocity vector of node
    \item {\bf Coordinates}: coordinates vector of node
    \item {\bf Coordinates\_t}: velocity coordinates vector of node
\end{itemize}
The item NodeRigidBodyEP has the following parameters:
%reference manual TABLE
\begin{center}
  \footnotesize
  \begin{longtable}{| p{4.5cm} | p{2.5cm} | p{0.5cm} | p{2.5cm} | p{6cm} |}
    \hline
    \bf Name & \bf type & \bf size & \bf default value & \bf description \\ \hline
    \multicolumn{4}{l}{\parbox{10cm}{type = 'RigidBodyEP'}} & \multicolumn{1}{l}{\parbox{6cm}{\it item typename for initialization}}\\ \hline
    name &     String &      &     '' &     node"s unique name\\ \hline
    referenceCoordinates &     Vector7D &     7 &     [0.,0.,0., 0.,0.,0.,0.] &     reference coordinates (x-pos,y-pos,z-pos and 4 Euler parameters) of node ==> e.g. ref. coordinates for finite elements or reference position of rigid body (e.g. for definition of joints)\\ \hline
    initialDisplacements &     Vector7D &     7 &     [0.,0.,0., 0.,0.,0.,0.] &     initial displacement coordinates: ux,uy,uz and 4 Euler parameters relative to reference coordinates\\ \hline
    initialVelocities &     Vector7D &     7 &     [0.,0.,0., 0.,0.,0.,0.] &     initial velocity coordinate: time derivatives of ux,uy,uz and 4 Euler parameters\\ \hline
    visualization & VNodeRigidBodyEP & & & parameters for visualization of item \\ \hline
	  \end{longtable}
	\end{center}
The item VNodeRigidBodyEP has the following parameters:
%reference manual TABLE
\begin{center}
  \footnotesize
  \begin{longtable}{| p{4.5cm} | p{2.5cm} | p{0.5cm} | p{2.5cm} | p{6cm} |}
    \hline
    \bf Name & \bf type & \bf size & \bf default value & \bf description \\ \hline
    show &     bool &      &     True &     set true, if item is shown in visualization and false if it is not shown\\ \hline
    drawSize &     float &      &     -1. &     drawing size (diameter, dimensions of underlying cube, etc.)  for item; size == -1.f means that default size is used\\ \hline
    color &     Float4 &     4 &     [-1.,-1.,-1.,-1.] &     Default RGBA color for nodes; 4th value is alpha-transparency; R=-1.f means, that default color is used\\ \hline
	  \end{longtable}
	\end{center}

%+++++++++++++++++++++++++++++++++++
\mysubsubsection{NodeRigidBodyRxyz}
A 3D rigid body node based on Euler / Tait-Bryan angles for rigid bodies or beams; the node has 3 displacement coordinates (displacements of center of mass - COM: $[u_x,u_y,u_z]$) and three rotation coordinates (angles $[\varphi_x,\varphi_y,\varphi_z]$ for rotations around x,y, and z-axis); all coordinates lead to second order differential equations; The rotation matrix $\Am=\Rm_x \Rm_y \Rm_z$ transforms local (body-fixed) 3D positions $\pv_{loc} = [p^x_{loc}\;\;p^y_{loc}\;\;p^z_{loc}]^T$ to global 3D positions $\pv_{glob} = [p^x_{glob}\;\;p^y_{glob}\;\;p^z_{glob}]^T$, \be \pv_{glob} = \Am \pv_{loc} \ee; the transformation matrix $\mathbf{G}$ between time derivatives of Euler angles and angular velocities is defined as according to the book of Nikravesh. NOTE that this node has a singularity if the second rotation parameter reaches $k \times \pi/2$, with $k \in \{...,-3,-1,1,3,... \}$.
 \\
{\bf Short name} for Python: {\bf RigidRxyz}
 \\\\ 
{\bf Output variables} (chose type, e.g., OutputVariableType.Position): 
\begin{itemize}
    \item {\bf Position}: global 3D position vector of node (=displacement+reference position)
    \item {\bf Displacement}: global 3D displacement vector of node
    \item {\bf RotationMatrix}: vector with 9 components of the rotation matrix (row-major format)
    \item {\bf Rotation}: vector with 3 components of the Euler angles in xyz-sequence (R=Rx*Ry*Rz)
    \item {\bf Velocity}: global 3D velocity vector of node
    \item {\bf AngularVelocity}: global 3D velocity vector of node
    \item {\bf AngularVelocityLocal}: local (body-fixed) 3D velocity vector of node
    \item {\bf Coordinates}: coordinates vector of node
    \item {\bf Coordinates\_t}: velocity coordinates vector of node
\end{itemize}
The item NodeRigidBodyRxyz has the following parameters:
%reference manual TABLE
\begin{center}
  \footnotesize
  \begin{longtable}{| p{4.5cm} | p{2.5cm} | p{0.5cm} | p{2.5cm} | p{6cm} |}
    \hline
    \bf Name & \bf type & \bf size & \bf default value & \bf description \\ \hline
    \multicolumn{4}{l}{\parbox{10cm}{type = 'RigidBodyRxyz'}} & \multicolumn{1}{l}{\parbox{6cm}{\it item typename for initialization}}\\ \hline
    name &     String &      &     '' &     node"s unique name\\ \hline
    referenceCoordinates &     Vector6D &     3 &     [0.,0.,0., 0.,0.,0.] &     reference coordinates (x-pos,y-pos,z-pos and 3 xyz Euler angles) of node ==> e.g. ref. coordinates for finite elements or reference position of rigid body (e.g. for definition of joints)\\ \hline
    initialDisplacements &     Vector6D &     3 &     [0.,0.,0., 0.,0.,0.] &     initial displacement coordinates: ux,uy,uz and 3 Euler angles (xyz) relative to reference coordinates\\ \hline
    initialVelocities &     Vector6D &     3 &     [0.,0.,0., 0.,0.,0.] &     initial velocity coordinate: time derivatives of ux,uy,uz and of 3 Euler angles (xyz)\\ \hline
    visualization & VNodeRigidBodyRxyz & & & parameters for visualization of item \\ \hline
	  \end{longtable}
	\end{center}
The item VNodeRigidBodyRxyz has the following parameters:
%reference manual TABLE
\begin{center}
  \footnotesize
  \begin{longtable}{| p{4.5cm} | p{2.5cm} | p{0.5cm} | p{2.5cm} | p{6cm} |}
    \hline
    \bf Name & \bf type & \bf size & \bf default value & \bf description \\ \hline
    show &     bool &      &     True &     set true, if item is shown in visualization and false if it is not shown\\ \hline
    drawSize &     float &      &     -1. &     drawing size (diameter, dimensions of underlying cube, etc.)  for item; size == -1.f means that default size is used\\ \hline
    color &     Float4 &     4 &     [-1.,-1.,-1.,-1.] &     Default RGBA color for nodes; 4th value is alpha-transparency; R=-1.f means, that default color is used\\ \hline
	  \end{longtable}
	\end{center}

%+++++++++++++++++++++++++++++++++++
\mysubsubsection{NodeRigidBodyRotVecLG}
A 3D rigid body node based on rotation vector and Lie group methods for rigid bodies or beams; the node has 3 displacement coordinates (displacements of center of mass - COM: $[u_x,u_y,u_z]$) and three rotation coordinates (rotation vector $\mathbf{v} = [v_x,v_y,v_z]^T = \varphi \cdot \mathbf{n}$, defining the rotation axis $\mathbf{n}$ and the angle $\varphi$ for rotations around x,y, and z-axis); the velocity coordinates are based on the translational (global) velocity and the (local/body-fixed) angular velocity vector; this node can only be integrated using special Lie group integrators; NOTE that this node has a singularity if the rotation is zero or multiple of $2\pi$.
 \\
{\bf Short name} for Python: {\bf RigidRotVecLG}
 \\\\ 
{\bf Output variables} (chose type, e.g., OutputVariableType.Position): 
\begin{itemize}
    \item {\bf Position}: global 3D position vector of node (=displacement+reference position)
    \item {\bf Displacement}: global 3D displacement vector of node
    \item {\bf RotationMatrix}: vector with 9 components of the rotation matrix (row-major format)
    \item {\bf Rotation}: vector with 3 components of the Euler angles in xyz-sequence (R=Rx*Ry*Rz)
    \item {\bf Velocity}: global 3D velocity vector of node
    \item {\bf AngularVelocity}: global 3D velocity vector of node
    \item {\bf AngularVelocityLocal}: local (body-fixed) 3D velocity vector of node
    \item {\bf Coordinates}: coordinates vector of node
    \item {\bf Coordinates\_t}: velocity coordinates vector of node
\end{itemize}
The item NodeRigidBodyRotVecLG has the following parameters:
%reference manual TABLE
\begin{center}
  \footnotesize
  \begin{longtable}{| p{4.5cm} | p{2.5cm} | p{0.5cm} | p{2.5cm} | p{6cm} |}
    \hline
    \bf Name & \bf type & \bf size & \bf default value & \bf description \\ \hline
    \multicolumn{4}{l}{\parbox{10cm}{type = 'RigidBodyRotVecLG'}} & \multicolumn{1}{l}{\parbox{6cm}{\it item typename for initialization}}\\ \hline
    name &     String &      &     '' &     node"s unique name\\ \hline
    referenceCoordinates &     Vector6D &     3 &     [0.,0.,0., 0.,0.,0.] &     reference coordinates (x-pos,y-pos,z-pos and rotation vector) of node ==> e.g. ref. coordinates for finite elements or reference position of rigid body (e.g. for definition of joints)\\ \hline
    initialDisplacements &     Vector6D &     3 &     [0.,0.,0., 0.,0.,0.] &     initial displacement coordinates: ux,uy,uz and rotation vector relative to reference coordinates\\ \hline
    initialVelocities &     Vector6D &     3 &     [0.,0.,0., 0.,0.,0.] &     initial velocity coordinate: time derivatives of ux,uy,uz and angular velocity vector\\ \hline
    visualization & VNodeRigidBodyRotVecLG & & & parameters for visualization of item \\ \hline
	  \end{longtable}
	\end{center}
The item VNodeRigidBodyRotVecLG has the following parameters:
%reference manual TABLE
\begin{center}
  \footnotesize
  \begin{longtable}{| p{4.5cm} | p{2.5cm} | p{0.5cm} | p{2.5cm} | p{6cm} |}
    \hline
    \bf Name & \bf type & \bf size & \bf default value & \bf description \\ \hline
    show &     bool &      &     True &     set true, if item is shown in visualization and false if it is not shown\\ \hline
    drawSize &     float &      &     -1. &     drawing size (diameter, dimensions of underlying cube, etc.)  for item; size == -1.f means that default size is used\\ \hline
    color &     Float4 &     4 &     [-1.,-1.,-1.,-1.] &     Default RGBA color for nodes; 4th value is alpha-transparency; R=-1.f means, that default color is used\\ \hline
	  \end{longtable}
	\end{center}

%+++++++++++++++++++++++++++++++++++
\mysubsubsection{NodeRigidBody2D}
A 2D rigid body node for rigid bodies or beams; the node has 2 displacement degrees of freedom (displacement of center of mass - COM: ux,uy) and one rotation coordinate (rotation around z-axis: uphi); all coordinates lead to second order differential equations; The rotation matrix $\Am$, transforming local (body-fixed) 3D positions $\pv_{loc} = [p^x_{loc}\;\;p^y_{loc}\;\;0]^T$ to global 3D positions $\pv_{glob} = [p^x_{glob}\;\;p^y_{glob}\;\;p^z_{glob}]^T$, \be \pv_{glob} = \Am \pv_{loc}, \ee is defined as \be \Am = \mp{\cos(\varphi)}{-\sin(\varphi)}{\sin(\varphi)}{\cos(\varphi)}.\ee
 \\
{\bf Short name} for Python: {\bf Rigid2D}
 \\\\ 
{\bf Output variables} (chose type, e.g., OutputVariableType.Position): 
\begin{itemize}
    \item {\bf Position}: global 3D position vector of node (=displacement+reference position)
    \item {\bf Displacement}: global 3D displacement vector of node
    \item {\bf Velocity}: global 3D velocity vector of node
    \item {\bf Coordinates}: coordinates vector of node
    \item {\bf Coordinates\_t}: velocity coordinates vector of node
\end{itemize}
The item NodeRigidBody2D has the following parameters:
%reference manual TABLE
\begin{center}
  \footnotesize
  \begin{longtable}{| p{4.5cm} | p{2.5cm} | p{0.5cm} | p{2.5cm} | p{6cm} |}
    \hline
    \bf Name & \bf type & \bf size & \bf default value & \bf description \\ \hline
    \multicolumn{4}{l}{\parbox{10cm}{type = 'RigidBody2D'}} & \multicolumn{1}{l}{\parbox{6cm}{\it item typename for initialization}}\\ \hline
    name &     String &      &     '' &     node"s unique name\\ \hline
    referenceCoordinates &     Vector3D &     3 &     [0.,0.,0.] &     reference coordinates (x-pos,y-pos and rotation phi) of node ==> e.g. ref. coordinates for finite elements; global position of node without displacement\\ \hline
    initialDisplacements &     Vector3D &     3 &     [0.,0.,0.] &     initial displacement coordinates: ux, uy and uphi\\ \hline
    initialVelocities &     Vector3D &     3 &     [0.,0.,0.] &     initial velocity coordinate: vx, vy, omega\\ \hline
    visualization & VNodeRigidBody2D & & & parameters for visualization of item \\ \hline
	  \end{longtable}
	\end{center}
The item VNodeRigidBody2D has the following parameters:
%reference manual TABLE
\begin{center}
  \footnotesize
  \begin{longtable}{| p{4.5cm} | p{2.5cm} | p{0.5cm} | p{2.5cm} | p{6cm} |}
    \hline
    \bf Name & \bf type & \bf size & \bf default value & \bf description \\ \hline
    show &     bool &      &     True &     set true, if item is shown in visualization and false if it is not shown\\ \hline
    drawSize &     float &      &     -1. &     drawing size (diameter, dimensions of underlying cube, etc.)  for item; size == -1.f means that default size is used\\ \hline
    color &     Float4 &     4 &     [-1.,-1.,-1.,-1.] &     Default RGBA color for nodes; 4th value is alpha-transparency; R=-1.f means, that default color is used\\ \hline
	  \end{longtable}
	\end{center}

%+++++++++++++++++++++++++++++++++++
\mysubsubsection{NodePoint2DSlope1}
A 2D point/slope vector node for planar Bernoulli-Euler ANCF (absolute nodal coordinate formulation) beam elements; the node has 4 displacement degrees of freedom (2 for displacement of point node and 2 for the slope vector 'slopex'); all coordinates lead to second order differential equations; the slope vector defines the directional derivative w.r.t the local axial (x) coordinate, denoted as $()^\prime$; in straight configuration aligned at the global x-axis, the slope vector reads $\rv^\prime=[r_x^\prime\;\;r_y^\prime]^T=[1\;\;0]^T$.
 \\
{\bf Short name} for Python: {\bf Point2DS1}
 \\\\ 
{\bf Output variables} (chose type, e.g., OutputVariableType.Position): 
\begin{itemize}
    \item {\bf Position}: global 3D position vector of node (=displacement+reference position)
    \item {\bf Displacement}: global 3D displacement vector of node
    \item {\bf Velocity}: global 3D velocity vector of node
    \item {\bf Coordinates}: coordinates vector of node (2 displacement coordinates + 2 slope vector coordinates)
    \item {\bf Coordinates\_t}: velocity coordinates vector of node (derivative of the 2 displacement coordinates + 2 slope vector coordinates)
\end{itemize}
The item NodePoint2DSlope1 has the following parameters:
%reference manual TABLE
\begin{center}
  \footnotesize
  \begin{longtable}{| p{4.5cm} | p{2.5cm} | p{0.5cm} | p{2.5cm} | p{6cm} |}
    \hline
    \bf Name & \bf type & \bf size & \bf default value & \bf description \\ \hline
    \multicolumn{4}{l}{\parbox{10cm}{type = 'Point2DSlope1'}} & \multicolumn{1}{l}{\parbox{6cm}{\it item typename for initialization}}\\ \hline
    name &     String &      &     '' &     node"s unique name\\ \hline
    referenceCoordinates &     Vector4D &     4 &     [0.,0.,1.,0.] &     reference coordinates (x-pos,y-pos; x-slopex, y-slopex) of node; global position of node without displacement\\ \hline
    initialDisplacements &     Vector4D &     4 &     [0.,0.,0.,0.] &     initial displacement coordinates: ux, uy and x/y "displacements" of slopex\\ \hline
    initialVelocities &     Vector4D &     4 &     [0.,0.,0.,0.] &     initial velocity coordinates\\ \hline
    visualization & VNodePoint2DSlope1 & & & parameters for visualization of item \\ \hline
	  \end{longtable}
	\end{center}
The item VNodePoint2DSlope1 has the following parameters:
%reference manual TABLE
\begin{center}
  \footnotesize
  \begin{longtable}{| p{4.5cm} | p{2.5cm} | p{0.5cm} | p{2.5cm} | p{6cm} |}
    \hline
    \bf Name & \bf type & \bf size & \bf default value & \bf description \\ \hline
    show &     bool &      &     True &     set true, if item is shown in visualization and false if it is not shown\\ \hline
    drawSize &     float &      &     -1. &     drawing size (diameter, dimensions of underlying cube, etc.)  for item; size == -1.f means that default size is used\\ \hline
    color &     Float4 &     4 &     [-1.,-1.,-1.,-1.] &     Default RGBA color for nodes; 4th value is alpha-transparency; R=-1.f means, that default color is used\\ \hline
	  \end{longtable}
	\end{center}

%+++++++++++++++++++++++++++++++++++
\mysubsubsection{NodeGenericODE2}
A node containing a number of ODE2 variables; use e.g. for scalar dynamic equations (Mass1D) or for the ALECable element.
 \\\\ 
{\bf Output variables} (chose type, e.g., OutputVariableType.Position): 
\begin{itemize}
    \item {\bf Coordinates}: coordinates vector of node
    \item {\bf Coordinates\_t}: velocity coordinates vector of node
\end{itemize}
The item NodeGenericODE2 has the following parameters:
%reference manual TABLE
\begin{center}
  \footnotesize
  \begin{longtable}{| p{4.5cm} | p{2.5cm} | p{0.5cm} | p{2.5cm} | p{6cm} |}
    \hline
    \bf Name & \bf type & \bf size & \bf default value & \bf description \\ \hline
    \multicolumn{4}{l}{\parbox{10cm}{type = 'GenericODE2'}} & \multicolumn{1}{l}{\parbox{6cm}{\it item typename for initialization}}\\ \hline
    name &     String &      &     '' &     node"s unique name\\ \hline
    referenceCoordinates &     Vector &      &     [] &     generic reference coordinates of node; must be consistent with numberOfODE2Coordinates\\ \hline
    initialCoordinates &     Vector &      &     [] &     initial displacement coordinates; must be consistent with numberOfODE2Coordinates\\ \hline
    initialCoordinates\_t &     Vector &      &     [] &     initial velocity coordinates; must be consistent with numberOfODE2Coordinates\\ \hline
    numberOfODE2Coordinates &     Index &      &     0 &     number of generic ODE2 coordinates\\ \hline
    visualization & VNodeGenericODE2 & & & parameters for visualization of item \\ \hline
	  \end{longtable}
	\end{center}
The item VNodeGenericODE2 has the following parameters:
%reference manual TABLE
\begin{center}
  \footnotesize
  \begin{longtable}{| p{4.5cm} | p{2.5cm} | p{0.5cm} | p{2.5cm} | p{6cm} |}
    \hline
    \bf Name & \bf type & \bf size & \bf default value & \bf description \\ \hline
    show &     bool &      &     False &     set true, if item is shown in visualization and false if it is not shown\\ \hline
	  \end{longtable}
	\end{center}

%+++++++++++++++++++++++++++++++++++
\mysubsubsection{NodeGenericData}
A node containing a number of data (history) variables; use e.g. for contact (active set), friction or plasticity (history variable).
 \\\\ 
{\bf Output variables} (chose type, e.g., OutputVariableType.Position): 
\begin{itemize}
    \item {\bf Coordinates}: data coordinates (history variables) vector of node
\end{itemize}
The item NodeGenericData has the following parameters:
%reference manual TABLE
\begin{center}
  \footnotesize
  \begin{longtable}{| p{4.5cm} | p{2.5cm} | p{0.5cm} | p{2.5cm} | p{6cm} |}
    \hline
    \bf Name & \bf type & \bf size & \bf default value & \bf description \\ \hline
    \multicolumn{4}{l}{\parbox{10cm}{type = 'GenericData'}} & \multicolumn{1}{l}{\parbox{6cm}{\it item typename for initialization}}\\ \hline
    name &     String &      &     '' &     node"s unique name\\ \hline
    initialCoordinates &     Vector &      &     [] &     initial data coordinates\\ \hline
    numberOfDataCoordinates &     Index &      &     0 &     number of generic data coordinates (history variables)\\ \hline
    visualization & VNodeGenericData & & & parameters for visualization of item \\ \hline
	  \end{longtable}
	\end{center}
The item VNodeGenericData has the following parameters:
%reference manual TABLE
\begin{center}
  \footnotesize
  \begin{longtable}{| p{4.5cm} | p{2.5cm} | p{0.5cm} | p{2.5cm} | p{6cm} |}
    \hline
    \bf Name & \bf type & \bf size & \bf default value & \bf description \\ \hline
    show &     bool &      &     False &     set true, if item is shown in visualization and false if it is not shown\\ \hline
	  \end{longtable}
	\end{center}

%+++++++++++++++++++++++++++++++++++
\mysubsubsection{NodePointGround}
A 3D point node fixed to ground. The node can be used as NodePoint, but does not lead to equations. Applied or reaction forces do not have any effect.
 \\
{\bf Short name} for Python: {\bf PointGround}
 \\\\ 
{\bf Output variables} (chose type, e.g., OutputVariableType.Position): 
\begin{itemize}
    \item {\bf Position}: global 3D position vector of node (=reference position)
    \item {\bf Displacement}: zero 3D vector
    \item {\bf Velocity}: zero 3D vector
    \item {\bf Coordinates}: vector of length zero
    \item {\bf Coordinates\_t}: vector of length zero
\end{itemize}
The item NodePointGround has the following parameters:
%reference manual TABLE
\begin{center}
  \footnotesize
  \begin{longtable}{| p{4.5cm} | p{2.5cm} | p{0.5cm} | p{2.5cm} | p{6cm} |}
    \hline
    \bf Name & \bf type & \bf size & \bf default value & \bf description \\ \hline
    \multicolumn{4}{l}{\parbox{10cm}{type = 'PointGround'}} & \multicolumn{1}{l}{\parbox{6cm}{\it item typename for initialization}}\\ \hline
    name &     String &      &     '' &     node"s unique name\\ \hline
    referenceCoordinates &     Vector3D &     3 &     [0.,0.,0.] &     reference coordinates of node ==> e.g. ref. coordinates for finite elements; global position of node without displacement\\ \hline
    visualization & VNodePointGround & & & parameters for visualization of item \\ \hline
	  \end{longtable}
	\end{center}
The item VNodePointGround has the following parameters:
%reference manual TABLE
\begin{center}
  \footnotesize
  \begin{longtable}{| p{4.5cm} | p{2.5cm} | p{0.5cm} | p{2.5cm} | p{6cm} |}
    \hline
    \bf Name & \bf type & \bf size & \bf default value & \bf description \\ \hline
    show &     bool &      &     True &     set true, if item is shown in visualization and false if it is not shown\\ \hline
    drawSize &     float &      &     -1. &     drawing size (diameter, dimensions of underlying cube, etc.)  for item; size == -1.f means that default size is used\\ \hline
    color &     Float4 &     4 &     [-1.,-1.,-1.,-1.] &     Default RGBA color for nodes; 4th value is alpha-transparency; R=-1.f means, that default color is used\\ \hline
	  \end{longtable}
	\end{center}

%+++++++++++++++++++++++++++++++
%+++++++++++++++++++++++++++++++
\mysubsection{Objects}

%+++++++++++++++++++++++++++++++++++
\mysubsubsection{ObjectMassPoint}
A 3D mass point which is attached to a position-based node. Equations of motion with the displacements $[u_x\;\; u_y\;\; u_z]^T$, the mass $m$ and the residual of all forces $[R_x\;\; R_y\;\; R_z]^T$ are given as \be \vr{m \cdot \ddot u_x}{m \cdot \ddot u_y}{m \cdot \ddot u_z} = \vr{R_x}{R_y}{R_z}.\ee
 \\
{\bf Short name} for Python: {\bf MassPoint}
 \\\\ 
{\bf Output variables} (chose type, e.g., OutputVariableType.Position): 
\begin{itemize}
    \item {\bf Position}: global position vector of translated local position
    \item {\bf Displacement}: global displacement vector of center point
    \item {\bf Velocity}: global velocity vector of center point
\end{itemize}
The item ObjectMassPoint has the following parameters:
%reference manual TABLE
\begin{center}
  \footnotesize
  \begin{longtable}{| p{4.5cm} | p{2.5cm} | p{0.5cm} | p{2.5cm} | p{6cm} |}
    \hline
    \bf Name & \bf type & \bf size & \bf default value & \bf description \\ \hline
    \multicolumn{4}{l}{\parbox{10cm}{type = 'MassPoint'}} & \multicolumn{1}{l}{\parbox{6cm}{\it item typename for initialization}}\\ \hline
    name &     String &      &     '' &     objects"s unique name\\ \hline
    physicsMass &     UReal &      &     0. &     mass [SI:kg] of mass point\\ \hline
    nodeNumber &     Index &      &     MAXINT &     node number for mass point\\ \hline
    visualization & VObjectMassPoint & & & parameters for visualization of item \\ \hline
	  \end{longtable}
	\end{center}
The item VObjectMassPoint has the following parameters:
%reference manual TABLE
\begin{center}
  \footnotesize
  \begin{longtable}{| p{4.5cm} | p{2.5cm} | p{0.5cm} | p{2.5cm} | p{6cm} |}
    \hline
    \bf Name & \bf type & \bf size & \bf default value & \bf description \\ \hline
    show &     bool &      &     True &     set true, if item is shown in visualization and false if it is not shown\\ \hline
    graphicsData &     BodyGraphicsData &     \tabnewline  &      &     Structure contains data for body visualization; data is defined in special list / dictionary structure\\ \hline
	  \end{longtable}
	\end{center}

%+++++++++++++++++++++++++++++++++++
\mysubsubsection{ObjectMassPoint2D}
A 2D mass point which is attached to a position-based 2D node. Equations of motion with the displacements $[u_x\;\; u_y]^T$, the mass $m$ and the residual of all forces $[R_x\;\; R_y]^T$ are given as \be \vp{m \cdot \ddot u_x}{m \cdot \ddot u_y} = \vp{R_x}{R_y}.\ee
 \\
{\bf Short name} for Python: {\bf MassPoint2D}
 \\\\ 
{\bf Output variables} (chose type, e.g., OutputVariableType.Position): 
\begin{itemize}
    \item {\bf Position}: global position vector of translated local position
    \item {\bf Displacement}: global displacement vector of center point
    \item {\bf Velocity}: global velocity vector of center point
\end{itemize}
The item ObjectMassPoint2D has the following parameters:
%reference manual TABLE
\begin{center}
  \footnotesize
  \begin{longtable}{| p{4.5cm} | p{2.5cm} | p{0.5cm} | p{2.5cm} | p{6cm} |}
    \hline
    \bf Name & \bf type & \bf size & \bf default value & \bf description \\ \hline
    \multicolumn{4}{l}{\parbox{10cm}{type = 'MassPoint2D'}} & \multicolumn{1}{l}{\parbox{6cm}{\it item typename for initialization}}\\ \hline
    name &     String &      &     '' &     objects"s unique name\\ \hline
    physicsMass &     UReal &      &     0. &     mass [SI:kg] of mass point\\ \hline
    nodeNumber &     Index &      &     MAXINT &     node number for mass point\\ \hline
    visualization & VObjectMassPoint2D & & & parameters for visualization of item \\ \hline
	  \end{longtable}
	\end{center}
The item VObjectMassPoint2D has the following parameters:
%reference manual TABLE
\begin{center}
  \footnotesize
  \begin{longtable}{| p{4.5cm} | p{2.5cm} | p{0.5cm} | p{2.5cm} | p{6cm} |}
    \hline
    \bf Name & \bf type & \bf size & \bf default value & \bf description \\ \hline
    show &     bool &      &     True &     set true, if item is shown in visualization and false if it is not shown\\ \hline
    graphicsData &     BodyGraphicsData &     \tabnewline  &      &     Structure contains data for body visualization; data is defined in special list / dictionary structure\\ \hline
	  \end{longtable}
	\end{center}

%+++++++++++++++++++++++++++++++++++
\mysubsubsection{ObjectRigidBody}
A 3D rigid body which is attached to a 3D rigid body node. Equations of motion with the displacements $[u_x\;\; u_y\;\; u_z]^T$ of the center of mass and the rotation parameters (Euler parameters) $\mathbf{q}$, the mass $m$, inertia $\mathbf{J} = [J_{xx}, J_{xy}, J_{xz}; J_{yx}, J_{yy}, J_{yz}; J_{zx}, J_{zy}, J_{zz}]$ and the residual of all forces and moments $[R_x\;\; R_y\;\; R_z\;\; R_{q0}\;\; R_{q1}\;\; R_{q2}\;\; R_{q3}]^T$ are given as ...
 \\
{\bf Short name} for Python: {\bf RigidBody}
 \\\\ 
{\bf Output variables} (chose type, e.g., OutputVariableType.Position): 
\begin{itemize}
    \item {\bf Position}: global position vector of rotated and translated local position
    \item {\bf Displacement}: global displacement vector of local position
    \item {\bf RotationMatrix}: vector with 9 components of the rotation matrix (row-major format)
    \item {\bf Rotation}: vector with 3 components of the Euler angles in xyz-sequence (R=Rx*Ry*Rz), recomputed from rotation matrix
    \item {\bf Velocity}: global velocity vector of local position
    \item {\bf AngularVelocity}: angular velocity of body
    \item {\bf AngularVelocityLocal}: local (body-fixed) 3D velocity vector of node
\end{itemize}
The item ObjectRigidBody has the following parameters:
%reference manual TABLE
\begin{center}
  \footnotesize
  \begin{longtable}{| p{4.5cm} | p{2.5cm} | p{0.5cm} | p{2.5cm} | p{6cm} |}
    \hline
    \bf Name & \bf type & \bf size & \bf default value & \bf description \\ \hline
    \multicolumn{4}{l}{\parbox{10cm}{type = 'RigidBody'}} & \multicolumn{1}{l}{\parbox{6cm}{\it item typename for initialization}}\\ \hline
    name &     String &      &     '' &     objects"s unique name\\ \hline
    physicsMass &     UReal &      &     0. &     mass [SI:kg] of mass point\\ \hline
    physicsInertia &     Vector6D &      &     [0.,0.,0., 0.,0.,0.] &     inertia components [SI:kgm$^2$]: $[J_{xx}, J_{yy}, J_{zz}, J_{yz}, J_{xz}, J_{xy}]$ of rigid body w.r.t. center of mass\\ \hline
    nodeNumber &     Index &      &     MAXINT &     node number for rigid body node\\ \hline
    visualization & VObjectRigidBody & & & parameters for visualization of item \\ \hline
	  \end{longtable}
	\end{center}
The item VObjectRigidBody has the following parameters:
%reference manual TABLE
\begin{center}
  \footnotesize
  \begin{longtable}{| p{4.5cm} | p{2.5cm} | p{0.5cm} | p{2.5cm} | p{6cm} |}
    \hline
    \bf Name & \bf type & \bf size & \bf default value & \bf description \\ \hline
    show &     bool &      &     True &     set true, if item is shown in visualization and false if it is not shown\\ \hline
    graphicsData &     BodyGraphicsData &     \tabnewline  &      &     Structure contains data for body visualization; data is defined in special list / dictionary structure\\ \hline
	  \end{longtable}
	\end{center}

%+++++++++++++++++++++++++++++++++++
\mysubsubsection{ObjectRigidBody2D}
A 2D rigid body which is attached to a rigid body 2D node. Equations of motion with the displacements $[u_x\;\; u_y]^T$ of the center of mass and the rotation $\varphi$ (positive rotation around z-axis), the mass $m$, inertia around z-axis $J$ and the residual of all forces and moments $[R_x\;\; R_y\;\; R_\varphi]^T$ are given as \be \vr{m \cdot \ddot u_x}{m \cdot \ddot u_y}{J \varphi} = \vr{R_x}{R_y}{R_\varphi}.\ee
 \\
{\bf Short name} for Python: {\bf RigidBody2D}
 \\\\ 
{\bf Output variables} (chose type, e.g., OutputVariableType.Position): 
\begin{itemize}
    \item {\bf Position}: global position vector of rotated and translated local position
    \item {\bf Displacement}: global displacement vector of local position
    \item {\bf Velocity}: global velocity vector of local position
    \item {\bf Rotation}: scalar rotation angle of body
    \item {\bf AngularVelocity}: angular velocity of body
    \item {\bf RotationMatrix}: rotation matrix in vector form (stored in row-major order)
\end{itemize}
The item ObjectRigidBody2D has the following parameters:
%reference manual TABLE
\begin{center}
  \footnotesize
  \begin{longtable}{| p{4.5cm} | p{2.5cm} | p{0.5cm} | p{2.5cm} | p{6cm} |}
    \hline
    \bf Name & \bf type & \bf size & \bf default value & \bf description \\ \hline
    \multicolumn{4}{l}{\parbox{10cm}{type = 'RigidBody2D'}} & \multicolumn{1}{l}{\parbox{6cm}{\it item typename for initialization}}\\ \hline
    name &     String &      &     '' &     objects"s unique name\\ \hline
    physicsMass &     UReal &      &     0. &     mass [SI:kg] of mass point\\ \hline
    physicsInertia &     UReal &      &     0. &     inertia [SI:kgm$^2$] of rigid body w.r.t. center of mass\\ \hline
    nodeNumber &     Index &      &     MAXINT &     node number for 2D rigid body node\\ \hline
    visualization & VObjectRigidBody2D & & & parameters for visualization of item \\ \hline
	  \end{longtable}
	\end{center}
The item VObjectRigidBody2D has the following parameters:
%reference manual TABLE
\begin{center}
  \footnotesize
  \begin{longtable}{| p{4.5cm} | p{2.5cm} | p{0.5cm} | p{2.5cm} | p{6cm} |}
    \hline
    \bf Name & \bf type & \bf size & \bf default value & \bf description \\ \hline
    show &     bool &      &     True &     set true, if item is shown in visualization and false if it is not shown\\ \hline
    graphicsData &     BodyGraphicsData &     \tabnewline  &      &     Structure contains data for body visualization; data is defined in special list / dictionary structure\\ \hline
	  \end{longtable}
	\end{center}

%+++++++++++++++++++++++++++++++++++
\mysubsubsection{ObjectANCFCable2D}
A 2D cable finite element using 2 nodes of type NodePoint2DSlope1; the element has 8 coordinates and uses cubic polynomials for position interpolation; the Bernoulli-Euler beam is capable of large deformation as it employs the material measure of curvature for the bending.
 \\
{\bf Short name} for Python: {\bf Cable2D}
 \\\\ 
{\bf Output variables} (chose type, e.g., OutputVariableType.Position): 
\begin{itemize}
    \item {\bf Position}: global position vector of local axis (1) and cross section (2) position
    \item {\bf Displacement}: global displacement vector of local axis (1) and cross section (2) position
    \item {\bf Velocity}: global velocity vector of local axis (1) and cross section (2) position
    \item {\bf Director1}: (axial) slope vector of local axis position
    \item {\bf Strain}: axial strain (scalar)
    \item {\bf Curvature}: axial strain (scalar)
    \item {\bf Force}: (local) section normal force (scalar)
    \item {\bf Torque}: (local) bending moment (scalar)
\end{itemize}
The item ObjectANCFCable2D has the following parameters:
%reference manual TABLE
\begin{center}
  \footnotesize
  \begin{longtable}{| p{4.5cm} | p{2.5cm} | p{0.5cm} | p{2.5cm} | p{6cm} |}
    \hline
    \bf Name & \bf type & \bf size & \bf default value & \bf description \\ \hline
    \multicolumn{4}{l}{\parbox{10cm}{type = 'ANCFCable2D'}} & \multicolumn{1}{l}{\parbox{6cm}{\it item typename for initialization}}\\ \hline
    name &     String &      &     '' &     objects"s unique name\\ \hline
    physicsLength &     UReal &      &     0. &     reference length $L$ [SI:m] of beam; such that the total volume (e.g. for volume load) gives $\rho A L$\\ \hline
    physicsMassPerLength &     UReal &      &     0. &     mass $\rho A$ [SI:kg/m$^2$] of beam\\ \hline
    physicsBendingStiffness &     UReal &      &     0. &     bending stiffness $EI$ [SI:Nm$^2$] of beam; the bending moment is $m = EI (\kappa - \kappa_0)$, in which $\kappa$ is the material measure of curvature\\ \hline
    physicsAxialStiffness &     UReal &      &     0. &     axial stiffness $EA$ [SI:N] of beam; the axial force is $f_{ax} = EA (\varepsilon -\varepsilon_0)$, in which $\varepsilon = |\rv^\prime|-1$ is the axial strain\\ \hline
    physicsBendingDamping &     UReal &      &     0. &     bending damping $d_{EI}$ [SI:Nm$^2$/s] of beam; the additional virtual work due to damping is $\delta W_{\dot \kappa} = \int_0^L \dot \kappa \delta \kappa dx$\\ \hline
    physicsAxialDamping &     UReal &      &     0. &     axial stiffness $d_{EA}$ [SI:N/s] of beam; the additional virtual work due to damping is $\delta W_{\dot\varepsilon} = \int_0^L \dot \varepsilon \delta \varepsilon dx$\\ \hline
    physicsReferenceAxialStrain &     UReal &      &     0. &     reference axial strain of beam (pre-deformation) $\varepsilon_0$ [SI:1] of beam; without external loading the beam will statically keep the reference axial strain value\\ \hline
    physicsReferenceCurvature &     UReal &      &     0. &     reference curvature of beam (pre-deformation) $\kappa_0$ [SI:1/m] of beam; without external loading the beam will statically keep the reference curvature value\\ \hline
    nodeNumbers &     Index2 &      &     [MAXINT, MAXINT] &     two node numbers ANCF cable element\\ \hline
    useReducedOrderIntegration &     Bool &      &     False &     false: use Gauss order 9 integration for virtual work of axial forces, order 5 for virtual work of bending moments; true: use Gauss order 7 integration for virtual work of axial forces, order 3 for virtual work of bending moments\\ \hline
    visualization & VObjectANCFCable2D & & & parameters for visualization of item \\ \hline
	  \end{longtable}
	\end{center}
The item VObjectANCFCable2D has the following parameters:
%reference manual TABLE
\begin{center}
  \footnotesize
  \begin{longtable}{| p{4.5cm} | p{2.5cm} | p{0.5cm} | p{2.5cm} | p{6cm} |}
    \hline
    \bf Name & \bf type & \bf size & \bf default value & \bf description \\ \hline
    show &     bool &      &     True &     set true, if item is shown in visualization and false if it is not shown\\ \hline
    drawHeight &     float &      &     0. &     if beam is drawn with rectangular shape, this is the drawing height\\ \hline
    color &     Float4 &      &     [-1.,-1.,-1.,-1.] &     RGBA color of the object; if R==-1, use default color\\ \hline
	  \end{longtable}
	\end{center}

%+++++++++++++++++++++++++++++++++++
\mysubsubsection{ObjectALEANCFCable2D}
A 2D cable finite element using 2 nodes of type NodePoint2DSlope1 and a axially moving coordinate of type NodeGenericODE2; the element has 8+1 coordinates and uses cubic polynomials for position interpolation; the element in addition to ANCFCable2D adds an Eulerian axial velocity by the GenericODE2 coordiante
 \\
{\bf Short name} for Python: {\bf ALECable2D}
 \\\\ 
{\bf Output variables} (chose type, e.g., OutputVariableType.Position): 
\begin{itemize}
    \item {\bf Position}: global position vector of local axis (1) and cross section (2) position
    \item {\bf Displacement}: global displacement vector of local axis (1) and cross section (2) position
    \item {\bf Velocity}: global velocity vector of local axis (1) and cross section (2) position
    \item {\bf Director1}: (axial) slope vector of local axis position
    \item {\bf Strain}: axial strain (scalar)
    \item {\bf Curvature}: axial strain (scalar)
    \item {\bf Force}: (local) section normal force (scalar)
    \item {\bf Torque}: (local) bending moment (scalar)
\end{itemize}
The item ObjectALEANCFCable2D has the following parameters:
%reference manual TABLE
\begin{center}
  \footnotesize
  \begin{longtable}{| p{4.5cm} | p{2.5cm} | p{0.5cm} | p{2.5cm} | p{6cm} |}
    \hline
    \bf Name & \bf type & \bf size & \bf default value & \bf description \\ \hline
    \multicolumn{4}{l}{\parbox{10cm}{type = 'ALEANCFCable2D'}} & \multicolumn{1}{l}{\parbox{6cm}{\it item typename for initialization}}\\ \hline
    name &     String &      &     '' &     objects"s unique name\\ \hline
    physicsLength &     UReal &      &     0. &     reference length $L$ [SI:m] of beam; such that the total volume (e.g. for volume load) gives $\rho A L$\\ \hline
    physicsMassPerLength &     UReal &      &     0. &     mass $\rho A$ [SI:kg/m$^2$] of beam\\ \hline
    physicsMovingMassFactor &     UReal &      &     1. &     this factor denotes the amount of $\rho A$ which is moving; physicsMovingMassFactor=1 means, that all mass is moving; physicsMovingMassFactor=0 means, that no mass is moving; factor can be used to simulate e.g. pipe conveying fluid, in which $\rho A$ is the mass of the pipe+fluid, while $physicsMovingMassFactor \cdot \rho A$ is the mass per unit length of the fluid\\ \hline
    physicsBendingStiffness &     UReal &      &     0. &     bending stiffness $EI$ [SI:Nm$^2$] of beam; the bending moment is $m = EI (\kappa - \kappa_0)$, in which $\kappa$ is the material measure of curvature\\ \hline
    physicsAxialStiffness &     UReal &      &     0. &     axial stiffness $EA$ [SI:N] of beam; the axial force is $f_{ax} = EA (\varepsilon -\varepsilon_0)$, in which $\varepsilon = |\rv^\prime|-1$ is the axial strain\\ \hline
    physicsBendingDamping &     UReal &      &     0. &     bending damping $d_{EI}$ [SI:Nm$^2$/s] of beam; the additional virtual work due to damping is $\delta W_{\dot \kappa} = \int_0^L \dot \kappa \delta \kappa dx$\\ \hline
    physicsAxialDamping &     UReal &      &     0. &     axial stiffness $d_{EA}$ [SI:N/s] of beam; the additional virtual work due to damping is $\delta W_{\dot\varepsilon} = \int_0^L \dot \varepsilon \delta \varepsilon dx$\\ \hline
    physicsReferenceAxialStrain &     UReal &      &     0. &     reference axial strain of beam (pre-deformation) $\varepsilon_0$ [SI:1] of beam; without external loading the beam will statically keep the reference axial strain value\\ \hline
    physicsReferenceCurvature &     UReal &      &     0. &     reference curvature of beam (pre-deformation) $\kappa_0$ [SI:1/m] of beam; without external loading the beam will statically keep the reference curvature value\\ \hline
    physicsUseCouplingTerms &     bool &      &     True &     true: correct case, where all coupling terms due to moving mass are respected; false: only include constant mass for ALE node coordinate, but deactivate other coupling terms (behaves like ANCFCable2D then)\\ \hline
    nodeNumbers &     Index3 &      &     [MAXINT, MAXINT, MAXINT] &     two node numbers ANCF cable element, third node=ALE GenericODE2 node\\ \hline
    useReducedOrderIntegration &     Bool &      &     False &     false: use Gauss order 9 integration for virtual work of axial forces, order 5 for virtual work of bending moments; true: use Gauss order 7 integration for virtual work of axial forces, order 3 for virtual work of bending moments\\ \hline
    visualization & VObjectALEANCFCable2D & & & parameters for visualization of item \\ \hline
	  \end{longtable}
	\end{center}
The item VObjectALEANCFCable2D has the following parameters:
%reference manual TABLE
\begin{center}
  \footnotesize
  \begin{longtable}{| p{4.5cm} | p{2.5cm} | p{0.5cm} | p{2.5cm} | p{6cm} |}
    \hline
    \bf Name & \bf type & \bf size & \bf default value & \bf description \\ \hline
    show &     bool &      &     True &     set true, if item is shown in visualization and false if it is not shown\\ \hline
    drawHeight &     float &      &     0. &     if beam is drawn with rectangular shape, this is the drawing height\\ \hline
    color &     Float4 &      &     [-1.,-1.,-1.,-1.] &     RGBA color of the object; if R==-1, use default color\\ \hline
	  \end{longtable}
	\end{center}

%+++++++++++++++++++++++++++++++++++
\mysubsubsection{ObjectGround}
A ground object behaving like a rigid body, but having no degrees of freedom; used to attach body-connectors without an action.
 \\\\ 
{\bf Output variables} (chose type, e.g., OutputVariableType.Position): 
\begin{itemize}
    \item {\bf Position}: global position vector of rotated and translated local position
    \item {\bf Displacement}: global displacement vector of local position
    \item {\bf Velocity}: global velocity vector of local position
    \item {\bf AngularVelocity}: angular velocity of body
    \item {\bf RotationMatrix}: rotation matrix in vector form (stored in row-major order)
\end{itemize}
The item ObjectGround has the following parameters:
%reference manual TABLE
\begin{center}
  \footnotesize
  \begin{longtable}{| p{4.5cm} | p{2.5cm} | p{0.5cm} | p{2.5cm} | p{6cm} |}
    \hline
    \bf Name & \bf type & \bf size & \bf default value & \bf description \\ \hline
    \multicolumn{4}{l}{\parbox{10cm}{type = 'Ground'}} & \multicolumn{1}{l}{\parbox{6cm}{\it item typename for initialization}}\\ \hline
    name &     String &      &     '' &     objects"s unique name\\ \hline
    referencePosition &     Vector3D &     3 &     [0.,0.,0.] &     reference position for ground object; local position is added on top of reference position for a ground object\\ \hline
    visualization & VObjectGround & & & parameters for visualization of item \\ \hline
	  \end{longtable}
	\end{center}
The item VObjectGround has the following parameters:
%reference manual TABLE
\begin{center}
  \footnotesize
  \begin{longtable}{| p{4.5cm} | p{2.5cm} | p{0.5cm} | p{2.5cm} | p{6cm} |}
    \hline
    \bf Name & \bf type & \bf size & \bf default value & \bf description \\ \hline
    show &     bool &      &     True &     set true, if item is shown in visualization and false if it is not shown\\ \hline
    color &     Float4 &      &     [-1.,-1.,-1.,-1.] &     RGB node color; if R==-1, use default color\\ \hline
    graphicsData &     BodyGraphicsData &     \tabnewline  &      &     Structure contains data for body visualization; data is defined in special list / dictionary structure\\ \hline
	  \end{longtable}
	\end{center}

%+++++++++++++++++++++++++++++++++++
\mysubsubsection{ObjectConnectorSpringDamper}
An simple spring-damper element with additional force; connects to position-based markers.
 \\
{\bf Short name} for Python: {\bf SpringDamper}
 \\  {\bf Requested marker type} = Marker::Position \\ 
The item ObjectConnectorSpringDamper has the following parameters:
%reference manual TABLE
\begin{center}
  \footnotesize
  \begin{longtable}{| p{4.5cm} | p{2.5cm} | p{0.5cm} | p{2.5cm} | p{6cm} |}
    \hline
    \bf Name & \bf type & \bf size & \bf default value & \bf description \\ \hline
    \multicolumn{4}{l}{\parbox{10cm}{type = 'ConnectorSpringDamper'}} & \multicolumn{1}{l}{\parbox{6cm}{\it item typename for initialization}}\\ \hline
    name &     String &      &     '' &     connector"s unique name\\ \hline
    markerNumbers &     ArrayIndex &      &     [ MAXINT, MAXINT ] &     list of markers used in connector\\ \hline
    referenceLength &     UReal &      &     0. &     reference length [SI:m] of spring\\ \hline
    stiffness &     UReal &      &     0. &     stiffness [SI:N/m] of spring; acts against (length-initialLength)\\ \hline
    damping &     UReal &      &     0. &     damping [SI:N/(m s)] of damper; acts against d/dt(length)\\ \hline
    force &     UReal &      &     0. &     added constant force [SI:N] of spring; scalar force; f=1 is equivalent to reducing initialLength by 1/stiffness; f > 0: tension; f < 0: compression\\ \hline
    activeConnector &     bool &      &     True &     flag, which determines, if the connector is active; used to deactivate (temorarily) a connector or constraint\\ \hline
    springForceUserFunction &     PyFunctionScalar5 &     \tabnewline  &     0 &     A python function which defines the spring force with parameters (deltaL, deltaL\_t, Real stiffness, Real damping, Real springForce); the parameters are provided to the function using the current values of the SpringDamper object; The python function will only be evaluated, if activeConnector is true, otherwise the SpringDamper is inactive; Example for python function: def f(u, v, k, d, F0): return k*u + d*v + F0\\ \hline
    visualization & VObjectConnectorSpringDamper & & & parameters for visualization of item \\ \hline
	  \end{longtable}
	\end{center}
The item VObjectConnectorSpringDamper has the following parameters:
%reference manual TABLE
\begin{center}
  \footnotesize
  \begin{longtable}{| p{4.5cm} | p{2.5cm} | p{0.5cm} | p{2.5cm} | p{6cm} |}
    \hline
    \bf Name & \bf type & \bf size & \bf default value & \bf description \\ \hline
    show &     Bool &      &     True &     set true, if item is shown in visualization and false if it is not shown\\ \hline
    drawSize &     float &      &     -1. &     drawing size = diameter of spring; size == -1.f means that default connector size is used\\ \hline
    color &     Float4 &      &     [-1.,-1.,-1.,-1.] &     RGB connector color; if R==-1, use default color\\ \hline
	  \end{longtable}
	\end{center}

%+++++++++++++++++++++++++++++++++++
\mysubsubsection{ObjectConnectorCartesianSpringDamper}
An 3D spring-damper element acting accordingly in three directions (x,y,z); connects to position-based markers; represents a penalty-based spherical joint; the resulting force in the spring-damper reads ($m0 = marker[0]$ and $m1 = marker[1]$): \be force_x = (m1.position_x - m0.position_x - offset_x)\cdot stiffness_x + (m1.velocity_x - m0.velocity_x)\cdot damping_x, etc. \ee.
 \\
{\bf Short name} for Python: {\bf CartesianSpringDamper}
 \\  {\bf Requested marker type} = Marker::Position \\ 
The item ObjectConnectorCartesianSpringDamper has the following parameters:
%reference manual TABLE
\begin{center}
  \footnotesize
  \begin{longtable}{| p{4.5cm} | p{2.5cm} | p{0.5cm} | p{2.5cm} | p{6cm} |}
    \hline
    \bf Name & \bf type & \bf size & \bf default value & \bf description \\ \hline
    \multicolumn{4}{l}{\parbox{10cm}{type = 'ConnectorCartesianSpringDamper'}} & \multicolumn{1}{l}{\parbox{6cm}{\it item typename for initialization}}\\ \hline
    name &     String &      &     '' &     connector"s unique name\\ \hline
    markerNumbers &     ArrayIndex &      &     [ MAXINT, MAXINT ] &     list of markers used in connector\\ \hline
    stiffness &     Vector3D &      &     [0.,0.,0.] &     stiffness [SI:N/m] of springs; act against relative displacements in x, y, and z-direction\\ \hline
    damping &     Vector3D &      &     [0.,0.,0.] &     damping [SI:N/(m s)] of dampers; act against relative velocities in x, y, and z-direction\\ \hline
    offset &     Vector3D &      &     [0.,0.,0.] &     offset between two springs\\ \hline
    activeConnector &     bool &      &     True &     flag, which determines, if the connector is active; used to deactivate (temorarily) a connector or constraint\\ \hline
    visualization & VObjectConnectorCartesianSpringDamper & & & parameters for visualization of item \\ \hline
	  \end{longtable}
	\end{center}
The item VObjectConnectorCartesianSpringDamper has the following parameters:
%reference manual TABLE
\begin{center}
  \footnotesize
  \begin{longtable}{| p{4.5cm} | p{2.5cm} | p{0.5cm} | p{2.5cm} | p{6cm} |}
    \hline
    \bf Name & \bf type & \bf size & \bf default value & \bf description \\ \hline
    show &     Bool &      &     True &     set true, if item is shown in visualization and false if it is not shown\\ \hline
    drawSize &     float &      &     -1. &     drawing size = diameter of spring; size == -1.f means that default connector size is used\\ \hline
    color &     Float4 &      &     [-1.,-1.,-1.,-1.] &     RGB connector color; if R==-1, use default color\\ \hline
	  \end{longtable}
	\end{center}

%+++++++++++++++++++++++++++++++++++
\mysubsubsection{ObjectConnectorRigidBodySpringDamper}
An 3D spring-damper element acting on relative displacements and relative rotations of two rigid body (position+orientation) markers; connects to (position+orientation)-based markers; represents a penalty-based rigid joint; the resulting force in the spring-damper reads ($m0 = marker[0]$ and $m1 = marker[1]$): \be force_x = (A0loc \cdot A0) \cdot stiffness_x \cdot (A0loc \cdot A0)^T(m1.position_x - m0.position_x - offset_x) + (A0loc \cdot A0) \cdot damping_x \cdot (A0loc \cdot A0)^T (m1.velocity_x - m0.velocity_x), etc. \ee and accordingly for rotation coordinates, which act on $(rotationMarker0 \cdot Rxyz0)^T \cdot (rotationMarker1 \cdot Rxyz1) $ rotations (0...rotation of marker0, 1...rotation of marker1).
 \\
{\bf Short name} for Python: {\bf RigidBodySpringDamper}
 \\  {\bf Requested marker type} = (Marker::Type)((Index)Marker::Position + (Index)Marker::Orientation) \\ 
The item ObjectConnectorRigidBodySpringDamper has the following parameters:
%reference manual TABLE
\begin{center}
  \footnotesize
  \begin{longtable}{| p{4.5cm} | p{2.5cm} | p{0.5cm} | p{2.5cm} | p{6cm} |}
    \hline
    \bf Name & \bf type & \bf size & \bf default value & \bf description \\ \hline
    \multicolumn{4}{l}{\parbox{10cm}{type = 'ConnectorRigidBodySpringDamper'}} & \multicolumn{1}{l}{\parbox{6cm}{\it item typename for initialization}}\\ \hline
    name &     String &      &     '' &     connector"s unique name\\ \hline
    markerNumbers &     ArrayIndex &      &     [ MAXINT, MAXINT ] &     list of markers used in connector\\ \hline
    stiffness &     Matrix6D &      &     Matrix6D[6,6,0.] &     stiffness [SI:N/m or Nm/rad] of translational, torsional and coupled springs; act against relative displacements in x, y, and z-direction as well as the relative angles (calculated as Euler angles)\\ \hline
    damping &     Matrix6D &      &     Matrix6D[6,6,0.] &     damping [SI:N/(m/s) or Nm/(rad/s)] of translational, torsional and coupled dampers; \\ \hline
    rotationMarker0 &     Matrix3D &      &     EXUmath::unitMatrix3D &     local rotation matrix for marker 0; stiffness, damping, etc. components are measured in local coordinates relative to rotationMarker0\\ \hline
    rotationMarker1 &     Matrix3D &      &     EXUmath::unitMatrix3D &     local rotation matrix for marker 1; stiffness, damping, etc. components are measured in local coordinates relative to rotationMarker1\\ \hline
    offset &     Vector6D &      &     [0.,0.,0.,0.,0.,0.] &     translational and rotational offset considered in the spring force calculation\\ \hline
    activeConnector &     bool &      &     True &     flag, which determines, if the connector is active; used to deactivate (temorarily) a connector or constraint\\ \hline
    visualization & VObjectConnectorRigidBodySpringDamper & & & parameters for visualization of item \\ \hline
	  \end{longtable}
	\end{center}
The item VObjectConnectorRigidBodySpringDamper has the following parameters:
%reference manual TABLE
\begin{center}
  \footnotesize
  \begin{longtable}{| p{4.5cm} | p{2.5cm} | p{0.5cm} | p{2.5cm} | p{6cm} |}
    \hline
    \bf Name & \bf type & \bf size & \bf default value & \bf description \\ \hline
    show &     Bool &      &     True &     set true, if item is shown in visualization and false if it is not shown\\ \hline
    drawSize &     float &      &     -1. &     drawing size = diameter of spring; size == -1.f means that default connector size is used\\ \hline
    color &     Float4 &      &     [-1.,-1.,-1.,-1.] &     RGB connector color; if R==-1, use default color\\ \hline
	  \end{longtable}
	\end{center}

%+++++++++++++++++++++++++++++++++++
\mysubsubsection{ObjectConnectorCoordinateSpringDamper}
A 1D (scalar) spring-damper element acting on single ODE2 coordinates; connects to coordinate-based markers; NOTE that the coordinate markers only measure the coordinate (=displacement), but the reference position is not included as compared to position-based markers!; the spring-damper can also act on rotational coordinates; the resulting force in the spring-damper reads ($m0 = marker[0]$ and $m1 = marker[1]$): \be f = (m1.coordinate - m0.coordinate - offset)\cdot stiffness + (m1.coordinate_t - m0.coordinate_t)\cdot damping \ee If dry (Coulomb) friction is non-zero, an additional term \be \text{sign}(m1.coordinate_t - m0.coordinate_t)\cdot dryFriction \ee is added to the force $f$.
 \\
{\bf Short name} for Python: {\bf CoordinateSpringDamper}
 \\  {\bf Requested marker type} = Marker::Coordinate \\ 
The item ObjectConnectorCoordinateSpringDamper has the following parameters:
%reference manual TABLE
\begin{center}
  \footnotesize
  \begin{longtable}{| p{4.5cm} | p{2.5cm} | p{0.5cm} | p{2.5cm} | p{6cm} |}
    \hline
    \bf Name & \bf type & \bf size & \bf default value & \bf description \\ \hline
    \multicolumn{4}{l}{\parbox{10cm}{type = 'ConnectorCoordinateSpringDamper'}} & \multicolumn{1}{l}{\parbox{6cm}{\it item typename for initialization}}\\ \hline
    name &     String &      &     '' &     connector"s unique name\\ \hline
    markerNumbers &     ArrayIndex &      &     [ MAXINT, MAXINT ] &     list of markers used in connector\\ \hline
    stiffness &     Real &      &     0. &     stiffness [SI:N/m] of spring; acts against relative value of coordinates\\ \hline
    damping &     Real &      &     0. &     damping [SI:N/(m s)] of damper; acts against relative velocity of coordinates\\ \hline
    offset &     Real &      &     0. &     offset between two coordinates (reference length of springs), see equation\\ \hline
    dryFriction &     Real &      &     0. &     dry friction coefficient against relative velocity\\ \hline
    dryFrictionProportionalZone &     Real &      &     0. &     limit velocity [m/s] up to which the friction is proportional to velocity (for regularization / avoid numerical oscillations)\\ \hline
    activeConnector &     bool &      &     True &     flag, which determines, if the connector is active; used to deactivate (temorarily) a connector or constraint\\ \hline
    springForceUserFunction &     PyFunctionScalar7 &     \tabnewline  &     0 &     A python function which defines the spring force with parameters (deltaL, deltaL$_t$, Real stiffness, Real damping, Real offset, Real dryFriction, Real dryFrictionProportionalZone); the parameters are provided to the function using the current values of the SpringDamper object; note that $u=(m1.coordinate - m0.coordinate)$, not including the offset; The python function will only be evaluated, if activeConnector is true, otherwise the SpringDamper is inactive; Example for python function: def f(u, v, k, d, offset, mu, muProp): return k*u + d*v + F0\\ \hline
    visualization & VObjectConnectorCoordinateSpringDamper & & & parameters for visualization of item \\ \hline
	  \end{longtable}
	\end{center}
The item VObjectConnectorCoordinateSpringDamper has the following parameters:
%reference manual TABLE
\begin{center}
  \footnotesize
  \begin{longtable}{| p{4.5cm} | p{2.5cm} | p{0.5cm} | p{2.5cm} | p{6cm} |}
    \hline
    \bf Name & \bf type & \bf size & \bf default value & \bf description \\ \hline
    show &     Bool &      &     True &     set true, if item is shown in visualization and false if it is not shown\\ \hline
    drawSize &     float &      &     -1. &     drawing size = diameter of spring; size == -1.f means that default connector size is used\\ \hline
    color &     Float4 &      &     [-1.,-1.,-1.,-1.] &     RGB connector color; if R==-1, use default color\\ \hline
	  \end{longtable}
	\end{center}

%+++++++++++++++++++++++++++++++++++
\mysubsubsection{ObjectConnectorDistance}
Connector which enforces constant or prescribed distance between two bodies/nodes.
 \\
{\bf Short name} for Python: {\bf DistanceConstraint}
 \\  {\bf Requested marker type} = Marker::Position \\ 
The item ObjectConnectorDistance has the following parameters:
%reference manual TABLE
\begin{center}
  \footnotesize
  \begin{longtable}{| p{4.5cm} | p{2.5cm} | p{0.5cm} | p{2.5cm} | p{6cm} |}
    \hline
    \bf Name & \bf type & \bf size & \bf default value & \bf description \\ \hline
    \multicolumn{4}{l}{\parbox{10cm}{type = 'ConnectorDistance'}} & \multicolumn{1}{l}{\parbox{6cm}{\it item typename for initialization}}\\ \hline
    name &     String &      &     '' &     constraints"s unique name\\ \hline
    markerNumbers &     ArrayIndex &      &     [ MAXINT, MAXINT ] &     list of markers used in connector\\ \hline
    distance &     UReal &      &     0. &     prescribed distance [SI:m] of the used markers\\ \hline
    activeConnector &     bool &      &     True &     flag, which determines, if the connector is active; used to deactivate (temorarily) a connector or constraint\\ \hline
    visualization & VObjectConnectorDistance & & & parameters for visualization of item \\ \hline
	  \end{longtable}
	\end{center}
The item VObjectConnectorDistance has the following parameters:
%reference manual TABLE
\begin{center}
  \footnotesize
  \begin{longtable}{| p{4.5cm} | p{2.5cm} | p{0.5cm} | p{2.5cm} | p{6cm} |}
    \hline
    \bf Name & \bf type & \bf size & \bf default value & \bf description \\ \hline
    show &     bool &      &     True &     set true, if item is shown in visualization and false if it is not shown\\ \hline
    drawSize &     float &      &     -1. &     drawing size = link size; size == -1.f means that default connector size is used\\ \hline
    color &     Float4 &      &     [-1.,-1.,-1.,-1.] &     RGB connector color; if R==-1, use default color\\ \hline
	  \end{longtable}
	\end{center}

%+++++++++++++++++++++++++++++++++++
\mysubsubsection{ObjectConnectorCoordinate}
A coordinate constraint which constrains two (scalar) coordinates of Marker[Node|Body]Coordinates attached to nodes or bodies. The constraint must fulfill the condition: \be factorValue1*marker[1].value-marker[0].value - offset = 0 \ee
 \\
{\bf Short name} for Python: {\bf CoordinateConstraint}
 \\  {\bf Requested marker type} = Marker::Coordinate \\ 
The item ObjectConnectorCoordinate has the following parameters:
%reference manual TABLE
\begin{center}
  \footnotesize
  \begin{longtable}{| p{4.5cm} | p{2.5cm} | p{0.5cm} | p{2.5cm} | p{6cm} |}
    \hline
    \bf Name & \bf type & \bf size & \bf default value & \bf description \\ \hline
    \multicolumn{4}{l}{\parbox{10cm}{type = 'ConnectorCoordinate'}} & \multicolumn{1}{l}{\parbox{6cm}{\it item typename for initialization}}\\ \hline
    name &     String &      &     '' &     constraints"s unique name\\ \hline
    markerNumbers &     ArrayIndex &      &     [ MAXINT, MAXINT ] &     list of markers used in connector\\ \hline
    offset &     UReal &      &     0. &     An offset between the two values\\ \hline
    factorValue1 &     UReal &      &     1. &     An additional factor multiplied with value1 used in algebraic equation\\ \hline
    velocityLevel &     bool &      &     False &     If true: connector constrains velocities (only works for ODE2 coordinates!); offset is used between velocities; in this case, the offsetUserFunction\_t is considered and offsetUserFunction is ignored\\ \hline
    offsetUserFunction &     PyFunctionScalar2 &     \tabnewline  &     0 &     A python function which defines the time-dependent offset with parameters (t, offset); the offset represents the current value of the object; it is highly RECOMMENDED to use sufficiently smooth functions, having consistent initial offsets with initial configuration of bodies, zero or compatible initial offset-velocity, and no accelerations; Example for python function: def f(t, offset): return offset*(1-np.cos(t*10*2*np.pi))\\ \hline
    offsetUserFunction\_t &     PyFunctionScalar2 &     \tabnewline  &     0 &     time derivative of offsetUserFunction; needed for "velocityLevel=True", or for index2 time integration and for computation of initial accelerations in SecondOrderImplicit integrators\\ \hline
    activeConnector &     bool &      &     True &     flag, which determines, if the connector is active; used to deactivate (temorarily) a connector or constraint\\ \hline
    visualization & VObjectConnectorCoordinate & & & parameters for visualization of item \\ \hline
	  \end{longtable}
	\end{center}
The item VObjectConnectorCoordinate has the following parameters:
%reference manual TABLE
\begin{center}
  \footnotesize
  \begin{longtable}{| p{4.5cm} | p{2.5cm} | p{0.5cm} | p{2.5cm} | p{6cm} |}
    \hline
    \bf Name & \bf type & \bf size & \bf default value & \bf description \\ \hline
    show &     bool &      &     True &     set true, if item is shown in visualization and false if it is not shown\\ \hline
    drawSize &     float &      &     -1. &     drawing size = link size; size == -1.f means that default connector size is used\\ \hline
    color &     Float4 &      &     [-1.,-1.,-1.,-1.] &     RGB connector color; if R==-1, use default color\\ \hline
	  \end{longtable}
	\end{center}

%+++++++++++++++++++++++++++++++++++
\mysubsubsection{ObjectContactCoordinate}
A penalty-based contact condition for one coordinate; the contact gap $g$ is defined as $g=marker.value[1]- marker.value[0] - offset$; the contact force $f_c$ is zero for $gap>0$ and otherwise computed from $f_c = g*contactStiffness + \dot g*contactDamping$; during Newton iterations, the contact force is actived only, if $dataCoordinate[0] <= 0$; dataCoordinate is set equal to gap in nonlinear iterations, but not modified in Newton iterations.
 \\  {\bf Requested marker type} = Marker::Coordinate \\ 
The item ObjectContactCoordinate has the following parameters:
%reference manual TABLE
\begin{center}
  \footnotesize
  \begin{longtable}{| p{4.5cm} | p{2.5cm} | p{0.5cm} | p{2.5cm} | p{6cm} |}
    \hline
    \bf Name & \bf type & \bf size & \bf default value & \bf description \\ \hline
    \multicolumn{4}{l}{\parbox{10cm}{type = 'ContactCoordinate'}} & \multicolumn{1}{l}{\parbox{6cm}{\it item typename for initialization}}\\ \hline
    name &     String &      &     '' &     connector"s unique name\\ \hline
    markerNumbers &     ArrayIndex &      &     [ MAXINT, MAXINT ] &     markers define contact gap\\ \hline
    nodeNumber &     Index &      &     MAXINT &     node number of a NodeGenericData for 1 dataCoordinate (used for active set strategy ==> holds the gap of the last discontinuous iteration)\\ \hline
    contactStiffness &     UReal &      &     0. &     contact (penalty) stiffness [SI:N/m]; acts only upon penetration\\ \hline
    contactDamping &     UReal &      &     0. &     contact damping [SI:N/(m s)]; acts only upon penetration\\ \hline
    offset &     UReal &      &     0. &     offset [SI:m] of contact\\ \hline
    activeConnector &     bool &      &     True &     flag, which determines, if the connector is active; used to deactivate (temorarily) a connector or constraint\\ \hline
    visualization & VObjectContactCoordinate & & & parameters for visualization of item \\ \hline
	  \end{longtable}
	\end{center}
The item VObjectContactCoordinate has the following parameters:
%reference manual TABLE
\begin{center}
  \footnotesize
  \begin{longtable}{| p{4.5cm} | p{2.5cm} | p{0.5cm} | p{2.5cm} | p{6cm} |}
    \hline
    \bf Name & \bf type & \bf size & \bf default value & \bf description \\ \hline
    show &     Bool &      &     True &     set true, if item is shown in visualization and false if it is not shown\\ \hline
    drawSize &     float &      &     -1. &     drawing size = diameter of spring; size == -1.f means that default connector size is used\\ \hline
    color &     Float4 &      &     [-1.,-1.,-1.,-1.] &     RGB connector color; if R==-1, use default color\\ \hline
	  \end{longtable}
	\end{center}

%+++++++++++++++++++++++++++++++++++
\mysubsubsection{ObjectContactCircleCable2D}
A very specialized penalty-based contact condition between a 2D circle (=marker0, any Position-marker) on a body and an ANCFCable2DShape (=marker1, Marker: BodyCable2DShape), in xy-plane; a node NodeGenericData is required with the number of cordinates according to the number of contact segments; the contact gap $g$ is integrated (piecewise linear) along the cable and circle; the contact force $f_c$ is zero for $gap>0$ and otherwise computed from $f_c = g*contactStiffness + \dot g*contactDamping$; during Newton iterations, the contact force is actived only, if $dataCoordinate[0] <= 0$; dataCoordinate is set equal to gap in nonlinear iterations, but not modified in Newton iterations.
 \\  {\bf Requested marker type} = Marker::None \\ 
The item ObjectContactCircleCable2D has the following parameters:
%reference manual TABLE
\begin{center}
  \footnotesize
  \begin{longtable}{| p{4.5cm} | p{2.5cm} | p{0.5cm} | p{2.5cm} | p{6cm} |}
    \hline
    \bf Name & \bf type & \bf size & \bf default value & \bf description \\ \hline
    \multicolumn{4}{l}{\parbox{10cm}{type = 'ContactCircleCable2D'}} & \multicolumn{1}{l}{\parbox{6cm}{\it item typename for initialization}}\\ \hline
    name &     String &      &     '' &     connector"s unique name\\ \hline
    markerNumbers &     ArrayIndex &      &     [ MAXINT, MAXINT ] &     markers define contact gap\\ \hline
    nodeNumber &     Index &      &     MAXINT &     node number of a NodeGenericData for nSegments dataCoordinates (used for active set strategy ==> hold the gap of the last discontinuous iteration and the friction state)\\ \hline
    numberOfContactSegments &     Index &      &     3 &     number of linear contact segments to determine contact; each segment is a line and is associated to a data (history) variable; must be same as in according marker\\ \hline
    contactStiffness &     UReal &      &     0. &     contact (penalty) stiffness [SI:N/m/(contact segment)]; the stiffness is per length of the beam axis; specific contact forces (per length) $f_N$ act in contact normal direction only upon penetration\\ \hline
    contactDamping &     UReal &      &     0. &     contact damping [SI:N/(m s)/(contact segment)]; the damping is per length of the beam axis; acts in contact normal direction only upon penetration\\ \hline
    circleRadius &     UReal &      &     0. &     radius [SI:m] of contact circle\\ \hline
    offset &     UReal &      &     0. &     offset [SI:m] of contact, e.g. to include thickness of cable element\\ \hline
    activeConnector &     bool &      &     True &     flag, which determines, if the connector is active; used to deactivate (temorarily) a connector or constraint\\ \hline
    visualization & VObjectContactCircleCable2D & & & parameters for visualization of item \\ \hline
	  \end{longtable}
	\end{center}
The item VObjectContactCircleCable2D has the following parameters:
%reference manual TABLE
\begin{center}
  \footnotesize
  \begin{longtable}{| p{4.5cm} | p{2.5cm} | p{0.5cm} | p{2.5cm} | p{6cm} |}
    \hline
    \bf Name & \bf type & \bf size & \bf default value & \bf description \\ \hline
    show &     Bool &      &     True &     set true, if item is shown in visualization and false if it is not shown\\ \hline
    drawSize &     float &      &     -1. &     drawing size = diameter of spring; size == -1.f means that default connector size is used\\ \hline
    color &     Float4 &      &     [-1.,-1.,-1.,-1.] &     RGB connector color; if R==-1, use default color\\ \hline
	  \end{longtable}
	\end{center}

%+++++++++++++++++++++++++++++++++++
\mysubsubsection{ObjectContactFrictionCircleCable2D}
A very specialized penalty-based contact/friction condition between a 2D circle (=marker0, any Position-marker) on a body and an ANCFCable2DShape (=marker1, Marker: BodyCable2DShape), in xy-plane; a node NodeGenericData is required with 3$\times$(number of contact segments) -- containing per segment: [contact gap, stick/slip (stick=1), last friction position]; the contact gap $g$ is integrated (piecewise linear) along the cable and circle; the contact force $f_c$ is zero for $gap>0$ and otherwise computed from $f_c = g*contactStiffness + \dot g*contactDamping$; during Newton iterations, the contact force is actived only, if $dataCoordinate[0] <= 0$; dataCoordinate is set equal to gap in nonlinear iterations, but not modified in Newton iterations.
 \\  {\bf Requested marker type} = Marker::None \\ 
The item ObjectContactFrictionCircleCable2D has the following parameters:
%reference manual TABLE
\begin{center}
  \footnotesize
  \begin{longtable}{| p{4.5cm} | p{2.5cm} | p{0.5cm} | p{2.5cm} | p{6cm} |}
    \hline
    \bf Name & \bf type & \bf size & \bf default value & \bf description \\ \hline
    \multicolumn{4}{l}{\parbox{10cm}{type = 'ContactFrictionCircleCable2D'}} & \multicolumn{1}{l}{\parbox{6cm}{\it item typename for initialization}}\\ \hline
    name &     String &      &     '' &     connector"s unique name\\ \hline
    markerNumbers &     ArrayIndex &      &     [ MAXINT, MAXINT ] &     markers define contact gap\\ \hline
    nodeNumber &     Index &      &     MAXINT &     node number of a NodeGenericData with 3 $\times$ nSegments dataCoordinates (used for active set strategy ==> hold the gap of the last discontinuous iteration and the friction state)\\ \hline
    numberOfContactSegments &     Index &      &     3 &     number of linear contact segments to determine contact; each segment is a line and is associated to a data (history) variable; must be same as in according marker\\ \hline
    contactStiffness &     UReal &      &     0. &     contact (penalty) stiffness [SI:N/m/(contact segment)]; the stiffness is per length of the beam axis; specific contact forces (per length) $f_N$ act in contact normal direction only upon penetration\\ \hline
    contactDamping &     UReal &      &     0. &     contact damping [SI:N/(m s)/(contact segment)]; the damping is per length of the beam axis; acts in contact normal direction only upon penetration\\ \hline
    frictionVelocityPenalty &     UReal &      &     0. &     velocity dependent penalty coefficient for friction [SI:N/(m s)/(contact segment)]; the coefficient causes tangential (contact) forces against relative tangential velocities in the contact area\\ \hline
    frictionStiffness &     UReal &      &     0. &     CURRENTLY NOT IMPLEMENTED: displacement dependent penalty/stiffness coefficient for friction [SI:N/m/(contact segment)]; the coefficient causes tangential (contact) forces against relative tangential displacements in the contact area\\ \hline
    frictionCoefficient &     UReal &      &     0. &     friction coefficient $\mu$ [SI: 1]; tangential specific friction forces (per length) $f_T$ must fulfill the condition $f_T \le \mu f_N$\\ \hline
    circleRadius &     UReal &      &     0. &     radius [SI:m] of contact circle\\ \hline
    offset &     UReal &      &     0. &     offset [SI:m] of contact, e.g. to include thickness of cable element\\ \hline
    activeConnector &     bool &      &     True &     flag, which determines, if the connector is active; used to deactivate (temorarily) a connector or constraint\\ \hline
    visualization & VObjectContactFrictionCircleCable2D & & & parameters for visualization of item \\ \hline
	  \end{longtable}
	\end{center}
The item VObjectContactFrictionCircleCable2D has the following parameters:
%reference manual TABLE
\begin{center}
  \footnotesize
  \begin{longtable}{| p{4.5cm} | p{2.5cm} | p{0.5cm} | p{2.5cm} | p{6cm} |}
    \hline
    \bf Name & \bf type & \bf size & \bf default value & \bf description \\ \hline
    show &     Bool &      &     True &     set true, if item is shown in visualization and false if it is not shown\\ \hline
    drawSize &     float &      &     -1. &     drawing size = diameter of spring; size == -1.f means that default connector size is used\\ \hline
    color &     Float4 &      &     [-1.,-1.,-1.,-1.] &     RGB connector color; if R==-1, use default color\\ \hline
	  \end{longtable}
	\end{center}

%+++++++++++++++++++++++++++++++++++
\mysubsubsection{ObjectJointSliding2D}
A specialized sliding joint (without rotation) in 2D between a Cable2D (marker1) and a position-based marker (marker0); the data coordinates provide [0] the current index in slidingMarkerNumbers, and [1] the local position in the cable element at the beginning of the timestep; the algebraic variables are \be \qv_{AE}=[\lambda_x\;\; \lambda_y \;\; s]^T \ee in which $\lambda_x$ and $\lambda_y$ are the Lagrange multipliers for the position of the sliding joint and $s$ is the (algebraic) sliding coordinate relative to the value at the beginning at the solution step; the data coordinates are \be \qv_{Data} = [i_{marker} \;\; s_{0}]^T \ee in which $i_{marker}$ is the current local index to the slidingMarkerNumber list and  $s_{0}$ is the sliding coordinate (which is the total sliding length along all cable elements in the cableMarkerNumber list) at the beginning of the solution step.
 \\
{\bf Short name} for Python: {\bf SlidingJoint2D}
 \\\\ 
{\bf Output variables} (chose type, e.g., OutputVariableType.Position): 
\begin{itemize}
    \item {\bf Position}: position vector of joint given by marker0
    \item {\bf Velocity}: velocity vector of joint given by marker0
    \item {\bf SlidingCoordinate}: global sliding coordinate along all elements; the maximum sliding coordinate is equivalent to the reference lengths of all sliding elements
    \item {\bf Force}: joint force vector (3D)
\end{itemize}
  {\bf Requested marker type} = Marker::None \\ 
The item ObjectJointSliding2D has the following parameters:
%reference manual TABLE
\begin{center}
  \footnotesize
  \begin{longtable}{| p{4.5cm} | p{2.5cm} | p{0.5cm} | p{2.5cm} | p{6cm} |}
    \hline
    \bf Name & \bf type & \bf size & \bf default value & \bf description \\ \hline
    \multicolumn{4}{l}{\parbox{10cm}{type = 'JointSliding2D'}} & \multicolumn{1}{l}{\parbox{6cm}{\it item typename for initialization}}\\ \hline
    name &     String &      &     '' &     constraints"s unique name\\ \hline
    markerNumbers &     ArrayIndex &      &     [ MAXINT, MAXINT ] &     marker0: position-marker of mass point or rigid body; marker1: updated marker to Cable2D element, where the sliding joint currently is attached to; must be initialized with an appropriate (global) marker number according to the starting position of the sliding object; this marker changes with time (PostNewtonStep)\\ \hline
    slidingMarkerNumbers &     ArrayIndex &      &     [] &     these markers are used to update marker1, if the sliding position exceeds the current cable"s range; the markers must be sorted such that marker(i) at x=cable.length is equal to marker(i+1) at x=0\\ \hline
    slidingMarkerOffsets &     Vector &      &     [] &     this list contains the offsets of every sliding object (given by slidingMarkerNumbers) w.r.t. to the initial position (0): marker0: offset=0, marker1: offset=Length(cable0), marker2: offset=Length(cable0)+Length(cable1), ...\\ \hline
    nodeNumber &     Index &      &     MAXINT &     node number of a NodeGenericData for 1 dataCoordinate showing the according marker number which is currently active and the initial (global) sliding position\\ \hline
    activeConnector &     bool &      &     True &     flag, which determines, if the connector is active; used to deactivate (temorarily) a connector or constraint\\ \hline
    visualization & VObjectJointSliding2D & & & parameters for visualization of item \\ \hline
	  \end{longtable}
	\end{center}
The item VObjectJointSliding2D has the following parameters:
%reference manual TABLE
\begin{center}
  \footnotesize
  \begin{longtable}{| p{4.5cm} | p{2.5cm} | p{0.5cm} | p{2.5cm} | p{6cm} |}
    \hline
    \bf Name & \bf type & \bf size & \bf default value & \bf description \\ \hline
    show &     bool &      &     True &     set true, if item is shown in visualization and false if it is not shown\\ \hline
    drawSize &     float &      &     -1. &     drawing size = radius of revolute joint; size == -1.f means that default connector size is used\\ \hline
    color &     Float4 &      &     [-1.,-1.,-1.,-1.] &     RGB connector color; if R==-1, use default color\\ \hline
	  \end{longtable}
	\end{center}

%+++++++++++++++++++++++++++++++++++
\mysubsubsection{ObjectJointALEMoving2D}
A specialized axially moving joint (without rotation) in 2D between a ALE Cable2D (marker1) and a position-based marker (marker0); the data coordinate [0] provides the current index in slidingMarkerNumbers, and the ODE2 coordinate [0] provides the (given) moving coordinate in the cable element; the algebraic variables are \be \qv_{AE}=[\lambda_x\;\; \lambda_y]^T \ee, in which $\lambda_x$ and $\lambda_y$ are the Lagrange multipliers for the position constraint of the moving joint; the data coordinate is \be \qv_{Data} = [i_{marker}]^T \ee in which $i_{marker}$ is the current local index to the slidingMarkerNumber list.
 \\
{\bf Short name} for Python: {\bf ALEMovingJoint2D}
 \\\\ 
{\bf Output variables} (chose type, e.g., OutputVariableType.Position): 
\begin{itemize}
    \item {\bf Position}: position vector of joint given by marker0
    \item {\bf Velocity}: velocity vector of joint given by marker0
    \item {\bf SlidingCoordinate}: global sliding coordinate along all elements + slidingOffset
    \item {\bf Coordinates}: provides two values: [0] = current sliding marker index, [1] = ALE sliding coordinate
    \item {\bf Coordinates\_t}: provides one value: [0] ALE sliding velocity
    \item {\bf Force}: joint force vector (3D)
\end{itemize}
  {\bf Requested marker type} = Marker::None \\ 
The item ObjectJointALEMoving2D has the following parameters:
%reference manual TABLE
\begin{center}
  \footnotesize
  \begin{longtable}{| p{4.5cm} | p{2.5cm} | p{0.5cm} | p{2.5cm} | p{6cm} |}
    \hline
    \bf Name & \bf type & \bf size & \bf default value & \bf description \\ \hline
    \multicolumn{4}{l}{\parbox{10cm}{type = 'JointALEMoving2D'}} & \multicolumn{1}{l}{\parbox{6cm}{\it item typename for initialization}}\\ \hline
    name &     String &      &     '' &     constraints"s unique name\\ \hline
    markerNumbers &     ArrayIndex &      &     [ MAXINT, MAXINT ] &     marker0: position-marker of mass point or rigid body; marker1: updated marker to Cable2D element, where the sliding joint currently is attached to; must be initialized with an appropriate (global) marker number according to the starting position of the sliding object; this marker changes with time (PostNewtonStep)\\ \hline
    slidingMarkerNumbers &     ArrayIndex &      &     [] &     these markers are used to update marker1, if the sliding position exceeds the current cable"s range; the markers must be sorted such that marker(i) at x=cable.length is equal to marker(i+1) at x=0\\ \hline
    slidingMarkerOffsets &     Vector &      &     [] &     this list contains the offsets of every sliding object (given by slidingMarkerNumbers) w.r.t. to the initial position (0): marker0: offset=0, marker1: offset=Length(cable0), marker2: offset=Length(cable0)+Length(cable1), ...\\ \hline
    slidingOffset &     Real &      &     0. &     offset [SI:m] used set the sliding position relative to the chosen Eulerian (NodeGenericODE2) coordinate; the following relation is used: $slidingPosition = posALE + slidingOffset$\\ \hline
    nodeNumbers &     ArrayIndex &      &     [ MAXINT, MAXINT ] &     node numbers of: [0] NodeGenericData for 1 dataCoordinate showing the according marker number which is currently active; [1] of the GenericNodeODE2 of the ALE sliding coordinate\\ \hline
    activeConnector &     bool &      &     True &     flag, which determines, if the connector is active; used to deactivate (temorarily) a connector or constraint\\ \hline
    visualization & VObjectJointALEMoving2D & & & parameters for visualization of item \\ \hline
	  \end{longtable}
	\end{center}
The item VObjectJointALEMoving2D has the following parameters:
%reference manual TABLE
\begin{center}
  \footnotesize
  \begin{longtable}{| p{4.5cm} | p{2.5cm} | p{0.5cm} | p{2.5cm} | p{6cm} |}
    \hline
    \bf Name & \bf type & \bf size & \bf default value & \bf description \\ \hline
    show &     bool &      &     True &     set true, if item is shown in visualization and false if it is not shown\\ \hline
    drawSize &     float &      &     -1. &     drawing size = radius of revolute joint; size == -1.f means that default connector size is used\\ \hline
    color &     Float4 &      &     [-1.,-1.,-1.,-1.] &     RGB connector color; if R==-1, use default color\\ \hline
	  \end{longtable}
	\end{center}

%+++++++++++++++++++++++++++++++++++
\mysubsubsection{ObjectJointGeneric}
A generic joint in 3D; constrains components of the absolute position and rotations of two points given by PointMarkers or RigidMarkers; an additional local rotation can be used to define three rotation axes and/or sliding axes
 \\
{\bf Short name} for Python: {\bf GenericJoint}
 \\  {\bf Requested marker type} = (Marker::Type)((Index)Marker::Position + (Index)Marker::Orientation) \\ 
The item ObjectJointGeneric has the following parameters:
%reference manual TABLE
\begin{center}
  \footnotesize
  \begin{longtable}{| p{4.5cm} | p{2.5cm} | p{0.5cm} | p{2.5cm} | p{6cm} |}
    \hline
    \bf Name & \bf type & \bf size & \bf default value & \bf description \\ \hline
    \multicolumn{4}{l}{\parbox{10cm}{type = 'JointGeneric'}} & \multicolumn{1}{l}{\parbox{6cm}{\it item typename for initialization}}\\ \hline
    name &     String &      &     '' &     constraints"s unique name\\ \hline
    markerNumbers &     ArrayIndex &     2 &     [ MAXINT, MAXINT ] &     list of markers used in connector\\ \hline
    constrainedAxes &     ArrayIndex &     6 &     [1,1,1,1,1,1] &     flag, which determines which translation (0,1,2) and rotation (3,4,5) axes are constrained; 0=free, 1=constrained\\ \hline
    rotationMarker0 &     Matrix3D &      &     EXUmath::unitMatrix3D &     local rotation matrix for marker 0; translation and rotation axes for marker0 are defined in the local body coordinate system and additionally transformed by rotationMarker0\\ \hline
    rotationMarker1 &     Matrix3D &      &     EXUmath::unitMatrix3D &     local rotation matrix for marker 1; translation and rotation axes for marker1 are defined in the local body coordinate system and additionally transformed by rotationMarker1\\ \hline
    activeConnector &     bool &      &     True &     flag, which determines, if the connector is active; used to deactivate (temorarily) a connector or constraint\\ \hline
    forceTorqueUserFunctionParameters &     Vector6D &      &     [0.,0.,0.,0.,0.,0.] &     vector of 6 parameters for joint"s forceTorqueUserFunction\\ \hline
    offsetUserFunctionParameters &     Vector6D &      &     [0.,0.,0.,0.,0.,0.] &     vector of 6 parameters for joint"s offsetUserFunction\\ \hline
    forceTorqueUserFunction &     PyFunctionVector6DScalarVector6D &     \tabnewline  &     \tabnewline 0 &     A python function which defines the time-dependent force (indices 0,1,2) and torque (indices 3,4,5) joint coordinates with parameters (t, forceTorqueUserFunctionParameters); the offset represents the current value of the object; it is highly RECOMMENDED to use sufficiently smooth functions, having consistent initial offsets with initial configuration of bodies, zero or compatible initial offset-velocity, and no accelerations; Example for python function: def f(t, forceTorqueUserFunctionParameters): return [forceTorqueUserFunctionParameters[0]*(1 - np.cos(t*10*2*np.pi)), 0,0,0,0,0]\\ \hline
    offsetUserFunction &     PyFunctionVector6DScalarVector6D &     \tabnewline  &     \tabnewline 0 &     A python function which defines the time-dependent (fixed) offset of translation (indices 0,1,2) and rotation (indices 3,4,5) joint coordinates with parameters (t, offsetUserFunctionParameters); the offset represents the current value of the object; it is highly RECOMMENDED to use sufficiently smooth functions, having consistent initial offsets with initial configuration of bodies, zero or compatible initial offset-velocity, and no accelerations; Example for python function: def f(t, offsetUserFunctionParameters): return [offsetUserFunctionParameters[0]*(1 - np.cos(t*10*2*np.pi)), 0,0,0,0,0]\\ \hline
    offsetUserFunction\_t &     PyFunctionVector6DScalarVector6D &     \tabnewline  &     \tabnewline 0 &     time derivative of offsetUserFunction using the same parameters; needed for "velocityLevel=True", or for index2 time integration and for computation of initial accelerations in SecondOrderImplicit integrators\\ \hline
    visualization & VObjectJointGeneric & & & parameters for visualization of item \\ \hline
	  \end{longtable}
	\end{center}
The item VObjectJointGeneric has the following parameters:
%reference manual TABLE
\begin{center}
  \footnotesize
  \begin{longtable}{| p{4.5cm} | p{2.5cm} | p{0.5cm} | p{2.5cm} | p{6cm} |}
    \hline
    \bf Name & \bf type & \bf size & \bf default value & \bf description \\ \hline
    show &     bool &      &     True &     set true, if item is shown in visualization and false if it is not shown\\ \hline
    axesRadius &     float &      &     0.1 &     radius of joint axes to draw\\ \hline
    axesLength &     float &      &     0.4 &     length of joint axes to draw\\ \hline
    color &     Float4 &      &     [-1.,-1.,-1.,-1.] &     RGB connector color; if R==-1, use default color\\ \hline
	  \end{longtable}
	\end{center}

%+++++++++++++++++++++++++++++++++++
\mysubsubsection{ObjectJointRevolute2D}
A revolute joint in 2D; constrains the absolute 2D position of two points given by PointMarkers or RigidMarkers
 \\
{\bf Short name} for Python: {\bf RevoluteJoint2D}
 \\  {\bf Requested marker type} = Marker::Position \\ 
The item ObjectJointRevolute2D has the following parameters:
%reference manual TABLE
\begin{center}
  \footnotesize
  \begin{longtable}{| p{4.5cm} | p{2.5cm} | p{0.5cm} | p{2.5cm} | p{6cm} |}
    \hline
    \bf Name & \bf type & \bf size & \bf default value & \bf description \\ \hline
    \multicolumn{4}{l}{\parbox{10cm}{type = 'JointRevolute2D'}} & \multicolumn{1}{l}{\parbox{6cm}{\it item typename for initialization}}\\ \hline
    name &     String &      &     '' &     constraints"s unique name\\ \hline
    markerNumbers &     ArrayIndex &      &     [ MAXINT, MAXINT ] &     list of markers used in connector\\ \hline
    activeConnector &     bool &      &     True &     flag, which determines, if the connector is active; used to deactivate (temorarily) a connector or constraint\\ \hline
    visualization & VObjectJointRevolute2D & & & parameters for visualization of item \\ \hline
	  \end{longtable}
	\end{center}
The item VObjectJointRevolute2D has the following parameters:
%reference manual TABLE
\begin{center}
  \footnotesize
  \begin{longtable}{| p{4.5cm} | p{2.5cm} | p{0.5cm} | p{2.5cm} | p{6cm} |}
    \hline
    \bf Name & \bf type & \bf size & \bf default value & \bf description \\ \hline
    show &     bool &      &     True &     set true, if item is shown in visualization and false if it is not shown\\ \hline
    drawSize &     float &      &     -1. &     drawing size = radius of revolute joint; size == -1.f means that default connector size is used\\ \hline
    color &     Float4 &      &     [-1.,-1.,-1.,-1.] &     RGB connector color; if R==-1, use default color\\ \hline
	  \end{longtable}
	\end{center}

%+++++++++++++++++++++++++++++++++++
\mysubsubsection{ObjectJointPrismatic2D}
A prismatic joint in 2D; allows the relative motion of two bodies, using two RigidMarkers; the vector $\tv_0$ = axisMarker0 is given in local coordinates of the first marker's (body) frame and defines the prismatic axis; the vector $\mathbf{n}_1$ = normalMarker1 is given in the second marker's (body) frame and is the normal vector to the prismatic axis; using the global position vector $\pv_0$ and rotation matrix $\Am_0$ of marker0 and the global position vector $\pv_1$ rotation matrix $\Am_1$ of marker1, the equations for the prismatic joint follow as \be (\pv_1-\pv_0)^T\cdot \Am_1 \cdot \mathbf{n}_1 = 0 \ee  \be (\Am_0 \cdot \tv_0)^T \cdot \Am_1 \cdot \mathbf{n}_1 = 0\ee The lagrange multipliers follow for these two equations $[\lambda_0,\lambda_1]$, in which $\lambda_0$ is the transverse force and $\lambda_1$ is the torque in the joint.
 \\
{\bf Short name} for Python: {\bf PrismaticJoint2D}
 \\  {\bf Requested marker type} = (Marker::Type)(Marker::Position + Marker::Orientation) \\ 
The item ObjectJointPrismatic2D has the following parameters:
%reference manual TABLE
\begin{center}
  \footnotesize
  \begin{longtable}{| p{4.5cm} | p{2.5cm} | p{0.5cm} | p{2.5cm} | p{6cm} |}
    \hline
    \bf Name & \bf type & \bf size & \bf default value & \bf description \\ \hline
    \multicolumn{4}{l}{\parbox{10cm}{type = 'JointPrismatic2D'}} & \multicolumn{1}{l}{\parbox{6cm}{\it item typename for initialization}}\\ \hline
    name &     String &      &     '' &     constraints"s unique name\\ \hline
    markerNumbers &     ArrayIndex &      &     [ MAXINT, MAXINT ] &     list of markers used in connector\\ \hline
    axisMarker0 &     Vector3D &      &     [1.,0.,0.] &     direction of prismatic axis, given as a 3D vector in Marker0 frame\\ \hline
    normalMarker1 &     Vector3D &      &     [0.,1.,0.] &     direction of normal to prismatic axis, given as a 3D vector in Marker1 frame\\ \hline
    constrainRotation &     bool &      &     True &     flag, which determines, if the connector also constrains the relative rotation of the two objects; if set to false, the constraint will keep an algebraic equation set equal zero\\ \hline
    activeConnector &     bool &      &     True &     flag, which determines, if the connector is active; used to deactivate (temorarily) a connector or constraint\\ \hline
    visualization & VObjectJointPrismatic2D & & & parameters for visualization of item \\ \hline
	  \end{longtable}
	\end{center}
The item VObjectJointPrismatic2D has the following parameters:
%reference manual TABLE
\begin{center}
  \footnotesize
  \begin{longtable}{| p{4.5cm} | p{2.5cm} | p{0.5cm} | p{2.5cm} | p{6cm} |}
    \hline
    \bf Name & \bf type & \bf size & \bf default value & \bf description \\ \hline
    show &     bool &      &     True &     set true, if item is shown in visualization and false if it is not shown\\ \hline
    drawSize &     float &      &     -1. &     drawing size = radius of revolute joint; size == -1.f means that default connector size is used\\ \hline
    color &     Float4 &      &     [-1.,-1.,-1.,-1.] &     RGB connector color; if R==-1, use default color\\ \hline
	  \end{longtable}
	\end{center}

%+++++++++++++++++++++++++++++++
%+++++++++++++++++++++++++++++++
\mysubsection{Markers}

%+++++++++++++++++++++++++++++++++++
\mysubsubsection{MarkerBodyMass}
A marker attached to the body mass; use this marker to apply a body-load (e.g. gravitational force).
 \\The item MarkerBodyMass has the following parameters:
%reference manual TABLE
\begin{center}
  \footnotesize
  \begin{longtable}{| p{4.5cm} | p{2.5cm} | p{0.5cm} | p{2.5cm} | p{6cm} |}
    \hline
    \bf Name & \bf type & \bf size & \bf default value & \bf description \\ \hline
    \multicolumn{4}{l}{\parbox{10cm}{type = 'BodyMass'}} & \multicolumn{1}{l}{\parbox{6cm}{\it item typename for initialization}}\\ \hline
    name &     String &      &     '' &     marker"s unique name\\ \hline
    bodyNumber &     Index &      &     MAXINT &     body number to which marker is attached to\\ \hline
    visualization & VMarkerBodyMass & & & parameters for visualization of item \\ \hline
	  \end{longtable}
	\end{center}
The item VMarkerBodyMass has the following parameters:
%reference manual TABLE
\begin{center}
  \footnotesize
  \begin{longtable}{| p{4.5cm} | p{2.5cm} | p{0.5cm} | p{2.5cm} | p{6cm} |}
    \hline
    \bf Name & \bf type & \bf size & \bf default value & \bf description \\ \hline
    show &     bool &      &     True &     set true, if item is shown in visualization and false if it is not shown\\ \hline
	  \end{longtable}
	\end{center}

%+++++++++++++++++++++++++++++++++++
\mysubsubsection{MarkerBodyPosition}
A position body-marker attached to local position (x,y,z) of the body.
 \\The item MarkerBodyPosition has the following parameters:
%reference manual TABLE
\begin{center}
  \footnotesize
  \begin{longtable}{| p{4.5cm} | p{2.5cm} | p{0.5cm} | p{2.5cm} | p{6cm} |}
    \hline
    \bf Name & \bf type & \bf size & \bf default value & \bf description \\ \hline
    \multicolumn{4}{l}{\parbox{10cm}{type = 'BodyPosition'}} & \multicolumn{1}{l}{\parbox{6cm}{\it item typename for initialization}}\\ \hline
    name &     String &      &     '' &     marker"s unique name\\ \hline
    bodyNumber &     Index &      &     MAXINT &     body number to which marker is attached to\\ \hline
    localPosition &     Vector3D &     3 &     [0.,0.,0.] &     local body position of marker; e.g. local (body-fixed) position where force is applied to\\ \hline
    bodyFixed &     Bool &      &     False &     if bodyFixed is true, the force/sensor is using body-fixed coordinates (orientation); otherwise, it uses global coordinates\\ \hline
    visualization & VMarkerBodyPosition & & & parameters for visualization of item \\ \hline
	  \end{longtable}
	\end{center}
The item VMarkerBodyPosition has the following parameters:
%reference manual TABLE
\begin{center}
  \footnotesize
  \begin{longtable}{| p{4.5cm} | p{2.5cm} | p{0.5cm} | p{2.5cm} | p{6cm} |}
    \hline
    \bf Name & \bf type & \bf size & \bf default value & \bf description \\ \hline
    show &     bool &      &     True &     set true, if item is shown in visualization and false if it is not shown\\ \hline
	  \end{longtable}
	\end{center}

%+++++++++++++++++++++++++++++++++++
\mysubsubsection{MarkerBodyRigid}
A rigid-body (position+orientation) body-marker attached to local position (x,y,z) of the body.
 \\The item MarkerBodyRigid has the following parameters:
%reference manual TABLE
\begin{center}
  \footnotesize
  \begin{longtable}{| p{4.5cm} | p{2.5cm} | p{0.5cm} | p{2.5cm} | p{6cm} |}
    \hline
    \bf Name & \bf type & \bf size & \bf default value & \bf description \\ \hline
    \multicolumn{4}{l}{\parbox{10cm}{type = 'BodyRigid'}} & \multicolumn{1}{l}{\parbox{6cm}{\it item typename for initialization}}\\ \hline
    name &     String &      &     '' &     marker"s unique name\\ \hline
    bodyNumber &     Index &      &     MAXINT &     body number to which marker is attached to\\ \hline
    localPosition &     Vector3D &     3 &     [0.,0.,0.] &     local body position of marker; e.g. local (body-fixed) position where force is applied to\\ \hline
    bodyFixed &     Bool &      &     False &     if bodyFixed is true, the force/sensor is using body-fixed coordinates (orientation); otherwise, it uses global coordinates\\ \hline
    visualization & VMarkerBodyRigid & & & parameters for visualization of item \\ \hline
	  \end{longtable}
	\end{center}
The item VMarkerBodyRigid has the following parameters:
%reference manual TABLE
\begin{center}
  \footnotesize
  \begin{longtable}{| p{4.5cm} | p{2.5cm} | p{0.5cm} | p{2.5cm} | p{6cm} |}
    \hline
    \bf Name & \bf type & \bf size & \bf default value & \bf description \\ \hline
    show &     bool &      &     True &     set true, if item is shown in visualization and false if it is not shown\\ \hline
	  \end{longtable}
	\end{center}

%+++++++++++++++++++++++++++++++++++
\mysubsubsection{MarkerNodePosition}
A node-Marker attached to a position-based node.
 \\The item MarkerNodePosition has the following parameters:
%reference manual TABLE
\begin{center}
  \footnotesize
  \begin{longtable}{| p{4.5cm} | p{2.5cm} | p{0.5cm} | p{2.5cm} | p{6cm} |}
    \hline
    \bf Name & \bf type & \bf size & \bf default value & \bf description \\ \hline
    \multicolumn{4}{l}{\parbox{10cm}{type = 'NodePosition'}} & \multicolumn{1}{l}{\parbox{6cm}{\it item typename for initialization}}\\ \hline
    name &     String &      &     '' &     marker"s unique name\\ \hline
    nodeNumber &     Index &      &     MAXINT &     node number to which marker is attached to\\ \hline
    visualization & VMarkerNodePosition & & & parameters for visualization of item \\ \hline
	  \end{longtable}
	\end{center}
The item VMarkerNodePosition has the following parameters:
%reference manual TABLE
\begin{center}
  \footnotesize
  \begin{longtable}{| p{4.5cm} | p{2.5cm} | p{0.5cm} | p{2.5cm} | p{6cm} |}
    \hline
    \bf Name & \bf type & \bf size & \bf default value & \bf description \\ \hline
    show &     bool &      &     True &     set true, if item is shown in visualization and false if it is not shown\\ \hline
	  \end{longtable}
	\end{center}

%+++++++++++++++++++++++++++++++++++
\mysubsubsection{MarkerNodeRigid}
A rigid-body (position+orientation) node-marker attached to a rigid-body node.
 \\The item MarkerNodeRigid has the following parameters:
%reference manual TABLE
\begin{center}
  \footnotesize
  \begin{longtable}{| p{4.5cm} | p{2.5cm} | p{0.5cm} | p{2.5cm} | p{6cm} |}
    \hline
    \bf Name & \bf type & \bf size & \bf default value & \bf description \\ \hline
    \multicolumn{4}{l}{\parbox{10cm}{type = 'NodeRigid'}} & \multicolumn{1}{l}{\parbox{6cm}{\it item typename for initialization}}\\ \hline
    name &     String &      &     '' &     marker"s unique name\\ \hline
    nodeNumber &     Index &      &     MAXINT &     node number to which marker is attached to\\ \hline
    visualization & VMarkerNodeRigid & & & parameters for visualization of item \\ \hline
	  \end{longtable}
	\end{center}
The item VMarkerNodeRigid has the following parameters:
%reference manual TABLE
\begin{center}
  \footnotesize
  \begin{longtable}{| p{4.5cm} | p{2.5cm} | p{0.5cm} | p{2.5cm} | p{6cm} |}
    \hline
    \bf Name & \bf type & \bf size & \bf default value & \bf description \\ \hline
    show &     bool &      &     True &     set true, if item is shown in visualization and false if it is not shown\\ \hline
	  \end{longtable}
	\end{center}

%+++++++++++++++++++++++++++++++++++
\mysubsubsection{MarkerNodeCoordinate}
A node-Marker attached to a ODE2 coordinate of a node; for other coordinates (ODE1,...) other markers need to be defined.
 \\The item MarkerNodeCoordinate has the following parameters:
%reference manual TABLE
\begin{center}
  \footnotesize
  \begin{longtable}{| p{4.5cm} | p{2.5cm} | p{0.5cm} | p{2.5cm} | p{6cm} |}
    \hline
    \bf Name & \bf type & \bf size & \bf default value & \bf description \\ \hline
    \multicolumn{4}{l}{\parbox{10cm}{type = 'NodeCoordinate'}} & \multicolumn{1}{l}{\parbox{6cm}{\it item typename for initialization}}\\ \hline
    name &     String &      &     '' &     marker"s unique name\\ \hline
    nodeNumber &     Index &      &     MAXINT &     node number to which marker is attached to\\ \hline
    coordinate &     Index &      &     MAXINT &     coordinate of node to which marker is attached to\\ \hline
    visualization & VMarkerNodeCoordinate & & & parameters for visualization of item \\ \hline
	  \end{longtable}
	\end{center}
The item VMarkerNodeCoordinate has the following parameters:
%reference manual TABLE
\begin{center}
  \footnotesize
  \begin{longtable}{| p{4.5cm} | p{2.5cm} | p{0.5cm} | p{2.5cm} | p{6cm} |}
    \hline
    \bf Name & \bf type & \bf size & \bf default value & \bf description \\ \hline
    show &     bool &      &     True &     set true, if item is shown in visualization and false if it is not shown\\ \hline
	  \end{longtable}
	\end{center}

%+++++++++++++++++++++++++++++++++++
\mysubsubsection{MarkerBodyCable2DShape}
A special Marker attached to a 2D ANCF beam finite element with cubic interpolation and 8 coordinates.
 \\The item MarkerBodyCable2DShape has the following parameters:
%reference manual TABLE
\begin{center}
  \footnotesize
  \begin{longtable}{| p{4.5cm} | p{2.5cm} | p{0.5cm} | p{2.5cm} | p{6cm} |}
    \hline
    \bf Name & \bf type & \bf size & \bf default value & \bf description \\ \hline
    \multicolumn{4}{l}{\parbox{10cm}{type = 'BodyCable2DShape'}} & \multicolumn{1}{l}{\parbox{6cm}{\it item typename for initialization}}\\ \hline
    name &     String &      &     '' &     marker"s unique name\\ \hline
    bodyNumber &     Index &      &     MAXINT &     body number to which marker is attached to\\ \hline
    numberOfSegments &     Index &      &     3 &     number of number of segments; each segment is a line and is associated to a data (history) variable; must be same as in according contact element\\ \hline
    visualization & VMarkerBodyCable2DShape & & & parameters for visualization of item \\ \hline
	  \end{longtable}
	\end{center}
The item VMarkerBodyCable2DShape has the following parameters:
%reference manual TABLE
\begin{center}
  \footnotesize
  \begin{longtable}{| p{4.5cm} | p{2.5cm} | p{0.5cm} | p{2.5cm} | p{6cm} |}
    \hline
    \bf Name & \bf type & \bf size & \bf default value & \bf description \\ \hline
    show &     bool &      &     True &     set true, if item is shown in visualization and false if it is not shown\\ \hline
	  \end{longtable}
	\end{center}

%+++++++++++++++++++++++++++++++++++
\mysubsubsection{MarkerBodyCable2DCoordinates}
A special Marker attached to the coordinates of a 2D ANCF beam finite element with cubic interpolation.
 \\The item MarkerBodyCable2DCoordinates has the following parameters:
%reference manual TABLE
\begin{center}
  \footnotesize
  \begin{longtable}{| p{4.5cm} | p{2.5cm} | p{0.5cm} | p{2.5cm} | p{6cm} |}
    \hline
    \bf Name & \bf type & \bf size & \bf default value & \bf description \\ \hline
    \multicolumn{4}{l}{\parbox{10cm}{type = 'BodyCable2DCoordinates'}} & \multicolumn{1}{l}{\parbox{6cm}{\it item typename for initialization}}\\ \hline
    name &     String &      &     '' &     marker"s unique name\\ \hline
    bodyNumber &     Index &      &     MAXINT &     body number to which marker is attached to\\ \hline
    visualization & VMarkerBodyCable2DCoordinates & & & parameters for visualization of item \\ \hline
	  \end{longtable}
	\end{center}
The item VMarkerBodyCable2DCoordinates has the following parameters:
%reference manual TABLE
\begin{center}
  \footnotesize
  \begin{longtable}{| p{4.5cm} | p{2.5cm} | p{0.5cm} | p{2.5cm} | p{6cm} |}
    \hline
    \bf Name & \bf type & \bf size & \bf default value & \bf description \\ \hline
    show &     bool &      &     True &     set true, if item is shown in visualization and false if it is not shown\\ \hline
	  \end{longtable}
	\end{center}

%+++++++++++++++++++++++++++++++
%+++++++++++++++++++++++++++++++
\mysubsection{Loads}

%+++++++++++++++++++++++++++++++++++
\mysubsubsection{LoadForceVector}
Load with (3D) force vector; attached to position-based marker.
 \\
{\bf Short name} for Python: {\bf Force}
 \\  {\bf Requested marker type} = Marker::Position \\ 
The item LoadForceVector has the following parameters:
%reference manual TABLE
\begin{center}
  \footnotesize
  \begin{longtable}{| p{4.5cm} | p{2.5cm} | p{0.5cm} | p{2.5cm} | p{6cm} |}
    \hline
    \bf Name & \bf type & \bf size & \bf default value & \bf description \\ \hline
    \multicolumn{4}{l}{\parbox{10cm}{type = 'ForceVector'}} & \multicolumn{1}{l}{\parbox{6cm}{\it item typename for initialization}}\\ \hline
    name &     String &      &     '' &     load"s unique name\\ \hline
    markerNumber &     Index &      &     MAXINT &     marker"s number to which load is applied\\ \hline
    loadVector &     Vector3D &      &     [0.,0.,0.] &     vector-valued load [SI:N]\\ \hline
    loadVectorUserFunction &     PyFunctionVector3DScalarVector3D &     \tabnewline  &     \tabnewline 0 &     A python function which defines the time-dependent load with parameters (Real t, Vector3D load); the load represents the current value of the load; WARNING: this factor does not work in combination with static computation (loadFactor); Example for python function: def f(t, loadVector): return [loadVector[0]*np.sin(t*10*2*3.1415),0,0]\\ \hline
    visualization & VLoadForceVector & & & parameters for visualization of item \\ \hline
	  \end{longtable}
	\end{center}
The item VLoadForceVector has the following parameters:
%reference manual TABLE
\begin{center}
  \footnotesize
  \begin{longtable}{| p{4.5cm} | p{2.5cm} | p{0.5cm} | p{2.5cm} | p{6cm} |}
    \hline
    \bf Name & \bf type & \bf size & \bf default value & \bf description \\ \hline
    show &     bool &      &     True &     set true, if item is shown in visualization and false if it is not shown\\ \hline
	  \end{longtable}
	\end{center}

%+++++++++++++++++++++++++++++++++++
\mysubsubsection{LoadTorqueVector}
Load with (3D) torque vector; attached to rigidbody-based marker.
 \\
{\bf Short name} for Python: {\bf Torque}
 \\  {\bf Requested marker type} = Marker::Orientation \\ 
The item LoadTorqueVector has the following parameters:
%reference manual TABLE
\begin{center}
  \footnotesize
  \begin{longtable}{| p{4.5cm} | p{2.5cm} | p{0.5cm} | p{2.5cm} | p{6cm} |}
    \hline
    \bf Name & \bf type & \bf size & \bf default value & \bf description \\ \hline
    \multicolumn{4}{l}{\parbox{10cm}{type = 'TorqueVector'}} & \multicolumn{1}{l}{\parbox{6cm}{\it item typename for initialization}}\\ \hline
    name &     String &      &     '' &     load"s unique name\\ \hline
    markerNumber &     Index &      &     MAXINT &     marker"s number to which load is applied\\ \hline
    loadVector &     Vector3D &      &     [0.,0.,0.] &     vector-valued load [SI:N]\\ \hline
    loadVectorUserFunction &     PyFunctionVector3DScalarVector3D &     \tabnewline  &     \tabnewline 0 &     A python function which defines the time-dependent load with parameters (Real t, Vector3D load); the load represents the current value of the load; WARNING: this factor does not work in combination with static computation (loadFactor); Example for python function: def f(t, loadVector): return [loadVector[0]*np.sin(t*10*2*3.1415),0,0]\\ \hline
    visualization & VLoadTorqueVector & & & parameters for visualization of item \\ \hline
	  \end{longtable}
	\end{center}
The item VLoadTorqueVector has the following parameters:
%reference manual TABLE
\begin{center}
  \footnotesize
  \begin{longtable}{| p{4.5cm} | p{2.5cm} | p{0.5cm} | p{2.5cm} | p{6cm} |}
    \hline
    \bf Name & \bf type & \bf size & \bf default value & \bf description \\ \hline
    show &     bool &      &     True &     set true, if item is shown in visualization and false if it is not shown\\ \hline
	  \end{longtable}
	\end{center}

%+++++++++++++++++++++++++++++++++++
\mysubsubsection{LoadMassProportional}
Load attached to BodyMass-based marker, applying a 3D vector load (e.g. the vector [0,-g,0] is used to apply gravitational loading of size g in negative y-direction).
 \\
{\bf Short name} for Python: {\bf Gravity}
 \\  {\bf Requested marker type} = Marker::BodyMass \\ 
The item LoadMassProportional has the following parameters:
%reference manual TABLE
\begin{center}
  \footnotesize
  \begin{longtable}{| p{4.5cm} | p{2.5cm} | p{0.5cm} | p{2.5cm} | p{6cm} |}
    \hline
    \bf Name & \bf type & \bf size & \bf default value & \bf description \\ \hline
    \multicolumn{4}{l}{\parbox{10cm}{type = 'MassProportional'}} & \multicolumn{1}{l}{\parbox{6cm}{\it item typename for initialization}}\\ \hline
    name &     String &      &     '' &     load"s unique name\\ \hline
    markerNumber &     Index &      &     MAXINT &     marker"s number to which load is applied\\ \hline
    loadVector &     Vector3D &      &     [0.,0.,0.] &     vector-valued load [SI:N/kg = m/s$^2$] \\ \hline
    loadVectorUserFunction &     PyFunctionVector3DScalarVector3D &     \tabnewline  &     \tabnewline 0 &     A python function which defines the time-dependent load with parameters (Real t, Vector3D load); the load represents the current value of the load; WARNING: this factor does not work in combination with static computation (loadFactor); Example for python function: def f(t, loadVector): return [loadVector[0]*np.sin(t*10*2*3.1415),0,0]\\ \hline
    visualization & VLoadMassProportional & & & parameters for visualization of item \\ \hline
	  \end{longtable}
	\end{center}
The item VLoadMassProportional has the following parameters:
%reference manual TABLE
\begin{center}
  \footnotesize
  \begin{longtable}{| p{4.5cm} | p{2.5cm} | p{0.5cm} | p{2.5cm} | p{6cm} |}
    \hline
    \bf Name & \bf type & \bf size & \bf default value & \bf description \\ \hline
    show &     bool &      &     True &     set true, if item is shown in visualization and false if it is not shown\\ \hline
	  \end{longtable}
	\end{center}

%+++++++++++++++++++++++++++++++++++
\mysubsubsection{LoadCoordinate}
Load with scalar value, which is attached to a coordinate-based marker; the load can be used e.g. to apply a force to a single axis of a body, a nodal coordinate of a finite element  or a torque to the rotatory DOF of a rigid body.
 \\  {\bf Requested marker type} = Marker::Coordinate \\ 
The item LoadCoordinate has the following parameters:
%reference manual TABLE
\begin{center}
  \footnotesize
  \begin{longtable}{| p{4.5cm} | p{2.5cm} | p{0.5cm} | p{2.5cm} | p{6cm} |}
    \hline
    \bf Name & \bf type & \bf size & \bf default value & \bf description \\ \hline
    \multicolumn{4}{l}{\parbox{10cm}{type = 'Coordinate'}} & \multicolumn{1}{l}{\parbox{6cm}{\it item typename for initialization}}\\ \hline
    name &     String &      &     '' &     load"s unique name\\ \hline
    markerNumber &     Index &      &     MAXINT &     marker"s number to which load is applied\\ \hline
    load &     Real &      &     0. &     scalar load [SI:N]\\ \hline
    loadUserFunction &     PyFunctionScalar2 &     \tabnewline  &     0 &     A python function which defines the time-dependent load with parameters (Real t, Real load); the load represents the current value of the load; WARNING: this factor does not work in combination with static computation (loadFactor); Example for python function: def f(t, load): return load*np.sin(t*10*2*3.1415)\\ \hline
    visualization & VLoadCoordinate & & & parameters for visualization of item \\ \hline
	  \end{longtable}
	\end{center}
The item VLoadCoordinate has the following parameters:
%reference manual TABLE
\begin{center}
  \footnotesize
  \begin{longtable}{| p{4.5cm} | p{2.5cm} | p{0.5cm} | p{2.5cm} | p{6cm} |}
    \hline
    \bf Name & \bf type & \bf size & \bf default value & \bf description \\ \hline
    show &     bool &      &     True &     set true, if item is shown in visualization and false if it is not shown\\ \hline
	  \end{longtable}
	\end{center}

%+++++++++++++++++++++++++++++++
%+++++++++++++++++++++++++++++++
\mysubsection{Sensors}

%+++++++++++++++++++++++++++++++++++
\mysubsubsection{SensorNode}
A sensor attached to a node. The sensor measures OutputVariables and outputs values into a file, showing time, sensorValue[0], sensorValue[1], ... . A user function can be attached to modify sensor values accordingly.
 \\The item SensorNode has the following parameters:
%reference manual TABLE
\begin{center}
  \footnotesize
  \begin{longtable}{| p{4.5cm} | p{2.5cm} | p{0.5cm} | p{2.5cm} | p{6cm} |}
    \hline
    \bf Name & \bf type & \bf size & \bf default value & \bf description \\ \hline
    \multicolumn{4}{l}{\parbox{10cm}{type = 'Node'}} & \multicolumn{1}{l}{\parbox{6cm}{\it item typename for initialization}}\\ \hline
    name &     String &      &     '' &     marker"s unique name\\ \hline
    nodeNumber &     Index &      &     MAXINT &     node number to which sensor is attached to\\ \hline
    writeToFile &     bool &      &     True &     true: write sensor output to file\\ \hline
    fileName &     String &      &     '' &     directory and file name for sensor file output; default: empty string generates sensor + sensorNumber + outputVariableType\\ \hline
    outputVariableType &     OutputVariableType &     \tabnewline  &     OutputVariableType::None &     OutputVariableType for sensor\\ \hline
    visualization & VSensorNode & & & parameters for visualization of item \\ \hline
	  \end{longtable}
	\end{center}
The item VSensorNode has the following parameters:
%reference manual TABLE
\begin{center}
  \footnotesize
  \begin{longtable}{| p{4.5cm} | p{2.5cm} | p{0.5cm} | p{2.5cm} | p{6cm} |}
    \hline
    \bf Name & \bf type & \bf size & \bf default value & \bf description \\ \hline
    show &     bool &      &     True &     set true, if item is shown in visualization and false if it is not shown\\ \hline
	  \end{longtable}
	\end{center}

%+++++++++++++++++++++++++++++++++++
\mysubsubsection{SensorBody}
A sensor attached to a body with local position. As a difference to other ObjectSensors, the body sensor has a local position at which the sensor is attached to. The sensor measures OutputVariableBody and outputs values into a file, showing time, sensorValue[0], sensorValue[1], ... . A user function can be attached to postprocess sensor values accordingly.
 \\The item SensorBody has the following parameters:
%reference manual TABLE
\begin{center}
  \footnotesize
  \begin{longtable}{| p{4.5cm} | p{2.5cm} | p{0.5cm} | p{2.5cm} | p{6cm} |}
    \hline
    \bf Name & \bf type & \bf size & \bf default value & \bf description \\ \hline
    \multicolumn{4}{l}{\parbox{10cm}{type = 'Body'}} & \multicolumn{1}{l}{\parbox{6cm}{\it item typename for initialization}}\\ \hline
    name &     String &      &     '' &     marker"s unique name\\ \hline
    bodyNumber &     Index &      &     MAXINT &     body (=object) number to which sensor is attached to\\ \hline
    localPosition &     Vector3D &     3 &     [0.,0.,0.] &     local (body-fixed) body position of sensor\\ \hline
    writeToFile &     bool &      &     True &     true: write sensor output to file\\ \hline
    fileName &     String &      &     '' &     directory and file name for sensor file output; default: empty string generates sensor + sensorNumber + outputVariableType\\ \hline
    outputVariableType &     OutputVariableType &     \tabnewline  &     OutputVariableType::None &     OutputVariableType for sensor\\ \hline
    visualization & VSensorBody & & & parameters for visualization of item \\ \hline
	  \end{longtable}
	\end{center}
The item VSensorBody has the following parameters:
%reference manual TABLE
\begin{center}
  \footnotesize
  \begin{longtable}{| p{4.5cm} | p{2.5cm} | p{0.5cm} | p{2.5cm} | p{6cm} |}
    \hline
    \bf Name & \bf type & \bf size & \bf default value & \bf description \\ \hline
    show &     bool &      &     True &     set true, if item is shown in visualization and false if it is not shown\\ \hline
	  \end{longtable}
	\end{center}
