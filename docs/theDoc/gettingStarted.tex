
%+++++++++++++++++++++++++++++++++++++++++++++++++++++++++++++++++++++++++++++++
%\mysection{Getting Started}
The documentation for \codeName\ is split into this introductory section, including a quick start up, code structure and important hints, 
as well as a couple of sections containing references to the available Python interfaces to interact with \codeName\ and finally some information on theory (e.g., 'Solver').

\codeName\ is hosted on GitHub \cite{EXUDYNgit}:
\bi
 \item web: \texttt{https://github.com/jgerstmayr/EXUDYN/wiki}
\ei
For any comments, requests, issues, bug reports, send an email to: 
\bi
  \item email: \texttt{reply.exudyn@gmail.com}
\ei
Thanks for your contribution!

\mysubsection{Getting started}
This section will show:
\bn
	\item What is \codeName ?
	\item Who is developing \codeName ?
	\item How to install \codeName\ 
	\item How to link \codeName\ and Python
	\item Goals of \codeName
	\item Run a simple example in Spyder
	\item FAQ -- Frequently asked questions
\en
%
\mysubsubsection{What is \codeName ?}
\codeName -- {\small (fl{\bf EX}ible m{\bf U}ltibody {\bf DYN}amics  -- {\bf EX}tend yo{\bf U}r {\bf DYN}amics)}\vspace{6pt}\\
\noindent \codeName\ is a C++ based Python library for efficient simulation of flexible multibody dynamics systems.
It is the follow up code of the previously developed multibody code HOTINT, which Johannes Gerstmayr started during his PhD-thesis.
The open source code HOTINT reached limits of further (efficient) development and it seemed impossible to continue from this code as it is outdated regarding programming techniques and the numerical formulation.

\codeName\ is designed to easily set up complex multibody models, consisting of rigid and flexible bodies with joints, loads and other components. It shall enable automatized model setup and parameter variations, which are often necessary for system design but also for analysis of technical problems. The broad usability of python allows to couple a multibody simulation with environments such as optimization, statistics, data analysis, machine learning and others.

The multibody formulation is mainly based on redundant coordinates. This means that computational objects (rigid bodies, flexible bodies, ...) are added as independent bodies to the system. Hereafter, connectors (e.g., springs or constraints) are used to interconnect the bodies. The connectors are using Markers on the bodies as interfaces, in order to transfer forces and displacements.
For details on the interaction of nodes, objects, markers and loads see Section \ref{sec_items}.
\vspace{6pt}\\
%
\mysubsubsection{Who is developing \codeName ?}
\codeName\ is currently (\the\month-\the\year) developed at the University of Innsbruck.
In the first phase most of the core code is written by Johannes Gerstmayr, implementing ideas that followed out of the project HOTINT. 15 years of development led to a lot of lessions learned and after 20 years, a code must be re-designed.

Some specific codes regarding Pybind11 (by Jakob Wenzel, \texttt{https://github.com/pybind/pybind11}, thanks a lot!!!) interface and parallelization have been written by Stefan Holzinger, who also supported the upload to GitLab.

Important discussions with researchers from the community were important for the design and development of \codeName , where we like to mention Joachim Sch{\"o}berl from TU-Vienna who boosted the design of the code with great concepts. 
%During a Comet-K2 cooperation project, several discussions with the TMECH/LCM group in Linz influenced the code development.

The cooperation and funding within the EU H2020-MSCA-ITN project 'Joint Training on Numerical Modelling of Highly Flexible Structures for Industrial Applications' contributes to the development of the code.

The following people have contributed to the examples:
\bi
	\item Stefan Holzinger, Michael Pieber, Joachim Sch{\"o}berl, Manuel Schieferle, Martin Knapp, Lukas March, Dominik Sponring, David Wibmer, Andreas Zw{\"o}lfer
\ei
-- thanks a lot! --
%
%++++++++++++++++++++++++++++++++++++++++++++++++++++++++++++++++
%++++++++++++++++++++++++++++++++++++++++++++++++++++++++++++++++
%++++++++++++++++++++++++++++++++++++++++++++++++++++++++++++++++
\mysubsubsection{How to install \codeName ?}

In order to run \codeName , you need an appropriate Python installation.
We recommend to use
\bi
  \item Anaconda, 32bit, Python 3.6.5 or Anaconda 64bit, Python 3.6.5)\footnote{Anaconda 32/64bit with Python3.6 can be downloaded via the repository archive \texttt{https://repo.anaconda.com/archive/} choosing \texttt{Anaconda3-5.2.0-Windows-x86.exe} or \texttt{Anaconda3-5.2.0-Windows-x86\_64.exe} for 64bit.}
	\item Spyder 3.2.8 with Python 3.6.5 32 bit (alternatively 64bit), which is included in the Anaconda installation\footnote{It is important that Spyder, python and exudyn are {\bf either} 32bit {\bf or} 64bit. There will be a strange .DLL error, if you mix up 32/64bit. It is possible to install both, Anaconda 32bit and Anaconda 64bit -- then you should follow the recommendations of paths as suggested by Anaconda installer}
%Anaconda3-5.2.0-Windows-x86\_64.exe
\ei
If you plan to extend the C++ code, we recommend to use VS2017\footnote{previously, VS2019 was recommended: However, VS2019 has problems with the library 'Eigen' and therefore leads to erroneous results with the sparse solver. VS2017 can also be configured with Python 3.7 now.} to compile your code, which offers Python 3.7 compatibility.
However, you should know that Python versions and the version of the module must be identical (e.g., Python 3.6 32 bit {\bf both} in the \codeName\ module and in Spyder).

%The simplest way to start is, to copy the files (and possibly further files that are needed)
%\bi
  %\item \texttt{exudynUtilities.py}
  %\item \texttt{itemInterface.py}
  %\item \texttt{exudyn.pyd}
%\ei
%to your working directory and directly import the modules as described in tutorials and examples.
%+++++++++++++++++++++++++++++++++++++
\mysubsubsection{Install with Windows MSI installer}
The simplest way on Windows 10 (and maybe also Windows 7), which works well {\bf if you installed only one python version} and if you installed Anaconda with the option {\bf 'Register Anaconda as my default Python 3.x'} or similar, then you can use the provided \texttt{.msi} installers in the \texttt{main/dist} directory:
\bi
  \item For the 64bits python 3.6 version, double click on (version may differ):\\ \texttt{exudyn-1.0.8.win-amd64-py3.6.msi}
	\item Follow the instructions of the installer
	\item If python / Anaconda is not found by the installer, provide the 'python directory' as the installation directory of Anaconda3, which usually is installed in:\\
	\texttt{C:$\backslash$ProgramData$\backslash$Anaconda3}
\ei

%+++++++++++++++++++++++++++++++++++++
\mysubsubsection{Install from Wheel}
The {\bf standard way to install} the python package \codeName\ is to use the so-called 'wheels' (file ending \texttt{.whl}) provided at the directory wheels in the \codeName\ repository. 
First, open an Anaconda prompt:
\bi
  \item EITHER calling: START->Anaconda->... OR go to anaconda/Scripts folder and call activate.bat
	\item You can check your python version then, by running \texttt{python}\footnote{\texttt{python3} under UBUNTU 18.04}, the output reads like:
	\bi
	  \item[] \texttt{Python 3.6.5 $|$Anaconda, Inc.$|$ (default, Mar 29 2018, 13:32:41) $[$MSC v.1900 64 bit (AMD64)$]$ on win32}
		\item[] ...
	\ei
	\item $\ra$ type \texttt{exit()} to close python
\ei

\noindent {\bf Go to the folder \texttt{Exudyn\_git/main}} (where \texttt{setup.py} lies) and choose the wheel in subdirectory \texttt{main/dist} according to your system (windows/UBUNTU), python version (3.6 or 3.7) and 32 or 64 bits.

For Windows this may read (version number 1.0.8 may be different):
\bi
  \item \texttt{Python 3.6, 32bit}: pip install dist$\backslash$exudyn-1.0.8-cp36-cp36m-win32.whl
  \item \texttt{Python 3.6, 64bit}: pip install dist$\backslash$exudyn-1.0.8-cp36-cp36m-win\_amd64.whl
  \item \texttt{Python 3.7, 64bit}: pip install dist$\backslash$exudyn-1.0.8-cp37-cp37m-win\_amd64.whl
\ei
For UBUNTU18.04 this may read (version number 1.0.8 may be different):
\bi
  \item \texttt{Python 3.6, 64bit}: pip3 install dist$\backslash$exudyn-1.0.8-cp36-cp36m-linux\_x86\_64.whl
\ei

%+++++++++++++++++++++++++++++++++++++
\mysubsubsection{Work without installation and editing \texttt{sys.path}}
The {\bf uncommon and old way} ($\ra$ not recommended for \codeName\ versions $\ge$ 1.0.0) is to use Python's \texttt{sys} module to link to your \texttt{exudyn} (previously \texttt{WorkingRelease}) directory, for example:\vspace{6pt}\\
\pythonstyle
\begin{lstlisting}[language=Python, firstnumber=1]
  import sys
  sys.path.append('C:/DATA/cpp/EXUDYN_git/bin/EXUDYN32bitsPython36')
\end{lstlisting}
%  sys.path.append('C:/DATA/cpp/EXUDYN_git/main/bin/WorkingRelease')
%
The folder \texttt{EXUDYN32bitsPython36} needs to be adapted to the location of the according \codeName\ package.
%In case of 64bit, it must be changed to \texttt{.... /bin/WorkingRelease64}.

%In the future, there will also be a possibility to install the module using pip commands -- we are happy, if somebody could do this!

%++++++++++++++++++++++++++++++++++++++++++++++++++++++++++++++++++++++++++++++
%++++++++++++++++++++++++++++++++++++++++++++++++++++++++++++++++++++++++++++++
\mysubsubsection{Build and install \codeName\ under Windows 10?}

Note that there are a couple of pre-requisites, depending on your system and installed libraries. For Windows 10, the following steps proved to work:
\bi
  \item install your Anaconda distribution including Spyder
  \item close all Python programs (e.g. Spyder, Jupyter, ...)
	\item run an Anaconda prompt (may need to be run as administrator)
	\item if you cannot run Anaconda prompt directly, do:
	\bi
	  \item open windows shell (cmd.exe) as administrator (START $\ra$ search for cmd.exe $\ra$ right click on app $\ra$ 'run as administrator' if necessary)
		\item go to your Scripts folder inside the Anaconda folder (e.g. \texttt{C:$\backslash$ProgramData$\backslash$Anaconda$\backslash$Scripts})
	  \item run 'activate.bat'
	\ei
	\item go to 'main' of your cloned github folder of exudyn
	\item run: \texttt{python setup.py install}
	\item read the output; if there are errors, try to solve them by installing appropriate modules
\ei
You can also create your own wheels, doing the above steps to activate the according python version and then calling (requires installation of Microsoft Visual Studio; recommended: VS2017):
\bi
  \item[] \texttt{python setup.py bdist\_wheel}
\ei
This will add a wheel in the \texttt{dist} folder.

%++++++++++++++++++++++++++++++++++++++++++++++++++++++++++++++++++++++++++++++
%++++++++++++++++++++++++++++++++++++++++++++++++++++++++++++++++++++++++++++++
\mysubsubsection{Build and install \codeName\ under UBUNTU?}

Having a new UBUNTU 18.04 standard installation (e.g. using a VM virtual box environment), the following steps need to be done (python {\bf 3.6} is already installed on UBUNTU18.04, otherwise use \texttt{sudo apt install python3})\footnote{see also the youtube video: \texttt{https://www.youtube.com/playlist?list=PLZduTa9mdcmOh5KVUqatD9GzVg\_jtl6fx}}:

\noindent First update ...
\begin{lstlisting}[firstnumber=1]
sudo apt-get update
\end{lstlisting}

\noindent Install necessary python libraries and pip3; \texttt{matplotlib} and\texttt{scipy} are not required for installation but used in \codeName\ examples:
\begin{lstlisting}[firstnumber=1]
sudo dpkg --configure -a
sudo apt install python3-pip
pip3 install numpy
pip3 install matplotlib
pip3 install scipy
\end{lstlisting}

\noindent Install pybind11 (needed for running the setup.py file derived from the pybind11 example):
\begin{lstlisting}[firstnumber=1]
pip3 install pybind11
\end{lstlisting}

\noindent If graphics is used (\texttt{\#define USE\_GLFW\_GRAPHICS} in \texttt{BasicDefinitions.h}), you must install the according GLFW and OpenGL libs:
\begin{lstlisting}[firstnumber=1]
sudo apt-get install freeglut3 freeglut3-dev
sudo apt-get install mesa-common-dev
sudo apt-get install libglfw3 libglfw3-dev
sudo apt-get install libx11-dev xorg-dev libglew1.5 libglew1.5-dev libglu1-mesa libglu1-mesa-dev libgl1-mesa-glx libgl1-mesa-dev
\end{lstlisting}

\noindent With all of these libs, you can run the setup.py installer (go to \texttt{Exudyn\_git/main} folder), which takes some minutes for compilation (the --user option is used to install in local user folder):
\begin{lstlisting}[firstnumber=1]
sudo python3 setup.py install --user
\end{lstlisting}

\noindent Congratulation! {\bf Now, run a test example} (will also open an OpenGL window if successful):
\bi
  \item[] \texttt{python3 pythonDev/Examples/rigid3Dexample.py}
\ei

\noindent You can also create a UBUNTU wheel which can be easily installed on the same machine (x64), same operating system (UBUNTU18.04) and with same python version (e.g., 3.6):
\bi
  \item[] \texttt{sudo pip3 install wheel}
  \item[] \texttt{sudo python3 setup.py bdist\_wheel}
\ei

%++++++++++++++++++++++++++++++++++++++++++++++++++++++++++++++++++++++++++++++
\mysubsubsection{Uninstall \codeName }

To uninstall exudyn under Windows, run (may require admin rights):
\bi
  \item[] \texttt{pip uninstall exudyn}
\ei
\noindent To uninstall under UBUNTU, run:
\bi
  \item[] \texttt{sudo pip3 uninstall exudyn}
\ei

If you upgrade to a newer version, uninstall is usually not necessary!
%
%++++++++++++++++++++++++++++++++++++++++++++++++++++++++++++++++
\mysubsubsection{How to install \codeName\ and using the C++ code (advanced)?}
\codeName\ is still under intensive development of core modules.
There are several ways to using the code, but you {\bf cannot} install \codeName\ as compared to other executable programs and apps.
\vspace{6pt}\\
%{\bf Ways to use \codeName }:
In order to make full usage of the C++ code and extending it, you can use:
\bi
	\item Windows / Microsoft Visual Studio 2017 and above:
	\bi
		\item get the files from git
		\item put them into a local directory (recommended: \texttt{C:/DATA/cpp/EXUDYN\_git})
		\item start \texttt{main\_sln.sln} with Visual Studio
		\item compile the code and run \texttt{main/pythonDev/pytest.py} example code
		\item adapt \texttt{pytest.py} for your applications
		\item extend the C++ source code
		\item link it to your own code
		\item NOTE: on Linux systems, you mostly need to replace '$/$' with '$\backslash$'
	\ei
	\item Linux, etc.: not fully supported yet; however, all external libraries are Linux-compatible and thus should run with minimum adaptation efforts.
\ei
%
%++++++++++++++++++++++++++++++++++++++++++++++++++++++++++++++++
%++++++++++++++++++++++++++++++++++++++++++++++++++++++++++++++++
\mysubsubsection{Goals of \codeName}
After the first development phase (planned in Q4/2021), it will
\bi
  \item be a small multibody library, which can be easily linked to other projects,
	\item allow to efficiently simulate small scale systems (compute 100000s time steps per second for systems with $n_{DOF}<10$),
	%\item allow to efficiently simulate medium scaled systems for problems with $n_{DOF} < 1\,000\,000$ (planned: Q4 2020),
	%\item use multi-threaded parallel computing techniques (planned: Q4 2020),
	\item safe and widely accessible module for Python,
	\item allow to add user defined objects in C++,
	%\item allow to add user defined objects in Python (planned: 2021),
	\item allow to add user defined solvers in Python).
\ei
%
%++++++++++++++++++++++++++++++++++++++++++++++++++++++++++++++++
%++++++++++++++++++++++++++++++++++++++++++++++++++++++++++++++++
\mysubsubsection{Run a simple example in Spyder}
After performing the steps of the previous section, this section shows a simplistic model which helps you to check if \codeName\ runs on your computer.

In order to start, run the python interpreter Spyder.
For the following example, either 
\bi
	\item open Spyder and copy the example provided in Listing \ref{lst:firstexample} into a new file, or
	\item open \texttt{myFirstExample.py} from your \texttt{EXUDYN32bitsPython36}\footnote{or any other directory according to your python version} directory
\ei
Hereafter, press the play button or \texttt{F5} in Spyder.
%\lstinputlisting[language=Python]{../../main/bin/WorkingRelease/myFirstExample.py}
\pythonexternal[language=Python, frame=single, float, label=lst:firstexample, caption=My first example]{../../main/pythonDev/Examples/myFirstExample.py}

If successful, the IPython Console of Spyder will print something like:
\plainlststyle
{\ttfamily \footnotesize
\begin{lstlisting}
runfile('C:/DATA/cpp/EXUDYN_git/main/bin/EXUDYN32bitsPython36/myFirstExample.py', 
  wdir='C:/DATA/cpp/EXUDYN_git/main/bin/EXUDYN32bitsPython36')
+++++++++++++++++++++++++++++++
EXUDYN V1.0.1 solver: implicit second order time integration
STEP100, t = 1 sec, timeToGo = 0 sec, Nit/step = 1
solver finished after 0.0007824 seconds.
\end{lstlisting}
  %runfile('C:/DATA/cpp/EXUDYN_git/main/bin/WorkingRelease/myFirstExample.py', 
	  %wdir='C:/DATA/cpp/EXUDYN_git/main/bin/WorkingRelease')
}

If you check your current directory (where \texttt{myFirstExample.py} lies), you will find a new file \texttt{coordinatesSolution.txt}, which contains the results of your computation (with default values for time integration).
The beginning and end of the file should look like: \vspace{6pt}\\
{\ttfamily \footnotesize
\begin{lstlisting}[breaklines=true]
  #Exudyn generalized alpha solver solution file
  #simulation started=2019-11-14,20:35:12
  #columns contain: time, ODE2 displacements, ODE2 velocities, ODE2 accelerations, AE coordinates, ODE2 velocities
  #number of system coordinates [nODE2, nODE1, nAlgebraic, nData] = [2,0,0,0]
  #number of written coordinates [nODE2, nVel2, nAcc2, nODE1, nVel1, nAlgebraic, nData] = [2,2,2,0,0,0,0]
  #total columns exported  (excl. time) = 6
  #number of time steps (planned) = 100
  #
  0,0,0,0,0,0.0001,0
  0.02,2e-08,0,2e-06,0,0.0001,0
  0.03,4.5e-08,0,3e-06,0,0.0001,0
  0.04,8e-08,0,4e-06,0,0.0001,0
  0.05,1.25e-07,0,5e-06,0,0.0001,0

  ...

  0.96,4.608e-05,0,9.6e-05,0,0.0001,0
  0.97,4.7045e-05,0,9.7e-05,0,0.0001,0
  0.98,4.802e-05,0,9.8e-05,0,0.0001,0
  0.99,4.9005e-05,0,9.9e-05,0,0.0001,0
  1,5e-05,0,0.0001,0,0.0001,0
  #simulation finished=2019-11-14,20:35:12
  #Solver Info: errorOccurred=0,converged=1,solutionDiverged=0,total time steps=100,total Newton iterations=100,total Newton jacobians=100
\end{lstlisting}
}

Within this file, the first column shows the simulation time and the following columns provide solution of coordinates, their derivatives and Lagrange multipliers on system level. As expected, the $x$-coordinate of the point mass has constant acceleration $a=f/m=0.001/10=0.0001$, the velocity grows up to $0.0001$ after 1 second and the point mass moves $0.00005$ along the $x$-axis.
%
%++++++++++++++++++++++++++++++++++++++++++++++++++++++++++++++++
%++++++++++++++++++++++++++++++++++++++++++++++++++++++++++++++++
\mysubsection{Trouble shooting and FAQ}

{\bf Known issues}:
\bn
  %++++++++++++++++++++++++++++++++++++++++++++++++++++++++++++++++++++++++++++++++++++++++
  %++++++++++++++++++++++++++++++++++++++++++++++++++++++++++++++++++++++++++++++++++++++++
  \item Sometimes the \texttt{exudyn} module cannot be loaded into Python. There are several reasons and workarounds:
	\bi
	  \item You mixed up 32 and 64 bits (see below) version
		\item You are using an exudyn version for Python $x_1.y_1$ (e.g., 3.6.$z_1$) different from the Python $x_2.y_2$ version in your Anaconda (e.g., 3.7.$z_2$); note that $x_1=x_2$ and $y_1=y_2$ must be obeyed while $z_1$ and $z_2$ may be different
		\item \texttt{ModuleNotFoundError: No module named 'exudynCPP'}:\\
		A known reason is that your CPU\footnote{modern Intel Core-i3, Core-i5 and Core-i7 processors as well as AMD processors, espcially Zen and Zen-2 architectures should have no problems with AVX; however, low-cost Celeron and Pentium processors {\bf will not support AVX}, e.g.,  Intel Celeron G3900, Intel core 2 quad q6600, Intel Pentium Gold G5400T; check the system settings of your computer to find out the processor type; typical CPU manufacturer pages or Wikipedia provide information on this} does not support AVX2, while exudyn is compiled with the AVX2 option; \\
		$\ra$ {\bf workaround} to solve the AVX problem: use the Python 3.6 32bits version, which is compiled without AVX2; you can also compile for your specific Python version without AVX if you adjust the \texttt{setup.py} file in the \texttt{main} folder.
	  \item[] The \texttt{ModuleNotFoundError} may also happen if something went wrong during installation (paths, problems with Anaconda, ..) $\ra$ very often a new installation of Anaconda and \codeName\ helps.
	\ei
  %++++++++++++++++++++++++++++++++++++++++++++++++++++++++++++++++++++++++++++++++++++++++
  %++++++++++++++++++++++++++++++++++++++++++++++++++++++++++++++++++++++++++++++++++++++++
  \item When importing \codeName\ in python (windows) I get the error (or similar):\\
{\ttfamily \footnotesize
\begin{lstlisting}[breaklines=true]
Traceback (most recent call last):
  File "C:\DATA\cpp\EXUDYN_git\main\pythonDev\pytest.py", line 18, in <module>
    import exudyn as exu
  ImportError: DLL load failed: %1 is no valid Win32 application.
\end{lstlisting}}
	$\ra$ probably this is a 32/64bit problem. Your Python installation and \codeName\ need to be {\bf BOTH} either 64bit OR 32bit (Check in your python help; exudyn in WorkingRelease64 is the 64bit version, in WorkingRelease it is the 32bit version) and the Python installation and \codeName\ need to have {\bf BOTH} the same version and 1$\mathrm{st}$ subversion number (e.g., 3.6.5 should be compatible with 3.6.2).
  %++++++++++++++++++++++++++++++++++++++++++++++++++++++++++++++++++++++++++++++++++++++++
  %++++++++++++++++++++++++++++++++++++++++++++++++++++++++++++++++++++++++++++++++++++++++
  \item I do not understand the python errors -- how can I find the reason of the error or crash?
	\bi
	\item[$\ra$] First, you should read all error messages and warnings: from the very first to the last message. Very often, there is a definite line number which shows the error. Note, that if you are executing a string (or module) as a python code, the line numbers refer to the local line number inside the script or module.
	\item[$\ra$] If everything fails, try to execute only part of the code to find out where the first error occurs. By omiting parts of the code, you should find the according source of the error.
	\item[$\ra$] If you think, it is a bug: send an email with a representative code snippet, version, etc.\ to \texttt{ reply.exudyn@gmail.com}
	\ei
  %++++++++++++++++++++++++++++++++++++++++++++++++++++++++++++++++++++++++++++++++++++++++
  %++++++++++++++++++++++++++++++++++++++++++++++++++++++++++++++++++++++++++++++++++++++++
  \item Spyder console hangs up, does not show error messages, ...:\\
	$\ra$ very often a new start of Spyder helps; most times, it is sufficient to restart the kernel or to just press the 'x' in your IPython console, which closes the current session and restarts the kernel (this is much faster than restarting Spyder); \\
	$\ra$ restarting the IPython console also brings back all error messages
	
\en
{\bf List of Frequently asked questions}:
\bn
  %++++++++++++++++++++++++++++++++++++++++++++++++++++++++++++++++++++++++++++++++++++++++
  %++++++++++++++++++++++++++++++++++++++++++++++++++++++++++++++++++++++++++++++++++++++++
  \item Where do I find the '.exe' file?
	\bi
	\item[$\ra$] \codeName\ is only available via the python interface as exudyn.pyd library, which is located in folder: \texttt{main/bin/WorkingRelease}. This means that you need to run python (best: Spyder) and import the \codeName\ module.
	\ei

	%
  %++++++++++++++++++++++++++++++++++++++++++++++++++++++++++++++++++++++++++++++++++++++++
  %++++++++++++++++++++++++++++++++++++++++++++++++++++++++++++++++++++++++++++++++++++++++
	\item I get the error message 'check potential mixing of different (object, node, marker, ...) indices', what does it mean?
	\bi
	\item[$\ra$] probably you used wrong item indices, see beginning of Section \ref{sec_PCpp_command_interface}. 
	\item E.g., an object number \texttt{oNum = mbs.AddObject(...)} is used at a place where a \texttt{NodeIndex} is expected:
	\texttt{mbs.AddObject(MassPoint(nodeNumber=oNum, ...))}
	\item Usually, this is an ERROR in your code, it does not make sense to mix up these indices!
	\item In the exceptional case, that you want to convert numbers, see beginning of Section \ref{sec_PCpp_command_interface}.
	\ei
	%
  %++++++++++++++++++++++++++++++++++++++++++++++++++++++++++++++++++++++++++++++++++++++++
  %++++++++++++++++++++++++++++++++++++++++++++++++++++++++++++++++++++++++++++++++++++++++
	\item Why does type auto completion does not work for mbs (Main system)?
	\bi
	\item[$\ra$] UPDATE 2020-06-01: with Spyder 4, using Python 3.7, type auto completion works much better, but may find too many completions.
	\item[$\ra$] most python environments (e.g., with Spyder 3) only have information up to the first sub-structure, e.g., \texttt{SC=exu.SystemContainer()} provides full access to SC in the type completion, but \texttt{mbs=SC.AddSystem()} is at the second sub-structure of the module and is not accessible.\\
	WORKAROUND: type \texttt{mbs=MainSystem()} {\bf before} the \texttt{mbs=SC.AddSystem()} command and the interpreter will know what type mbs is. This also works for settings, e.g., simulation settings 'Newton'.
	\ei
	%
  %++++++++++++++++++++++++++++++++++++++++++++++++++++++++++++++++++++++++++++++++++++++++
  %++++++++++++++++++++++++++++++++++++++++++++++++++++++++++++++++++++++++++++++++++++++++
  \item How to add graphics?
	\bi
	\item[$\ra$] Graphics (lines, text, 3D triangular / STL mesh) can be added to all BodyGraphicsData items in objects. Graphics objects which are fixed with the background can be attached to a ObjectGround object.
	Moving objects must be attached to the BodyGraphicsData of a moving body. Other moving bodies can be realized, e.g., by adding a ObjectGround and changing its reference with time.
	\ei
  %++++++++++++++++++++++++++++++++++++++++++++++++++++++++++++++++++++++++++++++++++++++++
  %++++++++++++++++++++++++++++++++++++++++++++++++++++++++++++++++++++++++++++++++++++++++
  \item What is the difference between MarkerBodyPosition and MarkerBodyRigid?
	\bi
	\item[$\ra$] Position markers (and nodes) do not have information on the orientation (rotation). For that reason, there is a difference between position based and rigid-body based markers. In case of a rigid body attached to ground with a SpringDamper, you can use both, MarkerBodyPosition or MarkerBodyRigid, markers. For a prismatic joint, you will need a MarkerBodyRigid.
	\ei
  %++++++++++++++++++++++++++++++++++++++++++++++++++++++++++++++++++++++++++++++++++++++++
  %++++++++++++++++++++++++++++++++++++++++++++++++++++++++++++++++++++++++++++++++++++++++
  \item I get an error in \texttt{SC.TimeIntegrationSolve(mbs, 'GeneralizedAlpha', simulationSettings)} but no further information -- how can I solve it?
	\bi
	\item[$\ra$] Typical time integration errors may look like:\\
	{\footnotesize
  \texttt{File "C:/DATA/cpp/EXUDYN\_git/main/pythonDev/...<file name>", line XXX, in <module>}\\  
	\texttt{SC.TimeIntegrationSolve(mbs, 'GeneralizedAlpha', simulationSettings)}\\
  \texttt{SystemError: <built-in method TimeIntegrationSolve of PyCapsule object at 0x0CC63590> returned a result with an error set}}
	\item[$\ra$] The prechecks, which are performed to enable a crash-free simulation are insufficient for your model.
	\item[$\ra$] As a first try, restart the IPython console in order to get all error messages, which may be blocked due to a previous run of \codeName.
	\item[$\ra$] Very likely, you are using python user functions inside EXUDYN: They lead to an internal python error, which is not catched by EXUDYN. However, you can just check all your user functions, if they will run without EXUDYN. E.g., a load user function UFload(t,load), which tries to access load[4] will fail internally.
	\item[$\ra$] Use the print(...) command in python at many places to find a possible error in user functions (e.g., put \texttt{print("Start user function XYZ")} at the beginning of every user function.
	\item[$\ra$] It is also possible, that you are using inconsistent data, which leads to the crash. In that case, you should try to change your model: omit parts and find out which part is causing your error
	\item[$\ra$] see also {\it I do not understand the python errors -- how can I find the cause?}
 	\ei


  %++++++++++++++++++++++++++++++++++++++++++++++++++++++++++++++++++++++++++++++++++++++++
  %++++++++++++++++++++++++++++++++++++++++++++++++++++++++++++++++++++++++++++++++++++++++
  \item Why can't I get the focus of the simulation window on startup (render window hidden)?
	\bi
	\item[$\ra$] Starting \codeName\ out of Spyder might not bring the simulation window to front, because of specific settings in Spyder(version 3.2.8), e.g., Tools$\ra$Preferences$\ra$Editor$\ra$Advanced settings: uncheck 'Maintain focus in the Editor after running cells or selections'; Alternatively, set \texttt{SC.visualizationSettings.window.alwaysOnTop=True} {\bf before} starting the renderer with \texttt{exu.StartRenderer()}
	\ei
	%
  %++++++++++++++++++++++++++++++++++++++++++++++++++++++++++++++++++++++++++++++++++++++++
  %++++++++++++++++++++++++++++++++++++++++++++++++++++++++++++++++++++++++++++++++++++++++
\en


