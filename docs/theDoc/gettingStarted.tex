
%+++++++++++++++++++++++++++++++++++++++++++++++++++++++++++++++++++++++++++++++
%\mysection{Getting Started}
The documentation for \codeName\ is split into this introductory section, including a quick start up, code structure and important hints, 
as well as a couple of sections containing references to the available Python interfaces to interact with \codeName\ and finally some information on theory (e.g., 'Solver').

\codeName\ is hosted on GitHub:
\bi
 \item web: \texttt{https://github.com/jgerstmayr/EXUDYN/wiki}
\ei
For any comments, requests, issues, bug reports, send an email to: 
\bi
  \item email: \texttt{reply.exudyn@gmail.com}
\ei
Thanks for your contribution!

\mysubsection{Getting started}
This section will show:
\bn
	\item What is \codeName ?
	\item Who is developing \codeName ?
	\item How to install \codeName\ 
	\item How to link \codeName\ and Python
	\item Goals of \codeName
	\item Run a simple example in Spyder
	\item FAQ -- Frequently asked questions
\en
%
\mysubsubsection{What is \codeName ?}
\codeName\ is a C++ based Python library for efficient simulation of flexible multibody dynamics systems.
It is designed to easily set up complex multibody models, consisting of rigid and flexible bodies with joints, loads and other components.

The formulation is mostly based on redundant coordinates. This means that computational objects (rigid bodies, flexible bodies, ...) are added as independent bodies to the system. Hereafter, connectors (e.g., springs or constraints) are used to interconnect the bodies. The connectors are using Markers on the bodies as interfaces, in order to transfer forces and displacements.
For details on the interaction of nodes, objects, markers and loads see Section \ref{sec_items}.
\vspace{6pt}\\
%
\mysubsubsection{Who is developing \codeName ?}
\codeName\ is currently (\the\month-\the\year) developed at the University of Innsbruck.
In the first phase most of the core code has been (and still is) written by Johannes Gerstmayr, implementing ideas of earlier developments of HOTINT. 15 years of development led to a lot of lessions learned.

Some specific codes regarding pybind interface and parallelization have been written by Stefan Holzinger, who also supports the upload to GitLab.
Important discussions with researchers from the community where important for the design and development of \codeName , where we like to mention Joachim Sch{\"o}berl from TU-Vienna who influenced the design of the code. During a Comet-K2 cooperation project, several discussions with the TMECH/LCM group in Linz influenced the code development.

The cooperation and funding within the EU H2020-MSCA-ITN project 'Joint Training on Numerical Modelling of Highly Flexible Structures for Industrial Applications' will support the further development of the code.
%
%++++++++++++++++++++++++++++++++++++++++++++++++++++++++++++++++
%++++++++++++++++++++++++++++++++++++++++++++++++++++++++++++++++
%++++++++++++++++++++++++++++++++++++++++++++++++++++++++++++++++
\mysubsubsection{How to link \codeName\ with Python (recommended for beginners)?}

In order to run \codeName , you need an appropriate Python installation.
We recommend to use
\bi
  \item Anaconda, 32bit, Python 3.6.5
	\item Spyder 3.2.8 (Python 3.6.5 32 bits)	
\ei
If you plan to use 64bit and newer Python versions, we recommend to use VS2019 to compile your code, which offers Python 3.7 compatibility.
However, you should know that Python versions and the version of the module must be identical (e.g., Python 3.6 32 {\bf both} in the exudyn module and in Spyder).

The simplest way to start is, to copy the files (and possibly further files that are needed)
\bi
  \item \texttt{exudynUtilities.py}
  \item \texttt{itemInterface.py}
  \item \texttt{exudyn.pyd}
\ei
to your working directory and directly import the modules as described in tutorials and examples.
The second way ({\bf recommended}) is to use Python's \texttt{sys} module to link to your \texttt{WorkingRelease} directory, for example:\vspace{6pt}\\
\pythonstyle
\begin{lstlisting}[language=Python, firstnumber=1]
  import sys
  sys.path.append('C:\\DATA\\cpp\\EXUDYN_git\\main\\bin\\WorkingRelease')
\end{lstlisting}
In the future, there will also be a possibility to install the module using pip commands -- we are happy, if somebody could do this!
%
%++++++++++++++++++++++++++++++++++++++++++++++++++++++++++++++++
\mysubsubsection{How to install \codeName\ and using the C++ code (advanced)?}
\codeName\ is still under intensive development of core modules.
There are several ways to using the code, but you {\bf cannot} install \codeName\ as compared to other executable programs and apps.
\vspace{6pt}\\
%{\bf Ways to use \codeName }:
In order to make full usage of the C++ code and extending it, you can use:
\bi
	\item Windows / Microsoft Visual Studio 17 and above:
	\bi
		\item get the files from git
		\item put them into a local directory (recommended: \texttt{C:/DATA/cpp/EXUDYN\_git})
		\item start \texttt{main\_sln.sln} with Visual Studio
		\item compile the code and run \texttt{main/pythonDev/pytest.py} example code
		\item adapt \texttt{pytest.py} for your applications
		\item extend the C++ source code
		\item link it to your own code
		\item NOTE: on some systems, you might need to replace '$/$' with '$\backslash$'
	\ei
	\item Linux, etc.: not fully supported yet; however, all external libraries are Linux-compatible and thus should run with minimum adaptation efforts.
\ei
%
%++++++++++++++++++++++++++++++++++++++++++++++++++++++++++++++++
%++++++++++++++++++++++++++++++++++++++++++++++++++++++++++++++++
\mysubsubsection{Goals of \codeName}
After the first development phase (planned in Q4/2021), it will
\bi
  \item be a small multibody library, which can be easily linked to other projects,
	\item allow to efficiently simulate small scale systems (compute 100000s time steps per second for systems with $n_{DOF}<10$),
	\item allow to efficiently simulate medium scaled systems for problems with $n_{DOF} < 1\,000\,000$ (planned: Q4 2020),
	\item use multi-threaded parallel computing techniques (planned: Q4 2020),
	\item be accessible safe and at a wide range via the Python interface,
	\item allow to add user defined objects in C++,
	\item allow to add user defined objects in Python (planned: 2021),
	\item allow to add user defined solvers in Python (finished: Q1 2020).
\ei
%
%++++++++++++++++++++++++++++++++++++++++++++++++++++++++++++++++
%++++++++++++++++++++++++++++++++++++++++++++++++++++++++++++++++
\mysubsubsection{Run a simple example in Spyder}
After performing the steps of the previous section, this section shows a simplistic model which helps you to check if \codeName\ runs on your computer.

In order to start, run the python interpreter Spyder.
For the following example, either 
\bi
	\item open Spyder and copy the example provided in Listing \ref{lst:firstexample} into a new file, or
	\item open \texttt{myFirstExample.py} from your \texttt{WorkingRelease} directory
\ei
Hereafter, press the play button or \texttt{F5} in Spyder.
%\lstinputlisting[language=Python]{../../main/bin/WorkingRelease/myFirstExample.py}
\pythonexternal[language=Python, frame=single, float, label=lst:firstexample, caption=My first example]{../../main/bin/WorkingRelease/myFirstExample.py}

If successful, the IPython Console of Spyder will print something like:
\plainlststyle
{\ttfamily \footnotesize
\begin{lstlisting}
  runfile('C:/DATA/cpp/EXUDYN_git/main/bin/WorkingRelease/myFirstExample.py', 
	  wdir='C:/DATA/cpp/EXUDYN_git/main/bin/WorkingRelease')
  Assemble nodes:
  Set initial system coordinates (for ODE2, ODE1 and Data coordinates) ...
  Time integration finished after 0.0011442 seconds.
\end{lstlisting}
}

If you check your current directory (where \texttt{myFirstExample.py} lies), you will find a new file \texttt{coordinatesSolution.txt}, which contains the results of your computation (with default values for time integration).
The beginning and end of the file should look like: \vspace{6pt}\\
{\ttfamily \footnotesize
\begin{lstlisting}[breaklines=true]
  #Exudyn generalized alpha solver solution file
  #simulation started=2019-11-14,20:35:12
  #columns contain: time, ODE2 displacements, ODE2 velocities, ODE2 accelerations, AE coordinates, ODE2 velocities
  #number of system coordinates [nODE2, nODE1, nAlgebraic, nData] = [2,0,0,0]
  #number of written coordinates [nODE2, nVel2, nAcc2, nODE1, nVel1, nAlgebraic, nData] = [2,2,2,0,0,0,0]
  #total columns exported  (excl. time) = 6
  #number of time steps (planned) = 100
  #
  0,0,0,0,0,0.0001,0
  0.02,2e-08,0,2e-06,0,0.0001,0
  0.03,4.5e-08,0,3e-06,0,0.0001,0
  0.04,8e-08,0,4e-06,0,0.0001,0
  0.05,1.25e-07,0,5e-06,0,0.0001,0

  ...

  0.96,4.608e-05,0,9.6e-05,0,0.0001,0
  0.97,4.7045e-05,0,9.7e-05,0,0.0001,0
  0.98,4.802e-05,0,9.8e-05,0,0.0001,0
  0.99,4.9005e-05,0,9.9e-05,0,0.0001,0
  1,5e-05,0,0.0001,0,0.0001,0
  #simulation finished=2019-11-14,20:35:12
  #Solver Info: errorOccurred=0,converged=1,solutionDiverged=0,total time steps=100,total Newton iterations=100,total Newton jacobians=100
\end{lstlisting}
}

Within this file, the first column shows the simulation time and the following columns provide solution of coordinates, their derivatives and Lagrange multipliers on system level. As expected, the $x$-coordinate of the point mass has constant acceleration $a=f/m=0.001/10=0.0001$, the velocity grows up to $0.0001$ after 1 second and the point mass moves $0.00005$ along the $x$-axis.
%
%++++++++++++++++++++++++++++++++++++++++++++++++++++++++++++++++
%++++++++++++++++++++++++++++++++++++++++++++++++++++++++++++++++
\mysubsection{FAQ -- Frequently asked questions}

\bn
  \item Where do I find the '.exe' file?
	\bi
	\item[$\ra$] \codeName\ is only available via the python interface as exudyn.pyd library, which is located in folder: \texttt{main/bin/WorkingRelease}. This means that you need to run python (best: Spyder) and import the \codeName\ module.
	\ei
	%
  \item What is the difference between MarkerBodyPosition and MarkerBodyRigid?
	\bi
	\item[$\ra$] Position markers (and nodes) do not have information on the orientation (rotation). For that reason, there is a difference between position based and rigid-body based markers. In case of a rigid body attached to ground with a SpringDamper, you can use both, MarkerBodyPosition or MarkerBodyRigid, markers. For a prismatic joint, you will need a MarkerBodyRigid.
	\ei
  \item Why can't I get the focus of the simulation window on startup (render window hidden)?
	\bi
	\item[$\ra$] Starting \codeName\ out of Spyder might not bring the simulation window to front, because of specific settings in Spyder(version 3.2.8), e.g., Tools$\ra$Preferences$\ra$Editor$\ra$Advanced settings: uncheck 'Maintain focus in the Editor after running cells or selections'; Alternatively, set \texttt{SC.visualizationSettings.window.alwaysOnTop=True} {\bf before} starting the renderer with \texttt{exu.StartRenderer()}
	\ei
	%
\en


