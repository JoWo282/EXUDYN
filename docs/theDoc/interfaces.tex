% definition of structures

%++++++++++++++++++++++++++++++++++++++
\mysubsection{Simulation settings}
This section includes hierarchical structures for simulation settings, e.g., time integration, static solver, Newton iteration and solution file export.

%+++++++++++++++++++++++++++++++++++
\mysubsubsection{SolutionSettings} \label{sec:SolutionSettings}
General settings for exporting the solution (results) of a simulation.\\ 
%
SolutionSettings has the following items:
%reference manual TABLE
\begin{center}
  \footnotesize
  \begin{longtable}{| p{4.2cm} | p{2.5cm} | p{0.3cm} | p{3.0cm} | p{6cm} |}
    \hline
    \bf Name & \bf type / function return type & \bf size & \bf default value / function args & \bf description \\ \hline
    writeSolutionToFile &     bool &      &     True &     flag (true/false), which determines if (global) solution vector is written to file; standard quantities that are written are: solution is written as displacements and coordinatesODE1; for additional coordinates in the solution file, see the options below\\ \hline
    appendToFile &     bool &      &     False &     flag (true/false); if true, solution and solverInformation is appended to existing file (otherwise created)\\ \hline
    writeFileHeader &     bool &      &     True &     flag (true/false); if true, file header is written (turn off, e.g. for multiple runs of time integration)\\ \hline
    writeFileFooter &     bool &      &     True &     flag (true/false); if true, information at end of simulation is written: convergence, total solution time, statistics\\ \hline
    solutionWritePeriod &     UReal &      &     0.01 &     time span (period), determines how often the solution is written during a simulation\\ \hline
    exportVelocities &     bool &      &     True &     add ODE2 velocities to solution file\\ \hline
    exportAccelerations &     bool &      &     True &     add ODE2 accelerations to solution file\\ \hline
    exportODE1Velocities &     bool &      &     True &     add coordinatesODE1\_t to solution file\\ \hline
    exportAlgebraicCoordinates &     bool &      &     True &     add algebraicCoordinates (=Lagrange multipliers) to solution file\\ \hline
    exportDataCoordinates &     bool &      &     True &     add DataCoordinates to solution file\\ \hline
    coordinatesSolutionFileName &     FileName &      &     'coordinatesSolution.txt' &     \tabnewline filename and (relative) path of solution file containing all coordinates versus time; directory will be created if it does not exist; character encoding of string is up to your filesystem, but for compatibility, it is recommended to use letters, numbers and '\_' only\\ \hline
    sensorsAppendToFile &     bool &      &     False &     flag (true/false); if true, sensor output is appended to existing file (otherwise created)\\ \hline
    sensorsWriteFileHeader &     bool &      &     True &     flag (true/false); if true, file header is written for sensor output (turn off, e.g. for multiple runs of time integration)\\ \hline
    sensorsWritePeriod &     UReal &      &     0.01 &     time span (period), determines how often the sensor output is written during a simulation\\ \hline
    solverInformationFileName &     FileName &      &     'solverInformation.txt' &     \tabnewline filename and (relative) path of text file showing detailed information during solving; detail level according to yourSolver.verboseModeFile; if solutionSettings.appendToFile is true, the information is appended in every solution step; directory will be created if it does not exist; character encoding of string is up to your filesystem, but for compatibility, it is recommended to use letters, numbers and '\_' only\\ \hline
    solutionInformation &     String &      &     '' &     special information added to header of solution file (e.g. parameters and settings, modes, ...); character encoding my be UTF-8, restricted to characters in \refSection{sec:utf8}, but for compatibility, it is recommended to use ASCII characters only (95 characters, see wiki)\\ \hline
    outputPrecision &     UInt &      &     10 &     precision for floating point numbers written to solution and sensor files\\ \hline
    recordImagesInterval &     Real &      &     -1. &     record frames (images) during solving: amount of time to wait until next image (frame) is recorded; set recordImages = -1. if no images shall be recorded; set, e.g., recordImages = 0.01 to record an image every 10 milliseconds (requires that the time steps / load steps are sufficiently small!); for file names, etc., see VisualizationSettings.exportImages\\ \hline
	  \end{longtable}
	\end{center}

%+++++++++++++++++++++++++++++++++++
\mysubsubsection{NumericalDifferentiationSettings} \label{sec:NumericalDifferentiationSettings}
Settings for numerical differentiation of a function (needed for computation of numerical jacobian e.g. in implizit integration); HOTINT1: relativeEpsilon * Maximum(minimumCoordinateSize, fabs(x(i))).\\ 
%
NumericalDifferentiationSettings has the following items:
%reference manual TABLE
\begin{center}
  \footnotesize
  \begin{longtable}{| p{4.2cm} | p{2.5cm} | p{0.3cm} | p{3.0cm} | p{6cm} |}
    \hline
    \bf Name & \bf type / function return type & \bf size & \bf default value / function args & \bf description \\ \hline
    relativeEpsilon &     UReal &      &     1e-7 &     relative differentiation parameter epsilon; the numerical differentiation parameter $\varepsilon$ follows from the formula ($\varepsilon = \varepsilon_\mathrm{relative}*max(q_{min}, |q_i + [q^{Ref}_i]|)$, with $\varepsilon_\mathrm{relative}$=relativeEpsilon, $q_{min} = $minimumCoordinateSize, $q_i$ is the current coordinate which is differentiated, and $qRef_i$ is the reference coordinate of the current coordinate\\ \hline
    minimumCoordinateSize &     UReal &      &     1e-2 &     minimum size of coordinates in relative differentiation parameter\\ \hline
    doSystemWideDifferentiation &     bool &      &     False &     True: system wide differentiation (e.g. all ODE2 equations w.r.t. all ODE2 coordinates); False: only local (object) differentiation\\ \hline
    addReferenceCoordinatesToEpsilon &     \tabnewline bool &      &     False &     True: for the size estimation of the differentiation parameter, the reference coordinate $q^{Ref}_i$ is added to ODE2 coordinates --> see; False: only the current coordinate is used for size estimation of the differentiation parameter\\ \hline
	  \end{longtable}
	\end{center}

%+++++++++++++++++++++++++++++++++++
\mysubsubsection{DiscontinuousSettings} \label{sec:DiscontinuousSettings}
Settings for discontinuous iterations, as in contact, friction, plasticity and general switching phenomena.\\ 
%
DiscontinuousSettings has the following items:
%reference manual TABLE
\begin{center}
  \footnotesize
  \begin{longtable}{| p{4.2cm} | p{2.5cm} | p{0.3cm} | p{3.0cm} | p{6cm} |}
    \hline
    \bf Name & \bf type / function return type & \bf size & \bf default value / function args & \bf description \\ \hline
    maxIterations &     UInt &      &     5 &     maximum number of discontinuous (post Newton) iterations\\ \hline
    ignoreMaxIterations &     bool &      &     True &     continue solver if maximum number of discontinuous (post Newton) iterations is reached (ignore tolerance)\\ \hline
    iterationTolerance &     UReal &      &     1 &     absolute tolerance for discontinuous (post Newton) iterations; the errors represent absolute residuals and can be quite high\\ \hline
	  \end{longtable}
	\end{center}

%+++++++++++++++++++++++++++++++++++
\mysubsubsection{NewtonSettings} \label{sec:NewtonSettings}
Settings for Newton method used in static or dynamic simulation.\\ 
%
NewtonSettings has the following items:
%reference manual TABLE
\begin{center}
  \footnotesize
  \begin{longtable}{| p{4.2cm} | p{2.5cm} | p{0.3cm} | p{3.0cm} | p{6cm} |}
    \hline
    \bf Name & \bf type / function return type & \bf size & \bf default value / function args & \bf description \\ \hline
    numericalDifferentiation &     NumericalDifferentiationSettings &      &      &     numerical differentiation parameters for numerical jacobian (e.g. Newton in static solver or implicit time integration)\\ \hline
    useNumericalDifferentiation &     bool &      &     False &     flag (true/false); false = perform direct computation of jacobian, true = use numerical differentiation for jacobian\\ \hline
    useNewtonSolver &     bool &      &     True &     flag (true/false); false = linear computation, true = use Newton solver for nonlinear solution\\ \hline
    relativeTolerance &     UReal &      &     1e-8 &     relative tolerance of residual for Newton (general goal of Newton is to decrease the residual by this factor)\\ \hline
    absoluteTolerance &     UReal &      &     1e-10 &     absolute tolerance of residual for Newton (needed e.g. if residual is fulfilled right at beginning); condition: sqrt(q*q)/numberOfCoordinates <= absoluteTolerance\\ \hline
    weightTolerancePerCoordinate &     bool &      &     False &     flag (true/false); false = compute error as L2-Norm of residual; true = compute error as (L2-Norm of residual) / (sqrt(number of coordinates)), which can help to use common tolerance independent of system size\\ \hline
    newtonResidualMode &     UInt &      &     0 &     0 ... use residual for computation of error (standard); 1 ... use ODE2 and ODE1 newton increment for error (set relTol and absTol to same values!) ==> may be advantageous if residual is zero, e.g., in kinematic analysis; TAKE CARE with this flag\\ \hline
    adaptInitialResidual &     bool &      &     True &     flag (true/false); false = standard; True: if initialResidual is very small (or zero), it may increas dramatically in first step; to achieve relativeTolerance, the initialResidual will by updated by a higher residual within the first Newton iteration\\ \hline
    modifiedNewtonContractivity &     PReal &      &     0.5 &     maximum contractivity (=reduction of error in every Newton iteration) accepted by modified Newton; if contractivity is greater, a Jacobian update is computed\\ \hline
    useModifiedNewton &     bool &      &     False &     True: compute Jacobian only at first step; no Jacobian updates per step; False: Jacobian computed in every step\\ \hline
    modifiedNewtonJacUpdatePerStep &     \tabnewline bool &      &     False &     True: compute Jacobian at every time step, but not in every iteration (except for bad convergence ==> switch to full Newton)\\ \hline
    maxIterations &     UInt &      &     25 &     maximum number of iterations (including modified + restart Newton steps); after that iterations, the static/dynamic solver stops with error\\ \hline
    maxModifiedNewtonIterations &     UInt &      &     8 &     maximum number of iterations for modified Newton (without Jacobian update); after that number of iterations, the modified Newton method gets a jacobian update and is further iterated\\ \hline
    maxModifiedNewtonRestartIterations &     \tabnewline UInt &      &     7 &     maximum number of iterations for modified Newton after aJacobian update; after that number of iterations, the full Newton method is started for this step\\ \hline
    maximumSolutionNorm &     UReal &      &     1e38 &     this is the maximum allowed value for solutionU.L2NormSquared() which is the square of the square norm (value=$u_1^2$+$u_2^2$+...), and solutionV/A...; if the norm of solution vectors are larger, Newton method is stopped; the default value is chosen such that it would still work for single precision numbers (float)\\ \hline
	  \end{longtable}
	\end{center}

%+++++++++++++++++++++++++++++++++++
\mysubsubsection{GeneralizedAlphaSettings} \label{sec:GeneralizedAlphaSettings}
Settings for generalized-alpha, implicit trapezoidal or Newmark time integration methods.\\ 
%
GeneralizedAlphaSettings has the following items:
%reference manual TABLE
\begin{center}
  \footnotesize
  \begin{longtable}{| p{4.2cm} | p{2.5cm} | p{0.3cm} | p{3.0cm} | p{6cm} |}
    \hline
    \bf Name & \bf type / function return type & \bf size & \bf default value / function args & \bf description \\ \hline
    newmarkBeta &     UReal &      &     0.25 &     value beta for Newmark method; default value beta = $\frac 1 4$ corresponds to (undamped) trapezoidal rule\\ \hline
    newmarkGamma &     UReal &      &     0.5 &     value gamma for Newmark method; default value gamma = $\frac 1 2$ corresponds to (undamped) trapezoidal rule\\ \hline
    useIndex2Constraints &     bool &      &     False &     set useIndex2Constraints = true in order to use index2 (velocity level constraints) formulation\\ \hline
    useNewmark &     bool &      &     False &     if true, use Newmark method with beta and gamma instead of generalized-Alpha\\ \hline
    spectralRadius &     UReal &      &     0.9 &     spectral radius for Generalized-alpha solver; set this value to 1 for no damping or to 0 < spectralRadius < 1 for damping of high-frequency dynamics; for position-level constraints (index 3), spectralRadius must be < 1\\ \hline
    computeInitialAccelerations &     bool &      &     True &     True: compute initial accelerations from system EOM in acceleration form; NOTE that initial accelerations that are following from user functions in constraints are not considered for now! False: use zero accelerations\\ \hline
	  \end{longtable}
	\end{center}

%+++++++++++++++++++++++++++++++++++
\mysubsubsection{ExplicitIntegrationSettings} \label{sec:ExplicitIntegrationSettings}
Settings for generalized-alpha, implicit trapezoidal or Newmark time integration methods.\\ 
%
ExplicitIntegrationSettings has the following items:
%reference manual TABLE
\begin{center}
  \footnotesize
  \begin{longtable}{| p{4.2cm} | p{2.5cm} | p{0.3cm} | p{3.0cm} | p{6cm} |}
    \hline
    \bf Name & \bf type / function return type & \bf size & \bf default value / function args & \bf description \\ \hline
    eliminateConstraints &     bool &      &     True &     True: make explicit solver work for simple CoordinateConstraints, which are eliminated for ground constraints (e.g. fixed nodes in finite element models). False: incompatible constraints are ignored (BE CAREFUL)!\\ \hline
    useLieGroupIntegration &     bool &      &     True &     True: use Lie group integration for rigid body nodes; must be turned on for Lie group nodes, but also improves integration of other rigid body nodes. Only available for RK44 integrator.\\ \hline
    dynamicSolverType &     DynamicSolverType &      &     DynamicSolverType::DOPRI5 &     \tabnewline selection of explicit solver type (DOPRI5, ExplicitEuler, ExplicitMidpoint, RK44, RK67, ...), for detailed description see DynamicSolverType, \refSection{sec:DynamicSolverType}, but only referring to explicit solvers.\\ \hline
	  \end{longtable}
	\end{center}

%+++++++++++++++++++++++++++++++++++
\mysubsubsection{TimeIntegrationSettings} \label{sec:TimeIntegrationSettings}
General parameters used in time integration; specific parameters are provided in the according solver settings, e.g. for generalizedAlpha.\\ 
%
TimeIntegrationSettings has the following items:
%reference manual TABLE
\begin{center}
  \footnotesize
  \begin{longtable}{| p{4.2cm} | p{2.5cm} | p{0.3cm} | p{3.0cm} | p{6cm} |}
    \hline
    \bf Name & \bf type / function return type & \bf size & \bf default value / function args & \bf description \\ \hline
    newton &     NewtonSettings &      &      &     parameters for Newton method; used for implicit time integration methods only\\ \hline
    discontinuous &     DiscontinuousSettings &      &      &     parameters for treatment of discontinuities\\ \hline
    startTime &     UReal &      &     0 &     $t_{start}$: start time of time integration (usually set to zero)\\ \hline
    endTime &     UReal &      &     1 &     $t_{end}$: end time of time integration\\ \hline
    numberOfSteps &     PInt &      &     100 &     $n_{steps}$: number of steps in time integration; (maximum) stepSize $h$ is computed from $h = \frac{t_{end} - t_{start}}{n_{steps}}$; for automatic stepsize control, this stepSize is the maximum steps size, $h_{max} = h$\\ \hline
    adaptiveStep &     bool &      &     True &     True: the step size may be reduced if step fails; no automatic stepsize control\\ \hline
    adaptiveStepRecoverySteps &     UInt &      &     10 &     Number of steps needed after which steps will be increased after previous step reduction due to discontinuousIteration or Newton errors\\ \hline
    adaptiveStepIncrease &     UReal &      &     2 &     Multiplicative factor (MUST BE > 1) for step size to increase after previous step reduction due to discontinuousIteration or Newton errors\\ \hline
    adaptiveStepDecrease &     UReal &      &     0.5 &     Multiplicative factor (MUST BE: 0 < factor < 1) for step size to decrese due to discontinuousIteration or Newton errors\\ \hline
    automaticStepSize &     bool &      &     True &     True: for specific integrators with error control (e.g., DOPRI5), compute automatic step size based on error estimation; False: constant step size (step may be reduced if adaptiveStep=True); the maximum stepSize reads $h = h_{max} = \frac{t_{end} - t_{start}}{n_{steps}}$\\ \hline
    minimumStepSize &     PReal &      &     1e-8 &     $h_{min}$: if automaticStepSize=True or adaptiveStep=True: lower limit of time step size, before integrator stops with adaptiveStep; lower limit of automaticStepSize control (continues but raises warning)\\ \hline
    initialStepSize &     UReal &      &     0 &     $h_{init}$: if automaticStepSize=True, initial step size; if initialStepSize==0, max. stepSize, which is (endTime-startTime)/numberOfSteps, is used as initial guess; a good choice of initialStepSize may help the solver to start up faster.\\ \hline
    absoluteTolerance &     UReal &      &     1e-8 &     $a_{tol}$: if automaticStepSize=True, absolute tolerance for the error control; must fulfill $a_{tol} > 0$; see \refSection{sec:ExplicitSolver}\\ \hline
    relativeTolerance &     UReal &      &     1e-8 &     $r_{tol}$: if automaticStepSize=True, relative tolerance for the error control; must fulfill $r_{tol} \ge 0$; see \refSection{sec:ExplicitSolver}\\ \hline
    stepSizeSafety &     UReal &      &     0.90 &     $r_{sfty}$: if automaticStepSize=True, a safety factor added to estimated optimal step size, in order to prevent from many rejected steps, see \refSection{sec:ExplicitSolver}. Make this factor smaller if many steps are rejected.\\ \hline
    stepSizeMaxIncrease &     UReal &      &     2 &     $f_{maxInc}$: if automaticStepSize=True, maximum increase of step size per step, see \refSection{sec:ExplicitSolver}; make this factor smaller (but $> 1$) if too many rejected steps\\ \hline
    preStepPyExecute &     String &      &     '' &     DEPRECATED, use mbs.SetPreStepUserFunction(...); Python code to be executed prior to every step and after last step, e.g. for postprocessing\\ \hline
    simulateInRealtime &     bool &      &     False &     True: simulate in realtime; the solver waits for computation of the next step until the CPU time reached the simulation time; if the simulation is slower than realtime, it simply continues\\ \hline
    realtimeFactor &     PReal &      &     1 &     if simulateInRealtime=True, this factor is used to make the simulation slower than realtime (factor < 1) or faster than realtime (factor > 1)\\ \hline
    verboseMode &     UInt &      &     0 &     0 ... no output, 1 ... show short step information every 2 seconds (error), 2 ... show every step information, 3 ... show also solution vector, 4 ... show also mass matrix and jacobian (implicit methods), 5 ... show also Jacobian inverse (implicit methods)\\ \hline
    verboseModeFile &     UInt &      &     0 &     same behaviour as verboseMode, but outputs all solver information to file\\ \hline
    stepInformation &     UInt &      &     2 &     0 ... only current step time, 1 ... show time to go, 2 ... show newton iterations (Nit) per step, 3 ... show discontinuous iterations (Dit) and newton jacobians (jac) per step\\ \hline
    generalizedAlpha &     GeneralizedAlphaSettings &      &      &     parameters for generalized-alpha, implicit trapezoidal rule or Newmark (options only apply for these methods)\\ \hline
    explicitIntegration &     ExplicitIntegrationSettings &      &      &     special parameters for explicit time integration\\ \hline
	  \end{longtable}
	\end{center}

%+++++++++++++++++++++++++++++++++++
\mysubsubsection{StaticSolverSettings} \label{sec:StaticSolverSettings}
Settings for static solver linear or nonlinear (Newton).\\ 
%
StaticSolverSettings has the following items:
%reference manual TABLE
\begin{center}
  \footnotesize
  \begin{longtable}{| p{4.2cm} | p{2.5cm} | p{0.3cm} | p{3.0cm} | p{6cm} |}
    \hline
    \bf Name & \bf type / function return type & \bf size & \bf default value / function args & \bf description \\ \hline
    newton &     NewtonSettings &      &      &     parameters for Newton method (e.g. in static solver or time integration)\\ \hline
    discontinuous &     DiscontinuousSettings &      &      &     parameters for treatment of discontinuities\\ \hline
    numberOfLoadSteps &     PInt &      &     1 &     number of load steps; if numberOfLoadSteps=1, no load steps are used and full forces are applied at once\\ \hline
    loadStepDuration &     PReal &      &     1 &     quasi-time for all load steps (added to current time in load steps)\\ \hline
    loadStepStart &     UReal &      &     0 &     a quasi time, which can be used for the output (first column) as well as for time-dependent forces; quasi-time is increased in every step i by loadStepDuration/numberOfLoadSteps; loadStepTime = loadStepStart + i*loadStepDuration/numberOfLoadSteps, but loadStepStart untouched ==> increment by user\\ \hline
    loadStepGeometric &     bool &      &     False &     if loadStepGeometric=false, the load steps are incremental (arithmetic series, e.g. 0.1,0.2,0.3,...); if true, the load steps are increased in a geometric series, e.g. for $n=8$ numberOfLoadSteps and $d = 1000$ loadStepGeometricRange, it follows: $1000^{1/8}/1000=0.00237$, $1000^{2/8}/1000=0.00562$, $1000^{3/8}/1000=0.0133$, ..., $1000^{7/8}/1000=0.422$, $1000^{8/8}/1000=1$\\ \hline
    loadStepGeometricRange &     PReal &      &     1000 &     if loadStepGeometric=true, the load steps are increased in a geometric series, see loadStepGeometric\\ \hline
    useLoadFactor &     bool &      &     True &     True: compute a load factor $\in [0,1]$ from static step time; all loads are scaled by the load factor; False: loads are always scaled with 1 -- use this option if time dependent loads use a userFunction\\ \hline
    stabilizerODE2term &     UReal &      &     0 &     add mass-proportional stabilizer term in ODE2 part of jacobian for stabilization (scaled ), e.g. of badly conditioned problems; the diagnoal terms are scaled with $stabilizer = (1-loadStepFactor^2)$, and go to zero at the end of all load steps: $loadStepFactor=1$ -> $stabilizer = 0$\\ \hline
    adaptiveStep &     bool &      &     True &     True: use step reduction if step fails; False: fixed step size\\ \hline
    adaptiveStepRecoverySteps &     UInt &      &     4 &     Number of steps needed after which steps will be increased after previous step reduction due to discontinuousIteration or Newton errors\\ \hline
    adaptiveStepIncrease &     UReal &      &     2 &     Multiplicative factor (MUST BE > 1) for step size to increase after previous step reduction due to discontinuousIteration or Newton errors\\ \hline
    adaptiveStepDecrease &     UReal &      &     0.25 &     Multiplicative factor (MUST BE: 0 < factor < 1) for step size to decrese due to discontinuousIteration or Newton errors\\ \hline
    minimumStepSize &     PReal &      &     1e-8 &     lower limit of step size, before nonlinear solver stops\\ \hline
    verboseMode &     UInt &      &     1 &     0 ... no output, 1 ... show errors and load steps, 2 ... show short Newton step information (error), 3 ... show also solution vector, 4 ... show also jacobian, 5 ... show also Jacobian inverse\\ \hline
    verboseModeFile &     UInt &      &     0 &     same behaviour as verboseMode, but outputs all solver information to file\\ \hline
    stepInformation &     UInt &      &     2 &     0 ... only current step time, 1 ... show time to go, 2 ... show newton iterations (Nit) per step, 3 ... show discontinuous iterations (Dit) and newton jacobians (jac) per step\\ \hline
    preStepPyExecute &     String &      &     '' &     DEPRECATED, use mbs.SetPreStepUserFunction(...); Python code to be executed prior to every load step and after last step, e.g. for postprocessing\\ \hline
	  \end{longtable}
	\end{center}

%+++++++++++++++++++++++++++++++++++
\mysubsubsection{LinearSolverSettings} \label{sec:LinearSolverSettings}
Settings for linear solver, both dense and sparse (Eigen).\\ 
%
LinearSolverSettings has the following items:
%reference manual TABLE
\begin{center}
  \footnotesize
  \begin{longtable}{| p{4.2cm} | p{2.5cm} | p{0.3cm} | p{3.0cm} | p{6cm} |}
    \hline
    \bf Name & \bf type / function return type & \bf size & \bf default value / function args & \bf description \\ \hline
    pivotTreshold &     PReal &      &     0 &     treshold for dense linear solver, can be used to detect close to singular solutions, setting this to, e.g., 1e-12; solver then reports on equations that are causing close to singularity\\ \hline
    ignoreRedundantConstraints &     bool &      &     False &     [ONLY implemented for dense matrices] False: standard way, fails if redundant equations or singular matrices occur; True: if redundant constraints appear, the solver tries to resolve them by setting according Lagrange multipliers to zero; in case of redundant constraints, this may help, but it may lead to erroneous behaviour\\ \hline
    ignoreSingularJacobian &     bool &      &     False &     [ONLY implemented for dense matrices] False: standard way, fails if jacobian is singular; True: if singularities appear in jacobian (e.g. no equation attributed to a node, redundant equations, zero mass matrix, zero eigenvalue for static problem, etc.), the jacobian inverse is resolved such that according solution variables are set to zero; this may help, but it MAY LEAD TO ERRONEOUS BEHAVIOUR; for static problems, this may suppress static motion or resolve problems in case of instabilities, but should in general be considered with care!\\ \hline
    showCausingItems &     bool &      &     True &     False: no output, if solver fails; True: if redundant equations appear, they are resolved such that according solution variables are set to zero; in case of redundant constraints, this may help, but it may lead to erroneous behaviour; for static problems, this may suppress static motion or resolve problems in case of instabilities, but should in general be considered with care!\\ \hline
	  \end{longtable}
	\end{center}

%+++++++++++++++++++++++++++++++++++
\mysubsubsection{SimulationSettings} \label{sec:SimulationSettings}
General Settings for simulation; according settings for solution and solvers are given in subitems of this structure. \\ 
%
SimulationSettings has the following items:
%reference manual TABLE
\begin{center}
  \footnotesize
  \begin{longtable}{| p{4.2cm} | p{2.5cm} | p{0.3cm} | p{3.0cm} | p{6cm} |}
    \hline
    \bf Name & \bf type / function return type & \bf size & \bf default value / function args & \bf description \\ \hline
    timeIntegration &     TimeIntegrationSettings &      &      &     time integration parameters\\ \hline
    solutionSettings &     SolutionSettings &      &      &     settings for solution files\\ \hline
    staticSolver &     StaticSolverSettings &      &      &     static solver parameters\\ \hline
    linearSolverSettings &     LinearSolverSettings &      &      &     linear solver parameters (used for dense and sparse solvers)\\ \hline
    linearSolverType &     LinearSolverType &      &     LinearSolverType::EXUdense &     \tabnewline selection of numerical linear solver: exu.LinearSolverType.EXUdense (dense matrix inverse), exu.LinearSolverType.EigenSparse (sparse matrix LU-factorization), ... (enumeration type)\\ \hline
    cleanUpMemory &     bool &      &     False &     True: solvers will free memory at exit (recommended for large systems); False: keep allocated memory for repeated computations to increase performance\\ \hline
    displayStatistics &     bool &      &     False &     display general computation information at end of time step (steps, iterations, function calls, step rejections, ...\\ \hline
    displayComputationTime &     bool &      &     False &     display computation time statistics at end of solving\\ \hline
    pauseAfterEachStep &     bool &      &     False &     pause after every time step or static load step(user press SPACE)\\ \hline
    outputPrecision &     UInt &      &     6 &     precision for floating point numbers written to console; e.g. values written by solver\\ \hline
    numberOfThreads &     PInt &      &     1 &     number of threads used for parallel computation (1 == scalar processing); not yet implemented (status: Nov 2019)\\ \hline
	  \end{longtable}
	\end{center}

%++++++++++++++++++++++++++++++++++++++
\mysubsection{Visualization settings}
This section includes hierarchical structures for visualization settings, e.g., drawing of nodes, bodies, connectors, loads and markers and furthermore openGL, window and save image options.

%+++++++++++++++++++++++++++++++++++
\mysubsubsection{VSettingsGeneral} \label{sec:VSettingsGeneral}
General settings for visualization.\\ 
%
VSettingsGeneral has the following items:
%reference manual TABLE
\begin{center}
  \footnotesize
  \begin{longtable}{| p{4.2cm} | p{2.5cm} | p{0.3cm} | p{3.0cm} | p{6cm} |}
    \hline
    \bf Name & \bf type / function return type & \bf size & \bf default value / function args & \bf description \\ \hline
    graphicsUpdateInterval &     float &      &     0.1 &     interval of graphics update during simulation in seconds; 0.1 = 10 frames per second; low numbers might slow down computation speed\\ \hline
    autoFitScene &     bool &      &     True &     automatically fit scene within first second after StartRenderer()\\ \hline
    textSize &     float &      &     12. &     general text size (font size) in pixels if not overwritten; if useWindowsMonitorScaleFactor=True, the the textSize is multplied with the windows monitor scaling factor for larger texts on on high resolution monitors; for bitmap fonts, the maximum size of any font (standard/large/huge) is limited to 256 (which is not recommended, especially if you do not have a powerful graphics card)\\ \hline
    textColor &     Float4 &     4 &     [0.,0.,0.,1.0] &     \tabnewline general text color (default); used for system texts in render window\\ \hline
    useWindowsMonitorScaleFactor &     bool &      &     True &     the windows monitor scaling is used for increased visibility of texts on high resolution monitors; based on GLFW glfwGetWindowContentScale\\ \hline
    useBitmapText &     bool &      &     True &     if true, texts are displayed using pre-defined bitmaps for the text; may increase the complexity of your scene, e.g., if many (>10000) node numbers shown\\ \hline
    minSceneSize &     float &      &     0.1 &     minimum scene size for initial scene size and for autoFitScene, to avoid division by zero; SET GREATER THAN ZERO\\ \hline
    backgroundColor &     Float4 &     4 &     [1.0,1.0,1.0,1.0] &     \tabnewline red, green, blue and alpha values for background color of render window (white=[1,1,1,1]; black = [0,0,0,1])\\ \hline
    backgroundColorBottom &     Float4 &     4 &     [0.8,0.8,1.0,1.0] &     \tabnewline red, green, blue and alpha values for bottom background color in case that useGradientBackground = True\\ \hline
    useGradientBackground &     bool &      &     False &     true = use vertical gradient for background; \\ \hline
    coordinateSystemSize &     float &      &     5. &     size of coordinate system relative to font size\\ \hline
    drawCoordinateSystem &     bool &      &     True &     false = no coordinate system shown\\ \hline
    drawWorldBasis &     bool &      &     False &     true = draw world basis coordinate system at (0,0,0)\\ \hline
    worldBasisSize &     float &      &     1.0 &     size of world basis coordinate system\\ \hline
    showHelpOnStartup &     PInt &      &     5 &     seconds to show help message on startup (0=deactivate)\\ \hline
    showComputationInfo &     bool &      &     True &     true = show (hide) all computation information including EXUDYN and version\\ \hline
    showSolutionInformation &     bool &      &     True &     true = show solution information (from simulationSettings.solution)\\ \hline
    showSolverInformation &     bool &      &     True &     true = solver name and further information shown in render window\\ \hline
    showSolverTime &     bool &      &     True &     true = solver current time shown in render window\\ \hline
    renderWindowString &     String &      &     '' &     string shown in render window (use this, e.g., for debugging, etc.; written below EXUDYN, similar to solutionInformation in SimulationSettings.solutionSettings)\\ \hline
    pointSize &     float &      &     0.01 &     global point size (absolute)\\ \hline
    circleTiling &     PInt &      &     16 &     global number of segments for circles; if smaller than 2, 2 segments are used (flat)\\ \hline
    cylinderTiling &     PInt &      &     16 &     global number of segments for cylinders; if smaller than 2, 2 segments are used (flat)\\ \hline
    sphereTiling &     PInt &      &     6 &     global number of segments for spheres; if smaller than 2, 2 segments are used (flat)\\ \hline
    axesTiling &     PInt &      &     12 &     global number of segments for drawing axes cylinders and cones (reduce this number, e.g. to 4, if many axes are drawn)\\ \hline
    threadSafeGraphicsUpdate &     bool &      &     True &     true = updating of visualization is threadsafe, but slower for complicated models; deactivate this to speed up computation, but activate for generation of animations; may be improved in future by adding a safe visualizationUpdate state\\ \hline
    useMultiThreadedRendering &     bool &      &     True &     true = rendering is done in separate thread; false = no separate thread, which may be more stable but has lagging interaction for large models (do not interact with models during simulation); set this parameter before call to exudyn.StartRenderer(); MAC OS: uses always false, because MAC OS does not support multi threaded GLFW\\ \hline
	  \end{longtable}
	\end{center}

%+++++++++++++++++++++++++++++++++++
\mysubsubsection{VSettingsContour} \label{sec:VSettingsContour}
Settings for contour plots; use these options to visualize field data, such as displacements, stresses, strains, etc. for bodies, nodes and finite elements.\\ 
%
VSettingsContour has the following items:
%reference manual TABLE
\begin{center}
  \footnotesize
  \begin{longtable}{| p{4.2cm} | p{2.5cm} | p{0.3cm} | p{3.0cm} | p{6cm} |}
    \hline
    \bf Name & \bf type / function return type & \bf size & \bf default value / function args & \bf description \\ \hline
    outputVariableComponent &     Int &     1 &     0 &     select the component of the chosen output variable; e.g., for displacements, 3 components are available: 0 == x, 1 == y, 2 == z component; for stresses, 6 components are available, see OutputVariableType description; to draw the norm of a outputVariable, set component to -1; if a certain component is not available by certain objects or nodes, no value is drawn (using default color)\\ \hline
    outputVariable &     OutputVariableType &      &     OutputVariableType::\_None &     \tabnewline selected contour plot output variable type; select OutputVariableType.\_None to deactivate contour plotting.\\ \hline
    minValue &     float &     1 &     0 &     minimum value for contour plot; set manually, if automaticRange == False\\ \hline
    maxValue &     float &     1 &     1 &     maximum value for contour plot; set manually, if automaticRange == False\\ \hline
    automaticRange &     bool &      &     True &     if true, the contour plot value range is chosen automatically to the maximum range\\ \hline
    reduceRange &     bool &      &     True &     if true, the contour plot value range is also reduced; better for static computation; in dynamic computation set this option to false, it can reduce visualization artifacts; you should also set minVal to max(float) and maxVal to min(float)\\ \hline
    showColorBar &     bool &      &     True &     show the colour bar with minimum and maximum values for the contour plot\\ \hline
    colorBarTiling &     PInt &     1 &     12 &     number of tiles (segements) shown in the colorbar for the contour plot\\ \hline
	  \end{longtable}
	\end{center}

%+++++++++++++++++++++++++++++++++++
\mysubsubsection{VSettingsNodes} \label{sec:VSettingsNodes}
Visualization settings for nodes.\\ 
%
VSettingsNodes has the following items:
%reference manual TABLE
\begin{center}
  \footnotesize
  \begin{longtable}{| p{4.2cm} | p{2.5cm} | p{0.3cm} | p{3.0cm} | p{6cm} |}
    \hline
    \bf Name & \bf type / function return type & \bf size & \bf default value / function args & \bf description \\ \hline
    show &     bool &      &     True &     flag to decide, whether the nodes are shown\\ \hline
    showNumbers &     bool &      &     False &     flag to decide, whether the node number is shown\\ \hline
    drawNodesAsPoint &     bool &      &     True &     simplified/faster drawing of nodes; uses general->pointSize as drawing size; if drawNodesAsPoint==True, the basis of the node will be drawn with lines\\ \hline
    showBasis &     bool &      &     False &     show basis (three axes) of coordinate system in 3D nodes\\ \hline
    basisSize &     float &      &     0.2 &     size of basis for nodes\\ \hline
    tiling &     PInt &      &     4 &     tiling for node if drawn as sphere; used to lower the amount of triangles to draw each node; if drawn as circle, this value is multiplied with 4\\ \hline
    defaultSize &     float &      &     -1. &     global node size; if -1.f, node size is relative to openGL.initialMaxSceneSize\\ \hline
    defaultColor &     Float4 &     4 &     [0.2,0.2,1.,1.] &     \tabnewline default cRGB olor for nodes; 4th value is alpha-transparency\\ \hline
    showNodalSlopes &     UInt &      &     False &     draw nodal slope vectors, e.g. in ANCF beam finite elements\\ \hline
	  \end{longtable}
	\end{center}

%+++++++++++++++++++++++++++++++++++
\mysubsubsection{VSettingsBeams} \label{sec:VSettingsBeams}
Visualization settings for beam finite elements.\\ 
%
VSettingsBeams has the following items:
%reference manual TABLE
\begin{center}
  \footnotesize
  \begin{longtable}{| p{4.2cm} | p{2.5cm} | p{0.3cm} | p{3.0cm} | p{6cm} |}
    \hline
    \bf Name & \bf type / function return type & \bf size & \bf default value / function args & \bf description \\ \hline
    axialTiling &     PInt &      &     8 &     number of segments to discretise the beams axis\\ \hline
	  \end{longtable}
	\end{center}

%+++++++++++++++++++++++++++++++++++
\mysubsubsection{VSettingsBodies} \label{sec:VSettingsBodies}
Visualization settings for bodies.\\ 
%
VSettingsBodies has the following items:
%reference manual TABLE
\begin{center}
  \footnotesize
  \begin{longtable}{| p{4.2cm} | p{2.5cm} | p{0.3cm} | p{3.0cm} | p{6cm} |}
    \hline
    \bf Name & \bf type / function return type & \bf size & \bf default value / function args & \bf description \\ \hline
    show &     bool &      &     True &     flag to decide, whether the bodies are shown\\ \hline
    showNumbers &     bool &      &     False &     flag to decide, whether the body(=object) number is shown\\ \hline
    defaultSize &     Float3 &     3 &     [1.,1.,1.] &     \tabnewline global body size of xyz-cube\\ \hline
    defaultColor &     Float4 &     4 &     [0.3,0.3,1.,1.] &     \tabnewline default cRGB olor for bodies; 4th value is \\ \hline
    deformationScaleFactor &     float &      &     1 &     global deformation scale factor; also applies to nodes, if drawn; used for scaled drawing of (linear) finite elements, beams, etc.\\ \hline
    beams &     VSettingsBeams &      &      &     visualization settings for beams (e.g. ANCFCable or other beam elements)\\ \hline
	  \end{longtable}
	\end{center}

%+++++++++++++++++++++++++++++++++++
\mysubsubsection{VSettingsConnectors} \label{sec:VSettingsConnectors}
Visualization settings for connectors.\\ 
%
VSettingsConnectors has the following items:
%reference manual TABLE
\begin{center}
  \footnotesize
  \begin{longtable}{| p{4.2cm} | p{2.5cm} | p{0.3cm} | p{3.0cm} | p{6cm} |}
    \hline
    \bf Name & \bf type / function return type & \bf size & \bf default value / function args & \bf description \\ \hline
    show &     bool &      &     True &     flag to decide, whether the connectors are shown\\ \hline
    showNumbers &     bool &      &     False &     flag to decide, whether the connector(=object) number is shown\\ \hline
    defaultSize &     float &      &     0.1 &     global connector size; if -1.f, connector size is relative to maxSceneSize\\ \hline
    showJointAxes &     bool &      &     False &     flag to decide, whether contact joint axes of 3D joints are shown\\ \hline
    jointAxesLength &     float &      &     0.2 &     global joint axes length\\ \hline
    jointAxesRadius &     float &      &     0.02 &     global joint axes radius\\ \hline
    showContact &     bool &      &     False &     flag to decide, whether contact points, lines, etc. are shown\\ \hline
    springNumberOfWindings &     PInt &      &     8 &     number of windings for springs drawn as helical spring\\ \hline
    contactPointsDefaultSize &     float &      &     0.02 &     global contact points size; if -1.f, connector size is relative to maxSceneSize\\ \hline
    defaultColor &     Float4 &     4 &     [0.2,0.2,1.,1.] &     \tabnewline default cRGB olor for connectors; 4th value is alpha-transparency\\ \hline
	  \end{longtable}
	\end{center}

%+++++++++++++++++++++++++++++++++++
\mysubsubsection{VSettingsMarkers} \label{sec:VSettingsMarkers}
Visualization settings for markers.\\ 
%
VSettingsMarkers has the following items:
%reference manual TABLE
\begin{center}
  \footnotesize
  \begin{longtable}{| p{4.2cm} | p{2.5cm} | p{0.3cm} | p{3.0cm} | p{6cm} |}
    \hline
    \bf Name & \bf type / function return type & \bf size & \bf default value / function args & \bf description \\ \hline
    show &     bool &      &     True &     flag to decide, whether the markers are shown\\ \hline
    showNumbers &     bool &      &     False &     flag to decide, whether the marker numbers are shown\\ \hline
    drawSimplified &     bool &      &     True &     draw markers with simplified symbols\\ \hline
    defaultSize &     float &      &     -1. &     global marker size; if -1.f, marker size is relative to maxSceneSize\\ \hline
    defaultColor &     Float4 &     4 &     [0.1,0.5,0.1,1.] &     \tabnewline default cRGB olor for markers; 4th value is alpha-transparency\\ \hline
	  \end{longtable}
	\end{center}

%+++++++++++++++++++++++++++++++++++
\mysubsubsection{VSettingsLoads} \label{sec:VSettingsLoads}
Visualization settings for loads.\\ 
%
VSettingsLoads has the following items:
%reference manual TABLE
\begin{center}
  \footnotesize
  \begin{longtable}{| p{4.2cm} | p{2.5cm} | p{0.3cm} | p{3.0cm} | p{6cm} |}
    \hline
    \bf Name & \bf type / function return type & \bf size & \bf default value / function args & \bf description \\ \hline
    show &     bool &      &     True &     flag to decide, whether the loads are shown\\ \hline
    showNumbers &     bool &      &     False &     flag to decide, whether the load numbers are shown\\ \hline
    defaultSize &     float &      &     0.2 &     global load size; if -1.f, load size is relative to maxSceneSize\\ \hline
    defaultRadius &     float &      &     0.005 &     global radius of load axis if drawn in 3D\\ \hline
    fixedLoadSize &     bool &      &     True &     if true, the load is drawn with a fixed vector length in direction of the load vector, independently of the load size\\ \hline
    drawSimplified &     bool &      &     True &     draw markers with simplified symbols\\ \hline
    loadSizeFactor &     float &      &     0.1 &     if fixedLoadSize=false, then this scaling factor is used to draw the load vector\\ \hline
    defaultColor &     Float4 &     4 &     [0.7,0.1,0.1,1.] &     \tabnewline default cRGB olor for loads; 4th value is alpha-transparency\\ \hline
	  \end{longtable}
	\end{center}

%+++++++++++++++++++++++++++++++++++
\mysubsubsection{VSettingsSensors} \label{sec:VSettingsSensors}
Visualization settings for sensors.\\ 
%
VSettingsSensors has the following items:
%reference manual TABLE
\begin{center}
  \footnotesize
  \begin{longtable}{| p{4.2cm} | p{2.5cm} | p{0.3cm} | p{3.0cm} | p{6cm} |}
    \hline
    \bf Name & \bf type / function return type & \bf size & \bf default value / function args & \bf description \\ \hline
    show &     bool &      &     True &     flag to decide, whether the sensors are shown\\ \hline
    showNumbers &     bool &      &     False &     flag to decide, whether the sensor numbers are shown\\ \hline
    drawSimplified &     bool &      &     True &     draw sensors with simplified symbols\\ \hline
    defaultSize &     float &      &     -1. &     global sensor size; if -1.f, sensor size is relative to maxSceneSize\\ \hline
    defaultColor &     Float4 &     4 &     [0.6,0.6,0.1,1.] &     \tabnewline default cRGB olor for sensors; 4th value is alpha-transparency\\ \hline
	  \end{longtable}
	\end{center}

%+++++++++++++++++++++++++++++++++++
\mysubsubsection{VSettingsWindow} \label{sec:VSettingsWindow}
Window and interaction settings for visualization; handle changes with care, as they might lead to unexpected results or crashes.\\ 
%
VSettingsWindow has the following items:
%reference manual TABLE
\begin{center}
  \footnotesize
  \begin{longtable}{| p{4.2cm} | p{2.5cm} | p{0.3cm} | p{3.0cm} | p{6cm} |}
    \hline
    \bf Name & \bf type / function return type & \bf size & \bf default value / function args & \bf description \\ \hline
    renderWindowSize &     Index2 &     2 &     [1024,768] &     initial size of OpenGL render window in pixel\\ \hline
    startupTimeout &     PInt &      &     2500 &     OpenGL render window startup timeout in ms (change might be necessary if CPU is very slow)\\ \hline
    alwaysOnTop &     bool &      &     False &     True: OpenGL render window will be always on top of all other windows\\ \hline
    maximize &     bool &      &     False &     True: OpenGL render window will be maximized at startup\\ \hline
    showWindow &     bool &      &     True &     True: OpenGL render window is shown on startup; False: window will be iconified at startup (e.g. if you are starting multiple computations automatically)\\ \hline
    keyPressUserFunction &     KeyPressUserFunction &      &     0 &     add a Python function f(key, action, mods) here, which is called every time a key is pressed; Example: \tabnewline def f(key, action, mods):\tabnewline \phantom{XXX} print('key=',key);\tabnewline use chr(key) to convert key codes [32 ...96] to ascii; special key codes (>256) are provided in the exudyn.KeyCode enumeration type; key action needs to be checked (0=released, 1=pressed, 2=repeated); mods provide information (binary) for SHIFT (1), CTRL (2), ALT (4), Super keys (8), CAPSLOCK (16)\\ \hline
    showMouseCoordinates &     bool &      &     False &     True: show OpenGL coordinates and distance to last left mouse button pressed position; switched on/off with key 'F3'\\ \hline
    ignoreKeys &     bool &      &     False &     True: ignore keyboard input except escape and 'F2' keys; used for interactive mode, e.g., to perform kinematic analysis; This flag can be switched with key 'F2'\\ \hline
    ResetKeyPressUserFunction() &     void &      &      &     set keyPressUserFunction to zero (no function); because this cannot be assign to the variable itself\\ \hline
	  \end{longtable}
	\end{center}

%+++++++++++++++++++++++++++++++++++
\mysubsubsection{VSettingsOpenGL} \label{sec:VSettingsOpenGL}
OpenGL settings for 2D and 2D rendering. For further details, see the OpenGL functionality. \\ 
%
VSettingsOpenGL has the following items:
%reference manual TABLE
\begin{center}
  \footnotesize
  \begin{longtable}{| p{4.2cm} | p{2.5cm} | p{0.3cm} | p{3.0cm} | p{6cm} |}
    \hline
    \bf Name & \bf type / function return type & \bf size & \bf default value / function args & \bf description \\ \hline
    initialCenterPoint &     Float3 &     3 &     [0.,0.,0.] &     \tabnewline centerpoint of scene (3D) at renderer startup; overwritten if autoFitScene = True\\ \hline
    initialZoom &     float &      &     1. &     initial zoom of scene; overwritten/ignored if autoFitScene = True\\ \hline
    initialMaxSceneSize &     float &      &     1. &     initial maximum scene size (auto: diagonal of cube with maximum scene coordinates); used for 'zoom all' functionality and for visibility of objects; overwritten if autoFitScene = True\\ \hline
    initialModelRotation &     StdArray33F &     3x3 &     [Matrix3DF[3,3,1.,0.,0., 0.,1.,0., 0.,0.,1.]] &     \tabnewline initial model rotation matrix for OpenGl; in python use e.g.: initialModelRotation=[[1,0,0],[0,1,0],[0,0,1]]\\ \hline
    multiSampling &     PInt &     1 &     1 &     multi sampling turned off (<=1) or turned on to given values (2, 4, 8 or 16); increases the graphics buffers and might crash due to graphics card memory limitations; only works if supported by hardware; if it does not work, try to change 3D graphics hardware settings!\\ \hline
    lineWidth &     float &     1 &     1. &     width of lines used for representation of lines, circles, points, etc.\\ \hline
    lineSmooth &     bool &     1 &     True &     draw lines smooth\\ \hline
    textLineWidth &     float &     1 &     1. &     width of lines used for representation of text\\ \hline
    textLineSmooth &     bool &     1 &     False &     draw lines for representation of text smooth\\ \hline
    showFaces &     bool &     1 &     True &     show faces of triangles, etc.; using the options showFaces=false and showFaceEdges=true gives are wire frame representation\\ \hline
    facesTransparent &     bool &     1 &     False &     True: show faces transparent independent of transparency (A)-value in color of objects; allow to show otherwise hidden node/marker/object numbers\\ \hline
    showFaceEdges &     bool &     1 &     False &     show edges of faces; using the options showFaces=false and showFaceEdges=true gives are wire frame representation\\ \hline
    shadeModelSmooth &     bool &     1 &     True &     True: turn on smoothing for shaders, which uses vertex normals to smooth surfaces\\ \hline
    materialAmbientAndDiffuse &     Float4 &     4 &     [0.6,0.6,0.6,1.] &     \tabnewline 4f ambient color of material\\ \hline
    materialShininess &     float &     1 &     32. &     shininess of material\\ \hline
    materialSpecular &     Float4 &     4 &     [0.6,0.6,0.6,1.] &     \tabnewline 4f specular color of material\\ \hline
    enableLighting &     bool &     1 &     True &     generally enable lighting (otherwise, colors of objects are used); OpenGL: glEnable(GL\_LIGHTING)\\ \hline
    lightModelLocalViewer &     bool &     1 &     False &     select local viewer for light; maps to OpenGL glLightModeli(GL\_LIGHT\_MODEL\_LOCAL\_VIEWER,...)\\ \hline
    lightModelTwoSide &     bool &     1 &     True &     enlighten also backside of object; maps to OpenGL glLightModeli(GL\_LIGHT\_MODEL\_TWO\_SIDE,...)\\ \hline
    lightModelAmbient &     Float4 &     4 &     [0.,0.,0.,1.] &     \tabnewline global ambient light; maps to OpenGL glLightModeli(GL\_LIGHT\_MODEL\_AMBIENT,[r,g,b,a])\\ \hline
    enableLight0 &     bool &     1 &     True &     turn on/off light0\\ \hline
    light0position &     Float4 &     4 &     [0.2,0.2,10.,0.] &     \tabnewline 4f position vector of GL\_LIGHT0; 4th value should be 0 for lights like sun, but 1 for directional lights (and for attenuation factor being calculated); see opengl manuals\\ \hline
    light0ambient &     float &     1 &     0.3 &     ambient value of GL\_LIGHT0\\ \hline
    light0diffuse &     float &     1 &     0.6 &     diffuse value of GL\_LIGHT0\\ \hline
    light0specular &     float &     1 &     0.5 &     specular value of GL\_LIGHT0\\ \hline
    light0constantAttenuation &     float &     1 &     1.0 &     constant attenuation coefficient of GL\_LIGHT0, this is a constant factor that attenuates the light source; attenuation factor = 1/(kx +kl*d + kq*d*d); (kc,kl,kq)=(1,0,0) means no attenuation; only used for lights, where last component of light position is 1\\ \hline
    light0linearAttenuation &     float &     1 &     0.0 &     linear attenuation coefficient of GL\_LIGHT0, this is a linear factor for attenuation of the light source with distance\\ \hline
    light0quadraticAttenuation &     float &     1 &     0.0 &     quadratic attenuation coefficient of GL\_LIGHT0, this is a quadratic factor for attenuation of the light source with distance\\ \hline
    enableLight1 &     bool &     1 &     True &     turn on/off light1\\ \hline
    light1position &     Float4 &     4 &     [1.,1.,-10.,0.] &     \tabnewline 4f position vector of GL\_LIGHT0; 4th value should be 0 for lights like sun, but 1 for directional lights (and for attenuation factor being calculated); see opengl manuals\\ \hline
    light1ambient &     float &     1 &     0.0  &     ambient value of GL\_LIGHT1\\ \hline
    light1diffuse &     float &     1 &     0.5 &     diffuse value of GL\_LIGHT1\\ \hline
    light1specular &     float &     1 &     0.6 &     specular value of GL\_LIGHT1\\ \hline
    light1constantAttenuation &     float &     1 &     1.0 &     constant attenuation coefficient of GL\_LIGHT1, this is a constant factor that attenuates the light source; attenuation factor = 1/(kx +kl*d + kq*d*d); only used for lights, where last component of light position is 1\\ \hline
    light1linearAttenuation &     float &     1 &     0.0 &     linear attenuation coefficient of GL\_LIGHT1, this is a linear factor for attenuation of the light source with distance\\ \hline
    light1quadraticAttenuation &     float &     1 &     0.0 &     quadratic attenuation coefficient of GL\_LIGHT1, this is a quadratic factor for attenuation of the light source with distance\\ \hline
    drawFaceNormals &     bool &     1 &     False &     draws triangle normals, e.g. at center of triangles; used for debugging of faces\\ \hline
    drawVertexNormals &     bool &     1 &     False &     draws vertex normals; used for debugging\\ \hline
    drawNormalsLength &     float &     1 &     0.1 &     length of normals; used for debugging\\ \hline
	  \end{longtable}
	\end{center}

%+++++++++++++++++++++++++++++++++++
\mysubsubsection{VSettingsExportImages} \label{sec:VSettingsExportImages}
Functionality to export images to files (.tga format) which can be used to create animations; to activate image recording during the solution process, set SolutionSettings.recordImagesInterval accordingly.\\ 
%
VSettingsExportImages has the following items:
%reference manual TABLE
\begin{center}
  \footnotesize
  \begin{longtable}{| p{4.2cm} | p{2.5cm} | p{0.3cm} | p{3.0cm} | p{6cm} |}
    \hline
    \bf Name & \bf type / function return type & \bf size & \bf default value / function args & \bf description \\ \hline
    saveImageTimeOut &     PInt &      &     5000 &     timeout in milliseconds for saving a frame as image to disk; this is the amount of time waited for redrawing; increase for very complex scenes\\ \hline
    saveImageFileName &     FileName &      &     'images/frame' &     filename (without extension!) and (relative) path for image file(s) with consecutive numbering (e.g., frame0000.tga, frame0001.tga,...); ; directory will be created if it does not exist\\ \hline
    saveImageFileCounter &     UInt &      &     0 &     current value of the counter which is used to consecutively save frames (images) with consecutive numbers\\ \hline
    saveImageSingleFile &     bool &      &     False &     True: only save single files with given filename, not adding numbering; False: add numbering to files, see saveImageFileName\\ \hline
	  \end{longtable}
	\end{center}

%+++++++++++++++++++++++++++++++++++
\mysubsubsection{VSettingsInteractive} \label{sec:VSettingsInteractive}
Functionality to interact with render window; will include left and right mouse press actions and others in future.\\ 
%
VSettingsInteractive has the following items:
%reference manual TABLE
\begin{center}
  \footnotesize
  \begin{longtable}{| p{4.2cm} | p{2.5cm} | p{0.3cm} | p{3.0cm} | p{6cm} |}
    \hline
    \bf Name & \bf type / function return type & \bf size & \bf default value / function args & \bf description \\ \hline
    keypressRotationStep &     float &      &     5. &     rotation increment per keypress in degree (full rotation = 360 degree)\\ \hline
    mouseMoveRotationFactor &     float &      &     1. &     rotation increment per 1 pixel mouse movement in degree\\ \hline
    keypressTranslationStep &     float &      &     0.1 &     translation increment per keypress relative to window size\\ \hline
    zoomStepFactor &     float &      &     1.15 &     change of zoom per keypress (keypad +/-) or mouse wheel increment\\ \hline
    highlightItemIndex &     Int &      &     -1 &     index of item that shall be highlighted (e.g., need to find item due to errors); if set -1, no item is highlighted\\ \hline
    highlightItemType &     ItemType &      &     ItemType::\_None &     item type (Node, Object, ...) that shall be highlighted (e.g., need to find item due to errors)\\ \hline
    highlightMbsNumber &     UInt &      &     0 &     index of main system (mbs) for which the item shall be highlighted; number is related to the ID in SystemContainer (first mbs = 0, second = 1, ...)\\ \hline
    highlightColor &     Float4 &     4 &     [0.8,0.05,0.05,0.75] &     \tabnewline cRGB color for highlighted item; 4th value is alpha-transparency\\ \hline
    highlightOtherColor &     Float4 &     4 &     [0.5,0.5,0.5,0.4] &     \tabnewline cRGB color for other items (which are not highlighted); 4th value is alpha-transparency\\ \hline
    selectionLeftMouse &     bool &      &     True &     True: left mouse click on items and show basic information\\ \hline
    selectionRightMouse &     bool &      &     False &     True: right mouse click on items and show dictionary (read only!)\\ \hline
	  \end{longtable}
	\end{center}

%+++++++++++++++++++++++++++++++++++
\mysubsubsection{VisualizationSettings} \label{sec:VisualizationSettings}
Settings for visualization. \\ 
%
VisualizationSettings has the following items:
%reference manual TABLE
\begin{center}
  \footnotesize
  \begin{longtable}{| p{4.2cm} | p{2.5cm} | p{0.3cm} | p{3.0cm} | p{6cm} |}
    \hline
    \bf Name & \bf type / function return type & \bf size & \bf default value / function args & \bf description \\ \hline
    general &     VSettingsGeneral &      &      &     general visualization settings\\ \hline
    contour &     VSettingsContour &      &      &     contour plot visualization settings\\ \hline
    nodes &     VSettingsNodes &      &      &     node visualization settings\\ \hline
    bodies &     VSettingsBodies &      &      &     body visualization settings\\ \hline
    connectors &     VSettingsConnectors &      &      &     connector visualization settings\\ \hline
    markers &     VSettingsMarkers &      &      &     marker visualization settings\\ \hline
    loads &     VSettingsLoads &      &      &     load visualization settings\\ \hline
    sensors &     VSettingsSensors &      &      &     sensor visualization settings\\ \hline
    window &     VSettingsWindow &      &      &     visualization window and interaction settings\\ \hline
    openGL &     VSettingsOpenGL &      &      &     OpenGL rendering settings\\ \hline
    interactive &     VSettingsInteractive &      &      &     Settings for interaction with renderer\\ \hline
    exportImages &     VSettingsExportImages &      &      &     settings for exporting (saving) images to files in order to create animations\\ \hline
	  \end{longtable}
	\end{center}

%++++++++++++++++++++++++++++++++++++++
\mysubsection{Solver substructures}\label{sec:solverSubstructures}
This section includes structures contained in the solver, which can be accessed via the python interface during solution or for building a customized solver in python.

%+++++++++++++++++++++++++++++++++++
\mysubsubsection{CSolverTimer} \label{sec:CSolverTimer}
Structure for timing in solver. Each Real variable is used to measure the CPU time which certain parts of the solver need. This structure is only active if the code is not compiled with the \_\_FAST\_EXUDYN\_LINALG option and if displayComputationTime is set True. Timings will only be filled, if useTimer is True.\\ 
%
CSolverTimer has the following items:
%reference manual TABLE
\begin{center}
  \footnotesize
  \begin{longtable}{| p{4.2cm} | p{2.5cm} | p{0.3cm} | p{3.0cm} | p{6cm} |}
    \hline
    \bf Name & \bf type / function return type & \bf size & \bf default value / function args & \bf description \\ \hline
    useTimer &     bool &      &     True &     flag to decide, whether the timer is used (true) or not\\ \hline
    total &     Real &      &     0. &     total time measured between start and end of computation (static/dynamics)\\ \hline
    factorization &     Real &      &     0. &     solve or inverse\\ \hline
    newtonIncrement &     Real &      &     0. &     Jac$^{-1}$ * RHS; backsubstitution\\ \hline
    integrationFormula &     Real &      &     0. &     time spent for evaluation of integration formulas\\ \hline
    ODE2RHS &     Real &      &     0. &     time for residual evaluation of ODE2 right-hand-side\\ \hline
    ODE1RHS &     Real &      &     0. &     time for residual evaluation of ODE1 right-hand-side\\ \hline
    AERHS &     Real &      &     0. &     time for residual evaluation of algebraic equations right-hand-side\\ \hline
    totalJacobian &     Real &      &     0. &     time for all jacobian computations\\ \hline
    jacobianODE1 &     Real &      &     0. &     jacobian w.r.t. coordinates of ODE1 equations (not counted in sum)\\ \hline
    jacobianODE2 &     Real &      &     0. &     jacobian w.r.t. coordinates of ODE2 equations (not counted in sum)\\ \hline
    jacobianODE2\_t &     Real &      &     0. &     jacobian w.r.t. coordinates\_t of ODE2 equations (not counted in sum)\\ \hline
    jacobianAE &     Real &      &     0. &     jacobian of algebraic equations (not counted in sum)\\ \hline
    massMatrix &     Real &      &     0. &     mass matrix computation\\ \hline
    reactionForces &     Real &      &     0. &     CqT * lambda\\ \hline
    postNewton &     Real &      &     0. &     post newton step\\ \hline
    errorEstimator &     Real &      &     0. &     for explicit solvers, additional evaluation\\ \hline
    writeSolution &     Real &      &     0. &     time for writing solution\\ \hline
    overhead &     Real &      &     0. &     overhead, such as initialization, copying and some matrix-vector multiplication\\ \hline
    python &     Real &      &     0. &     time spent for python functions\\ \hline
    visualization &     Real &      &     0. &     time spent for visualization in computation thread\\ \hline
    Reset(...) &     void &      &     useSolverTimer &     reset solver timings to initial state by assigning default values; useSolverTimer sets the useTimer flag\\ \hline
    Sum() &     Real &      &      &     compute sum of all timers (except for those counted multiple, e.g., jacobians\\ \hline
    StartTimer(...) &     void &      &     value &     start timer function for a given variable; subtracts current CPU time from value\\ \hline
    StopTimer(...) &     void &      &     value &     stop timer function for a given variable; adds current CPU time to value\\ \hline
    ToString() &     String &      &      &     converts the current timings to a string\\ \hline
	  \end{longtable}
	\end{center}

%+++++++++++++++++++++++++++++++++++
\mysubsubsection{SolverLocalData} \label{sec:SolverLocalData}
Solver local data structure for solution vectors, system matrices and temporary vectors and data structures.\\ 
%
SolverLocalData has the following items:
%reference manual TABLE
\begin{center}
  \footnotesize
  \begin{longtable}{| p{4.2cm} | p{2.5cm} | p{0.3cm} | p{3.0cm} | p{6cm} |}
    \hline
    \bf Name & \bf type / function return type & \bf size & \bf default value / function args & \bf description \\ \hline
    nODE2 &     Index &      &     0 &     number of second order ordinary diff. eq. coordinates\\ \hline
    nODE1 &     Index &      &     0 &     number of first order ordinary diff. eq. coordinates\\ \hline
    nAE &     Index &      &     0 &     number of algebraic coordinates\\ \hline
    nData &     Index &      &     0 &     number of data coordinates\\ \hline
    nSys &     Index &      &     0 &     number of system (unknown) coordinates = nODE2+nODE1+nAE\\ \hline
    startAE &     Index &      &     0 &     start of algebraic coordinates, but set to zero if nAE==0\\ \hline
    systemResidual &     ResizableVector &      &      &     system residual vector (vectors will be linked to this vector!)\\ \hline
    newtonSolution &     ResizableVector &      &      &     Newton decrement (computed from residual and jacobian)\\ \hline
    tempODE2 &     ResizableVector &      &      &     temporary vector for ODE2 quantities; use in initial accelerations and during Newton\\ \hline
    temp2ODE2 &     ResizableVector &      &      &     second temporary vector for ODE2 quantities; use in static computation\\ \hline
    tempODE2F0 &     ResizableVector &      &      &     temporary vector for ODE2 Jacobian\\ \hline
    tempODE2F1 &     ResizableVector &      &      &     temporary vector for ODE2 Jacobian\\ \hline
    tempODE1F0 &     ResizableVector &      &      &     temporary vector for ODE1 Jacobian\\ \hline
    tempODE1F1 &     ResizableVector &      &      &     temporary vector for ODE1 Jacobian\\ \hline
    startOfStepStateAAlgorithmic &     ResizableVector &      &      &     additional term needed for generalized alpha (startOfStep state)\\ \hline
    aAlgorithmic &     ResizableVector &      &      &     additional term needed for generalized alpha (current state)\\ \hline
    CleanUpMemory() &     void &      &      &     if desired, temporary data is cleaned up to safe memory\\ \hline
    SetLinearSolverType(...) &     void &      &     linearSolverType &     set linear solver type and matrix version: links system matrices to according dense/sparse versions\\ \hline
    GetLinearSolverType() &     LinearSolverType &      &      &     return current linear solver type (dense/sparse)\\ \hline
	  \end{longtable}
	\end{center}

%+++++++++++++++++++++++++++++++++++
\mysubsubsection{SolverIterationData} \label{sec:SolverIterationData}
Solver internal structure for counters, steps, step size, time, etc.; solution vectors, residuals, etc. are SolverLocalData. The given default values are overwritten by the simulationSettings when initializing the solver.\\ 
%
SolverIterationData has the following items:
%reference manual TABLE
\begin{center}
  \footnotesize
  \begin{longtable}{| p{4.2cm} | p{2.5cm} | p{0.3cm} | p{3.0cm} | p{6cm} |}
    \hline
    \bf Name & \bf type / function return type & \bf size & \bf default value / function args & \bf description \\ \hline
    maxStepSize &     Real &      &     0. &     constant or maximum stepSize\\ \hline
    minStepSize &     Real &      &     0. &     minimum stepSize for static/dynamic solver; only used, if automaticStepSize is activated\\ \hline
    initialStepSize &     Real &      &     1e-6 &     initial stepSize for dynamic solver; only used, if automaticStepSize is activated\\ \hline
    lastStepSize &     Real &      &     0. &     stepSize suggested from last step or by initial step size; only used, if automaticStepSize is activated\\ \hline
    currentStepSize &     Real &      &     0. &     stepSize of current step\\ \hline
    recommendedStepSize &     Real &      &     -1. &     recommended step size $h_{recom}$ after PostNewton(...): $h_{recom} < 0$: no recommendation, $h_{recom}==0$: use minimum step size, $h_{recom}>0$: use specific step size, if no smaller size requested by other reason\\ \hline
    numberOfSteps &     Index &      &     0 &     number of time steps (if fixed size); $n$\\ \hline
    currentStepIndex &     Index &      &     0 &     current step index; $i$\\ \hline
    adaptiveStep &     bool &      &     True &     True: the step size may be reduced if step fails; no automatic stepsize control\\ \hline
    automaticStepSize &     bool &      &     True &     True: if timeIntegration.automaticStepSize == True AND chosen integrators supports automatic step size control (e.g., DOPRI5); False: constant step size used (step may be reduced if adaptiveStep=True)\\ \hline
    currentTime &     Real &      &     0. &     holds the current simulation time, copy of state.current.time; interval is [startTime,tEnd]; in static solver, duration is loadStepDuration\\ \hline
    startTime &     Real &      &     0. &     time at beginning of time integration\\ \hline
    endTime &     Real &      &     0. &     end time of static/dynamic solver\\ \hline
    discontinuousIteration &     Index &      &     0 &     number of current discontinuous iteration\\ \hline
    newtonSteps &     Index &      &     0 &     number of current newton steps\\ \hline
    newtonStepsCount &     Index &      &     0 &     count total Newton steps\\ \hline
    newtonJacobiCount &     Index &      &     0 &     count total Newton jacobian computations\\ \hline
    rejectedModifiedNewtonSteps &     Index &      &     0 &     count the number of rejected modified Newton steps (switch to full Newton)\\ \hline
    discontinuousIterationsCount &     Index &      &     0 &     count total number of discontinuous iterations (min. 1 per step)\\ \hline
    rejectedAutomaticStepSizeSteps &     \tabnewline Index &      &     0 &     count the number of rejected steps in case of automatic step size control (rejected steps are repeated with smaller step size)\\ \hline
    automaticStepSizeError &     Real &      &     0 &     estimated error (relative to atol + rtol*solution) of last step; must be $\le 1$  for a step to be accepted\\ \hline
    ToString() &     String &      &      &     convert iteration statistics to string; used for displayStatistics option\\ \hline
	  \end{longtable}
	\end{center}

%+++++++++++++++++++++++++++++++++++
\mysubsubsection{SolverConvergenceData} \label{sec:SolverConvergenceData}
Solver internal structure for convergence information: residua, iteration loop errors and error flags. For detailed behavior of these flags, visit the source code!. \\ 
%
SolverConvergenceData has the following items:
%reference manual TABLE
\begin{center}
  \footnotesize
  \begin{longtable}{| p{4.2cm} | p{2.5cm} | p{0.3cm} | p{3.0cm} | p{6cm} |}
    \hline
    \bf Name & \bf type / function return type & \bf size & \bf default value / function args & \bf description \\ \hline
    stepReductionFailed &     bool &      &     False &     true, if iterations over time/static steps failed (finally, cannot be recovered)\\ \hline
    discontinuousIterationSuccessful &     \tabnewline bool &      &     True &     true, if last discontinuous iteration had success (failure may be recovered by adaptive step)\\ \hline
    linearSolverFailed &     bool &      &     False &     true, if linear solver failed to factorize\\ \hline
    linearSolverCausingRow &     Index &      &     -1 &     -1 if successful, 0 ... n-1, the system equation (=coordinate) index which may have caused the problem, at which the linear solver failed\\ \hline
    newtonConverged &     bool &      &     False &     true, if Newton has (finally) converged\\ \hline
    newtonSolutionDiverged &     bool &      &     False &     true, if Newton diverged (may be recovered)\\ \hline
    jacobianUpdateRequested &     bool &      &     True &     true, if a jacobian update is requested in modified Newton (determined in previous step)\\ \hline
    massMatrixNotInvertible &     bool &      &     True &     true, if mass matrix is not invertable during initialization or solution (explicit solver)\\ \hline
    discontinuousIterationError &     Real &      &     0. &     error of discontinuous iterations (contact, friction, ...) outside of Newton iteration\\ \hline
    residual &     Real &      &     0. &     current Newton residual\\ \hline
    lastResidual &     Real &      &     0. &     last Newton residual to determine contractivity\\ \hline
    contractivity &     Real &      &     0. &     Newton contractivity = geometric decay of error in every step\\ \hline
    errorCoordinateFactor &     Real &      &     1. &     factor may include the number of system coordinates to reduce the residual\\ \hline
    InitializeData() &     void &      &      &     initialize SolverConvergenceData by assigning default values\\ \hline
	  \end{longtable}
	\end{center}

%+++++++++++++++++++++++++++++++++++
\mysubsubsection{SolverOutputData} \label{sec:SolverOutputData}
Solver internal structure for output modes, output timers and counters.\\ 
%
SolverOutputData has the following items:
%reference manual TABLE
\begin{center}
  \footnotesize
  \begin{longtable}{| p{4.2cm} | p{2.5cm} | p{0.3cm} | p{3.0cm} | p{6cm} |}
    \hline
    \bf Name & \bf type / function return type & \bf size & \bf default value / function args & \bf description \\ \hline
    finishedSuccessfully &     bool &      &     False &     flag is false until solver finshed successfully (can be used as external trigger)\\ \hline
    verboseMode &     Index &      &     0 &     this is a copy of the solvers verboseMode used for console output\\ \hline
    verboseModeFile &     Index &      &     0 &     this is a copy of the solvers verboseModeFile used for file\\ \hline
    stepInformation &     Index &      &     0 &     this is a copy of the solvers stepInformation used for console output\\ \hline
    writeToSolutionFile &     bool &      &     False &     if false, no solution file is generated and no file is written\\ \hline
    writeToSolverFile &     bool &      &     False &     if false, no solver output file is generated and no file is written\\ \hline
    sensorValuesTemp &     ResizableVector &      &      &     temporary vector for per sensor values (overwritten for every sensor; usually contains last sensor)\\ \hline
    lastSolutionWritten &     Real &      &     0. &     simulation time when last solution has been written\\ \hline
    lastSensorsWritten &     Real &      &     0. &     simulation time when last sensors have been written\\ \hline
    lastImageRecorded &     Real &      &     0. &     simulation time when last image has been recorded\\ \hline
    cpuStartTime &     Real &      &     0. &     CPU start time of computation (starts counting at computation of initial conditions)\\ \hline
    cpuLastTimePrinted &     Real &      &     0. &     CPU time when output has been printed last time\\ \hline
    InitializeData() &     void &      &      &     initialize SolverOutputData by assigning default values\\ \hline
	  \end{longtable}
	\end{center}

%+++++++++++++++++++++++++++++++++++
\mysubsubsection{MainSolverStatic} \label{sec:MainSolverStatic}
PyBind interface (trampoline) class for static solver. With this interface, the static solver and its substructures can be accessed via python. NOTE that except from SolveSystem(...), these functions are only intended for experienced users and they need to be handled with care, as unexpected crashes may happen if used inappropriate. Furthermore, the functions have a lot of overhead (performance much lower than internal solver) due to python interfaces, and should thus be used for small systems. To access the solver in python, write: \bi
 \item[] solver = MainSolverStatic() 
\ei
 and hereafter you can access all data and functions via 'solver'.\\ 
%
MainSolverStatic has the following items:
%reference manual TABLE
\begin{center}
  \footnotesize
  \begin{longtable}{| p{4.2cm} | p{2.5cm} | p{0.3cm} | p{3.0cm} | p{6cm} |}
    \hline
    \bf Name & \bf type / function return type & \bf size & \bf default value / function args & \bf description \\ \hline
    timer &     CSolverTimer &      &      &     timer which measures the CPU time of solver sub functions\\ \hline
    it &     SolverIterationData &      &      &     all information about iterations (steps, discontinuous iteration, newton,...)\\ \hline
    conv &     SolverConvergenceData &      &      &     all information about tolerances, errors and residua\\ \hline
    output &     SolverOutputData &      &      &     output modes and timers for exporting solver information and solution\\ \hline
    newton &     NewtonSettings &      &      &     copy of newton settings from timeint or staticSolver\\ \hline
    loadStepGeometricFactor &     Real &      &      &     multiplicative load step factor; this factor is computed from loadStepGeometric parameters in SolveSystem(...)\\ \hline
    CheckInitialized(...) &     bool &      &     mainSystem &     check if MainSolver and MainSystem are correctly initialized ==> otherwise raise SysError\\ \hline
    ComputeLoadFactor(...) &     Real &      &     simulationSettings &     for static solver, this is a factor in interval [0,1]; MUST be overwritten\\ \hline
    GetSolverName() &     std::string &      &      &     get solver name - needed for output file header and visualization window\\ \hline
    IsStaticSolver() &     bool &      &      &     return true, if static solver; needs to be overwritten in derived class\\ \hline
    GetSimulationEndTime(...) &     Real &      &     simulationSettings &     compute simulation end time (depends on static or time integration solver)\\ \hline
    ReduceStepSize(...) &     bool &      &     mainSystem, simulationSettings, severity &     reduce step size (1..normal, 2..severe problems); return true, if reduction was successful\\ \hline
    IncreaseStepSize(...) &     void &      &     mainSystem, simulationSettings &     increase step size if convergence is good\\ \hline
    InitializeSolver(...) &     bool &      &     mainSystem, simulationSettings &     initialize solverSpecific,data,it,conv; set/compute initial conditions (solver-specific!); initialize output files\\ \hline
    PreInitializeSolverSpecific(...) &     \tabnewline void &      &     mainSystem, simulationSettings &     pre-initialize for solver specific tasks; called at beginning of InitializeSolver, right after Solver data reset\\ \hline
    InitializeSolverOutput(...) &     void &      &     mainSystem, simulationSettings &     initialize output files; called from InitializeSolver()\\ \hline
    InitializeSolverPreChecks(...) &     \tabnewline bool &      &     mainSystem, simulationSettings &     check if system is solvable; initialize dense/sparse computation modes\\ \hline
    InitializeSolverData(...) &     void &      &     mainSystem, simulationSettings &     initialize all data,it,conv; called from InitializeSolver()\\ \hline
    InitializeSolverInitialConditions(...) &     \tabnewline void &      &     mainSystem, simulationSettings &     set/compute initial conditions (solver-specific!); called from InitializeSolver()\\ \hline
    PostInitializeSolverSpecific(...) &     \tabnewline void &      &     mainSystem, simulationSettings &     post-initialize for solver specific tasks; called at the end of InitializeSolver\\ \hline
    SolveSystem(...) &     bool &      &     mainSystem, simulationSettings &     solve System: InitializeSolver, SolveSteps, FinalizeSolver\\ \hline
    FinalizeSolver(...) &     void &      &     mainSystem, simulationSettings &     write concluding information (timer statistics, messages) and close files\\ \hline
    SolveSteps(...) &     bool &      &     mainSystem, simulationSettings &     main solver part: calls multiple InitializeStep(...)/ DiscontinuousIteration(...)/ FinishStep(...); do step reduction if necessary; return true if success, false else\\ \hline
    UpdateCurrentTime(...) &     void &      &     mainSystem, simulationSettings &     update currentTime (and load factor); MUST be overwritten in special solver class\\ \hline
    InitializeStep(...) &     void &      &     mainSystem, simulationSettings &     initialize static step / time step; python-functions; do some outputs, checks, etc.\\ \hline
    FinishStep(...) &     void &      &     mainSystem, simulationSettings &     finish static step / time step; write output of results to file\\ \hline
    DiscontinuousIteration(...) &     bool &      &     mainSystem, simulationSettings &     perform discontinuousIteration for static step / time step; CALLS ComputeNewtonResidual\\ \hline
    Newton(...) &     bool &      &     mainSystem, simulationSettings &     perform Newton method for given solver method\\ \hline
    ComputeNewtonResidual(...) &     Real &      &     mainSystem, simulationSettings &     compute residual for Newton method (e.g. static or time step); store residual vector in systemResidual and return scalar residual (specific computation may depend on solver types)\\ \hline
    ComputeNewtonUpdate(...) &     void &      &     mainSystem, simulationSettings, initial=true &     compute update for currentState from newtonSolution (decrement from residual and jacobian); if initial, this is for the initial update with newtonSolution=0\\ \hline
    ComputeNewtonJacobian(...) &     void &      &     mainSystem, simulationSettings &     compute jacobian for newton method of given solver method; store result in systemJacobian\\ \hline
    WriteSolutionFileHeader(...) &     void &      &     mainSystem, simulationSettings &     write unique file header, depending on static/ dynamic simulation\\ \hline
    WriteCoordinatesToFile(...) &     void &      &     mainSystem, simulationSettings &     write unique coordinates solution file\\ \hline
    IsVerboseCheck(...) &     bool &      &     level &     return true, if file or console output is at or above the given level\\ \hline
    VerboseWrite(...) &     void &      &     level, str &     write to console and/or file in case of level\\ \hline
    GetODE2size() &     Index &      &      &     number of ODE2 equations in solver\\ \hline
    GetODE1size() &     Index &      &      &     number of ODE1 equations in solver (not yet implemented)\\ \hline
    GetAEsize() &     Index &      &      &     number of algebraic equations in solver\\ \hline
    GetDataSize() &     Index &      &      &     number of data (history) variables in solver\\ \hline
    GetSystemJacobian() &     NumpyMatrix &      &      &     get locally stored / last computed system jacobian of solver\\ \hline
    GetSystemMassMatrix() &     NumpyMatrix &      &      &     get locally stored / last computed mass matrix of solver\\ \hline
    GetSystemResidual() &     NumpyVector &      &      &     get locally stored / last computed system residual\\ \hline
    GetNewtonSolution() &     NumpyVector &      &      &     get locally stored / last computed solution (=increment) of Newton\\ \hline
    SetSystemJacobian(...) &     void &      &     systemJacobian &     set locally stored system jacobian of solver; must have size nODE2+nODE1+nAE\\ \hline
    SetSystemMassMatrix(...) &     void &      &     systemMassMatrix &     set locally stored mass matrix of solver; must have size nODE2+nODE1+nAE\\ \hline
    SetSystemResidual(...) &     void &      &     systemResidual &     set locally stored system residual; must have size nODE2+nODE1+nAE\\ \hline
    ComputeMassMatrix(...) &     void &      &     mainSystem, scalarFactor=1. &     compute systemMassMatrix (multiplied with factor) in cSolver and return mass matrix\\ \hline
    ComputeJacobianODE2RHS(...) &     void &      &     mainSystem, scalarFactor=1. &     set systemJacobian to zero and add jacobian (multiplied with factor) of ODE2RHS to systemJacobian in cSolver\\ \hline
    ComputeJacobianODE2RHS\_t(...) &     \tabnewline void &      &     mainSystem, scalarFactor=1. &     add jacobian of ODE2RHS\_t (multiplied with factor) to systemJacobian in cSolver\\ \hline
    ComputeJacobianAE(...) &     void &      &     mainSystem, scalarFactor\_ODE2=1., scalarFactor\_ODE2\_t=1., velocityLevel=false &     add jacobian of algebraic equations (multiplied with factor) to systemJacobian in cSolver; the scalarFactors are scaling the derivatives w.r.t. ODE2 coordinates and w.r.t. ODE2\_t (velocity) coordinates; if velocityLevel == true, the constraints are evaluated at velocity level\\ \hline
    ComputeODE2RHS(...) &     void &      &     mainSystem &     compute the RHS of ODE2 equations in systemResidual in range(0,nODE2)\\ \hline
    ComputeAlgebraicEquations(...) &     \tabnewline void &      &     mainSystem, velocityLevel=false &     compute the algebraic equations in systemResidual in range(nODE2+nODE1, nODE2+nODE1+nAE)\\ \hline
	  \end{longtable}
	\end{center}

%+++++++++++++++++++++++++++++++++++
\mysubsubsection{MainSolverImplicitSecondOrder} \label{sec:MainSolverImplicitSecondOrder}
PyBind interface (trampoline) class for dynamic implicit solver. Note that this solver includes the classical Newmark method (set useNewmark True; with option of index 2 reduction) as well as the generalized-alpha method. With the interface, the dynamic implicit solver and its substructures can be accessed via python. NOTE that except from SolveSystem(...), these functions are only intended for experienced users and they need to be handled with care, as unexpected crashes may happen if used inappropriate. Furthermore, the functions have a lot of overhead (still fast, but performance much lower than internal solver) due to python interfaces, and should thus be used for small systems. To access the solver in python, write \bi
 \item[] solver = MainSolverImplicitSecondOrder() 
\ei
 and hereafter you can access all data and functions via 'solver'.
 In this solver, user functions are possible to extend the solver at certain parts, while keeping the overal C++ performance. User functions, which are added with SetUserFunction...(...), have the arguments (MainSolver, MainSystem, simulationSettings), except for ComputeNewtonUpdate which adds the initial flag as an additional argument and ComputeNewtonResidual, which returns the scalar residual.\\ 
%
MainSolverImplicitSecondOrder has the following items:
%reference manual TABLE
\begin{center}
  \footnotesize
  \begin{longtable}{| p{4.2cm} | p{2.5cm} | p{0.3cm} | p{3.0cm} | p{6cm} |}
    \hline
    \bf Name & \bf type / function return type & \bf size & \bf default value / function args & \bf description \\ \hline
    timer &     CSolverTimer &      &      &     timer which measures the CPU time of solver sub functions; note that solver structures can only be written indirectly, e.g.,  timer=dynamicSolver.timer; timer.useTimer = False; dynamicSolver.timer=timer; however, dynamicSolver.timer.useTimer cannot be written.\\ \hline
    it &     SolverIterationData &      &      &     all information about iterations (steps, discontinuous iteration, newton,...)\\ \hline
    conv &     SolverConvergenceData &      &      &     all information about tolerances, errors and residua\\ \hline
    output &     SolverOutputData &      &      &     output modes and timers for exporting solver information and solution\\ \hline
    newton &     NewtonSettings &      &      &     copy of newton settings from timeint or staticSolver\\ \hline
    newmarkBeta &     Real &      &      &     copy of parameter in timeIntegration.generalizedAlpha\\ \hline
    newmarkGamma &     Real &      &      &     copy of parameter in timeIntegration.generalizedAlpha\\ \hline
    alphaM &     Real &      &      &     copy of parameter in timeIntegration.generalizedAlpha\\ \hline
    alphaF &     Real &      &      &     copy of parameter in timeIntegration.generalizedAlpha\\ \hline
    spectralRadius &     Real &      &      &     copy of parameter in timeIntegration.generalizedAlpha\\ \hline
    factJacAlgorithmic &     Real &      &      &     locally computed parameter from generalizedAlpha parameters\\ \hline
    CheckInitialized(...) &     bool &      &     mainSystem &     check if MainSolver and MainSystem are correctly initialized ==> otherwise raise SysError\\ \hline
    ComputeLoadFactor(...) &     Real &      &     simulationSettings &     for static solver, this is a factor in interval [0,1]; MUST be overwritten\\ \hline
    GetAAlgorithmic() &     NumpyVector &      &      &     get locally stored / last computed algorithmic accelerations\\ \hline
    GetStartOfStepStateAAlgorithmic() &     \tabnewline NumpyVector &      &      &     get locally stored / last computed algorithmic accelerations at start of step\\ \hline
    SetUserFunctionUpdateCurrentTime(...) &     \tabnewline void &      &     mainSystem, userFunction &     set user function\\ \hline
    SetUserFunctionInitializeStep(...) &     \tabnewline void &      &     mainSystem, userFunction &     set user function\\ \hline
    SetUserFunctionFinishStep(...) &     \tabnewline void &      &     mainSystem, userFunction &     set user function\\ \hline
    SetUserFunctionDiscontinuousIteration(...) &     \tabnewline void &      &     mainSystem, userFunction &     set user function\\ \hline
    SetUserFunctionNewton(...) &     void &      &     mainSystem, userFunction &     set user function\\ \hline
    SetUserFunctionComputeNewtonUpdate(...) &     \tabnewline void &      &     mainSystem, userFunction &     set user function\\ \hline
    SetUserFunctionComputeNewtonResidual(...) &     \tabnewline void &      &     mainSystem, userFunction &     set user function\\ \hline
    SetUserFunctionComputeNewtonJacobian(...) &     \tabnewline void &      &     mainSystem, userFunction &     set user function\\ \hline
    SetUserFunctionPostNewton(...) &     \tabnewline void &      &     mainSystem, userFunction &     set user function\\ \hline
    GetSolverName() &     std::string &      &      &     get solver name - needed for output file header and visualization window\\ \hline
    IsStaticSolver() &     bool &      &      &     return true, if static solver; needs to be overwritten in derived class\\ \hline
    GetSimulationEndTime(...) &     Real &      &     simulationSettings &     compute simulation end time (depends on static or time integration solver)\\ \hline
    ReduceStepSize(...) &     bool &      &     mainSystem, simulationSettings, severity &     reduce step size (1..normal, 2..severe problems); return true, if reduction was successful\\ \hline
    IncreaseStepSize(...) &     void &      &     mainSystem, simulationSettings &     increase step size if convergence is good\\ \hline
    InitializeSolver(...) &     bool &      &     mainSystem, simulationSettings &     initialize solverSpecific,data,it,conv; set/compute initial conditions (solver-specific!); initialize output files\\ \hline
    PreInitializeSolverSpecific(...) &     \tabnewline void &      &     mainSystem, simulationSettings &     pre-initialize for solver specific tasks; called at beginning of InitializeSolver, right after Solver data reset\\ \hline
    InitializeSolverOutput(...) &     void &      &     mainSystem, simulationSettings &     initialize output files; called from InitializeSolver()\\ \hline
    InitializeSolverPreChecks(...) &     \tabnewline bool &      &     mainSystem, simulationSettings &     check if system is solvable; initialize dense/sparse computation modes\\ \hline
    InitializeSolverData(...) &     void &      &     mainSystem, simulationSettings &     initialize all data,it,conv; called from InitializeSolver()\\ \hline
    InitializeSolverInitialConditions(...) &     \tabnewline void &      &     mainSystem, simulationSettings &     set/compute initial conditions (solver-specific!); called from InitializeSolver()\\ \hline
    PostInitializeSolverSpecific(...) &     \tabnewline void &      &     mainSystem, simulationSettings &     post-initialize for solver specific tasks; called at the end of InitializeSolver\\ \hline
    SolveSystem(...) &     bool &      &     mainSystem, simulationSettings &     solve System: InitializeSolver, SolveSteps, FinalizeSolver\\ \hline
    FinalizeSolver(...) &     void &      &     mainSystem, simulationSettings &     write concluding information (timer statistics, messages) and close files\\ \hline
    SolveSteps(...) &     bool &      &     mainSystem, simulationSettings &     main solver part: calls multiple InitializeStep(...)/ DiscontinuousIteration(...)/ FinishStep(...); do step reduction if necessary; return true if success, false else\\ \hline
    UpdateCurrentTime(...) &     void &      &     mainSystem, simulationSettings &     update currentTime (and load factor); MUST be overwritten in special solver class\\ \hline
    InitializeStep(...) &     void &      &     mainSystem, simulationSettings &     initialize static step / time step; python-functions; do some outputs, checks, etc.\\ \hline
    FinishStep(...) &     void &      &     mainSystem, simulationSettings &     finish static step / time step; write output of results to file\\ \hline
    DiscontinuousIteration(...) &     bool &      &     mainSystem, simulationSettings &     perform discontinuousIteration for static step / time step; CALLS ComputeNewtonResidual\\ \hline
    Newton(...) &     bool &      &     mainSystem, simulationSettings &     perform Newton method for given solver method\\ \hline
    PostNewton(...) &     Real &      &     mainSystem, simulationSettings &     call PostNewton for all relevant objects (contact, friction, ... iterations); returns error for discontinuous iteration\\ \hline
    ComputeNewtonResidual(...) &     Real &      &     mainSystem, simulationSettings &     compute residual for Newton method (e.g. static or time step); store residual vector in systemResidual and return scalar residual (specific computation may depend on solver types)\\ \hline
    ComputeNewtonUpdate(...) &     void &      &     mainSystem, simulationSettings, initial=true &     compute update for currentState from newtonSolution (decrement from residual and jacobian); if initial, this is for the initial update with newtonSolution=0\\ \hline
    ComputeNewtonJacobian(...) &     void &      &     mainSystem, simulationSettings &     compute jacobian for newton method of given solver method; store result in systemJacobian\\ \hline
    WriteSolutionFileHeader(...) &     void &      &     mainSystem, simulationSettings &     write unique file header, depending on static/ dynamic simulation\\ \hline
    WriteCoordinatesToFile(...) &     void &      &     mainSystem, simulationSettings &     write unique coordinates solution file\\ \hline
    IsVerboseCheck(...) &     bool &      &     level &     return true, if file or console output is at or above the given level\\ \hline
    VerboseWrite(...) &     void &      &     level, str &     write to console and/or file in case of level\\ \hline
    GetODE2size() &     Index &      &      &     number of ODE2 equations in solver\\ \hline
    GetODE1size() &     Index &      &      &     number of ODE1 equations in solver (not yet implemented)\\ \hline
    GetAEsize() &     Index &      &      &     number of algebraic equations in solver\\ \hline
    GetDataSize() &     Index &      &      &     number of data (history) variables in solver\\ \hline
    GetSystemJacobian() &     NumpyMatrix &      &      &     get locally stored / last computed system jacobian of solver\\ \hline
    GetSystemMassMatrix() &     NumpyMatrix &      &      &     get locally stored / last computed mass matrix of solver\\ \hline
    GetSystemResidual() &     NumpyVector &      &      &     get locally stored / last computed system residual\\ \hline
    GetNewtonSolution() &     NumpyVector &      &      &     get locally stored / last computed solution (=increment) of Newton\\ \hline
    SetSystemJacobian(...) &     void &      &     systemJacobian &     set locally stored system jacobian of solver; must have size nODE2+nODE1+nAE\\ \hline
    SetSystemMassMatrix(...) &     void &      &     systemMassMatrix &     set locally stored mass matrix of solver; must have size nODE2+nODE1+nAE\\ \hline
    SetSystemResidual(...) &     void &      &     systemResidual &     set locally stored system residual; must have size nODE2+nODE1+nAE\\ \hline
    ComputeMassMatrix(...) &     void &      &     mainSystem, scalarFactor=1. &     compute systemMassMatrix (multiplied with factor) in cSolver and return mass matrix\\ \hline
    ComputeJacobianODE2RHS(...) &     void &      &     mainSystem, scalarFactor=1. &     set systemJacobian to zero and add jacobian (multiplied with factor) of ODE2RHS to systemJacobian in cSolver\\ \hline
    ComputeJacobianODE2RHS\_t(...) &     \tabnewline void &      &     mainSystem, scalarFactor=1. &     add jacobian of ODE2RHS\_t (multiplied with factor) to systemJacobian in cSolver\\ \hline
    ComputeJacobianAE(...) &     void &      &     mainSystem, scalarFactor\_ODE2=1., scalarFactor\_ODE2\_t=1., velocityLevel=false &     add jacobian of algebraic equations (multiplied with factor) to systemJacobian in cSolver; the scalarFactors are scaling the derivatives w.r.t. ODE2 coordinates and w.r.t. ODE2\_t (velocity) coordinates; if velocityLevel == true, the constraints are evaluated at velocity level\\ \hline
    ComputeODE2RHS(...) &     void &      &     mainSystem &     compute the RHS of ODE2 equations in systemResidual in range(0,nODE2)\\ \hline
    ComputeODE1RHS(...) &     void &      &     mainSystem &     compute the RHS of ODE1 equations in systemResidual in range(0,nODE1)\\ \hline
    ComputeAlgebraicEquations(...) &     \tabnewline void &      &     mainSystem, velocityLevel=false &     compute the algebraic equations in systemResidual in range(nODE2+nODE1, nODE2+nODE1+nAE)\\ \hline
	  \end{longtable}
	\end{center}

%+++++++++++++++++++++++++++++++++++
\mysubsubsection{MainSolverExplicit} \label{sec:MainSolverExplicit}
PyBind interface (trampoline) class for dynamic explicit solver. Note that this solver includes the 1st order explicit Euler scheme and the 4th order Runge-Kutta scheme with 5th order error estimation (DOPRI5). With the interface, the solver and its substructures can be accessed via python. NOTE that except from SolveSystem(...), these functions are only intended for experienced users and they need to be handled with care, as unexpected crashes may happen if used inappropriate. Furthermore, the functions have a lot of overhead (still fast, but performance much lower than internal solver) due to python interfaces, and should thus be used for small systems. To access the solver in python, write \bi
 \item[] solver = MainSolverExplicit() 
\ei
 and hereafter you can access all data and functions via 'solver'.
 In this solver, no user functions are possible, but you can use SolverImplicitSecondOrder instead (turning off Newton gives explicit scheme ...).\\ 
%
MainSolverExplicit has the following items:
%reference manual TABLE
\begin{center}
  \footnotesize
  \begin{longtable}{| p{4.2cm} | p{2.5cm} | p{0.3cm} | p{3.0cm} | p{6cm} |}
    \hline
    \bf Name & \bf type / function return type & \bf size & \bf default value / function args & \bf description \\ \hline
    timer &     CSolverTimer &      &      &     timer which measures the CPU time of solver sub functions\\ \hline
    it &     SolverIterationData &      &      &     all information about iterations (steps, discontinuous iteration, newton,...)\\ \hline
    conv &     SolverConvergenceData &      &      &     all information about tolerances, errors and residua\\ \hline
    output &     SolverOutputData &      &      &     output modes and timers for exporting solver information and solution\\ \hline
    ComputeLoadFactor(...) &     Real &      &     simulationSettings &     for static solver, this is a factor in interval [0,1]; MUST be overwritten\\ \hline
    GetNumberOfStages() &     Index &      &      &     return number of stages in current method\\ \hline
    GetMethodOrder() &     Index &      &      &     return order of method (higher value in methods with automatic step size, e.g., DOPRI5=5)\\ \hline
    GetSolverName() &     std::string &      &      &     get solver name - needed for output file header and visualization window\\ \hline
    IsStaticSolver() &     bool &      &      &     return true, if static solver; needs to be overwritten in derived class\\ \hline
    GetSimulationEndTime(...) &     Real &      &     simulationSettings &     compute simulation end time (depends on static or time integration solver)\\ \hline
    ReduceStepSize(...) &     bool &      &     mainSystem, simulationSettings, severity &     reduce step size (1..normal, 2..severe problems); return true, if reduction was successful\\ \hline
    IncreaseStepSize(...) &     void &      &     mainSystem, simulationSettings &     increase step size if convergence is good\\ \hline
    InitializeSolver(...) &     bool &      &     mainSystem, simulationSettings &     initialize solverSpecific,data,it,conv; set/compute initial conditions (solver-specific!); initialize output files\\ \hline
    PreInitializeSolverSpecific(...) &     \tabnewline void &      &     mainSystem, simulationSettings &     pre-initialize for solver specific tasks; called at beginning of InitializeSolver, right after Solver data reset\\ \hline
    InitializeSolverOutput(...) &     void &      &     mainSystem, simulationSettings &     initialize output files; called from InitializeSolver()\\ \hline
    InitializeSolverPreChecks(...) &     \tabnewline bool &      &     mainSystem, simulationSettings &     check if system is solvable; initialize dense/sparse computation modes\\ \hline
    InitializeSolverData(...) &     void &      &     mainSystem, simulationSettings &     initialize all data,it,conv; called from InitializeSolver()\\ \hline
    InitializeSolverInitialConditions(...) &     \tabnewline void &      &     mainSystem, simulationSettings &     set/compute initial conditions (solver-specific!); called from InitializeSolver()\\ \hline
    PostInitializeSolverSpecific(...) &     \tabnewline void &      &     mainSystem, simulationSettings &     post-initialize for solver specific tasks; called at the end of InitializeSolver\\ \hline
    SolveSystem(...) &     bool &      &     mainSystem, simulationSettings &     solve System: InitializeSolver, SolveSteps, FinalizeSolver\\ \hline
    FinalizeSolver(...) &     void &      &     mainSystem, simulationSettings &     write concluding information (timer statistics, messages) and close files\\ \hline
    SolveSteps(...) &     bool &      &     mainSystem, simulationSettings &     main solver part: calls multiple InitializeStep(...)/ DiscontinuousIteration(...)/ FinishStep(...); do step reduction if necessary; return true if success, false else\\ \hline
    UpdateCurrentTime(...) &     void &      &     mainSystem, simulationSettings &     update currentTime (and load factor); MUST be overwritten in special solver class\\ \hline
    InitializeStep(...) &     void &      &     mainSystem, simulationSettings &     initialize static step / time step; python-functions; do some outputs, checks, etc.\\ \hline
    FinishStep(...) &     void &      &     mainSystem, simulationSettings &     finish static step / time step; write output of results to file\\ \hline
    DiscontinuousIteration(...) &     bool &      &     mainSystem, simulationSettings &     perform discontinuousIteration for static step / time step; CALLS ComputeNewtonResidual\\ \hline
    Newton(...) &     bool &      &     mainSystem, simulationSettings &     perform Newton method for given solver method\\ \hline
    ComputeNewtonResidual(...) &     Real &      &     mainSystem, simulationSettings &     compute residual for Newton method (e.g. static or time step); store residual vector in systemResidual and return scalar residual (specific computation may depend on solver types)\\ \hline
    ComputeNewtonUpdate(...) &     void &      &     mainSystem, simulationSettings, initial=true &     compute update for currentState from newtonSolution (decrement from residual and jacobian); if initial, this is for the initial update with newtonSolution=0\\ \hline
    ComputeNewtonJacobian(...) &     void &      &     mainSystem, simulationSettings &     compute jacobian for newton method of given solver method; store result in systemJacobian\\ \hline
    WriteSolutionFileHeader(...) &     void &      &     mainSystem, simulationSettings &     write unique file header, depending on static/ dynamic simulation\\ \hline
    WriteCoordinatesToFile(...) &     void &      &     mainSystem, simulationSettings &     write unique coordinates solution file\\ \hline
    IsVerboseCheck(...) &     bool &      &     level &     return true, if file or console output is at or above the given level\\ \hline
    VerboseWrite(...) &     void &      &     level, str &     write to console and/or file in case of level\\ \hline
    GetODE2size() &     Index &      &      &     number of ODE2 equations in solver\\ \hline
    GetODE1size() &     Index &      &      &     number of ODE1 equations in solver (not yet implemented)\\ \hline
    GetAEsize() &     Index &      &      &     number of algebraic equations in solver\\ \hline
    GetDataSize() &     Index &      &      &     number of data (history) variables in solver\\ \hline
    GetSystemMassMatrix() &     NumpyMatrix &      &      &     get locally stored / last computed mass matrix of solver\\ \hline
    GetSystemResidual() &     NumpyVector &      &      &     get locally stored / last computed system residual\\ \hline
    SetSystemMassMatrix(...) &     void &      &     systemMassMatrix &     set locally stored mass matrix of solver; must have size nODE2+nODE1+nAE\\ \hline
    SetSystemResidual(...) &     void &      &     systemResidual &     set locally stored system residual; must have size nODE2+nODE1+nAE\\ \hline
    ComputeMassMatrix(...) &     void &      &     mainSystem, scalarFactor=1. &     compute systemMassMatrix (multiplied with factor) in cSolver and return mass matrix\\ \hline
    ComputeODE2RHS(...) &     void &      &     mainSystem &     compute the RHS of ODE2 equations in systemResidual in range(0,nODE2)\\ \hline
    ComputeODE1RHS(...) &     void &      &     mainSystem &     compute the RHS of ODE1 equations in systemResidual in range(0,nODE1)\\ \hline
	  \end{longtable}
	\end{center}
