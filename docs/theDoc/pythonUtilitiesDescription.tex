% ++++++++++++++++++++++++++++++++++++++++++++++++++++++++++++++++++++++++% description of python utility functions; generated by Johannes Gerstmayr% ++++++++++++++++++++++++++++++++++++++++++++++++++++++++++++++++++++++++

\mysubsection{Module: basicUtilities}
\noindent {def {\bf DiagonalMatrix}}({\it rowsColumns}, {\it value}=1)
\setlength{\itemindent}{0.7cm}
\begin{itemize}[leftmargin=0.7cm]
  \item[--]  {\bf function description}: create a diagonal or identity matrix; used for interface.py, avoiding the need for numpy  \item[--]  {\bf input}: \vspace{-6pt}
  \begin{itemize}[leftmargin=1.2cm]
\setlength{\itemindent}{-0.7cm}
    \item[] {\it rowsColumns}: provides the number of rows and columns
    \item[] {\it        value}: initialization value for diagonal terms
  \ei
  \item[--]  {\bf output}: list of lists representing a matrix\vspace{12pt}\end{itemize}
%
\noindent\rule{8cm}{0.75pt}\vspace{1pt} \\ 
\noindent {def {\bf NormL2}}({\it vector})
\setlength{\itemindent}{0.7cm}
\begin{itemize}[leftmargin=0.7cm]
  \item[--]  {\bf function description}: compute L2 norm for vectors without switching to numpy or math module  \item[--]  {\bf input}: vector as list or in numpy format  \item[--]  {\bf output}: L2-norm of vector\vspace{12pt}\end{itemize}
%
\noindent\rule{8cm}{0.75pt}\vspace{1pt} \\ 
\noindent {def {\bf VSum}}({\it vector})
\setlength{\itemindent}{0.7cm}
\begin{itemize}[leftmargin=0.7cm]
  \item[--]  {\bf function description}: compute sum of all values of vector  \item[--]  {\bf input}: vector as list or in numpy format  \item[--]  {\bf output}: sum of all components of vector\vspace{12pt}\end{itemize}
%
\noindent\rule{8cm}{0.75pt}\vspace{1pt} \\ 
\noindent {def {\bf VAdd}}({\it v0}, {\it v1})
\setlength{\itemindent}{0.7cm}
\begin{itemize}[leftmargin=0.7cm]
  \item[--]  {\bf function description}: add two vectors instead using numpy  \item[--]  {\bf input}: vectors v0 and v1 as list or in numpy format  \item[--]  {\bf output}: component-wise sum of v0 and v1\vspace{12pt}\end{itemize}
%
\noindent\rule{8cm}{0.75pt}\vspace{1pt} \\ 
\noindent {def {\bf VSub}}({\it v0}, {\it v1})
\setlength{\itemindent}{0.7cm}
\begin{itemize}[leftmargin=0.7cm]
  \item[--]  {\bf function description}: subtract two vectors instead using numpy: result = v0-v1  \item[--]  {\bf input}: vectors v0 and v1 as list or in numpy format  \item[--]  {\bf output}: component-wise difference of v0 and v1\vspace{12pt}\end{itemize}
%
\noindent\rule{8cm}{0.75pt}\vspace{1pt} \\ 
\noindent {def {\bf VMult}}({\it v0}, {\it v1})
\setlength{\itemindent}{0.7cm}
\begin{itemize}[leftmargin=0.7cm]
  \item[--]  {\bf function description}: scalar multiplication of two vectors instead using numpy: result = v0'*v1  \item[--]  {\bf input}: vectors v0 and v1 as list or in numpy format  \item[--]  {\bf output}: sum of all component wise products: c0[0]*v1[0] + v0[1]*v1[0] + ...\vspace{12pt}\end{itemize}
%
\noindent\rule{8cm}{0.75pt}\vspace{1pt} \\ 
\noindent {def {\bf ScalarMult}}({\it scalar}, {\it v})
\setlength{\itemindent}{0.7cm}
\begin{itemize}[leftmargin=0.7cm]
  \item[--]  {\bf function description}: multiplication vectors with scalar: result = s*v  \item[--]  {\bf input}: value {\it scalar} and vector {\it v} as list or in numpy format  \item[--]  {\bf output}: scalar multiplication of all components of v: [scalar*v[0], scalar*v[1], ...]\vspace{12pt}\end{itemize}
%
\noindent\rule{8cm}{0.75pt}\vspace{1pt} \\ 
\noindent {def {\bf Normalize}}({\it v})
\setlength{\itemindent}{0.7cm}
\begin{itemize}[leftmargin=0.7cm]
  \item[--]  {\bf function description}: take a 3D vector and return a normalized 3D vector (L2Norm=1)  \item[--]  {\bf input}: vector v as list or in numpy format  \item[--]  {\bf output}: vector v multiplied with scalar such that L2-norm of vector is 1\vspace{12pt}\end{itemize}
%
\noindent\rule{8cm}{0.75pt}\vspace{1pt} \\ 
\noindent {def {\bf Vec2Tilde}}({\it v})
\setlength{\itemindent}{0.7cm}
\begin{itemize}[leftmargin=0.7cm]
  \item[--]  {\bf function description}: apply tilde operator (skew) to 3D-vector and return skew matrix  \item[--]  {\bf input}: 3D vector v as list or in numpy format  \item[--]  {\bf output}: matrix as list of lists containing the skew-symmetric matrix computed from v: $\mr{0}{-v[2]}{v[1]} {v[2]}{0}{-v[0]} {-v[1]}{v[0]}{0}$\vspace{12pt}\end{itemize}
%
\noindent\rule{8cm}{0.75pt}\vspace{1pt} \\ 
\noindent {def {\bf Tilde2Vec}}({\it m})
\setlength{\itemindent}{0.7cm}
\begin{itemize}[leftmargin=0.7cm]
  \item[--]  {\bf function description}: take skew symmetric matrix and return vector (inverse of Skew(...))  \item[--]  {\bf input}: list of lists containing a skew-symmetric matrix (3x3)  \item[--]  {\bf output}: list containing the vector v (inverse function of Vec2Tilde(...))\vspace{12pt}\end{itemize}
%
\mysubsection{Module: utilities}
\noindent {def {\bf PlotLineCode}}({\it index})
\setlength{\itemindent}{0.7cm}
\begin{itemize}[leftmargin=0.7cm]
  \item[--]  {\bf function description}: helper functions for matplotlib, returns a list of 28 line codes to be used in plot, e.g. 'r-' for red solid line  \item[--]  {\bf input}: index in range(0:28)  \item[--]  {\bf output}: a color and line style code for matplotlib plot\vspace{12pt}\end{itemize}
%
\noindent\rule{8cm}{0.75pt}\vspace{1pt} \\ 
\noindent {def {\bf FillInSubMatrix}}({\it subMatrix}, {\it destinationMatrix}, {\it destRow}, {\it destColumn})
\setlength{\itemindent}{0.7cm}
\begin{itemize}[leftmargin=0.7cm]
  \item[--]  {\bf function description}: fill submatrix into given destinationMatrix; all matrices must be numpy arrays  \item[--]  {\bf input}: \vspace{-6pt}
  \begin{itemize}[leftmargin=1.2cm]
\setlength{\itemindent}{-0.7cm}
    \item[] {\it subMatrix}: input matrix, which is filled into destinationMatrix
    \item[] {\it   destinationMatrix}: the subMatrix is entered here
    \item[] {\it   destRow}: row destination of subMatrix
    \item[] {\it   destColumn}: column destination of subMatrix
  \ei
  \item[--]  {\bf output}: destinationMatrix is changed after function call  \item[--]  {\bf notes}: may be erased in future!\vspace{12pt}\end{itemize}
%
\noindent\rule{8cm}{0.75pt}\vspace{1pt} \\ 
\noindent {def {\bf SweepSin}}({\it t}, {\it t1}, {\it f0}, {\it f1})
\setlength{\itemindent}{0.7cm}
\begin{itemize}[leftmargin=0.7cm]
  \item[--]  {\bf function description}: compute sin sweep at given time t  \item[--]  {\bf input}: \vspace{-6pt}
  \begin{itemize}[leftmargin=1.2cm]
\setlength{\itemindent}{-0.7cm}
    \item[] {\it t}: evaluate of sweep at time t
    \item[] {\it   t1}: end time of sweep frequency range
    \item[] {\it   f0}: start of frequency interval [f0,f1] in Hz
    \item[] {\it   f1}: end of frequency interval [f0,f1] in Hz
  \ei
  \item[--]  {\bf output}: evaluation of sin sweep (in range -1..+1)\vspace{12pt}\end{itemize}
%
\noindent\rule{8cm}{0.75pt}\vspace{1pt} \\ 
\noindent {def {\bf SweepCos}}({\it t}, {\it t1}, {\it f0}, {\it f1})
\setlength{\itemindent}{0.7cm}
\begin{itemize}[leftmargin=0.7cm]
  \item[--]  {\bf function description}: compute cos sweep at given time t  \item[--]  {\bf input}: \vspace{-6pt}
  \begin{itemize}[leftmargin=1.2cm]
\setlength{\itemindent}{-0.7cm}
    \item[] {\it t}: evaluate of sweep at time t
    \item[] {\it   t1}: end time of sweep frequency range
    \item[] {\it   f0}: start of frequency interval [f0,f1] in Hz
    \item[] {\it   f1}: end of frequency interval [f0,f1] in Hz
  \ei
  \item[--]  {\bf output}: evaluation of cos sweep (in range -1..+1)\vspace{12pt}\end{itemize}
%
\noindent\rule{8cm}{0.75pt}\vspace{1pt} \\ 
\noindent {def {\bf FrequencySweep}}({\it t}, {\it t1}, {\it f0}, {\it f1})
\setlength{\itemindent}{0.7cm}
\begin{itemize}[leftmargin=0.7cm]
  \item[--]  {\bf function description}: frequency according to given sweep functions SweepSin, SweepCos  \item[--]  {\bf input}: \vspace{-6pt}
  \begin{itemize}[leftmargin=1.2cm]
\setlength{\itemindent}{-0.7cm}
    \item[] {\it t}: evaluate of frequency at time t
    \item[] {\it   t1}: end time of sweep frequency range
    \item[] {\it   f0}: start of frequency interval [f0,f1] in Hz
    \item[] {\it   f1}: end of frequency interval [f0,f1] in Hz
  \ei
  \item[--]  {\bf output}: frequency in Hz\vspace{12pt}\end{itemize}
%
\noindent\rule{8cm}{0.75pt}\vspace{1pt} \\ 
\noindent {def {\bf RoundMatrix}}({\it matrix}, {\it treshold}=1e-14)
\setlength{\itemindent}{0.7cm}
\begin{itemize}[leftmargin=0.7cm]
  \item[--]  {\bf function description}: set all entries in matrix to zero which are smaller than given treshold; operates directly on matrix  \item[--]  {\bf input}: matrix as np.array, treshold as positive value  \item[--]  {\bf output}: changes matrix\vspace{12pt}\end{itemize}
%
\noindent\rule{8cm}{0.75pt}\vspace{1pt} \\ 
\noindent {def {\bf ComputeSkewMatrix}}({\it v})
\setlength{\itemindent}{0.7cm}
\begin{itemize}[leftmargin=0.7cm]
  \item[--]  {\bf function description}: compute (3 x 3*n) skew matrix from (3*n) vector; used for ObjectFFRF and CMS implementation  \item[--]  {\bf input}: a vector v in np.array format, containing 3*n components  \item[--]  {\bf output}: (3 x 3*n) skew matrix in np.array format\vspace{12pt}\end{itemize}
%
\noindent\rule{8cm}{0.75pt}\vspace{1pt} \\ 
\noindent {def {\bf CheckInputVector}}({\it vector}, {\it length}=-1)
\setlength{\itemindent}{0.7cm}
\begin{itemize}[leftmargin=0.7cm]
  \item[--]  {\bf function description}: check if input is list or array with according length; if length==-1, the length is not checked; raises Exception if the check fails  \item[--]  {\bf input}: \vspace{-6pt}
  \begin{itemize}[leftmargin=1.2cm]
\setlength{\itemindent}{-0.7cm}
    \item[] {\it vector}: a vector in np.array or list format
    \item[] {\it   length}: desired length of vector; if length=-1, it is ignored
  \ei
  \item[--]  {\bf output}: None\vspace{12pt}\end{itemize}
%
\noindent\rule{8cm}{0.75pt}\vspace{1pt} \\ 
\noindent {def {\bf CheckInputIndexArray}}({\it indexArray}, {\it length}=-1)
\setlength{\itemindent}{0.7cm}
\begin{itemize}[leftmargin=0.7cm]
  \item[--]  {\bf function description}: check if input is list or array with according length and positive indices; if length==-1, the length is not checked; raises Exception if the check fails  \item[--]  {\bf input}: \vspace{-6pt}
  \begin{itemize}[leftmargin=1.2cm]
\setlength{\itemindent}{-0.7cm}
    \item[] {\it indexArray}: a vector in np.array or list format
    \item[] {\it   length}: desired length of vector; if length=-1, it is ignored
  \ei
  \item[--]  {\bf output}: None\vspace{12pt}\end{itemize}
%
\noindent\rule{8cm}{0.75pt}\vspace{1pt} \\ 
\noindent {def {\bf LoadSolutionFile}}({\it fileName})
\setlength{\itemindent}{0.7cm}
\begin{itemize}[leftmargin=0.7cm]
  \item[--]  {\bf function description}: read coordinates solution file (exported during static or dynamic simulation with option exu.SimulationSettings().solutionSettings.coordinatesSolutionFileName='...') into dictionary:  \item[--]  {\bf input}: fileName: string containing directory and filename of stored coordinatesSolutionFile  \item[--]  {\bf output}: dictionary with 'data': the matrix of stored solution vectors, 'columnsExported': a list with binary values showing the exported columns [nODE2, nVel2, nAcc2, nODE1, nVel1, nAlgebraic, nData],'nColumns': the number of data columns and 'nRows': the number of data rows\vspace{12pt}\end{itemize}
%
\noindent\rule{8cm}{0.75pt}\vspace{1pt} \\ 
\noindent {def {\bf SetSolutionState}}({\it exu}, {\it mbs}, {\it solution}, {\it row}, {\it configuration})
\setlength{\itemindent}{0.7cm}
\begin{itemize}[leftmargin=0.7cm]
  \item[--]  {\bf function description}: load selected row of solution dictionary (previously loaded with LoadSolutionFile) into specific state\vspace{12pt}\end{itemize}
%
\noindent\rule{8cm}{0.75pt}\vspace{1pt} \\ 
\noindent {def {\bf SetVisualizationState}}({\it exu}, {\it mbs}, {\it solution}, {\it row})
\setlength{\itemindent}{0.7cm}
\begin{itemize}[leftmargin=0.7cm]
  \item[--]  {\bf function description}: load selected row of solution dictionary into visualization state and redraw  \item[--]  {\bf input}: \vspace{-6pt}
  \begin{itemize}[leftmargin=1.2cm]
\setlength{\itemindent}{-0.7cm}
    \item[] {\it exu}: the exudyn library
    \item[] {\it   mbs}: the system, where the state is applied to
    \item[] {\it   solution}: solution dictionary previously loaded with LoadSolutionFile
    \item[] {\it   row}: the according row of the solution file which is visualized
  \ei
  \item[--]  {\bf output}: renders the scene in mbs and changes the visualization state in mbs\vspace{12pt}\end{itemize}
%
\noindent\rule{8cm}{0.75pt}\vspace{1pt} \\ 
\noindent {def {\bf AnimateSolution}}({\it exu}, {\it SC}, {\it mbs}, {\it solution}, {\it rowIncrement}=1, {\it timeout}=0.04, {\it createImages}=False, {\it runLoop}=False)
\setlength{\itemindent}{0.7cm}
\begin{itemize}[leftmargin=0.7cm]
  \item[--]  {\bf function description}: consecutively load the rows of a solution file and visualize the result  \item[--]  {\bf input}: \vspace{-6pt}
  \begin{itemize}[leftmargin=1.2cm]
\setlength{\itemindent}{-0.7cm}
    \item[] {\it exu}: the exudyn library
    \item[] {\it   SC}: the system container, where the mbs lives in
    \item[] {\it   mbs}: the system used for animation
    \item[] {\it   solution}: solution dictionary previously loaded with LoadSolutionFile; will be played from first to last row
    \item[] {\it   rowIncrement}: can be set larger than 1 in order to skip solution frames: e.g. rowIncrement=10 visualizes every 10th row (frame)
    \item[] {\it   timeout}: in seconds is used between frames in order to limit the speed of animation; e.g. use timeout=0.04 to achieve approximately 25 frames per second
    \item[] {\it   createImages}: creates consecutively images from the animation, which can be converted into an animation
    \item[] {\it   runLoop}: if True, the animation is played in a loop until 'q' is pressed in render window
  \ei
  \item[--]  {\bf output}: renders the scene in mbs and changes the visualization state in mbs continuously\vspace{12pt}\end{itemize}
%
\noindent\rule{8cm}{0.75pt}\vspace{1pt} \\ 
\noindent {def {\bf DrawSystemGraph}}({\it mbs}, {\it showLoads}=True, {\it showSensors}=True, {\it useItemNames}=False, {\it useItemTypes}=False)
\setlength{\itemindent}{0.7cm}
\begin{itemize}[leftmargin=0.7cm]
  \item[--]  {\bf function description}: helper function which draws system graph of a MainSystem (mbs); several options let adjust the appearance of the graph  \item[--]  {\bf input}: \vspace{-6pt}
  \begin{itemize}[leftmargin=1.2cm]
\setlength{\itemindent}{-0.7cm}
    \item[] {\it showLoads}: toggle appearance of loads in mbs
    \item[] {\it    showSensors}: toggle appearance of sensors in mbs
    \item[] {\it    useItemNames}: if True, object names are shown instead of basic object types (Node, Load, ...)
    \item[] {\it    useItemTypes}: if True, object type names (ObjectMassPoint, ...) are shown instead of basic object types (Node, Load, ...)
  \ei
  \item[--]  {\bf output}: nothing\vspace{12pt}\end{itemize}
%
\noindent\rule{8cm}{0.75pt}\vspace{1pt} \\ 
\noindent {def {\bf GenerateStraightLineANCFCable2D}}({\it mbs}, {\it positionOfNode0}, {\it positionOfNode1}, {\it numberOfElements}, {\it cableTemplate}, {\it massProportionalLoad}=[0,0,0], {\it fixedConstraintsNode0}=[0,0,0,0], {\it fixedConstraintsNode1}=[0,0,0,0], {\it vALE}=0, {\it ConstrainAleCoordinate}=True)
\setlength{\itemindent}{0.7cm}
\begin{itemize}[leftmargin=0.7cm]
  \item[--]  {\bf function description}: generate cable elements along straight line with certain discretization  \item[--]  {\bf input}: \vspace{-6pt}
  \begin{itemize}[leftmargin=1.2cm]
\setlength{\itemindent}{-0.7cm}
    \item[] {\it mbs}: the system where ANCF cables are added
    \item[] {\it   positionOfNode0}: 3D position (list or np.array) for starting point of line
    \item[] {\it   positionOfNode1}: 3D position (list or np.array) for end point of line
    \item[] {\it   numberOfElements}: for discretization of line
    \item[] {\it   cableTemplate}: a ObjectANCFCable2D object, containing the desired cable properties; cable length and node numbers are set automatically
    \item[] {\it   massProportionalLoad}: a 3D list or np.array, containing the gravity vector or zero
    \item[] {\it   fixedConstraintsNode0}: a list of 4 binary values, indicating the coordinate contraints on the first node (x,y-position and x,y-slope)
    \item[] {\it   fixedConstraintsNode1}: a list of 4 binary values, indicating the coordinate contraints on the last node (x,y-position and x,y-slope)
    \item[] {\it   vALE}: used for ObjectALEANCFCable2D objects
  \ei
  \item[--]  {\bf output}: returns a list [cableNodeList, cableObjectList, loadList, cableNodePositionList, cableCoordinateConstraintList]\vspace{12pt}\end{itemize}
%
\noindent\rule{8cm}{0.75pt}\vspace{1pt} \\ 
\noindent {def {\bf GenerateSlidingJoint}}({\it mbs}, {\it cableObjectList}, {\it markerBodyPositionOfSlidingBody}, {\it localMarkerIndexOfStartCable}=0, {\it slidingCoordinateStartPosition}=0)
\setlength{\itemindent}{0.7cm}
\begin{itemize}[leftmargin=0.7cm]
  \item[--]  {\bf function description}: generate a sliding joint from a list of cables, marker to a sliding body, etc.\vspace{12pt}\end{itemize}
%
\noindent\rule{8cm}{0.75pt}\vspace{1pt} \\ 
\noindent {def {\bf GenerateAleSlidingJoint}}({\it mbs}, {\it cableObjectList}, {\it markerBodyPositionOfSlidingBody}, {\it AleNode}, {\it localMarkerIndexOfStartCable}=0, {\it AleSlidingOffset}=0, {\it activeConnector}=True, {\it penaltyStiffness}=0)
\setlength{\itemindent}{0.7cm}
\begin{itemize}[leftmargin=0.7cm]
  \item[--]  {\bf function description}: generate an ALE sliding joint from a list of cables, marker to a sliding body, etc.\vspace{12pt}\end{itemize}
%
\mysubsection{Module: graphicsDataUtilities}
\noindent {def {\bf GraphicsDataRectangle}}({\it xMin}, {\it yMin}, {\it xMax}, {\it yMax}, {\it color}=[0.,0.,0.,1.])
\setlength{\itemindent}{0.7cm}
\begin{itemize}[leftmargin=0.7cm]
  \item[--]  {\bf function description}: generate graphics data for 2D rectangle  \item[--]  {\bf input}: minimal and maximal cartesian coordinates in (x/y) plane; color provided as list of 4 RGBA values  \item[--]  {\bf output}: graphicsData dictionary, to be used in visualization of EXUDYN objects\vspace{12pt}\end{itemize}
%
\noindent\rule{8cm}{0.75pt}\vspace{1pt} \\ 
\noindent {def {\bf GraphicsDataOrthoCubeLines}}({\it xMin}, {\it yMin}, {\it zMin}, {\it xMax}, {\it yMax}, {\it zMax}, {\it color}=[0.,0.,0.,1.])
\setlength{\itemindent}{0.7cm}
\begin{itemize}[leftmargin=0.7cm]
  \item[--]  {\bf function description}: generate graphics data for orthogonal cube drawn with lines  \item[--]  {\bf input}: minimal and maximal cartesian coordinates for orthogonal cube; color provided as list of 4 RGBA values  \item[--]  {\bf output}: graphicsData dictionary, to be used in visualization of EXUDYN objects\vspace{12pt}\end{itemize}
%
\noindent\rule{8cm}{0.75pt}\vspace{1pt} \\ 
\noindent {def {\bf GraphicsDataOrthoCube}}({\it xMin}, {\it yMin}, {\it zMin}, {\it xMax}, {\it yMax}, {\it zMax}, {\it color}=[0.,0.,0.,1.])
\setlength{\itemindent}{0.7cm}
\begin{itemize}[leftmargin=0.7cm]
  \item[--]  {\bf function description}: generate graphics data for orthogonal 3D cube with min and max dimensions  \item[--]  {\bf input}: minimal and maximal cartesian coordinates for orthogonal cube; color provided as list of 4 RGBA values  \item[--]  {\bf output}: graphicsData dictionary, to be used in visualization of EXUDYN objects\vspace{12pt}\end{itemize}
%
\noindent\rule{8cm}{0.75pt}\vspace{1pt} \\ 
\noindent {def {\bf GraphicsDataOrthoCubePoint}}({\it centerPoint}, {\it size}, {\it color}=[0.,0.,0.,1.])
\setlength{\itemindent}{0.7cm}
\begin{itemize}[leftmargin=0.7cm]
  \item[--]  {\bf function description}: generate graphics data forfor orthogonal 3D cube with center point and size  \item[--]  {\bf input}: center point and size of cube (as 3D list or np.array); color provided as list of 4 RGBA values  \item[--]  {\bf output}: graphicsData dictionary, to be used in visualization of EXUDYN objects\vspace{12pt}\end{itemize}
%
\noindent\rule{8cm}{0.75pt}\vspace{1pt} \\ 
\noindent {def {\bf GraphicsDataCube}}({\it pList}, {\it color}=[0.,0.,0.,1.], {\it faces}=[1,1,1,1,1,1])
\setlength{\itemindent}{0.7cm}
\begin{itemize}[leftmargin=0.7cm]
  \item[--]  {\bf function description}: generate graphics data for general cube with endpoints, according to given vertex definition  \item[--]  {\bf input}: \vspace{-6pt}
  \begin{itemize}[leftmargin=1.2cm]
\setlength{\itemindent}{-0.7cm}
    \item[] {\it pList}: is a list of points [[x0,y0,z0],[x1,y11,z1],...]
    \item[] {\it   color}: provided as list of 4 RGBA values
    \item[] {\it   faces}: includes the list of six binary values (0/1), denoting active faces (value=1); set index to zero to hide face
  \ei
  \item[--]  {\bf output}: graphicsData dictionary, to be used in visualization of EXUDYN objects\vspace{12pt}\end{itemize}
%
\noindent\rule{8cm}{0.75pt}\vspace{1pt} \\ 
\noindent {def {\bf SwitchTripletOrder}}({\it vector})
\setlength{\itemindent}{0.7cm}
\begin{itemize}[leftmargin=0.7cm]
  \item[--]  {\bf function description}: switch order of three items in a list; mostly used for reverting normals in triangles  \item[--]  {\bf input}: 3D vector as list or as np.array  \item[--]  {\bf output}: interchanged 2nd and 3rd component of list\vspace{12pt}\end{itemize}
%
\noindent\rule{8cm}{0.75pt}\vspace{1pt} \\ 
\noindent {def {\bf GraphicsDataSphere}}({\it point}, {\it radius}, {\it color}=[0.,0.,0.,1.], {\it nTiles}=8)
\setlength{\itemindent}{0.7cm}
\begin{itemize}[leftmargin=0.7cm]
  \item[--]  {\bf function description}: generate graphics data for a sphere with point p and radius  \item[--]  {\bf input}: \vspace{-6pt}
  \begin{itemize}[leftmargin=1.2cm]
\setlength{\itemindent}{-0.7cm}
    \item[] {\it point}: center of sphere (3D list or np.array)
    \item[] {\it   radius}: positive value
    \item[] {\it   color}: provided as list of 4 RGBA values
    \item[] {\it   nTiles}: used to determine resolution of sphere >=3; use larger values for finer resolution
  \ei
  \item[--]  {\bf output}: graphicsData dictionary, to be used in visualization of EXUDYN objects\vspace{12pt}\end{itemize}
%
\noindent\rule{8cm}{0.75pt}\vspace{1pt} \\ 
\noindent {def {\bf GraphicsDataCylinder}}({\it pAxis}, {\it vAxis}, {\it radius}, {\it color}=[0.,0.,0.,1.], {\it nTiles}=16, {\it angleRange}=[0,2*np.pi], {\it lastFace}=True, {\it cutPlain}=True)
\setlength{\itemindent}{0.7cm}
\begin{itemize}[leftmargin=0.7cm]
  \item[--]  {\bf function description}: generate graphics data for a cylinder with given axis, radius and color; nFaces gives the number of tiles (minimum=3)  \item[--]  {\bf input}: \vspace{-6pt}
  \begin{itemize}[leftmargin=1.2cm]
\setlength{\itemindent}{-0.7cm}
    \item[] {\it pAxis}: axis point of one face of cylinder (3D list or np.array)
    \item[] {\it   vAxis}: vector representing the cylinder's axis (3D list or np.array)
    \item[] {\it   radius}: positive value representing radius of cylinder
    \item[] {\it   color}: provided as list of 4 RGBA values
    \item[] {\it   nTiles}: used to determine resolution of cylinder >=3; use larger values for finer resolution
    \item[] {\it   angleRange}: given in rad, to draw only part of cylinder (halfcylinder, etc.); for full range use [0..2 * pi]
    \item[] {\it   lastFace}: if angleRange != [0,2*pi], then the faces of the open cylinder are shown with lastFace = True
    \item[] {\it   cutPlain}: only used for angleRange != [0,2*pi]; if True, a plane is cut through the part of the cylinder; if False, the cylinder becomes a cake shape ...
  \ei
  \item[--]  {\bf output}: graphicsData dictionary, to be used in visualization of EXUDYN objects\vspace{12pt}\end{itemize}
%
\noindent\rule{8cm}{0.75pt}\vspace{1pt} \\ 
\noindent {def {\bf GraphicsDataRigidLink}}({\it p0}, {\it p1}, {\it axis0}=[0,0,0], {\it axis1}=[0,0,0], {\it radius}=[0.1,0.1], {\it thickness}=0.05, {\it width}=[0.05,0.05], {\it color}=[0.,0.,0.,1.], {\it nTiles}=16)
\setlength{\itemindent}{0.7cm}
\begin{itemize}[leftmargin=0.7cm]
  \item[--]  {\bf function description}: generate graphics data for a planar Link between the two joint positions, having two axes  \item[--]  {\bf input}: \vspace{-6pt}
  \begin{itemize}[leftmargin=1.2cm]
\setlength{\itemindent}{-0.7cm}
    \item[] {\it p0}: joint0 center position
    \item[] {\it   p1}: joint1 center position
    \item[] {\it   axis0}: direction of rotation axis at p0, if drawn as a cylinder; [0,0,0] otherwise
    \item[] {\it   axis1}: direction of rotation axis of p1, if drawn as a cylinder; [0,0,0] otherwise
    \item[] {\it   radius}: list of two radii [radius0, radius1], being the two radii of the joints drawn by a cylinder or sphere
    \item[] {\it   width}: list of two widths [width0, width1], being the two widths of the joints drawn by a cylinder; ignored for sphere
    \item[] {\it   thickness}: the thickness of the link (shaft) between the two joint positions; thickness in z-direction or diameter (cylinder)
    \item[] {\it   color}: provided as list of 4 RGBA values
    \item[] {\it   nTiles}: used to determine resolution of cylinder >=3; use larger values for finer resolution
  \ei
  \item[--]  {\bf output}: graphicsData dictionary, to be used in visualization of EXUDYN objects\vspace{12pt}\end{itemize}
%
\noindent\rule{8cm}{0.75pt}\vspace{1pt} \\ 
\noindent {def {\bf GraphicsDataFromSTLfileTxt}}({\it fileName}, {\it color}=[0.,0.,0.,1.], {\it verbose}=False)
\setlength{\itemindent}{0.7cm}
\begin{itemize}[leftmargin=0.7cm]
  \item[--]  {\bf function description}: generate graphics data from STL file (text format!) and use color for visualization  \item[--]  {\bf input}: \vspace{-6pt}
  \begin{itemize}[leftmargin=1.2cm]
\setlength{\itemindent}{-0.7cm}
    \item[] {\it fileName}: string containing directory and filename of STL-file (in text / SCII format) to load
    \item[] {\it   color}: provided as list of 4 RGBA values
    \item[] {\it   verbose}: if True, useful information is provided during reading
  \ei
  \item[--]  {\bf output}: interchanged 2nd and 3rd component of list\vspace{12pt}\end{itemize}
%
\mysubsection{Module: rigidBodyUtilities}
\noindent {def {\bf ComputeOrthonormalBasis}}({\it vector0})
\setlength{\itemindent}{0.7cm}
\begin{itemize}[leftmargin=0.7cm]
  \item[--]  {\bf function description}: compute orthogonal basis vectors (normal1, normal2) for given vector0 (non-unique solution!); if vector0 == [0,0,0], then any normal basis is returned\vspace{12pt}\end{itemize}
%
\noindent\rule{8cm}{0.75pt}\vspace{1pt} \\ 
\noindent {def {\bf GramSchmidt}}({\it vector0}, {\it vector1})
\setlength{\itemindent}{0.7cm}
\begin{itemize}[leftmargin=0.7cm]
  \item[--]  {\bf function description}: compute Gram-Schmidt projection of given 3D vector 1 on vector 0 and return normalized triad (vector0, vector1, vector0 x vector1)\vspace{12pt}\end{itemize}
%
\noindent\rule{8cm}{0.75pt}\vspace{1pt} \\ 
\noindent {def {\bf Skew}}({\it vector})
\setlength{\itemindent}{0.7cm}
\begin{itemize}[leftmargin=0.7cm]
  \item[--]  {\bf function description}: compute skew symmetric 3x3-matrix from 3x1- or 1x3-vector\vspace{12pt}\end{itemize}
%
\noindent\rule{8cm}{0.75pt}\vspace{1pt} \\ 
\noindent {def {\bf Skew2Vec}}({\it skew})
\setlength{\itemindent}{0.7cm}
\begin{itemize}[leftmargin=0.7cm]
  \item[--]  {\bf function description}: \vspace{-6pt}
  \begin{itemize}[leftmargin=1.2cm]
\setlength{\itemindent}{-0.7cm}
    \item[] convert skew symmetric matrix m to vector
    \item[] {\it def Skew2Vec(m)}:
  \ei
\vspace{12pt}\end{itemize}
%
\noindent\rule{8cm}{0.75pt}\vspace{1pt} \\ 
\noindent {def {\bf ComputeSkewMatrix}}({\it v})
\setlength{\itemindent}{0.7cm}
\begin{itemize}[leftmargin=0.7cm]
  \item[--]  {\bf function description}: compute (3 x 3*n) skew matrix from (3*n) vector\vspace{12pt}\end{itemize}
%
\noindent\rule{8cm}{0.75pt}\vspace{1pt} \\ 
\noindent {def {\bf EulerParameters2G}}({\it eulerParameters})
\setlength{\itemindent}{0.7cm}
\begin{itemize}[leftmargin=0.7cm]
  \item[--]  {\bf function description}: convert Euler parameters (ep) to G-matrix (=$\partial \tomega  / \partial \pv_t$)  \item[--]  {\bf input}: vector of 4 eulerParameters as list or np.array  \item[--]  {\bf output}: 3x4 matrix G as np.array\vspace{12pt}\end{itemize}
%
\noindent\rule{8cm}{0.75pt}\vspace{1pt} \\ 
\noindent {def {\bf EulerParameters2GLocal}}({\it eulerParameters})
\setlength{\itemindent}{0.7cm}
\begin{itemize}[leftmargin=0.7cm]
  \item[--]  {\bf function description}: convert Euler parameters (ep) to local G-matrix (=$\partial \LU{b}{\tomega} / \partial \pv_t$)  \item[--]  {\bf input}: vector of 4 eulerParameters as list or np.array  \item[--]  {\bf output}: 3x4 matrix G as np.array\vspace{12pt}\end{itemize}
%
\noindent\rule{8cm}{0.75pt}\vspace{1pt} \\ 
\noindent {def {\bf EulerParameters2RotationMatrix}}({\it eulerParameters})
\setlength{\itemindent}{0.7cm}
\begin{itemize}[leftmargin=0.7cm]
  \item[--]  {\bf function description}: compute rotation matrix from eulerParameters  \item[--]  {\bf input}: vector of 4 eulerParameters as list or np.array  \item[--]  {\bf output}: 3x3 rotation matrix as np.array\vspace{12pt}\end{itemize}
%
\noindent\rule{8cm}{0.75pt}\vspace{1pt} \\ 
\noindent {def {\bf RotationMatrix2EulerParameters}}({\it rotationMatrix})
\setlength{\itemindent}{0.7cm}
\begin{itemize}[leftmargin=0.7cm]
  \item[--]  {\bf function description}: compute Euler parameters from given rotation matrix  \item[--]  {\bf input}: 3x3 rotation matrix as list of lists or as np.array  \item[--]  {\bf output}: vector of 4 eulerParameters as np.array\vspace{12pt}\end{itemize}
%
\noindent\rule{8cm}{0.75pt}\vspace{1pt} \\ 
\noindent {def {\bf AngularVelocity2EulerParameters\_t}}({\it angularVelocity}, {\it eulerParameters})
\setlength{\itemindent}{0.7cm}
\begin{itemize}[leftmargin=0.7cm]
  \item[--]  {\bf function description}: \vspace{-6pt}
  \begin{itemize}[leftmargin=1.2cm]
\setlength{\itemindent}{-0.7cm}
    \item[] compute time derivative of Euler parameters from (global) angular velocity vector
    \item[] note that for Euler parameters $\pv$, we have $\tomega=\Gm \pv_t$ ==> $\Gm^T \tomega = \Gm^T\cdot \Gm\cdot \pv_t$ ==> $\Gm^T \Gm=4(\Im_{4x4} - \pv\cdot \pv^T)\pv_t = 4 (\Im_{4x4}) \pv_t$
  \ei
  \item[--]  {\bf input}: \vspace{-6pt}
  \begin{itemize}[leftmargin=1.2cm]
\setlength{\itemindent}{-0.7cm}
    \item[] {\it angularVelocity}: 3D vector of angular velocity in global frame, as lists or as np.array
    \item[] {\it   eulerParameters}: vector of 4 eulerParameters as np.array or list
  \ei
  \item[--]  {\bf output}: vector of time derivatives of 4 eulerParameters as np.array\vspace{12pt}\end{itemize}
%
\noindent\rule{8cm}{0.75pt}\vspace{1pt} \\ 
\noindent {def {\bf RotationVector2RotationMatrix}}({\it rotationVector})
\setlength{\itemindent}{0.7cm}
\begin{itemize}[leftmargin=0.7cm]
  \item[--]  {\bf function description}: rotaton matrix from rotation vector, see appendix B in \cite{Simo1988}  \item[--]  {\bf input}: 3D rotation vector as list or np.array  \item[--]  {\bf output}: 3x3 rotation matrix as np.array\vspace{12pt}\end{itemize}
%
\noindent\rule{8cm}{0.75pt}\vspace{1pt} \\ 
\noindent {def {\bf RotationMatrix2RotationVector}}({\it rotationMatrix})
\setlength{\itemindent}{0.7cm}
\begin{itemize}[leftmargin=0.7cm]
  \item[--]  {\bf function description}: compute rotation vector from rotation matrix  \item[--]  {\bf input}: 3x3 rotation matrix as list of lists or as np.array  \item[--]  {\bf output}: vector of 3 components of rotation vector as np.array\vspace{12pt}\end{itemize}
%
\noindent\rule{8cm}{0.75pt}\vspace{1pt} \\ 
\noindent {def {\bf ComputeRotationAxisFromRotationVector}}({\it rotationVector})
\setlength{\itemindent}{0.7cm}
\begin{itemize}[leftmargin=0.7cm]
  \item[--]  {\bf function description}: compute rotation axis from given rotation vector  \item[--]  {\bf input}: 3D rotation vector as np.array  \item[--]  {\bf output}: 3D vector as np.array representing the rotation axis\vspace{12pt}\end{itemize}
%
\noindent\rule{8cm}{0.75pt}\vspace{1pt} \\ 
\noindent {def {\bf RotXYZ2RotationMatrix}}({\it rot})
\setlength{\itemindent}{0.7cm}
\begin{itemize}[leftmargin=0.7cm]
  \item[--]  {\bf function description}: compute rotation matrix from consecutive xyz rotations (Tait-Bryan angles); A=Ax*Ay*Az; rot=[rotX, rotY, rotZ]  \item[--]  {\bf input}: 3D vector of Tait-Bryan rotation parameters [X,Y,Z] in radiant  \item[--]  {\bf output}: 3x3 rotation matrix as np.array\vspace{12pt}\end{itemize}
%
\noindent\rule{8cm}{0.75pt}\vspace{1pt} \\ 
\noindent {def {\bf RotationMatrix2RotXYZ}}({\it rotationMatrix})
\setlength{\itemindent}{0.7cm}
\begin{itemize}[leftmargin=0.7cm]
  \item[--]  {\bf function description}: convert rotation matrix to xyz Euler angles (Tait-Bryan angles);  A=Ax*Ay*Az;  \item[--]  {\bf input}: 3x3 rotation matrix as list of lists or np.array  \item[--]  {\bf output}: vector of Tait-Bryan rotation parameters [X,Y,Z] (in radiant) as np.array\vspace{12pt}\end{itemize}
%
\noindent\rule{8cm}{0.75pt}\vspace{1pt} \\ 
\noindent {def {\bf AngularVelocity2RotXYZ\_t}}({\it angularVelocity}, {\it rotation})
\setlength{\itemindent}{0.7cm}
\begin{itemize}[leftmargin=0.7cm]
  \item[--]  {\bf function description}: compute time derivatives of angles RotXYZ from (global) angular velocity vector and given rotation  \item[--]  {\bf input}: \vspace{-6pt}
  \begin{itemize}[leftmargin=1.2cm]
\setlength{\itemindent}{-0.7cm}
    \item[] {\it angularVelocity}: global angular velocity vector as list or np.array
    \item[] {\it   rotation}: 3D vector of Tait-Bryan rotation parameters [X,Y,Z] in radiant
  \ei
  \item[--]  {\bf output}: time derivative of vector of Tait-Bryan rotation parameters [X,Y,Z] (in radiant) as np.array\vspace{12pt}\end{itemize}
%
\noindent\rule{8cm}{0.75pt}\vspace{1pt} \\ 
\noindent {def {\bf RotXYZ2EulerParameters}}({\it alpha})
\setlength{\itemindent}{0.7cm}
\begin{itemize}[leftmargin=0.7cm]
  \item[--]  {\bf function description}: compute four Euler parameters from given RotXYZ angles, see \cite{Henderson1977}  \item[--]  {\bf input}: alpha: 3D vector as np.array containing RotXYZ angles  \item[--]  {\bf output}: \vspace{-6pt}
  \begin{itemize}[leftmargin=1.2cm]
\setlength{\itemindent}{-0.7cm}
    \item[] 4D vector as np.array containing four Euler parameters
    \item[]           entry zero of output represent the scalar part of Euler parameters
  \ei
\vspace{12pt}\end{itemize}
%
\noindent\rule{8cm}{0.75pt}\vspace{1pt} \\ 
\noindent {def {\bf RotationMatrixX}}({\it angleRad})
\setlength{\itemindent}{0.7cm}
\begin{itemize}[leftmargin=0.7cm]
  \item[--]  {\bf function description}: compute rotation matrix w.r.t. X-axis (first axis)  \item[--]  {\bf input}: angle around X-axis in radiant  \item[--]  {\bf output}: 3x3 rotation matrix as np.array\vspace{12pt}\end{itemize}
%
\noindent\rule{8cm}{0.75pt}\vspace{1pt} \\ 
\noindent {def {\bf RotationMatrixY}}({\it angleRad})
\setlength{\itemindent}{0.7cm}
\begin{itemize}[leftmargin=0.7cm]
  \item[--]  {\bf function description}: compute rotation matrix w.r.t. Y-axis (second axis)  \item[--]  {\bf input}: angle around Y-axis in radiant  \item[--]  {\bf output}: 3x3 rotation matrix as np.array\vspace{12pt}\end{itemize}
%
\noindent\rule{8cm}{0.75pt}\vspace{1pt} \\ 
\noindent {def {\bf RotationMatrixZ}}({\it angleRad})
\setlength{\itemindent}{0.7cm}
\begin{itemize}[leftmargin=0.7cm]
  \item[--]  {\bf function description}: compute rotation matrix w.r.t. Z-axis (third axis)  \item[--]  {\bf input}: angle around Z-axis in radiant  \item[--]  {\bf output}: 3x3 rotation matrix as np.array\vspace{12pt}\end{itemize}
%
\noindent\rule{8cm}{0.75pt}\vspace{1pt} \\ 
\noindent {def {\bf HomogeneousTransformation}}({\it A}, {\it r})
\setlength{\itemindent}{0.7cm}
\begin{itemize}[leftmargin=0.7cm]
  \item[--]  {\bf function description}: compute homogeneous transformation matrix from rotation matrix A and translation vector r\vspace{12pt}\end{itemize}
%
\noindent\rule{8cm}{0.75pt}\vspace{1pt} \\ 
\noindent {def {\bf HTtranslate}}({\it r})
\setlength{\itemindent}{0.7cm}
\begin{itemize}[leftmargin=0.7cm]
  \item[--]  {\bf function description}: homogeneous transformation for translation with vector r\vspace{12pt}\end{itemize}
%
\noindent\rule{8cm}{0.75pt}\vspace{1pt} \\ 
\noindent {def {\bf HT0}}({\it })
\setlength{\itemindent}{0.7cm}
\begin{itemize}[leftmargin=0.7cm]
  \item[--]  {\bf function description}: identity homogeneous transformation:\vspace{12pt}\end{itemize}
%
\noindent\rule{8cm}{0.75pt}\vspace{1pt} \\ 
\noindent {def {\bf HTrotateX}}({\it angle})
\setlength{\itemindent}{0.7cm}
\begin{itemize}[leftmargin=0.7cm]
  \item[--]  {\bf function description}: homogeneous transformation for rotation around axis X (first axis)\vspace{12pt}\end{itemize}
%
\noindent\rule{8cm}{0.75pt}\vspace{1pt} \\ 
\noindent {def {\bf HTrotateY}}({\it angle})
\setlength{\itemindent}{0.7cm}
\begin{itemize}[leftmargin=0.7cm]
  \item[--]  {\bf function description}: homogeneous transformation for rotation around axis X (first axis)\vspace{12pt}\end{itemize}
%
\noindent\rule{8cm}{0.75pt}\vspace{1pt} \\ 
\noindent {def {\bf HTrotateZ}}({\it angle})
\setlength{\itemindent}{0.7cm}
\begin{itemize}[leftmargin=0.7cm]
  \item[--]  {\bf function description}: homogeneous transformation for rotation around axis X (first axis)\vspace{12pt}\end{itemize}
%
\noindent\rule{8cm}{0.75pt}\vspace{1pt} \\ 
\noindent {def {\bf HT2translation}}({\it T})
\setlength{\itemindent}{0.7cm}
\begin{itemize}[leftmargin=0.7cm]
  \item[--]  {\bf function description}: return translation part of homogeneous transformation\vspace{12pt}\end{itemize}
%
\noindent\rule{8cm}{0.75pt}\vspace{1pt} \\ 
\noindent {def {\bf HT2rotationMatrix}}({\it T})
\setlength{\itemindent}{0.7cm}
\begin{itemize}[leftmargin=0.7cm]
  \item[--]  {\bf function description}: return rotation matrix of homogeneous transformation\vspace{12pt}\end{itemize}
%
\noindent\rule{8cm}{0.75pt}\vspace{1pt} \\ 
\noindent {def {\bf InverseHT}}({\it T})
\setlength{\itemindent}{0.7cm}
\begin{itemize}[leftmargin=0.7cm]
  \item[--]  {\bf function description}: return inverse homogeneous transformation such that inv(T)*T = np.eye(4)\vspace{12pt}\end{itemize}
%
\noindent\rule{8cm}{0.75pt}\vspace{1pt} \\ 
\noindent {def {\bf AddRigidBody}}({\it mainSys}, {\it inertia}, {\it nodeType}, {\it position}=[0,0,0], {\it velocity}=[0,0,0], {\it rotationMatrix}=[], {\it rotationParameters}=[], {\it angularVelocity}=[0,0,0], {\it gravity}=[0,0,0], {\it graphicsDataList}=[])
\setlength{\itemindent}{0.7cm}
\begin{itemize}[leftmargin=0.7cm]
  \item[--]  {\bf function description}: \vspace{-6pt}
  \begin{itemize}[leftmargin=1.2cm]
\setlength{\itemindent}{-0.7cm}
    \item[] adds a node (with str(exu.NodeType. ...)) and body for a given rigid body
    \item[] either the initial rotation is given by the rotationMatrix (while rotationParameters=[]) or by rotationParameters (while rotationMatrix=[]) (non empty)
    \item[] position ... initial position, etc.
    \item[] all quantities (esp. velocity and angular velocity) are given in global coordinates!
    \item[] returns node number and body number
    \item[] adds gravity force, i.e., m*gravity
  \ei
\vspace{12pt}\end{itemize}
%
\mysubsection{Module: lieGroupBasics}
\noindent {def {\bf Sinc}}({\it x})
\setlength{\itemindent}{0.7cm}
\begin{itemize}[leftmargin=0.7cm]
  \item[--]  {\bf function description}: compute the cardinal sine function in radians  \item[--]  {\bf input}: scalar float or int value  \item[--]  {\bf output}: float value in radians\vspace{12pt}\end{itemize}
%
\noindent\rule{8cm}{0.75pt}\vspace{1pt} \\ 
\noindent {def {\bf Cot}}({\it x})
\setlength{\itemindent}{0.7cm}
\begin{itemize}[leftmargin=0.7cm]
  \item[--]  {\bf function description}: compute the cotangent function cot(x)=1/tan(x) in radians  \item[--]  {\bf input}: scalar float or int value  \item[--]  {\bf output}: float value in radians\vspace{12pt}\end{itemize}
%
\noindent\rule{8cm}{0.75pt}\vspace{1pt} \\ 
\noindent {def {\bf R3xSO3Matrix2RotationMatrix}}({\it G})
\setlength{\itemindent}{0.7cm}
\begin{itemize}[leftmargin=0.7cm]
  \item[--]  {\bf function description}: computes 3x3 rotation matrix from 7x7 R3xSO(3) matrix, see \cite{Bruels2011}  \item[--]  {\bf input}: G: 7x7 matrix as np.array  \item[--]  {\bf output}: 3x3 rotation matrix as np.array\vspace{12pt}\end{itemize}
%
\noindent\rule{8cm}{0.75pt}\vspace{1pt} \\ 
\noindent {def {\bf R3xSO3Matrix2Translation}}({\it G})
\setlength{\itemindent}{0.7cm}
\begin{itemize}[leftmargin=0.7cm]
  \item[--]  {\bf function description}: computes translation part of R3xSO(3) matrix, see \cite{Bruels2011}  \item[--]  {\bf input}: G: 7x7 matrix as np.array  \item[--]  {\bf output}: 3D vector as np.array containg translational part of R3xSO(3)\vspace{12pt}\end{itemize}
%
\noindent\rule{8cm}{0.75pt}\vspace{1pt} \\ 
\noindent {def {\bf R3xSO3Matrix}}({\it x}, {\it R})
\setlength{\itemindent}{0.7cm}
\begin{itemize}[leftmargin=0.7cm]
  \item[--]  {\bf function description}: builds 7x7 matrix as element of the Lie group R3xSO(3), see \cite{Bruels2011}  \item[--]  {\bf input}: \vspace{-6pt}
  \begin{itemize}[leftmargin=1.2cm]
\setlength{\itemindent}{-0.7cm}
    \item[] {\it x}: 3D vector as np.array representing the translation part corresponding to R3
    \item[] {\it    R}: 3x3 rotation matrix as np.array
  \ei
  \item[--]  {\bf output}: 7x7 matrix as np.array\vspace{12pt}\end{itemize}
%
\noindent\rule{8cm}{0.75pt}\vspace{1pt} \\ 
\noindent {def {\bf ExpSO3}}({\it Omega})
\setlength{\itemindent}{0.7cm}
\begin{itemize}[leftmargin=0.7cm]
  \item[--]  {\bf function description}: compute the matrix exponential map on the Lie group SO(3), see \cite{Mueller2017}  \item[--]  {\bf input}: 3D rotation vector as np.array  \item[--]  {\bf output}: 3x3 matrix as np.array\vspace{12pt}\end{itemize}
%
\noindent\rule{8cm}{0.75pt}\vspace{1pt} \\ 
\noindent {def {\bf ExpS3}}({\it Omega})
\setlength{\itemindent}{0.7cm}
\begin{itemize}[leftmargin=0.7cm]
  \item[--]  {\bf function description}: compute the quaternion exponential map on the Lie group S(3), see \cite{Terze2016, Mueller2017}  \item[--]  {\bf input}: 3D rotation vector as np.array  \item[--]  {\bf output}: \vspace{-6pt}
  \begin{itemize}[leftmargin=1.2cm]
\setlength{\itemindent}{-0.7cm}
    \item[] 4D vector as np.array containing four Euler parameters
    \item[]           entry zero of output represent the scalar part of Euler parameters
  \ei
\vspace{12pt}\end{itemize}
%
\noindent\rule{8cm}{0.75pt}\vspace{1pt} \\ 
\noindent {def {\bf LogSO3}}({\it R})
\setlength{\itemindent}{0.7cm}
\begin{itemize}[leftmargin=0.7cm]
  \item[--]  {\bf function description}: compute the matrix logarithmic map on the Lie group SO(3), see \cite{Sonneville2014, Sonneville2017}  \item[--]  {\bf input}: 3x3 rotation matrix as np.array  \item[--]  {\bf output}: 3x3 skew symmetric matrix as np.array\vspace{12pt}\end{itemize}
%
\noindent\rule{8cm}{0.75pt}\vspace{1pt} \\ 
\noindent {def {\bf TExpSO3}}({\it Omega})
\setlength{\itemindent}{0.7cm}
\begin{itemize}[leftmargin=0.7cm]
  \item[--]  {\bf function description}: compute the tangent operator corresponding to ExpSO3, see \cite{Bruels2011}  \item[--]  {\bf input}: 3D rotation vector as np.array  \item[--]  {\bf output}: 3x3 matrix as np.array\vspace{12pt}\end{itemize}
%
\noindent\rule{8cm}{0.75pt}\vspace{1pt} \\ 
\noindent {def {\bf TExpSO3Inv}}({\it Omega})
\setlength{\itemindent}{0.7cm}
\begin{itemize}[leftmargin=0.7cm]
  \item[--]  {\bf function description}: \vspace{-6pt}
  \begin{itemize}[leftmargin=1.2cm]
\setlength{\itemindent}{-0.7cm}
    \item[] compute the inverse of the tangent operator TExpSO3, see \cite{Sonneville2014}
    \item[]             this function was improved, see coordinateMaps.pdf by Stefan Holzinger
  \ei
  \item[--]  {\bf input}: 3D rotation vector as np.array  \item[--]  {\bf output}: 3x3 matrix as np.array\vspace{12pt}\end{itemize}
%
\noindent\rule{8cm}{0.75pt}\vspace{1pt} \\ 
\noindent {def {\bf ExpSE3}}({\it x})
\setlength{\itemindent}{0.7cm}
\begin{itemize}[leftmargin=0.7cm]
  \item[--]  {\bf function description}: compute the matrix exponential map on the Lie group SE(3), see \cite{Bruels2011}  \item[--]  {\bf input}: 6D incremental motion vector as np.array  \item[--]  {\bf output}: 4x4 homogeneous transformation matrix as np.array\vspace{12pt}\end{itemize}
%
\noindent\rule{8cm}{0.75pt}\vspace{1pt} \\ 
\noindent {def {\bf LogSE3}}({\it H})
\setlength{\itemindent}{0.7cm}
\begin{itemize}[leftmargin=0.7cm]
  \item[--]  {\bf function description}: compute the matrix logarithm on the Lie group SE(3), see \cite{Sonneville2014}  \item[--]  {\bf input}: 4x4 homogeneous transformation matrix as np.array  \item[--]  {\bf output}: 4x4 skew symmetric matrix as np.array\vspace{12pt}\end{itemize}
%
\noindent\rule{8cm}{0.75pt}\vspace{1pt} \\ 
\noindent {def {\bf TExpSE3}}({\it x})
\setlength{\itemindent}{0.7cm}
\begin{itemize}[leftmargin=0.7cm]
  \item[--]  {\bf function description}: compute the tangent operator corresponding to ExpSE3, see \cite{Bruels2011}  \item[--]  {\bf input}: 6D incremental motion vector as np.array  \item[--]  {\bf output}: 6x6 matrix as np.array\vspace{12pt}\end{itemize}
%
\noindent\rule{8cm}{0.75pt}\vspace{1pt} \\ 
\noindent {def {\bf TExpSE3Inv}}({\it x})
\setlength{\itemindent}{0.7cm}
\begin{itemize}[leftmargin=0.7cm]
  \item[--]  {\bf function description}: compute the inverse of tangent operator TExpSE3, see \cite{Sonneville2014}  \item[--]  {\bf input}: 6D incremental motion vector as np.array  \item[--]  {\bf output}: 6x6 matrix as np.array\vspace{12pt}\end{itemize}
%
\noindent\rule{8cm}{0.75pt}\vspace{1pt} \\ 
\noindent {def {\bf ExpR3xSO3}}({\it x})
\setlength{\itemindent}{0.7cm}
\begin{itemize}[leftmargin=0.7cm]
  \item[--]  {\bf function description}: compute the matrix exponential map on the Lie group R3xSO(3), see \cite{Bruels2011}  \item[--]  {\bf input}: 6D incremental motion vector as np.array  \item[--]  {\bf output}: 7x7 matrix as np.array\vspace{12pt}\end{itemize}
%
\noindent\rule{8cm}{0.75pt}\vspace{1pt} \\ 
\noindent {def {\bf TExpR3xSO3}}({\it x})
\setlength{\itemindent}{0.7cm}
\begin{itemize}[leftmargin=0.7cm]
  \item[--]  {\bf function description}: compute the tangent operator corresponding to ExpR3xSO3, see \cite{Bruels2011}  \item[--]  {\bf input}: 6D incremental motion vector as np.array  \item[--]  {\bf output}: 6x6 matrix as np.array\vspace{12pt}\end{itemize}
%
\noindent\rule{8cm}{0.75pt}\vspace{1pt} \\ 
\noindent {def {\bf TExpR3xSO3Inv}}({\it x})
\setlength{\itemindent}{0.7cm}
\begin{itemize}[leftmargin=0.7cm]
  \item[--]  {\bf function description}: compute the inverse of tangent operator TExpR3xSO3  \item[--]  {\bf input}: 6D incremental motion vector as np.array  \item[--]  {\bf output}: 6x6 matrix as np.array\vspace{12pt}\end{itemize}
%
\noindent\rule{8cm}{0.75pt}\vspace{1pt} \\ 
\noindent {def {\bf CompositionRuleDirectProductR3AndS3}}({\it q0}, {\it incrementalMotionVector})
\setlength{\itemindent}{0.7cm}
\begin{itemize}[leftmargin=0.7cm]
  \item[--]  {\bf function description}: compute composition operation for pairs in the Lie group R3xS3  \item[--]  {\bf input}: \vspace{-6pt}
  \begin{itemize}[leftmargin=1.2cm]
\setlength{\itemindent}{-0.7cm}
    \item[] {\it q0}: 7D vector as np.array containing position coordinates and Euler parameters
    \item[] {\it   incrementalMotionVector}: 6D incremental motion vector as np.array
  \ei
  \item[--]  {\bf output}: 7D vector as np.array containing composed position coordinates and composed Euler parameters\vspace{12pt}\end{itemize}
%
\noindent\rule{8cm}{0.75pt}\vspace{1pt} \\ 
\noindent {def {\bf CompositionRuleSemiDirectProductR3AndS3}}({\it q0}, {\it incrementalMotionVector})
\setlength{\itemindent}{0.7cm}
\begin{itemize}[leftmargin=0.7cm]
  \item[--]  {\bf function description}: compute composition operation for pairs in the Lie group R3 semiTimes S3 (corresponds to SE(3))  \item[--]  {\bf input}: \vspace{-6pt}
  \begin{itemize}[leftmargin=1.2cm]
\setlength{\itemindent}{-0.7cm}
    \item[] {\it q0}: 7D vector as np.array containing position coordinates and Euler parameters
    \item[] {\it   incrementalMotionVector}: 6D incremental motion vector as np.array
  \ei
  \item[--]  {\bf output}: 7D vector as np.array containing composed position coordinates and composed Euler parameters\vspace{12pt}\end{itemize}
%
\noindent\rule{8cm}{0.75pt}\vspace{1pt} \\ 
\noindent {def {\bf CompositionRuleDirectProductR3AndR3RotVec}}({\it q0}, {\it incrementalMotionVector})
\setlength{\itemindent}{0.7cm}
\begin{itemize}[leftmargin=0.7cm]
  \item[--]  {\bf function description}: \vspace{-6pt}
  \begin{itemize}[leftmargin=1.2cm]
\setlength{\itemindent}{-0.7cm}
    \item[] compute composition operation for pairs in the group obtained from the direct product of R3 and R3, see \cite{Holzinger2020}
    \item[]             the rotation vector is used as rotation parametrizations
    \item[]             this composition operation can be used in formulations which represent the translational velocities in the global (inertial) frame
  \ei
  \item[--]  {\bf input}: \vspace{-6pt}
  \begin{itemize}[leftmargin=1.2cm]
\setlength{\itemindent}{-0.7cm}
    \item[] {\it q0}: 6D vector as np.array containing position coordinates and rotation vector
    \item[] {\it   incrementalMotionVector}: 6D incremental motion vector as np.array
  \ei
  \item[--]  {\bf output}: 7D vector as np.array containing composed position coordinates and composed rotation vector\vspace{12pt}\end{itemize}
%
\noindent\rule{8cm}{0.75pt}\vspace{1pt} \\ 
\noindent {def {\bf CompositionRuleSemiDirectProductR3AndR3RotVec}}({\it q0}, {\it incrementalMotionVector})
\setlength{\itemindent}{0.7cm}
\begin{itemize}[leftmargin=0.7cm]
  \item[--]  {\bf function description}: \vspace{-6pt}
  \begin{itemize}[leftmargin=1.2cm]
\setlength{\itemindent}{-0.7cm}
    \item[] compute composition operation for pairs in the group obtained from the direct product of R3 and R3.
    \item[]             the rotation vector is used as rotation parametrizations
    \item[]             this composition operation can be used in formulations which represent the translational velocities in the local (body-attached) frame
  \ei
  \item[--]  {\bf input}: \vspace{-6pt}
  \begin{itemize}[leftmargin=1.2cm]
\setlength{\itemindent}{-0.7cm}
    \item[] {\it q0}: 6D vector as np.array containing position coordinates and rotation vector
    \item[] {\it   incrementalMotionVector}: 6D incremental motion vector as np.array
  \ei
  \item[--]  {\bf output}: 6D vector as np.array containing composed position coordinates and composed rotation vector\vspace{12pt}\end{itemize}
%
\noindent\rule{8cm}{0.75pt}\vspace{1pt} \\ 
\noindent {def {\bf CompositionRuleDirectProductR3AndR3RotXYZAngles}}({\it q0}, {\it incrementalMotionVector})
\setlength{\itemindent}{0.7cm}
\begin{itemize}[leftmargin=0.7cm]
  \item[--]  {\bf function description}: \vspace{-6pt}
  \begin{itemize}[leftmargin=1.2cm]
\setlength{\itemindent}{-0.7cm}
    \item[] compute composition operation for pairs in the group obtained from the direct product of R3 and R3.
    \item[]             Cardan-Tait/Bryan (CTB) angles are used as rotation parametrizations
    \item[]             this composition operation can be used in formulations which represent the translational velocities in the global (inertial) frame
  \ei
  \item[--]  {\bf input}: \vspace{-6pt}
  \begin{itemize}[leftmargin=1.2cm]
\setlength{\itemindent}{-0.7cm}
    \item[] {\it q0}: 6D vector as np.array containing position coordinates and Cardan-Tait/Bryan angles
    \item[] {\it   incrementalMotionVector}: 6D incremental motion vector as np.array
  \ei
  \item[--]  {\bf output}: 6D vector as np.array containing composed position coordinates and composed Cardan-Tait/Bryan angles\vspace{12pt}\end{itemize}
%
\noindent\rule{8cm}{0.75pt}\vspace{1pt} \\ 
\noindent {def {\bf CompositionRuleSemiDirectProductR3AndR3RotXYZAngles}}({\it q0}, {\it incrementalMotionVector})
\setlength{\itemindent}{0.7cm}
\begin{itemize}[leftmargin=0.7cm]
  \item[--]  {\bf function description}: \vspace{-6pt}
  \begin{itemize}[leftmargin=1.2cm]
\setlength{\itemindent}{-0.7cm}
    \item[] compute composition operation for pairs in the group obtained from the direct product of R3 and R3.
    \item[]             Cardan-Tait/Bryan (CTB) angles are used as rotation parametrizations
    \item[]             this composition operation can be used in formulations which represent the translational velocities in the local (body-attached) frame
  \ei
  \item[--]  {\bf input}: \vspace{-6pt}
  \begin{itemize}[leftmargin=1.2cm]
\setlength{\itemindent}{-0.7cm}
    \item[] {\it q0}: 6D vector as np.array containing position coordinates and Cardan-Tait/Bryan angles
    \item[] {\it   incrementalMotionVector}: 6D incremental motion vector as np.array
  \ei
  \item[--]  {\bf output}: 6D vector as np.array containing composed position coordinates and composed Cardan-Tait/Bryan angles\vspace{12pt}\end{itemize}
%
\noindent\rule{8cm}{0.75pt}\vspace{1pt} \\ 
\noindent {def {\bf CompositionRuleForEulerParameters}}({\it q}, {\it p})
\setlength{\itemindent}{0.7cm}
\begin{itemize}[leftmargin=0.7cm]
  \item[--]  {\bf function description}: \vspace{-6pt}
  \begin{itemize}[leftmargin=1.2cm]
\setlength{\itemindent}{-0.7cm}
    \item[] compute composition operation for Euler parameters (unit quaternions)
    \item[]             this composition operation is quaternion multiplication, see \cite{Terze2016}
  \ei
  \item[--]  {\bf input}: \vspace{-6pt}
  \begin{itemize}[leftmargin=1.2cm]
\setlength{\itemindent}{-0.7cm}
    \item[] {\it q}: 4D vector as np.array containing Euler parameters
    \item[] {\it   p}: 4D vector as np.array containing Euler parameters
  \ei
  \item[--]  {\bf output}: 4D vector as np.array containing composed (multiplied) Euler parameters\vspace{12pt}\end{itemize}
%
\noindent\rule{8cm}{0.75pt}\vspace{1pt} \\ 
\noindent {def {\bf CompositionRuleForRotationVectors}}({\it v0}, {\it Omega})
\setlength{\itemindent}{0.7cm}
\begin{itemize}[leftmargin=0.7cm]
  \item[--]  {\bf function description}: compute composition operation for rotation vectors v0 and Omega, see \cite{Holzinger2021}  \item[--]  {\bf input}: \vspace{-6pt}
  \begin{itemize}[leftmargin=1.2cm]
\setlength{\itemindent}{-0.7cm}
    \item[] {\it v0}: 3D rotation vector as np.array
    \item[] {\it   Omega}: 3D (incremental) rotation vector as np.array
  \ei
  \item[--]  {\bf output}: 3D vector as np.array containing composed rotation vector v\vspace{12pt}\end{itemize}
%
\noindent\rule{8cm}{0.75pt}\vspace{1pt} \\ 
\noindent {def {\bf CompositionRuleRotXYZAnglesRotationVector}}({\it alpha0}, {\it Omega})
\setlength{\itemindent}{0.7cm}
\begin{itemize}[leftmargin=0.7cm]
  \item[--]  {\bf function description}: compute composition operation for RotXYZ angles, see \cite{Holzinger2021}  \item[--]  {\bf input}: \vspace{-6pt}
  \begin{itemize}[leftmargin=1.2cm]
\setlength{\itemindent}{-0.7cm}
    \item[] {\it alpha0}: 3D vector as np.array containing RotXYZ angles
    \item[] {\it   Omega}:  3D vector as np.array containing the (incremental) rotation vector
  \ei
  \item[--]  {\bf output}: 3D vector as np.array containing composed RotXYZ angles\vspace{12pt}\end{itemize}
%
\mysubsection{Module: FEM}
\noindent {def {\bf CompressedRowSparseToDenseMatrix}}({\it sparseData})
\setlength{\itemindent}{0.7cm}
\begin{itemize}[leftmargin=0.7cm]
  \item[--]  {\bf function description}: convert zero-based sparse matrix data to dense numpy matrix  \item[--]  {\bf input}: sparseData: format (per row): [row, column, value] ==> converted into dense format  \item[--]  {\bf output}: a dense matrix as np.array\vspace{12pt}\end{itemize}
%
\noindent\rule{8cm}{0.75pt}\vspace{1pt} \\ 
\noindent {def {\bf MapSparseMatrixIndices}}({\it matrix}, {\it sorting})
\setlength{\itemindent}{0.7cm}
\begin{itemize}[leftmargin=0.7cm]
  \item[--]  {\bf function description}: resort a sparse matrix (internal CSR format) with given sorting for rows and columns; changes matrix directly! used for ANSYS matrix import\vspace{12pt}\end{itemize}
%
\noindent\rule{8cm}{0.75pt}\vspace{1pt} \\ 
\noindent {def {\bf VectorDiadicUnitMatrix3D}}({\it v})
\setlength{\itemindent}{0.7cm}
\begin{itemize}[leftmargin=0.7cm]
  \item[--]  {\bf function description}: compute diadic product of vector v and a 3D unit matrix = diadic(v,I$_{3x3}$); used for ObjectFFRF and CMS implementation\vspace{12pt}\end{itemize}
%
\noindent\rule{8cm}{0.75pt}\vspace{1pt} \\ 
\noindent {def {\bf CyclicCompareReversed}}({\it list1}, {\it list2})
\setlength{\itemindent}{0.7cm}
\begin{itemize}[leftmargin=0.7cm]
  \item[--]  {\bf function description}: compare cyclic two lists, reverse second list; return True, if any cyclic shifted lists are same, False otherwise\vspace{12pt}\end{itemize}
%
\noindent\rule{8cm}{0.75pt}\vspace{1pt} \\ 
\noindent {def {\bf AddEntryToCompressedRowSparseArray}}({\it sparseData}, {\it row}, {\it column}, {\it value})
\setlength{\itemindent}{0.7cm}
\begin{itemize}[leftmargin=0.7cm]
  \item[--]  {\bf function description}: \vspace{-6pt}
  \begin{itemize}[leftmargin=1.2cm]
\setlength{\itemindent}{-0.7cm}
    \item[] add entry to compressedRowSparse matrix, avoiding duplicates
    \item[] value is either added to existing entry (avoid duplicates) or a new entry is appended
  \ei
\vspace{12pt}\end{itemize}
%
\noindent\rule{8cm}{0.75pt}\vspace{1pt} \\ 
\noindent {def {\bf CSRtoRowsAndColumns}}({\it sparseMatrixCSR})
\setlength{\itemindent}{0.7cm}
\begin{itemize}[leftmargin=0.7cm]
  \item[--]  {\bf function description}: compute rows and columns of a compressed sparse matrix and return as tuple: (rows,columns)\vspace{12pt}\end{itemize}
%
\noindent\rule{8cm}{0.75pt}\vspace{1pt} \\ 
\noindent {def {\bf CSRtoScipySparseCSR}}({\it sparseMatrixCSR})
\setlength{\itemindent}{0.7cm}
\begin{itemize}[leftmargin=0.7cm]
  \item[--]  {\bf function description}: convert internal compressed CSR to scipy.sparse csr matrix\vspace{12pt}\end{itemize}
%
\noindent\rule{8cm}{0.75pt}\vspace{1pt} \\ 
\noindent {def {\bf ScipySparseCSRtoCSR}}({\it scipyCSR})
\setlength{\itemindent}{0.7cm}
\begin{itemize}[leftmargin=0.7cm]
  \item[--]  {\bf function description}: convert scipy.sparse csr matrix to internal compressed CSR\vspace{12pt}\end{itemize}
%
\noindent\rule{8cm}{0.75pt}\vspace{1pt} \\ 
\noindent {def {\bf ResortIndicesOfCSRmatrix}}({\it mXXYYZZ}, {\it numberOfRows})
\setlength{\itemindent}{0.7cm}
\begin{itemize}[leftmargin=0.7cm]
  \item[--]  {\bf function description}: \vspace{-6pt}
  \begin{itemize}[leftmargin=1.2cm]
\setlength{\itemindent}{-0.7cm}
    \item[] resort indices of given CSR matrix in XXXYYYZZZ format to XYZXYZXYZ format; numberOfRows must be equal to columns
    \item[] needed for import from NGsolve
  \ei
\vspace{12pt}\end{itemize}
%
\noindent\rule{8cm}{0.75pt}\vspace{1pt} \\ 
\noindent {def {\bf ConvertHexToTrigs}}({\it nodeNumbers})
\setlength{\itemindent}{0.7cm}
\begin{itemize}[leftmargin=0.7cm]
  \item[--]  {\bf function description}: \vspace{-6pt}
  \begin{itemize}[leftmargin=1.2cm]
\setlength{\itemindent}{-0.7cm}
    \item[] convert list of Hex8/C3D8  element with 8 nodes in nodeNumbers into triangle-List
    \item[] also works for Hex20 elements, but does only take the corner nodes!
  \ei
\vspace{12pt}\end{itemize}
%
\noindent\rule{8cm}{0.75pt}\vspace{1pt} \\ 
\noindent {def {\bf ConvertDenseToCompressedRowMatrix}}({\it denseMatrix})
\setlength{\itemindent}{0.7cm}
\begin{itemize}[leftmargin=0.7cm]
  \item[--]  {\bf function description}: convert numpy.array dense matrix to (internal) compressed row format\vspace{12pt}\end{itemize}
%
\noindent\rule{8cm}{0.75pt}\vspace{1pt} \\ 
\noindent {def {\bf ReadMatrixFromAnsysMMF}}({\it fileName}, {\it verbose}=False)
\setlength{\itemindent}{0.7cm}
\begin{itemize}[leftmargin=0.7cm]
  \item[--]  {\bf function description}: \vspace{-6pt}
  \begin{itemize}[leftmargin=1.2cm]
\setlength{\itemindent}{-0.7cm}
    \item[] This function reads either the mass or stiffness matrix from an Ansys
    \item[]            Matrix Market Format (MMF). The corresponding matrix can either be exported
    \item[]            as dense matrix or sparse matrix.
  \ei
  \item[--]  {\bf input}: fileName of MMF file  \item[--]  {\bf output}: internal compressed row sparse matrix (as (nrows x 3) numpy array)  \item[--]  {\bf author}: \vspace{-6pt}
  \begin{itemize}[leftmargin=1.2cm]
\setlength{\itemindent}{-0.7cm}
    \item[] Stefan Holzinger
    \item[] {\it  Note}:
    \item[]    A MMF file can be created in Ansys by placing the following APDL code inside
    \item[] {\it    the solution tree in Ansys Workbench}:
    \item[] !!!!!!!!!!!!!!!!!!!!!!!!!!!!!!!!!!!!!!!!!!!!!!!!!!!!!!!!!!!!!!!!!!!!!!!!!!!!!!
    \item[]    ! APDL code that exports sparse stiffnes and mass matrix in MMF format. If
    \item[]    ! the dense matrix is needed, replace *SMAT with *DMAT in the following
    \item[]    ! APDL code.
    \item[]    ! Export the stiffness matrix in MMF format
    \item[]    *SMAT,MatKD,D,IMPORT,FULL,file.full,STIFF
    \item[]    *EXPORT,MatKD,MMF,fileNameStiffnessMatrix,,,
    \item[]    ! Export the mass matrix in MMF format
    \item[]    *SMAT,MatMD,D,IMPORT,FULL,file.full,MASS
    \item[]    *EXPORT,MatMD,MMF,fileNameMassMatrix,,,
    \item[] !!!!!!!!!!!!!!!!!!!!!!!!!!!!!!!!!!!!!!!!!!!!!!!!!!!!!!!!!!!!!!!!!!!!!!!!!!!!!!
    \item[]  In case a lumped mass matrix is needed, place the following APDL Code inside
    \item[] {\it  the Modal Analysis Tree}:
    \item[] !!!!!!!!!!!!!!!!!!!!!!!!!!!!!!!!!!!!!!!!!!!!!!!!!!!!!!!!!!!!!!!!!!!!!!!!!!!!!!
    \item[]  ! APDL code to force Ansys to use a lumped mass formulation (if available for
    \item[]  ! used elements)
    \item[]  LUMPM, ON, , 0
    \item[] !!!!!!!!!!!!!!!!!!!!!!!!!!!!!!!!!!!!!!!!!!!!!!!!!!!!!!!!!!!!!!!!!!!!!!!!!!!!!!
    \item[] ++++++++++++++++++++++++++++++++++++++++++++++++++++++++++++++++++++++++++++++
  \ei
\vspace{12pt}\end{itemize}
%
\noindent\rule{8cm}{0.75pt}\vspace{1pt} \\ 
\noindent {def {\bf ReadMatrixDOFmappingVectorFromAnsysTxt}}({\it fileName})
\setlength{\itemindent}{0.7cm}
\begin{itemize}[leftmargin=0.7cm]
  \item[--]  {\bf function description}: \vspace{-6pt}
  \begin{itemize}[leftmargin=1.2cm]
\setlength{\itemindent}{-0.7cm}
    \item[] read sorting vector for ANSYS mass and stiffness matrices and return sorting vector as np.array
    \item[]   the file contains sorting for nodes and applies this sorting to the DOF (assuming 3 DOF per node!)
    \item[]   the resulting sorted vector is already converted to 0-based indices
  \ei
\vspace{12pt}\end{itemize}
%
\noindent\rule{8cm}{0.75pt}\vspace{1pt} \\ 
\noindent {def {\bf ReadNodalCoordinatesFromAnsysTxt}}({\it fileName}, {\it verbose}=False)
\setlength{\itemindent}{0.7cm}
\begin{itemize}[leftmargin=0.7cm]
  \item[--]  {\bf function description}: This function reads the nodal coordinates exported from Ansys.  \item[--]  {\bf input}: fileName (file name ending must be .txt!)  \item[--]  {\bf output}: nodal coordinates as numpy array  \item[--]  {\bf author}: Stefan Holzinger  \item[--]  {\bf notes}: \vspace{-6pt}
  \begin{itemize}[leftmargin=1.2cm]
\setlength{\itemindent}{-0.7cm}
    \item[] The nodal coordinates can be exported from Ansys by creating a named selection
    \item[]    of the body whos mesh should to exported by choosing its geometry. Next,
    \item[]    create a second named selcetion by using a worksheet. Add the named selection
    \item[]    that was created first into the worksheet of the second named selection.
    \item[]    Inside the working sheet, choose 'convert' and convert the first created
    \item[]    named selection to 'mesh node' (Netzknoten in german) and click on generate
    \item[]    to create the second named selection. Next, right click on the second
    \item[]    named selection tha was created and choose 'export' and save the nodal
    \item[]    coordinates as .txt file.
    \item[] ++++++++++++++++++++++++++++++++++++++++++++++++++++++++++++++++++++++++++++++
  \ei
\vspace{12pt}\end{itemize}
%
\noindent\rule{8cm}{0.75pt}\vspace{1pt} \\ 
\noindent {def {\bf ReadElementsFromAnsysTxt}}({\it fileName}, {\it verbose}=False)
\setlength{\itemindent}{0.7cm}
\begin{itemize}[leftmargin=0.7cm]
  \item[--]  {\bf function description}: This function reads the nodal coordinates exported from Ansys.  \item[--]  {\bf input}: fileName (file name ending must be .txt!)  \item[--]  {\bf output}: element connectivity as numpy array  \item[--]  {\bf author}: Stefan Holzinger  \item[--]  {\bf notes}: \vspace{-6pt}
  \begin{itemize}[leftmargin=1.2cm]
\setlength{\itemindent}{-0.7cm}
    \item[] The elements can be exported from Ansys by creating a named selection
    \item[]    of the body whos mesh should to exported by choosing its geometry. Next,
    \item[]    create a second named selcetion by using a worksheet. Add the named selection
    \item[]    that was created first into the worksheet of the second named selection.
    \item[]    Inside the worksheet, choose 'convert' and convert the first created
    \item[]    named selection to 'mesh element' (Netzelement in german) and click on generate
    \item[]    to create the second named selection. Next, right click on the second
    \item[]    named selection tha was created and choose 'export' and save the elements
    \item[]    as .txt file.
    \item[] ++++++++++++++++++++++++++++++++++++++++++++++++++++++++++++++++++++++++++++++
  \ei
\vspace{12pt}\end{itemize}
%
\mysubsubsection{CLASS ObjectFFRFinterface (in module FEM)}
\noindent\textcolor{steelblue}{{\bf class description}}:  compute terms necessary for ObjectFFRF
class used internally in FEMinterface to compute ObjectFFRF object
this class holds all data for ObjectFFRF user functions
\vspace{9pt} \\ 
\noindent \textcolor{steelblue}{def {\bf \_\_init\_\_}}({\it self}, {\it femInterface})
\setlength{\itemindent}{0.7cm}
\begin{itemize}[leftmargin=0.7cm]
  \item[--]  \textcolor{steelblue}{\bf classFunction}: \vspace{-6pt}
  \begin{itemize}[leftmargin=1.2cm]
\setlength{\itemindent}{-0.7cm}
    \item[] initialize ObjectFFRFinterface with FEMinterface class
    \item[]   initializes the ObjectFFRFinterface with nodes, modes, surface description and systemmatrices from FEMinterface
    \item[]   data is then transfered to mbs object with classFunction AddObjectFFRF(...)
  \ei
\vspace{12pt}\end{itemize}
%
\noindent\rule{8cm}{0.75pt}\vspace{1pt} \\ 
\noindent \textcolor{steelblue}{def {\bf AddObjectFFRF}}({\it self}, {\it exu}, {\it mbs}, {\it positionRef}=[0,0,0], {\it eulerParametersRef}=[1,0,0,0], {\it initialVelocity}=[0,0,0], {\it initialAngularVelocity}=[0,0,0], {\it gravity}=[0,0,0], {\it constrainRigidBodyMotion}=True, {\it massProportionalDamping}=0, {\it stiffnessProportionalDamping}=0, {\it color}=[0.1,0.9,0.1,1.])
\setlength{\itemindent}{0.7cm}
\begin{itemize}[leftmargin=0.7cm]
  \item[--]  \textcolor{steelblue}{\bf classFunction}: add according nodes, objects and constraints for FFRF object to MainSystem mbs  \item[--]  \textcolor{steelblue}{\bf input}: \vspace{-6pt}
  \begin{itemize}[leftmargin=1.2cm]
\setlength{\itemindent}{-0.7cm}
    \item[] {\it exu}: the exudyn module
    \item[] {\it   mbs}: a MainSystem object
    \item[] {\it   positionRef}: reference position of created ObjectFFRF (set in rigid body node underlying to ObjectFFRF)
    \item[] {\it   eulerParametersRef}: reference euler parameters of created ObjectFFRF (set in rigid body node underlying to ObjectFFRF)
    \item[] {\it   initialVelocity}: initial velocity of created ObjectFFRF (set in rigid body node underlying to ObjectFFRF)
    \item[] {\it   initialAngularVelocity}: initial angular velocity of created ObjectFFRF (set in rigid body node underlying to ObjectFFRF)
    \item[] {\it   gravity}: set [0,0,0] if no gravity shall be applied, or to the gravity vector otherwise
    \item[] {\it   constrainRigidBodyMotion}: set True in order to add constraint (Tisserand frame) in order to suppress rigid motion of mesh nodes
    \item[] {\it   color}: provided as list of 4 RGBA values
    \item[] add object to mbs as well as according nodes
  \ei
\vspace{12pt}\end{itemize}
%
\noindent\rule{8cm}{0.75pt}\vspace{1pt} \\ 
\noindent \textcolor{steelblue}{def {\bf UFforce}}({\it self}, {\it exu}, {\it mbs}, {\it t}, {\it q}, {\it q\_t})
\setlength{\itemindent}{0.7cm}
\begin{itemize}[leftmargin=0.7cm]
  \item[--]  \textcolor{steelblue}{\bf classFunction}: optional forceUserFunction for ObjectFFRF (per default, this user function is ignored)\vspace{12pt}\end{itemize}
%
\noindent\rule{8cm}{0.75pt}\vspace{1pt} \\ 
\noindent \textcolor{steelblue}{def {\bf UFmassGenericODE2}}({\it self}, {\it exu}, {\it mbs}, {\it t}, {\it q}, {\it q\_t})
\setlength{\itemindent}{0.7cm}
\begin{itemize}[leftmargin=0.7cm]
  \item[--]  \textcolor{steelblue}{\bf classFunction}: optional massMatrixUserFunction for ObjectFFRF (per default, this user function is ignored)\vspace{12pt}\end{itemize}
%
\mysubsubsection{CLASS ObjectFFRFreducedOrderInterface (in module FEM)}
\noindent\textcolor{steelblue}{{\bf class description}}:  compute terms necessary for ObjectFFRFreducedOrder
  class used internally in FEMinterface to compute ObjectFFRFreducedOrder dictionary
  this class holds all data for ObjectFFRFreducedOrder user functions
\vspace{9pt} \\ 
\noindent \textcolor{steelblue}{def {\bf \_\_init\_\_}}({\it self}, {\it femInterface}, {\it roundMassMatrix}=1e-13, {\it roundStiffNessMatrix}=1e-13)
\setlength{\itemindent}{0.7cm}
\begin{itemize}[leftmargin=0.7cm]
  \item[--]  \textcolor{steelblue}{\bf classFunction}: \vspace{-6pt}
  \begin{itemize}[leftmargin=1.2cm]
\setlength{\itemindent}{-0.7cm}
    \item[] initialize ObjectFFRFreducedOrderInterface with FEMinterface class
    \item[]   initializes the ObjectFFRFreducedOrderInterface with nodes, modes, surface description and reduced system matrices from FEMinterface
    \item[]   data is then transfered to mbs object with classFunction AddObjectFFRFreducedOrderWithUserFunctions(...)
  \ei
  \item[--]  \textcolor{steelblue}{\bf input}: \vspace{-6pt}
  \begin{itemize}[leftmargin=1.2cm]
\setlength{\itemindent}{-0.7cm}
    \item[] {\it femInterface}: must provide nodes, surfaceTriangles, modeBasis, massMatrix, stiffness
    \item[] {\it   roundMassMatrix}: use this value to set entries of reduced mass matrix to zero which are below the treshold
    \item[] {\it   roundStiffNessMatrix}: use this value to set entries of reduced stiffness matrix to zero which are below the treshold
  \ei
\vspace{12pt}\end{itemize}
%
\noindent\rule{8cm}{0.75pt}\vspace{1pt} \\ 
\noindent \textcolor{steelblue}{def {\bf AddObjectFFRFreducedOrderWithUserFunctions}}({\it self}, {\it exu}, {\it mbs}, {\it positionRef}=[0,0,0], {\it eulerParametersRef}=[1,0,0,0], {\it initialVelocity}=[0,0,0], {\it initialAngularVelocity}=[0,0,0], {\it gravity}=[0,0,0], {\it UFforce}=0, {\it UFmassMatrix}=0, {\it massProportionalDamping}=0, {\it stiffnessProportionalDamping}=0, {\it color}=[0.1,0.9,0.1,1.])
\setlength{\itemindent}{0.7cm}
\begin{itemize}[leftmargin=0.7cm]
  \item[--]  \textcolor{steelblue}{\bf classFunction}: add according nodes, objects and constraints for ObjectFFRFreducedOrder object to MainSystem mbs  \item[--]  \textcolor{steelblue}{\bf input}: \vspace{-6pt}
  \begin{itemize}[leftmargin=1.2cm]
\setlength{\itemindent}{-0.7cm}
    \item[] {\it exu}: the exudyn module
    \item[] {\it   mbs}: a MainSystem object
    \item[] {\it   positionRef}: reference position of created ObjectFFRFreducedOrder (set in rigid body node underlying to ObjectFFRFreducedOrder)
    \item[] {\it   eulerParametersRef}: reference euler parameters of created ObjectFFRFreducedOrder (set in rigid body node underlying to ObjectFFRFreducedOrder)
    \item[] {\it   initialVelocity}: initial velocity of created ObjectFFRFreducedOrder (set in rigid body node underlying to ObjectFFRFreducedOrder)
    \item[] {\it   initialAngularVelocity}: initial angular velocity of created ObjectFFRFreducedOrder (set in rigid body node underlying to ObjectFFRFreducedOrder)
    \item[] {\it   gravity}: set [0,0,0] if no gravity shall be applied, or to the gravity vector otherwise
    \item[] {\it   UFforce}: provide a user function, which computes the quadratic velocity vector and applied forces; usually this function reads like:\\ \texttt{def UFforceFFRFreducedOrder(t, qReduced, qReduced\_t):\\ \phantom{XXXX}return cms.UFforceFFRFreducedOrder(exu, mbs, t, qReduced, qReduced\_t)}
    \item[] {\it   UFmassMatrix}: provide a user function, which computes the quadratic velocity vector and applied forces; usually this function reads like:\\ \texttt{def UFmassFFRFreducedOrder(t, qReduced, qReduced\_t):\\  \phantom{XXXX}return cms.UFmassFFRFreducedOrder(exu, mbs, t, qReduced, qReduced\_t)}
    \item[] {\it   massProportionalDamping}: Rayleigh damping factor for mass proportional damping, added to floating frame/modal coordinates only
    \item[] {\it   stiffnessProportionalDamping}: Rayleigh damping factor for stiffness proportional damping, added to floating frame/modal coordinates only
    \item[] {\it   color}: provided as list of 4 RGBA values
  \ei
\vspace{12pt}\end{itemize}
%
\noindent\rule{8cm}{0.75pt}\vspace{1pt} \\ 
\noindent \textcolor{steelblue}{def {\bf UFmassFFRFreducedOrder}}({\it self}, {\it exu}, {\it mbs}, {\it t}, {\it qReduced}, {\it qReduced\_t})
\setlength{\itemindent}{0.7cm}
\begin{itemize}[leftmargin=0.7cm]
  \item[--]  \textcolor{steelblue}{\bf classFunction}: CMS mass matrix user function; qReduced and qReduced\_t contain the coordiantes of the rigid body node and the modal coordinates in one vector!\vspace{12pt}\end{itemize}
%
\noindent\rule{8cm}{0.75pt}\vspace{1pt} \\ 
\noindent \textcolor{steelblue}{def {\bf UFforceFFRFreducedOrder}}({\it self}, {\it exu}, {\it mbs}, {\it t}, {\it qReduced}, {\it qReduced\_t})
\setlength{\itemindent}{0.7cm}
\begin{itemize}[leftmargin=0.7cm]
  \item[--]  \textcolor{steelblue}{\bf classFunction}: CMS force matrix user function; qReduced and qReduced\_t contain the coordiantes of the rigid body node and the modal coordinates in one vector!\vspace{12pt}\end{itemize}
%
\mysubsubsection{CLASS FEMinterface (in module FEM)}
\noindent\textcolor{steelblue}{{\bf class description}}:  general interface to different FEM / mesh imports and export to EXUDYN functions
         use this class to import meshes from different meshing or FEM programs (NETGEN/NGsolve, ABAQUS, ANSYS, ..) and store it in a unique format
         do mesh operations, compute eigenmodes and reduced basis, etc.
         load/store the data efficiently with LoadFromFile(...), SaveToFile(...)  if import functions are slow
         export to EXUDYN objects
\vspace{9pt} \\ 
\noindent \textcolor{steelblue}{def {\bf \_\_init\_\_}}({\it self})
\setlength{\itemindent}{0.7cm}
\begin{itemize}[leftmargin=0.7cm]
  \item[--]  \textcolor{steelblue}{\bf classFunction}: initalize all data of the FEMinterface by, e.g., \texttt{fem = FEMinterface()}\vspace{12pt}\end{itemize}
%
\noindent\rule{8cm}{0.75pt}\vspace{1pt} \\ 
\noindent \textcolor{steelblue}{def {\bf SaveToFile}}({\it self}, {\it fileName})
\setlength{\itemindent}{0.7cm}
\begin{itemize}[leftmargin=0.7cm]
  \item[--]  \textcolor{steelblue}{\bf classFunction}: save all data (nodes, elements, ...) to a data filename; this function is much faster than the text-based import functions  \item[--]  \textcolor{steelblue}{\bf input}: use filename without ending ==> ".npy" will be added\vspace{12pt}\end{itemize}
%
\noindent\rule{8cm}{0.75pt}\vspace{1pt} \\ 
\noindent \textcolor{steelblue}{def {\bf LoadFromFile}}({\it self}, {\it fileName})
\setlength{\itemindent}{0.7cm}
\begin{itemize}[leftmargin=0.7cm]
  \item[--]  \textcolor{steelblue}{\bf classFunction}: \vspace{-6pt}
  \begin{itemize}[leftmargin=1.2cm]
\setlength{\itemindent}{-0.7cm}
    \item[] load all data (nodes, elements, ...) from a data filename previously stored with SaveToFile(...).
    \item[] this function is much faster than the text-based import functions
  \ei
  \item[--]  \textcolor{steelblue}{\bf input}: use filename without ending ==> ".npy" will be added\vspace{12pt}\end{itemize}
%
\noindent\rule{8cm}{0.75pt}\vspace{1pt} \\ 
\noindent \textcolor{steelblue}{def {\bf ImportFromAbaqusInputFile}}({\it self}, {\it fileName}, {\it typeName}='Part', {\it name}='Part-1', {\it verbose}=False)
\setlength{\itemindent}{0.7cm}
\begin{itemize}[leftmargin=0.7cm]
  \item[--]  \textcolor{steelblue}{\bf classFunction}: \vspace{-6pt}
  \begin{itemize}[leftmargin=1.2cm]
\setlength{\itemindent}{-0.7cm}
    \item[] import nodes and elements from Abaqus input file and create surface elements
    \item[] node numbers in elements are converted from 1-based indices to python's 0-based indices
    \item[] only works for Hex8, Hex20, Tet4 and Tet10 (C3D4, C3D8, C3D10, C3D20) elements
    \item[] return node numbers as numpy array
  \ei
\vspace{12pt}\end{itemize}
%
\noindent\rule{8cm}{0.75pt}\vspace{1pt} \\ 
\noindent \textcolor{steelblue}{def {\bf ReadMassMatrixFromAbaqus}}({\it self}, {\it fileName}, {\it type}='SparseRowColumnValue')
\setlength{\itemindent}{0.7cm}
\begin{itemize}[leftmargin=0.7cm]
  \item[--]  \textcolor{steelblue}{\bf classFunction}: \vspace{-6pt}
  \begin{itemize}[leftmargin=1.2cm]
\setlength{\itemindent}{-0.7cm}
    \item[] {\it read mass matrix from compressed row text format (exported from Abaqus); in order to export system matrices, write the following lines in your Abaqus input file}:
    \item[] *STEP
    \item[] *MATRIX GENERATE, STIFFNESS, MASS
    \item[] *MATRIX OUTPUT, STIFFNESS, MASS, FORMAT=COORDINATE
    \item[] *End Step
  \ei
\vspace{12pt}\end{itemize}
%
\noindent\rule{8cm}{0.75pt}\vspace{1pt} \\ 
\noindent \textcolor{steelblue}{def {\bf ReadStiffnessMatrixFromAbaqus}}({\it self}, {\it fileName}, {\it type}='SparseRowColumnValue')
\setlength{\itemindent}{0.7cm}
\begin{itemize}[leftmargin=0.7cm]
  \item[--]  \textcolor{steelblue}{\bf classFunction}: read stiffness matrix from compressed row text format (exported from Abaqus)\vspace{12pt}\end{itemize}
%
\noindent\rule{8cm}{0.75pt}\vspace{1pt} \\ 
\noindent \textcolor{steelblue}{def {\bf ImportMeshFromNGsolve}}({\it self}, {\it mesh}, {\it density}, {\it youngsModulus}, {\it poissonsRatio}, {\it verbose}=False)
\setlength{\itemindent}{0.7cm}
\begin{itemize}[leftmargin=0.7cm]
  \item[--]  \textcolor{steelblue}{\bf classFunction}: import mesh from NETGEN/NGsolve and setup mechanical problem  \item[--]  \textcolor{steelblue}{\bf input}: \vspace{-6pt}
  \begin{itemize}[leftmargin=1.2cm]
\setlength{\itemindent}{-0.7cm}
    \item[] {\it mesh}: a previously created \texttt{ngs.mesh} (NGsolve mesh, see examples)
    \item[] {\it     youngsModulus}: Young's modulus used for mechanical model
    \item[] {\it     poissonsRatio}: Poisson's ratio used for mechanical model
    \item[] {\it     density}: density used for mechanical model
    \item[] {\it     verbose}: set True to print out some status information
  \ei
  \item[--]  \textcolor{steelblue}{\bf notes}: \vspace{-6pt}
  \begin{itemize}[leftmargin=1.2cm]
\setlength{\itemindent}{-0.7cm}
    \item[] The interface to NETGEN/NGsolve has been created together with Joachim Schöberl, main developer
    \item[]   of NETGEN/NGsolve; Thank's a lot!
    \item[] {\it   download NGsolve at}: https://ngsolve.org/
    \item[]   NGsolve needs Python 3.7 (64bit) ==> use according EXUDYN version!
    \item[]   note that node/element indices in the NGsolve mesh are 1-based and need to be converted to 0-base!
  \ei
\vspace{12pt}\end{itemize}
%
\noindent\rule{8cm}{0.75pt}\vspace{1pt} \\ 
\noindent \textcolor{steelblue}{def {\bf GetMassMatrix}}({\it self}, {\it sparse}=True)
\setlength{\itemindent}{0.7cm}
\begin{itemize}[leftmargin=0.7cm]
  \item[--]  \textcolor{steelblue}{\bf classFunction}: get sparse mass matrix in according format\vspace{12pt}\end{itemize}
%
\noindent\rule{8cm}{0.75pt}\vspace{1pt} \\ 
\noindent \textcolor{steelblue}{def {\bf GetStiffnessMatrix}}({\it self}, {\it sparse}=True)
\setlength{\itemindent}{0.7cm}
\begin{itemize}[leftmargin=0.7cm]
  \item[--]  \textcolor{steelblue}{\bf classFunction}: get sparse stiffness matrix in according format\vspace{12pt}\end{itemize}
%
\noindent\rule{8cm}{0.75pt}\vspace{1pt} \\ 
\noindent \textcolor{steelblue}{def {\bf NumberOfNodes}}({\it self})
\setlength{\itemindent}{0.7cm}
\begin{itemize}[leftmargin=0.7cm]
  \item[--]  \textcolor{steelblue}{\bf classFunction}: get total number of nodes\vspace{12pt}\end{itemize}
%
\noindent\rule{8cm}{0.75pt}\vspace{1pt} \\ 
\noindent \textcolor{steelblue}{def {\bf GetNodePositionsAsArray}}({\it self})
\setlength{\itemindent}{0.7cm}
\begin{itemize}[leftmargin=0.7cm]
  \item[--]  \textcolor{steelblue}{\bf classFunction}: get node points as array; only possible, if there exists only one type of Position nodes\vspace{12pt}\end{itemize}
%
\noindent\rule{8cm}{0.75pt}\vspace{1pt} \\ 
\noindent \textcolor{steelblue}{def {\bf NumberOfCoordinates}}({\it self})
\setlength{\itemindent}{0.7cm}
\begin{itemize}[leftmargin=0.7cm]
  \item[--]  \textcolor{steelblue}{\bf classFunction}: get number of total nodal coordinates\vspace{12pt}\end{itemize}
%
\noindent\rule{8cm}{0.75pt}\vspace{1pt} \\ 
\noindent \textcolor{steelblue}{def {\bf GetNodeAtPoint}}({\it self}, {\it point}, {\it tolerance}=1e-5, {\it raiseException}=True)
\setlength{\itemindent}{0.7cm}
\begin{itemize}[leftmargin=0.7cm]
  \item[--]  \textcolor{steelblue}{\bf classFunction}: \vspace{-6pt}
  \begin{itemize}[leftmargin=1.2cm]
\setlength{\itemindent}{-0.7cm}
    \item[] get node number for node at given point, e.g. p=[0.1,0.5,-0.2], using a tolerance (+/-) if coordinates are available only with reduced accuracy
    \item[] if not found, it returns an invalid index
  \ei
\vspace{12pt}\end{itemize}
%
\noindent\rule{8cm}{0.75pt}\vspace{1pt} \\ 
\noindent \textcolor{steelblue}{def {\bf GetNodesInPlane}}({\it self}, {\it point}, {\it normal}, {\it tolerance}=1e-5)
\setlength{\itemindent}{0.7cm}
\begin{itemize}[leftmargin=0.7cm]
  \item[--]  \textcolor{steelblue}{\bf classFunction}: \vspace{-6pt}
  \begin{itemize}[leftmargin=1.2cm]
\setlength{\itemindent}{-0.7cm}
    \item[] get node numbers in plane defined by point p and (normalized) normal vector n using a tolerance for the distance to the plane
    \item[] if not found, it returns an empty list
  \ei
\vspace{12pt}\end{itemize}
%
\noindent\rule{8cm}{0.75pt}\vspace{1pt} \\ 
\noindent \textcolor{steelblue}{def {\bf GetNodesOnCircle}}({\it self}, {\it point}, {\it normal}, {\it r}, {\it tolerance}=1e-5)
\setlength{\itemindent}{0.7cm}
\begin{itemize}[leftmargin=0.7cm]
  \item[--]  \textcolor{steelblue}{\bf classFunction}: \vspace{-6pt}
  \begin{itemize}[leftmargin=1.2cm]
\setlength{\itemindent}{-0.7cm}
    \item[] get node numbers on a circle, by point p, (normalized) normal vector n (which is the axis of the circle) and radius r
    \item[] using a tolerance for the distance to the plane
    \item[] if not found, it returns an empty list
  \ei
\vspace{12pt}\end{itemize}
%
\noindent\rule{8cm}{0.75pt}\vspace{1pt} \\ 
\noindent \textcolor{steelblue}{def {\bf GetSurfaceTriangles}}({\it self})
\setlength{\itemindent}{0.7cm}
\begin{itemize}[leftmargin=0.7cm]
  \item[--]  \textcolor{steelblue}{\bf classFunction}: return surface trigs as node number list (for drawing in EXUDYN)\vspace{12pt}\end{itemize}
%
\noindent\rule{8cm}{0.75pt}\vspace{1pt} \\ 
\noindent \textcolor{steelblue}{def {\bf VolumeToSurfaceElements}}({\it self}, {\it verbose}=False)
\setlength{\itemindent}{0.7cm}
\begin{itemize}[leftmargin=0.7cm]
  \item[--]  \textcolor{steelblue}{\bf classFunction}: \vspace{-6pt}
  \begin{itemize}[leftmargin=1.2cm]
\setlength{\itemindent}{-0.7cm}
    \item[] generate surface elements from volume elements
    \item[] stores the surface in self.surface
    \item[] only works for one element list and one type ('Hex8') of elements
  \ei
\vspace{12pt}\end{itemize}
%
\noindent\rule{8cm}{0.75pt}\vspace{1pt} \\ 
\noindent \textcolor{steelblue}{def {\bf GetGyroscopicMatrix}}({\it self}, {\it rotationAxis}=2, {\it sparse}=True)
\setlength{\itemindent}{0.7cm}
\begin{itemize}[leftmargin=0.7cm]
  \item[--]  \textcolor{steelblue}{\bf classFunction}: get gyroscopic matrix in according format; rotationAxis=[0,1,2] = [x,y,z]\vspace{12pt}\end{itemize}
%
\noindent\rule{8cm}{0.75pt}\vspace{1pt} \\ 
\noindent \textcolor{steelblue}{def {\bf ScaleMassMatrix}}({\it self}, {\it factor})
\setlength{\itemindent}{0.7cm}
\begin{itemize}[leftmargin=0.7cm]
  \item[--]  \textcolor{steelblue}{\bf classFunction}: scale (=multiply) mass matrix with factor\vspace{12pt}\end{itemize}
%
\noindent\rule{8cm}{0.75pt}\vspace{1pt} \\ 
\noindent \textcolor{steelblue}{def {\bf ScaleStiffnessMatrix}}({\it self}, {\it factor})
\setlength{\itemindent}{0.7cm}
\begin{itemize}[leftmargin=0.7cm]
  \item[--]  \textcolor{steelblue}{\bf classFunction}: scale (=multiply) stiffness matrix with factor\vspace{12pt}\end{itemize}
%
\noindent\rule{8cm}{0.75pt}\vspace{1pt} \\ 
\noindent \textcolor{steelblue}{def {\bf AddElasticSupportAtNode}}({\it self}, {\it nodeNumber}, {\it springStiffness}=[1e8,1e8,1e8])
\setlength{\itemindent}{0.7cm}
\begin{itemize}[leftmargin=0.7cm]
  \item[--]  \textcolor{steelblue}{\bf classFunction}: \vspace{-6pt}
  \begin{itemize}[leftmargin=1.2cm]
\setlength{\itemindent}{-0.7cm}
    \item[] modify stiffness matrix to add elastic support (joint, etc.) to a node; nodeNumber zero based (as everywhere in the code...)
    \item[] springStiffness must have length according to the node size
  \ei
\vspace{12pt}\end{itemize}
%
\noindent\rule{8cm}{0.75pt}\vspace{1pt} \\ 
\noindent \textcolor{steelblue}{def {\bf AddNodeMass}}({\it self}, {\it nodeNumber}, {\it addedMass})
\setlength{\itemindent}{0.7cm}
\begin{itemize}[leftmargin=0.7cm]
  \item[--]  \textcolor{steelblue}{\bf classFunction}: modify mass matrix by adding a mass to a certain node, modifying directly the mass matrix\vspace{12pt}\end{itemize}
%
\noindent\rule{8cm}{0.75pt}\vspace{1pt} \\ 
\noindent \textcolor{steelblue}{def {\bf ComputeEigenmodes}}({\it self}, {\it nModes}, {\it excludeRigidBodyModes}=0, {\it useSparseSolver}=True)
\setlength{\itemindent}{0.7cm}
\begin{itemize}[leftmargin=0.7cm]
  \item[--]  \textcolor{steelblue}{\bf classFunction}: \vspace{-6pt}
  \begin{itemize}[leftmargin=1.2cm]
\setlength{\itemindent}{-0.7cm}
    \item[] compute nModes smallest eigenvalues and eigenmodes from mass and stiffnessMatrix
    \item[] store mode vector in modeBasis, but exclude a number of 'excludeRigidBodyModes' rigid body modes from modeBasis
    \item[] if excludeRigidBodyModes > 0, then the computed modes is nModes + excludeRigidBodyModes, from which excludeRigidBodyModes smallest eigenvalues are excluded
  \ei
\vspace{12pt}\end{itemize}
%
\noindent\rule{8cm}{0.75pt}\vspace{1pt} \\ 
\noindent \textcolor{steelblue}{def {\bf GetEigenFrequenciesHz}}({\it self})
\setlength{\itemindent}{0.7cm}
\begin{itemize}[leftmargin=0.7cm]
  \item[--]  \textcolor{steelblue}{\bf classFunction}: return list of eigenvalues in Hz of previously computed eigenmodes\vspace{12pt}\end{itemize}
%
\noindent\rule{8cm}{0.75pt}\vspace{1pt} \\ 
\noindent \textcolor{steelblue}{def {\bf ComputeCampbellDiagram}}({\it self}, {\it terminalFrequency}, {\it nEigenfrequencies}=10, {\it frequencySteps}=25, {\it rotationAxis}=2, {\it plotDiagram}=False, {\it verbose}=False)
\setlength{\itemindent}{0.7cm}
\begin{itemize}[leftmargin=0.7cm]
  \item[--]  \textcolor{steelblue}{\bf classFunction}: \vspace{-6pt}
  \begin{itemize}[leftmargin=1.2cm]
\setlength{\itemindent}{-0.7cm}
    \item[] compute Campbell diagram for given mechanical system
    \item[] create a first order system Axd + Bx = 0 with x= [q,qd]' and compute eigenvalues
    \item[] takes mass M, stiffness K and gyroscopic matrix G from FEMinterface
    \item[] currently only uses dense matrices, so it is limited to approx. 5000 unknowns!
  \ei
  \item[--]  \textcolor{steelblue}{\bf input}: \vspace{-6pt}
  \begin{itemize}[leftmargin=1.2cm]
\setlength{\itemindent}{-0.7cm}
    \item[] {\it terminalFrequency}: frequency in Hz, up to which the campbell diagram is computed
    \item[] {\it   nEigenfrequencies}: gives the number of computed eigenfrequencies(modes), in addition to the rigid body mode 0
    \item[] {\it   frequencySteps}: gives the number of increments (gives frequencySteps+1 total points in campbell diagram)
    \item[] {\it   rotationAxis}:[0,1,2] = [x,y,z] provides rotation axis
    \item[] {\it   plotDiagram}: if True, plots diagram for nEigenfrequencies befor terminating
    \item[] {\it   verbose}: if True, shows progress of computation
  \ei
  \item[--]  \textcolor{steelblue}{\bf output}: \vspace{-6pt}
  \begin{itemize}[leftmargin=1.2cm]
\setlength{\itemindent}{-0.7cm}
    \item[] [listFrequencies, campbellFrequencies]
    \item[] {\it   listFrequencies}: list of computed frequencies
    \item[] {\it   campbellFrequencies}: array of campbell frequencies per eigenfrequency of system
  \ei
\vspace{12pt}\end{itemize}
%
\noindent\rule{8cm}{0.75pt}\vspace{1pt} \\ 
\noindent \textcolor{steelblue}{def {\bf CheckConsistency}}({\it self})
\setlength{\itemindent}{0.7cm}
\begin{itemize}[leftmargin=0.7cm]
  \item[--]  \textcolor{steelblue}{\bf classFunction}: perform some consistency checks\vspace{12pt}\end{itemize}
%
\noindent\rule{8cm}{0.75pt}\vspace{1pt} \\ 
\noindent \textcolor{steelblue}{def {\bf ReadMassMatrixFromAnsys}}({\it self}, {\it fileName}, {\it dofMappingVectorFile}, {\it sparse}=True, {\it verbose}=False)
\setlength{\itemindent}{0.7cm}
\begin{itemize}[leftmargin=0.7cm]
  \item[--]  \textcolor{steelblue}{\bf classFunction}: read mass matrix from CSV format (exported from Ansys)\vspace{12pt}\end{itemize}
%
\noindent\rule{8cm}{0.75pt}\vspace{1pt} \\ 
\noindent \textcolor{steelblue}{def {\bf ReadStiffnessMatrixFromAnsys}}({\it self}, {\it fileName}, {\it dofMappingVectorFile}, {\it sparse}=True, {\it verbose}=False)
\setlength{\itemindent}{0.7cm}
\begin{itemize}[leftmargin=0.7cm]
  \item[--]  \textcolor{steelblue}{\bf classFunction}: read stiffness matrix from CSV format (exported from Ansys)\vspace{12pt}\end{itemize}
%
\noindent\rule{8cm}{0.75pt}\vspace{1pt} \\ 
\noindent \textcolor{steelblue}{def {\bf ReadNodalCoordinatesFromAnsys}}({\it self}, {\it fileName}, {\it verbose}=False)
\setlength{\itemindent}{0.7cm}
\begin{itemize}[leftmargin=0.7cm]
  \item[--]  \textcolor{steelblue}{\bf classFunction}: read nodal coordinates (exported from Ansys as .txt-File)\vspace{12pt}\end{itemize}
%
\noindent\rule{8cm}{0.75pt}\vspace{1pt} \\ 
\noindent \textcolor{steelblue}{def {\bf ReadElementsFromAnsys}}({\it self}, {\it fileName}, {\it verbose}=False)
\setlength{\itemindent}{0.7cm}
\begin{itemize}[leftmargin=0.7cm]
  \item[--]  \textcolor{steelblue}{\bf classFunction}: read elements (exported from Ansys as .txt-File)\vspace{12pt}\end{itemize}
%
\mysubsection{Module: plot}
\noindent {def {\bf PlotSensor}}({\it mbs}, {\it sensorNumbers}, {\it components}=0, {\it **kwargs})
\setlength{\itemindent}{0.7cm}
\begin{itemize}[leftmargin=0.7cm]
  \item[--]  {\bf function description}: helper for matplotlib in order to easily visualize sensor output  \item[--]  {\bf input}: \vspace{-6pt}
  \begin{itemize}[leftmargin=1.2cm]
\setlength{\itemindent}{-0.7cm}
    \item[] {\it mbs}: must be a valid MainSystem (mbs)
    \item[] {\it   sensorNumbers}: consists of one or a list of sensor numbers (type SensorIndex) as returned by the mbs function AddSensor(...)
    \item[] {\it   components}: consists of one or a list of components according to the component of the sensor to be plotted;
    \item[] {\it   *kwargs}: additional options, e.g.:
    \item[]         xLabel -> string for text at x-axis
    \item[]         yLabel -> string for text at y-axis
  \ei
  \item[--]  {\bf output}: plots the sensor data  \item[--]  {\bf example}: \vspace{-6pt}
  \begin{itemize}[leftmargin=1.2cm]
\setlength{\itemindent}{-0.7cm}
    \item[] s0=mbs.AddSensor(SensorNode(nodeNumber=0))
    \item[]   s1=mbs.AddSensor(SensorNode(nodeNumber=1))
    \item[]   Plot(mbs, s0, 0)
    \item[]   Plot(mbs, sensorNumbers=[s0,s1], components=[0,2], xlabel='time in seconds')
  \ei
\vspace{12pt}\end{itemize}
%
\mysubsection{Module: processing}
\noindent {def {\bf ProcessParameterList}}({\it parameterFunction}, {\it parameterList}, {\it addComputationIndex}, {\it useMultiProcessing}, {\it **kwargs})
\setlength{\itemindent}{0.7cm}
\begin{itemize}[leftmargin=0.7cm]
  \item[--]  {\bf function description}: processes parameterFunction for given parameters in parameterList, see ParameterVariation  \item[--]  {\bf input}: \vspace{-6pt}
  \begin{itemize}[leftmargin=1.2cm]
\setlength{\itemindent}{-0.7cm}
    \item[] {\it parameterFunction}: function, which takes the form parameterFunction(parameterDict) and which returns any values that can be stored in a list (e.g., a floating point number)
    \item[] {\it     parameterList}: list of parameter sets (as dictionaries) which are fed into the parameter variation, e.g., [{'mass': 10}, {'mass':20}, ...]
    \item[] {\it     addComputationIndex}: if True, key 'computationIndex' is added to every parameterDict in the call to parameterFunction(), which allows to generate independent output files for every parameter, etc.
    \item[] {\it     useMultiProcessing}: if True, the multiprocessing lib is used for parallelized computation; WARNING: be aware that the function does not check if your function runs independently; DO NOT use GRAPHICS and DO NOT write to same output files, etc.!
    \item[] {\it     numberOfThreads}: default: same as number of cpus (threads); used for multiprocessing lib;
  \ei
  \item[--]  {\bf output}: returns values containing the results according to parameterList  \item[--]  {\bf notes}: options are passed from Parametervariation\vspace{12pt}\end{itemize}
%
\noindent\rule{8cm}{0.75pt}\vspace{1pt} \\ 
\noindent {def {\bf ParameterVariation}}({\it parameterFunction}, {\it parameters}, {\it **kwargs})
\setlength{\itemindent}{0.7cm}
\begin{itemize}[leftmargin=0.7cm]
  \item[--]  {\bf function description}: \vspace{-6pt}
  \begin{itemize}[leftmargin=1.2cm]
\setlength{\itemindent}{-0.7cm}
    \item[] calls successively the function parameterFunction(parameterDict) with variation of parameters in given range; parameterDict is a dictionary, containing the current values of parameters,
    \item[] {\it   e.g., parameterDict=['mass'}:13, 'stiffness':12000] to be computed and returns a value or a list of values which is then stored for each parameter
  \ei
  \item[--]  {\bf input}: \vspace{-6pt}
  \begin{itemize}[leftmargin=1.2cm]
\setlength{\itemindent}{-0.7cm}
    \item[] {\it parameterFunction}: function, which takes the form parameterFunction(parameterDict) and which returns any values that can be stored in a list (e.g., a floating point number)
    \item[] {\it     parameters}: given as a dictionary, consist of name and triple of (begin, end and steps) same as in np.linspace(...), e.g. 'mass':(10,50,10) for a mass varied from 10 to 50, using 10 steps
    \item[] {\it     useLogSpace}: (optional) if True, the parameters are varied at a logarithmic scale, e.g., [1, 10, 100] instead linear [1, 50.5, 100]
    \item[] {\it     debugMode}: if True, additional print out is done
    \item[] {\it     addComputationIndex}: if True, key 'computationIndex' is added to every parameterDict in the call to parameterFunction(), which allows to generate independent output files for every parameter, etc.
    \item[] {\it     useMultiProcessing}: if True, the multiprocessing lib is used for parallelized computation; WARNING: be aware that the function does not check if your function runs independently; DO NOT use GRAPHICS and DO NOT write to same output files, etc.!
    \item[] {\it     numberOfThreads}: default: same as number of cpus (threads); used for multiprocessing lib;
  \ei
  \item[--]  {\bf output}: \vspace{-6pt}
  \begin{itemize}[leftmargin=1.2cm]
\setlength{\itemindent}{-0.7cm}
    \item[] {\it returns [parameterList, values], containing, e.g., parameterList={'mass'}:[1,1,1,2,2,2,3,3,3], 'stiffness':[4,5,6, 4,5,6, 4,5,6]} and the result values of the parameter variation accoring to the parameterList,
    \item[]            values=[7,8,9 ,3,4,5, 6,7,8] (depends on solution of problem ..., can also contain tuples, etc.)
  \ei
  \item[--]  {\bf example}: ParameterVariation(parameters=\{'mass':(1,10,10), 'stiffness':(1000,10000,10)\}, parameterFunction=Test, useMultiProcessing=True)\vspace{12pt}\end{itemize}
%
\noindent\rule{8cm}{0.75pt}\vspace{1pt} \\ 
\noindent {def {\bf GeneticOptimization}}({\it objectiveFunction}, {\it parameters}, {\it **kwargs})
\setlength{\itemindent}{0.7cm}
\begin{itemize}[leftmargin=0.7cm]
  \item[--]  {\bf function description}: compute minimum of given objectiveFunction  \item[--]  {\bf input}: \vspace{-6pt}
  \begin{itemize}[leftmargin=1.2cm]
\setlength{\itemindent}{-0.7cm}
    \item[] {\it objectiveFunction}: function, which takes the form parameterFunction(parameterDict) and which returns a value or list (or numpy array) which reflects the size of the objective to be minimized
    \item[] {\it     parameters}: given as a dictionary, consist of name and tuple of (begin, end), e.g. 'mass':(10,50)
    \item[] {\it     initialPopulationSize}: number of random initial individuals (default: 100)
    \item[] {\it     numberOfGenerations}: number of generations (default: 10)
    \item[] {\it     numberOfChildren}: number childrens of surviving population (default: 8)
    \item[] {\it     useGenCrossing}: if True, the children are generated from parents by gen-crossover (default: True)
    \item[] {\it     survivingIndividuals}: number of surviving individuals after children are born (default: 8)
    \item[] {\it     rangeReductionFactor}: (not implemented yet!) reduction of mutation range relative to ranges of last step (default: 0.7)
    \item[] {\it     randomizerInitialization}: (not implemented yet!) initialize randomizer at beginning in order to get reproducible results (Default: True)
    \item[] {\it     distanceFactor}: (not implemented yet!) children only survive at a certain distance of the current range (Default: 0.1)
    \item[] {\it     debugMode}: if True, additional print out is done
    \item[] {\it     addComputationIndex}: if True, key 'computationIndex' is added to every parameterDict in the call to parameterFunction(), which allows to generate independent output files for every parameter, etc.
    \item[] {\it     useMultiProcessing}: if True, the multiprocessing lib is used for parallelized computation; WARNING: be aware that the function does not check if your function runs independently; DO NOT use GRAPHICS and DO NOT write to same output files, etc.!
    \item[] {\it     numberOfThreads}: default: same as number of cpus (threads); used for multiprocessing lib;
  \ei
  \item[--]  {\bf output}: \vspace{-6pt}
  \begin{itemize}[leftmargin=1.2cm]
\setlength{\itemindent}{-0.7cm}
    \item[] returns [optimumParameter, optimumValue, parametersAll, valuesAll], containing the optimum parameter set 'optimumParameter', optimum value 'optimumValue', the whole list of parameters parametersAll with according objective values 'valuesAll'
    \item[]            values=[7,8,9 ,3,4,5, 6,7,8] (depends on solution of problem ..., can also contain tuples, etc.)
  \ei
  \item[--]  {\bf notes}: This function is still under development and shows an experimental state!  \item[--]  {\bf example}: GeneticOptimization(objectiveFunction = fOpt, parameters=\{'mass':(1,10), 'stiffness':(1000,10000)\})\vspace{12pt}\end{itemize}
%
\noindent\rule{8cm}{0.75pt}\vspace{1pt} \\ 
\noindent {def {\bf PlotOptimizationResults}}({\it parameterList}, {\it valueList})
\setlength{\itemindent}{0.7cm}
\begin{itemize}[leftmargin=0.7cm]
  \item[--]  {\bf function description}: visualize results of optimization  \item[--]  {\bf input}: \vspace{-6pt}
  \begin{itemize}[leftmargin=1.2cm]
\setlength{\itemindent}{-0.7cm}
    \item[] {\it parameterList}: taken from output parametersAll of \texttt{GeneticOptimization}, containing a list of parameter set dictinaries
    \item[] {\it    valueList}: taken from output valuesAll of \texttt{GeneticOptimization}; containing a list of floats that result from the objective function
  \ei
  \item[--]  {\bf output}: creates a figure for every parameter in parameterList\vspace{12pt}\end{itemize}
%
