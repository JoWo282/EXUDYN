%solver

\mysubsection{Jacobian computation}

The computation of the global jacobian matrix is time consuming for the static solver or implicit time integration.
The equations are split into \SON, \FON and \AEN parts. From this structure, in the general non-symmetric case, 3 $\times$ 3 submatrices result for the jacobian.
Every submatrix of the jacobian has a certain meaning and needs to be computed individually.
Specifically, in implicit time integration the \SON $\times$ \SON term includes the (tangent) stiffness matrix and the mass matrix.

For efficient computation purpose, the elements provide a list of flags, which determine the dependencies as well as available (analytical) functions to compute the local (object) jacobian:
\bi
  \item ODE2\_ODE2 $\ldots$ derivative of ODE2 equations with respect to ODE2 variables
  \item ODE2\_ODE2\_t $\ldots$ derivative of ODE2 equations with respect to ODE2\_t (velocity) variables
  \item ODE1\_ODE1 $\ldots$ derivative of ODE1 equations with respect to ODE1 variables (NOT YET AVAILABLE)
  \item AE\_ODE2 $\ldots$ derivative of AE (algebraic) equations with respect to ODE2 variables
  \item AE\_ODE2\_t $\ldots$ derivative of AE (algebraic) equations with respect to ODE2\_t (velocity) variables (NOT YET AVAILABLE)
  \item AE\_ODE1 $\ldots$ derivative of AE (algebraic) equations with respect to ODE1 variables (NOT YET AVAILABLE)
  \item AE\_AE $\ldots$ derivative of AE (algebraic) equations with respect to AE variables
\ei
If one of these flags is set (binary; e.g.ODE2\_ODE2 + ODE2\_ODE2\_t), then the according local jacobian is computed and assembled into the global jacobian in the static or implicit dynamic solver.

Jacobians can also be supplied in analytical (function) form, which is indicated by an additional flag with the same name but an additional term '\_function', e.g.\ 'ODE2\_ODE2\_function' indicates that the derivative of ODE2 equations with respect to its ODE2 coordinates is provided in an analytical form (this is the tangent stiffness matrix).

Two {\bf object} functions are used to compute the local jacobians:
\bi
  \item \texttt{\bf ComputeJacobianODE2\_ODE2(Matrix\& jacobian, Matrix\& jacobian\_ODE2\_t)}: computes the \texttt{ODE2\_ODE2} and \texttt{ODE2\_ODE2\_t} jacobians
	\item \texttt{\bf ComputeJacobianAE(Matrix\& jacobian, Matrix\& jacobian\_AE)}: computes the \texttt{AE\_ODE2} and \texttt{AE\_AE} jacobians of the object ITSELF
\ei
Two {\bf connector} functions are used to compute the local jacobians, using \texttt{MarkerData}:
\bi
  \item \texttt{\bf ComputeJacobianODE2\_ODE2(Matrix\& jacobian, Matrix\& jacobian\_ODE2\_t, const MarkerDataStructure\& markerData)}: computes the \texttt{ODE2\_ODE2} and \texttt{ODE2\_ODE2\_t} jacobians of the connector; e.g.\ for spring-damper
	\item \texttt{\bf ComputeJacobianAE(Matrix\& jacobian, Matrix\& jacobian\_AE, const MarkerDataStructure\& markerData)}: computes the \texttt{AE\_ODE2} and \texttt{AE\_AE} jacobians of the connector; e.g.\ for coordinate constraint
\ei

The system jacobian has the structure (\SO = ODE2, \FO = ODE1, $\AE$ = AE; $\bar \fv_\SO$ = according system residual including dynamic (mass matrix) terms in time integration; $\gv_\AE$ = algebraic equations):
\be
  \bigmr
  {\frac{\partial \bar \fv_\SO}{\partial \qv_\SO}} {0} {\left(\frac{\partial \gv_\AE}{\partial \qv_\SO}\right)^T}
  {0} {\frac{\partial \fv_\FO}{\partial \qv_\FO}} {\left(\frac{\partial \gv_\AE}{\partial \qv_\FO}\right)^T}
	{\frac{\partial \gv_\AE}{\partial \qv_\SO}} {\frac{\partial \gv_\AE}{\partial \qv_\FO}} {\frac{\partial \gv_\AE}{\partial \qv_\AE}}
\ee

Two system jacobian functions are currently available:
\bi
  \item \texttt{\bf JacobianODE2RHS(temp, newton, factorODE2, factorODE2\_t, jacobian\_ODE2, jacobian\_ODE2\_t)}: compute analytical/numerical differentiation of ODE2RHS w.r.t. ODE2 and ODE2\_t coordinates; if analytical/functional version of jacobian is available and Newton flag 'useNumericalDifferentiation'=false, then the according jacobian is computed by its according function; results are 2 jacobians; the factors 'factor\_ODE2' and 'factor\_ODE2\_t' are used to scale the two jacobians; if a factor is zero, the according jacobian is not computed.
	\item \texttt{\bf JacobianAE(temp, newton, jacobian, factorODE2, velocityLevel, fillIntoSystemMatrix)}: 
		compute constraint jacobian of AE with respect to ODE2 ('fillIntoSystemMatrix'=true: also w.r.t. [ODE1] and AE) coordinates $\ra$ direct computation given by access functions; 'factorODE2' is used to scale the ODE2-part of the jacobian (to avoid postmultiplication); velocityLevel = true: velocityLevel constraints are used, if available; 'fillIntoSystemMatrix'=true: fill in both $\frac{\partial \bar \fv_\AE}{\partial \qv_\SO}$, $\frac{\partial \bar \fv_\AE}{\partial \qv_\SO}^T$ AND $\frac{\partial \bar \fv_\AE}{\partial \qv_\AE}$ at according locations into system matrix; 'fillIntoSystemMatrix'=false: (this is a temporary/WORKAROUND function):
\ei
The system jacobian functions compute the local jacobians either by means of a provided function or numerically, using the 'NumericalDifferentiation' settings of 'Newton'.

\mysubsection{Implicit trapezoidal rule solver}

This solver represents a class of solvers, which are based on the implicit trapezoidal rule. This integration includes the start value and the end value of a time step for the interpolation, thus being a trapezoidal integration rule. In some specializations, e.g. the Newmark method, the interpolation might only depend on the start value or the end value.

Most important representations of this rule:
\bi
  \item Trapezoidal rule ($=$ Newmark with $\beta = \frac 1 4$ and $\gamma = \frac 1 2$) 
	\item Newmark method
	\item Generalized-$\alpha$ method ($=$ generalized Newmark method with additional parameters
\ei

\mysubsection{Representation of coordinates and equations of motion}

Nomenclature:
\bi
	\item '\SO' $\ldots$ second order equations (usually of a mechanical system)
	\item '\FO' $\ldots$ first order equations (e.g.\ of a controller, fluid, etc.)
	\item '$\AE$' $\ldots$ algebraic equations (usually of joints)
	%
  \item $\Mm$ $\ldots$ mass matrix
  \item $\qv_\SO$ $\ldots$ 'displacement' coordinates of ODE2 equations
  \item $\dot \qv_\SO$ $\ldots$ 'velocity' coordinates of ODE2 equations
  \item $\ddot \qv_\SO$ $\ldots$ 'acceleration' coordinates of ODE2 equations
	%
  \item $\qv_\FO$ $\ldots$ coordinates of ODE1 equations
  \item $\dot \qv_\FO$ $\ldots$ 'velocity' coordinates of ODE1 equations
  \item $\qv_\AE$ $\ldots$ Lagrange multipliers
  \item $\fv_\SO$ $\ldots$ right-hand-side of ODE2 equations (except for action of joint reaction forces)
  \item $\fv_\FO$ $\ldots$ right-hand-side of ODE1 equations
	%
  \item $\gv$ $\ldots$ algebraic equations
  \item $\Km$ $\ldots$ (tangent) stiffness matrix
  \item $\Dm$ $\ldots$ damping/gyroscopic matrix
	\item $h$ $\ldots$ step size of time integration method
\ei

The equations of motion in \codeName\ are represented as 
\bea \label{eq_Newmark}
  \Mm \ddot \qv_\SO + \frac{\partial \gv}{\partial \qv_\SO^\mathrm{T}} \qv_\AE & = &\fv_\SO(\qv_\SO, \dot \qv_\SO, t) \\
  \dot \qv_\FO + \frac{\partial \gv}{\partial \qv_\FO^\mathrm{T}} \qv_\AE & = &\fv_\FO(\qv_\FO, t) \\
	\gv(\qv_\SO, \dot \qv_\SO, \qv_\FO, \qv_\AE, t) &=& 0
\eea
Note that the term $\frac{\partial \gv}{\partial \qv_\FO} \qv_\AE$ is not yet implemented, such that algebraic equations may not yet depend on \FON coordinates.

It is important to note, that for linear mechanical systems, $\fv_\SO$ becomes
\be
  \fv^{lin}_\SO = \fv^a - \Km \qv_\SO - \Dm \dot \qv_\FO
\ee
in which $\fv^a$ represents applied forces and $\Km$ and $\Dm$ become part of the system Jacobian for time integration.

\mysubsection{Newmark method}
The Newmark method obtains two parameters $\beta$ and $\gamma$. 
The main ideas are 
\bi
	\item Interpolate the displacements and the velocities linearly using the accelerations of the beginning of the time step (subindex '0') and the end of the time step (subindex 'T').
	\item Solve the system equations at the end of the time step for the unknown accelerations as well as for \FON\ and \AEN\  coordinates.
\ei

% ++++++++++++++++++++++++++++
% System equations for Newmark
\newcommand{\acc}{\av}
\newcommand{\vel}{\vv}
We abbreviate the unknown accelerations by $\ddot \qv = \acc$ and the unknown velocities $\dot \qv = \vel$. 
Thus, the equations at the end of the time step read (bring all terms to LHS):
\bea \label{eq_Nemark_acc}
  \fv^\mathrm{Newmark}_\SO &=& \Mm \acc^T_\SO + \frac{\partial \gv}{\partial \qv_\SO^\mathrm{T}} \qv^T_\AE - \fv_\SO(\qv^T_\SO, \dot \qv^T_\SO, t) = 0\\
  \fv^\mathrm{Newmark}_\FO &=& \vel^T_\FO + \frac{\partial \gv}{\partial \qv_\FO^\mathrm{T}} \qv^T_\AE - \fv_\FO(\qv^T_\FO, t) = 0\\
	\fv^\mathrm{Newmark}_\AE &=& \gv(\qv^T_\SO, \dot \qv^T_\SO, \qv^T_\FO, \qv^T_\AE, t) = 0
\eea
%
%
Within \eq{eq_Nemark_acc}, the \SON\ displacements and velocities and for \FON\ coordinates are given by
\bea \label{eq_Newmark_interpolation}
		   \qv^T_\SO & = &      \qv^0_\SO + h \dot \qv^0_\SO + h^2 (\frac 1 2 -\beta) \acc^0_\SO + h^2 \beta \acc^T_\SO \nonumber\\	
	\dot \qv^T_\SO & = & \dot \qv^0_\SO + h (1-\gamma) \acc^0_\SO + h\gamma \acc^T_\SO \nonumber\\
			 \qv^T_\FO & = & \qv^0_\FO + h (1-\gamma) \vel^0_\FO + h\gamma \vel^T_\FO
\eea
%
The unknowns for the Newton method are
\be \label{eq_Newton_unknowns}
  \qv^\mathrm{Newton} = \vr{\acc^T_\SO}{\vel^T_\FO}{\qv^T_\AE}
\ee
For the Newton method, we need to compute an update for the unknowns \eq{eq_Newton_unknowns}, using the known residual $\rv_{i-1}$ and the inverse of the Jacobian $\Jm_{i-1}$ of step $i-1$,
\be
  \qv^\mathrm{Newton}_{i} = \qv^\mathrm{Newton}_{i-1} - \Jm^{-1}\left( \qv^\mathrm{Newton}_{i-1}\right) \rv\left( \qv^\mathrm{Newton}_{i-1}\right)
%  \qv^\mathrm{Newton}_{i} = \qv^\mathrm{Newton}_{i-1} - \Jm^{-1}_{i-1} \Rm\left(\acc^T_{\SO,i-1},\,\vel^T_{\FO,i-1},\,\qv^T_{\AE,i-1} \right)
\ee

% +++++++++++++++++++++
% Jacobians for Newmark
The Jacobian has the following $3 \times 3$ structure,
\be
  \Jm = \mr{\Jm_{\SO\SO}}{\Jm_{\SO\FO}}{\Jm_{\SO\AE}}
           {\Jm_{\FO\SO}}{\Jm_{\FO\FO}}{\Jm_{\FO\AE}}
           {\Jm_{\AE\SO}}  {\Jm_{\AE\FO}}  {\Jm_{\AE\AE}}
\ee
Note that currently, all terms related to '$\FO$' are not implemented. The other terms are only evaluated in the specific jacobian computation, if according flags are set in GetAvailableJacobian(). 
Otherwise, the constraint needs to be implemented as object which can employ all kinds of coordinates, which do not depend on coordinates of markers.

The available Jacobians need to be rewritten in terms of the Newton unkowns \eqref{eq_Newton_unknowns}, and thus read
\bea
  \Jm_{\SO\SO}&=&\frac{\partial \fv^\mathrm{Newmark}_\SO}{\partial \acc_\SO^\mathrm{T}} 
							 = \frac{\partial \fv^\mathrm{Newmark}_\SO}{\partial \qv_\SO^\mathrm{T}} \frac{\qv_\SO}{\acc_\SO^\mathrm{T}} 
							   + \frac{\partial \fv^\mathrm{Newmark}_\SO}{\partial \dot \qv_\SO^\mathrm{T}} \frac{\dot \qv_\SO}{\acc_\SO^\mathrm{T}} 
							 = h^2 \beta \Km + h \gamma \Dm
							 \nonumber \\
	\Jm_{\SO\AE}&=&\frac{\partial \fv^\mathrm{Newmark}_\SO}{\partial \qv_\AE^\mathrm{T}} 
	             = \frac{\partial \gv}{\partial \qv_\SO^\mathrm{T}} \nonumber \\
	\Jm_{\AE\SO}&=&\frac{\partial \fv^\mathrm{Newmark}_\AE}{\partial \acc_\SO^\mathrm{T}} 
	             = \frac{\partial \gv}{\partial \acc_\SO^\mathrm{T}}
	             = \frac{\partial \gv}{\partial \qv_\SO^\mathrm{T}} \frac{\qv_\SO}{\acc_\SO^\mathrm{T}} + \frac{\partial \gv}{\partial \dot \qv_\SO^\mathrm{T}} \frac{\dot \qv_\SO}{\acc_\SO^\mathrm{T}}
							 = h^2 \beta \frac{\partial \gv}{\partial \qv_\SO^\mathrm{T}} 
							   + h \gamma \frac{\partial \gv}{\partial \dot \qv_\SO^\mathrm{T}}
							\nonumber \\
	\Jm_{\AE\AE}&=&\frac{\partial \fv^\mathrm{Newmark}_\AE}{\partial \qv_\AE^\mathrm{T}}
							 = \frac{\partial \gv}{\partial \qv_\AE^\mathrm{T}}
\eea
Note that the derivative $\frac{\qv_\SO}{\acc_\SO^\mathrm{T}}$ follows from the Newmark interpolation \eqref{eq_Newmark_interpolation} using the relation between $\qv^T_\SO$ and $\acc^T_\SO$. The tangent stiffness matrix $\Km$ must also include derivatives of applied forces $\fv^a$, which is currently not implemented.
Furthermore, the Jacobian is not symmetric, which could be obtained by according scaling.

%
% +++++++++++++++++++++
% Newton update formula
Once an update $\qv^\mathrm{Newton}_{i}$ has been computed, the interpolation formulas \eqref{eq_Newmark_interpolation} need to be evaluated before the next residual and Jacobian can be computed.


